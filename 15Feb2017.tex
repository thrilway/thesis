%        File: 15Feb2017.tex
%     Created: Wed Feb 15 09:00 AM 2017 E
% Last Change: Wed Feb 15 09:00 AM 2017 E
%
% arara: pdflatex
% arara: biber
% arara: pdflatex
% arara: pdflatex
\documentclass[letterpaper]{article}

\usepackage[margin=0.5in]{geometry}
\usepackage[backend=biber,style=authoryear-comp,useprefix=false]{biblatex}

\usepackage{stmaryrd}
\usepackage[]{amsmath}
\usepackage{amsfonts}
\usepackage{amssymb}
\usepackage{forest}
\usepackage{tabularx}
\usepackage{linguex}
\usepackage{centernot}

\addbibresource{Thesis.bib}
\title{On the learnability of \{*\}Resultative}
\author{Dan Milway}
\begin{document}
\maketitle
\section{Puzzles about direct acquisition}
\begin{itemize}
  \item There are two puzzles
\end{itemize}
\subsection{Resultatives and Depictives}
\begin{itemize}
  \item The string templates for resultatives and depictives are non-distinct
\end{itemize}
\ex. \textsc{Subj} V \textsc{Obj} Adj.

\begin{itemize}
  \item Some actual examples are ambiguous
\end{itemize}
\ex. 
\a. He fried the fish dry.
\a. $\approx$ He fried the fish once it was dry. (\textbf{Depictive})
\b. $\approx$ He fried the fish until it was dry. (\textbf{Resultative})
\z.
\b. She painted the barn red.
\a. $\approx$ The barn is red in her painting. (\textbf{Depictive})
\b. $\approx$ She applied a coat of red paint to the door. (\textbf{Resultative})
\z.

\begin{itemize}
  \item No obvious way, given an utterance of form \LLast, to assign it resultative or depictive interpretation.
  \item \textbf{Objection:} The learner isn't presented with the PLD in a vacuum, the context of an utterence would disambiguate.
  \item \textbf{Rejoinder:} Not necessarily:
    \begin{itemize}
      \item Displacement: We can and do talk about things outside of our direct perceptual experience.
      \item Even when we talk about the here and now, the mapping of words to extramental entities is far from obvious.
	\begin{itemize}
	  \item To read: Gleitman (1990), Carey (1984)
	\end{itemize}
      \item Two learners can have radically different nonlinguistic experiences, yet acquire the ``same'' language.
    \end{itemize}
\end{itemize}
\subsection{Resultatives and Causatives}
\begin{itemize}
  \item In +Resultative languages, resultative semantics can be expressed in two ways.
\end{itemize}
\ex. 
\a. She hammered the metal flat.
\b. She flattened the metal by hammering.

\begin{itemize}
  \item In -Resultative languages, only \Last[b] is available.
  \item Most parameters are largely an either/or choice. (cp. V-to-T raising)
  \item \Next[a] and \Next[b] ``compete'' to express Y/N questions.
\end{itemize}
\ex.
\a. \textsc{do Subj} V \textsc{Obj}? 
\b. V \textsc{Subj} \textsc{Obj}?

\begin{itemize}
  \item The learner can infer *\Last[b] from the presence of \Last[a] in the PLD and vice-versa.
  \item Not so with \LLast.
\end{itemize}
\section{Previous attempts}
\textit{*Here I need to discuss previous analyses of resultatives and how they are unlearnable*}
\end{document}


