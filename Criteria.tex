%        File: Criteria.tex
%     Created: Tue May 17 10:00 AM 2016 E
% Last Change: Tue May 17 10:00 AM 2016 E
%
% arara: pdflatex
% arara: biber
% arara: pdflatex
% arara: pdflatex
\documentclass[letterpaper]{article}

\usepackage[margin=1in]{geometry}
\usepackage[backend=biber,style=authoryear-comp,useprefix=false]{biblatex}

\usepackage{stmaryrd}
\usepackage[]{amsmath}
\usepackage{amsfonts}
\usepackage{amssymb}
\usepackage{forest}
\usepackage{tabularx}
\usepackage{linguex}
\usepackage{centernot}

\useforestlibrary{linguistics}
\forestset{tree defaults/.style={for tree={parent anchor=south, child anchor=north},every tree node/.style={align=center,anchor=north},level/.style={sibling distance=50mm/#1},baseline}}

\forestset{en/.style={parent anchor=center, child anchor=center}}
\forestset{em/.style={parent anchor=north west, child anchor=north west}}
\forestset{el/.style={parent anchor=north, child anchor=north}}

\usetikzlibrary{positioning}
\DeclareNameFormat{labelname:poss}{% Based on labelname from biblatex.def
  \ifcase\value{uniquename}%
  \usebibmacro{name:last}{#1}{#3}{#5}{#7}%
  \or
  \ifuseprefix
  {\usebibmacro{name:first-last}{#1}{#4}{#5}{#8}}
  {\usebibmacro{name:first-last}{#1}{#4}{#6}{#8}}%
  \or
  \usebibmacro{name:first-last}{#1}{#3}{#5}{#7}%
  \fi
  \usebibmacro{name:andothers}%
  \ifnumequal{\value{listcount}}{\value{liststop}}{'s}{}
}

\DeclareFieldFormat{shorthand:poss}{%
  \ifnameundef{labelname}{#1's}{#1}
}

\DeclareFieldFormat{citetitle:poss}{\mkbibemph{#1}'s}

\DeclareFieldFormat{label:poss}{#1's}

\newrobustcmd*{\posscitealias}{%
  \AtNextCite{%
    \DeclareNameAlias{labelname}{labelname:poss}%
    \DeclareFieldAlias{shorthand}{shorthand:poss}%
    \DeclareFieldAlias{citetitle}{citetitle:poss}%
    \DeclareFieldAlias{label}{label:poss}
  }
}

\newrobustcmd*{\posscite}{%
  \posscitealias%
  \textcite
}

\newrobustcmd*{\Posscite}{\bibsentence\posscite}

\newrobustcmd*{\posscites}{%
  \posscitealias%
  \textcites
}

\newcommand\quelle[1]{{%
  \unskip\nobreak\hfil\penalty50
  \hskip2em\hbox{}\nobreak\hfil#1%
  \parfillskip=0pt \finalhyphendemerits=0 \par
}
}

\bibliography{Thesis}

\begin{document}
\section{On Adjectives}
\begin{itemize}
  \item Much of my reading these past weeks has been about adjectives.
    \begin{itemize}
      \item Slightly aimless, trying to get the lay of the land.
    \end{itemize}
  \item Adjective literature seems to focus on attributive APs.
  \item Discussion of predicative and SC APs are in the copula/SC literature.
  \item A few papers seemed to have valuable insights.
\end{itemize}
\subsection{\textcite{bolinger1967adjectives}}
\begin{itemize}
  \item Two types of nominal modification: characteristic and occasional
\end{itemize}
\ex.
\a. The responsible man 
\a. $\approx$ the generally responsible man (characteristic)
\b. $\approx$ the man who is responsible for the present situation (occasional)
\z.
\b. The man responsible
\a. $\approx$ the man who is responsible for the present situation (occasional)
\b. *characteristic
\z.

\subsection{\textcite{larson1998events}}
\begin{itemize}
  \item Discussing the intersective/non-intersective ambiguity
\end{itemize}
\ex. Olga is a beautiful dancer

\begin{itemize}
  \item Proposes that nominals (like \textit{dancer}) carry event and entity variables.
  \item \textit{beautiful} can bind either variable.
  \item He discusses \textcite{bolinger1967adjectives}
    \begin{itemize}
      \item characteristic = i-level, closer to the N
      \item occasional = s-level, further from the N
    \end{itemize}
  \item \textcite{chierchia1995individual} argues that i-level predicates are inherent generics
    \begin{itemize}
      \item They have an event variable bound by a generic quantifier $\Gamma$
    \end{itemize}
  \item  Larson proposes the following structure for NPs
\end{itemize}
\ex. $\left[ \text{ ~~~~AP}_\text{s-level} \left[ \Gamma\text{e} \left[ \text{ ~~~~AP}_\text{i-level} \text{ ~~~~N ~~~~} \right] \right] \text{ ~~~~AP}_\text{s-level}  \right]$

\section{\textcite{rizzi2015notes}}
\begin{itemize}
  \item In recent work, Rizzi has been concerned with ``criterial positions,'' positions that constituents must raise to and cannot raise from.
\end{itemize}
\ex.
\a. I wonder which book Bill read \textit{t}.
\b.* Which book do you wonder \textit{t} Bill read \textit{t}?

\begin{itemize}
  \item For a WhP, Spec QuesP is the criterial position.
  \item Rizzi give a label based explanation.
    \begin{itemize}
      \item Label(X--YP) = X
      \item Label(XP--YP) =  ?
	\begin{itemize}
	  \item If XP and YP agree for some feature F, label XP--YP as F.
	  \item If they don't agree, move one XP or YP.
	\end{itemize}
    \end{itemize}
  \item Only maximal objects (XPs) can be moved.
  \item In \Last[b], the WhP and QuesP agree for feature Q, and the new object QP is the maximal projection of both.
\end{itemize}
\ex.  I wonder \dots
\begin{forest}
  tree defaults
  [Q
    [Q
      [Q\\which,align=center]
      [n
	[book]
	[n]
      ]
    ]
    [Q
      [Q]
      [I[Bill read \textit{t},roof]]
    ]
  ]
\end{forest}

\begin{itemize}
  \item For Rizzi:
    \begin{itemize}
      \item Complements can stay or move
      \item Specifiers must stay in criterial position, otherwise they must move
    \end{itemize}
  \item SCs present a problem for this.
\end{itemize}
\ex. 
\a. I consider [$_\alpha$ John intelligent].
\b. John is considered [$_\beta$ \textit{t} intelligent].
\c. A man who I consider [$_\beta$ \textit{t} intelligent]
 
\begin{itemize}
  \item Rizzi proposes that $\alpha \neq \beta$
  \item In $\alpha$, Spec is criterial.
  \item In $\beta$, Spec is not criterial.
  \item Verbs can select either or both
\end{itemize}
\ex. 
\a.* I think [$_\alpha$ John intelligent].
\b. John is thought [$_\beta$ \textit{t} intelligent].
\c. A man who I think [$_\beta$ \textit{t} intelligent]\hfill (Reported to Rizzi by Ian Roberts)

\begin{itemize}
  \item The two types of SCs are expected to have interpretive differences.
  \item Rizzi points to an Italian example for evidence of this
    \begin{itemize}
      \item Bare plurals seem to be imcompatible with the $\alpha$-type SCs
    \end{itemize}
\end{itemize}
\ex.
\a.* Gianni considera [$_\alpha$[amici][simpatici]]\\
``Gianni considers (some) friends nice''
\b. Gianni frequenta amici [che considera [$_\beta$ \textit{t} [simpatici]]\\
``Gianni sees friends that he considers nice''

\section{How this is applicable}
In my first generals paper I proposed (in different terms) that Spec SC in languages with Adjectival agreement in predicative position was criterial for resultatives.
This was untenable because movement from Spec SC was clearly allowed for depictives and also \Last[b]; agreement doesn't seem to be enough to bar movement.
Suppose, however, that mutual agreement is required for a criterial position, meaning [XP, YP] is criterial iff XP checks YP's F feature and YP checks XP's G feature.
In the case of Wh-movement, the Wh-element needs its Q-feature checked and the C$_Q$ needs its Wh feature checked.
In the case of SCs, supose, following Larson and Chierchia, that APs in Italian carry a $\Gamma$ feature, and NPs have an unvalued Quantifier/Operator feature.
In \Last[a], the BP checks the AP's phi-features, and the AP checks the BP's Op feature, yielding a criterial configuration.
Italian BPS, however, are only interpretable under existential quantification, rendering \Last[a] uninterpretable.

\printbibliography
\end{document}


