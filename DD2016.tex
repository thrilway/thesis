%        File: DD2016.tex
%     Created: Mon Aug 08 11:00 AM 2016 E
% Last Change: Mon Aug 08 11:00 AM 2016 E
%
% arara: pdflatex
% arara: biber
% arara: pdflatex
% arara: pdflatex
\documentclass[letterpaper]{article}

\usepackage[margin=1in]{geometry}
\usepackage[backend=biber,style=authoryear-comp,useprefix=false]{biblatex}

\usepackage{stmaryrd}
\usepackage[]{amsmath}
\usepackage{amsfonts}
\usepackage{amssymb}
\usepackage{forest}
\usepackage{tabularx}
\usepackage{linguex}
\usepackage{centernot}
\usepackage{adjustbox}
\newcommand*{\StrikeThruDistance}{0.15cm}%
\newcommand*{\StrikeThru}{\StrikeThruDistance,\StrikeThruDistance}%

\useforestlibrary{linguistics}


\forestset{tree defaults/.style={for tree={parent anchor=south, child anchor=north},every tree node/.style={align=center,anchor=north},level/.style={sibling distance=50mm/#1},baseline}}

\forestset{en/.style={parent anchor=center, child anchor=center}}
\forestset{em/.style={parent anchor=north west, child anchor=north west}}
\forestset{el/.style={parent anchor=north, child anchor=north}}

\usetikzlibrary{positioning}
\usetikzlibrary{calc}
\usetikzlibrary{arrows}
\usetikzlibrary{decorations.markings}

\tikzset{strike thru arrow/.style={
    decoration={markings, mark=at position 0.5 with {
        \draw [thick,-] 
            ++ (-\StrikeThruDistance,-\StrikeThruDistance) 
            -- ( \StrikeThruDistance, \StrikeThruDistance);
	\draw [thick,-]
	++ (-\StrikeThruDistance, \StrikeThruDistance)
	-- (\StrikeThruDistance, -\StrikeThruDistance);
	  }
    },
    postaction={decorate},
}}
%\DeclareNameFormat{labelname:poss}{% Based on labelname from biblatex.def
%  \ifcase\value{uniquename}%
%  \usebibmacro{name:last}{#1}{#3}{#5}{#7}%
%  \or
%  \ifuseprefix
%  {\usebibmacro{name:first-last}{#1}{#4}{#5}{#8}}
%  {\usebibmacro{name:first-last}{#1}{#4}{#6}{#8}}%
%  \or
%  \usebibmacro{name:first-last}{#1}{#3}{#5}{#7}%
%  \fi
%  \usebibmacro{name:andothers}%
%  \ifnumequal{\value{listcount}}{\value{liststop}}{'s}{}
%}
%
%\DeclareFieldFormat{shorthand:poss}{%
%  \ifnameundef{labelname}{#1's}{#1}
%}
%
%\DeclareFieldFormat{citetitle:poss}{\mkbibemph{#1}'s}
%
%\DeclareFieldFormat{label:poss}{#1's}
%
%\newrobustcmd*{\posscitealias}{%
%  \AtNextCite{%
%    \DeclareNameAlias{labelname}{labelname:poss}%
%    \DeclareFieldAlias{shorthand}{shorthand:poss}%
%    \DeclareFieldAlias{citetitle}{citetitle:poss}%
%    \DeclareFieldAlias{label}{label:poss}
%  }
%}
%
%\newrobustcmd*{\posscite}{%
%  \posscitealias%
%  \textcite
%}
%
%\newrobustcmd*{\Posscite}{\bibsentence\posscite}
%
%\newrobustcmd*{\posscites}{%
%  \posscitealias%
%  \textcites
%}
%
\newcommand\quelle[1]{{%
  \unskip\nobreak\hfil\penalty50
  \hskip2em\hbox{}\nobreak\hfil#1%
  \parfillskip=0pt \finalhyphendemerits=0 \par
}
}
\title{Subjects of Adjuncts and Labeling}
\author{Dan Milway}
\date{Dog Days Workshop\\17 August 2016}
\bibliography{Thesis}

\begin{document}
\maketitle
\section{Introduction}
This paper investigates a peculiar movement pattern shown by subjects ACC-ing clauses (ACs), which are non-finite clauses with accusative subjects and progressive main verbs.
These clauses are often embedded in direct perceptions reports as shown in \Last below.
\ex. 
\a. They saw her proving the theorem.
\b. She was seen proving the theorem.

The subjects of ACs cannot raise to the matrix clause if the AC is an argument, but must raise if the AC is an adjunct.
This pattern can be partially accounted for by assuming two label-based theories: (i) Labels are assigned algorithmically as a requirement of the Conceptual-Intentional (CI) interface \parencite{chomsky2013problems,chomsky2015problems}, and (ii) adjunction structures do not receive labels \parencite{chametzky1996theory,hornstein2009theory}.
These assumptions require a rethinking of the CI interface which can allow for a full account of the pattern.

This paper is structured as follows.
The analysis of AC subject movement patterns will be justified in section \ref{sec:analysis}.
A brief outline of label-theoretic notions will be given in \ref{sec:labels}
\section{The phenomenon: ACC-ing (and Pseudo-relative?) subjects}\label{sec:analysis}
\begin{itemize}
  \item In active sentences, the perception verb doesn't $\theta$-mark the ACC-ing subject.
  \item In passives, the perception verb does $\theta$-mark the ACC-ing subject
\end{itemize}
\ex.
\a. We heard it raining last night.
\b. We saw all hell breaking loose.
\b. We heard Jamie being slandered.

\ex.
\a.* It was heard raining last night.
\b.* All hell was seen breaking loose. (*idiomatic)
\b.* Jamie was heard being slandered.

\newpage
\begin{itemize}
  \item \textsc{theme}-marking occurs in Comp V.
  \item In actives, the AC occupies Comp V.
  \item In passives, the AC subject occupies Comp V.
    \begin{itemize}
      \item The AC is adjoined to VP.
    \end{itemize}
\end{itemize}
\begin{minipage}[t]{0.5\textwidth}
\ex. \textbf{Active/Argument ACC-ing}\\
\begin{forest}
  nice empty nodes,sn edges,baseline
  [TP
    [Mary]
    [
      [+pst]
      [\dots
	[,no edge]
	[VP
	  [see]
	  [ProgP
	    [Bill]
	    [[crying,roof]]
	  ]
	  ]
	]
      ]
    ]
\end{forest}

\end{minipage}
\begin{minipage}[t]{0.5\textwidth}
\ex. \textbf{Passive/Adjunct ACC-ing}\\
\begin{forest}
  nice empty nodes,sn edges,baseline
  [TP
  [Bill,name=fin subj]
  [
    [was]
    [\dots
      [,no edge]
      [VP
	[VP
	  [see]
	  [$\langle\text{Bill}\rangle$,name=theme]
	]
	[ProgP
	  [$\langle\text{Bill}\rangle$,name=prog subj]
	  [
	    [crying,roof]
	  ]
	]
      ]
    ]
  ]
]
\draw [->] (prog subj.south) --++ (0em,-4ex) -|(fin subj);
\draw [->] (theme.south) --++ (0em,-5ex) -|(fin subj);
\end{forest}

\end{minipage}
\begin{itemize}
  \item Strange result: \textbf{The ACC-ing subject is frozen in the Argument ACC-ing, but must move from the Adjunct ACC-ing.}
  \item cp. Raising-to-Object and Adjunct Islands
\end{itemize}
\begin{minipage}[t]{\textwidth}
  \ex. \textbf{Argument AC subjects do not move}\\
\begin{minipage}[t]{0.5\textwidth}
\begin{forest}
  nice empty nodes,sn edges
  [VoiceP
    [DP[Mary,roof]]
    [
      [Voice]
      [AgrOP
	[\textit{Obj},name=object]
	[
	  [AgrO]
	  [VP
	    [see]
	    [ProgP
	      [Bill,name=AC subj]
	      [
		[crying,roof]
	      ]
	    ]
	  ]
	]
      ]
    ]
  ]
  \draw [->,strike thru arrow] (AC subj) to[out=south west, in=south] (object);
\end{forest}
\end{minipage}
\begin{minipage}[t]{0.5\textwidth}
 \begin{forest}
  nice empty nodes,sn edges
  [TP
    [\textit{Subj},name=subject]
    [
      [T\\was,align=center]
      [\dots
	[,phantom]
	[VP
	    [see]
	    [ProgP
	      [Bill,name=AC subj]
	      [
		[crying,roof]
	      ]
	    ]
	  ]
	]
      ]
    ]
    \draw [->,strike thru arrow] (AC subj) to[out=south west, in=south] (subject);
\end{forest}
\end{minipage}

\end{minipage}
\begin{minipage}[t]{\textwidth}
  \ex. \textbf{Adjunct AC subjects must move}\\
  \begin{minipage}[t]{0.5\textwidth}
    \begin{forest}
      nice empty nodes,sn edges
      [VoiceP,name=voicep
	[DP[Mary,roof]]
	[
	  [Voice]
	  [VP
	    [VP
	      [V\\see,align=center]
	      [Jamie]
	    ]
	    [ProgP
	      [Bill,name=AC subj]
	      [
		[crying,roof]
	      ]
	    ]
	  ]
	]
      ]
      \node[left = 0.5cm of voicep] () {\Large *};
    \end{forest}
  \end{minipage}
  
\end{minipage}
\section{Label theory \parencite{chomsky2013problems,chomsky2015problems}}\label{sec:labels}
\subsection{The Theory}
\begin{itemize}
  \item The narrow syntax is (simplest) Merge
    \begin{itemize}
      \item Merge($\alpha, \beta$) = $\left\{ \alpha,\beta \right\}$
    \end{itemize}
  \item Since Merge doesn't specify the label of its output, and the narrow syntax is only Merge, labels must be determined at one of the interfaces.
  \item Specifically: The CI interface.
    \begin{itemize}
      \item Chomsky's CI primacy conclusion 
    \end{itemize}
  \item Labels are assigned by a special instance of Minimal Search, the Labelling Algorithm (LA), upon Transfer at the phase level.
  \item Unlabellable objects cause a crash.
\end{itemize}
But:
\begin{itemize}
  \item No current theory of semantics has any need for labels.
    \begin{itemize}
      \item For type-driven interpretation, only the content of syntactic objects is required
      \item For a neo-Davidsonian theory, only the merge order of arguments is required
    \end{itemize}
  \item If our current understanding of the syntax-semantics interface is correct, the proposal above must be wrong.
  \item $\therefore$ \textbf{If the proposal above is correct, our current understanding of the syntax semantics interface is wrong.}
\end{itemize}
\subsection{Labelling algorithm}
\begin{itemize}
  \item LA, when applied to a syntactic object SO, searches SO for its most prominent sub-object and assigns that as SO's label.
  \item \textcite{chomsky2013problems,chomsky2013problems} discusses the three logical possibilities:
\end{itemize}
\ex.
\a. $\text{LA}(\left\{ \text{X, YP} \right\}) = \text{X}$
\b. $\text{LA}(\left\{ \text{X, Y} \right\}) = 
	\begin{cases}
	  \text{X} & \text{if Y is a root, and X is not a root}\\
	  \text{Undefined} & \text{otherwise}
	\end{cases}
	$
\c. $\text{LA}(\left\{ \text{XP, YP} \right\}) = 
	\begin{cases}
	  \langle\text{F, F}\rangle & \text{if XP and YP agree for some feature F}\\
	  \text{LA(YP)} & \text{if XP is a lower copy in a chain}\\
	  \text{Undefined} & \text{otherwise}
	\end{cases}
	$

\section{Labelling explanation}\label{sec:explanation}
\begin{minipage}[t]{0.35\textwidth}
  \ex. \textbf{Argument ACs}
  \a. $\left\{ \text{DP, ProgP} \right\}$
  \b.* $\left\{ t, \text{ProgP} \right\}$
  \z.

\end{minipage}
\begin{minipage}[t]{0.35\textwidth}
  \ex. \textbf{Adjunct ACs}
  \a.* $\left\{ \text{DP, ProgP} \right\}$
  \b. $\left\{ t, \text{ProgP} \right\}$
  \z.

\end{minipage}

\subsection{Argument ACC-ing subjects cannot move}
\begin{itemize}
  \item Subject of argument ACs show \textbf{Criterial freezing} (in Rizzi's terms).
  \item \textcite{chomsky2015problems} proposes a labelling account for this type of freezing.
    \begin{itemize}
      \item $\left\{ \text{XP}_\text{F}, \left\{ \text{Y}_\text{F}, \text{ZP} \right\} \right\}$ is labelled $\langle\text{F,F}\rangle$.
      \item Y is ``too weak'' to label on its own.
      \item $\left\{ t, \left\{ \text{Y}_\text{F}, \text{ZP} \right\} \right\}$ is unlabellable so it yields a crash.
    \end{itemize}
  \item Replace XP with the AC subject and Y with Prog$^0$ and we have our account.
\end{itemize}
\ex.
\a.* Bill$_i$ was see-en [ $t_{see}$ [ $t_i$ throwing the ball]]
\b. 
\begin{minipage}[t]{0.5\textwidth}
  \textbf{Derivation}
\begin{enumerate}
  \item Merge(Bill, $\left\{ \text{Prog, YP} \right\}$)
  \item Transfer+Label(YP)\footnotemark  \item Merge(see, ProgP)\\
    \dots \\
    (Derive the finite clause)\\
    \dots
  \item (Internal-)Merge(Bill, T')
  \item Merge(C, TP)
  \item Transfer+Label(TP)\\
    *CRASH* \\
    ($\left\{ t, \text{Prog} \right\}$ is unlabellable.)
\end{enumerate}
\end{minipage}
\begin{minipage}[t]{0.5\textwidth}
 \begin{forest}
  baseline,nice empty nodes,sn edges
  [CP
    [C]
  [TP
    [Bill,name=subject]
    [
      [T\\was,align=center]
      [\dots
	[,phantom,ignore edge]
	[,l sep+=0.5cm
	    [see]
	    [ProgP
	      [Bill,name=AC subj]
		[Prog]
	    ]
	  ]
	]
      ]
    ]
  ]
    \draw [->] (AC subj) to[out=south west, in=south] (subject);
\end{forest}
\end{minipage}
\z.

\footnotetext{Assuming Prog$^0$ is a phase head, following \textcite{harwood2015being}}
\subsection{Adjunct ACC-ing subjects can move}
\begin{itemize}
  \item \textbf{Assumption:} Adjunction structures do not receive a label. \parencite{hornstein2009theory,chametzky1996theory}
    \begin{itemize}
	    \item If $\left\{ \text{XP, YP} \right\}$ is an adjunction structure, LA skips it, and moves on to the adjunction host.
    \end{itemize}
  \item The AC is an adjunct and thus, invisible the labelling algorithm.
  \item It follows that the internal structure of the AC is also invisible to LA.
  \item $\left\{ t, \text{Prog} \right\}$ is still unlabellable, but doesn't lead to a crash.
\begin{itemize}
	\item Crashes occur when LA fails
\end{itemize}
\end{itemize}
\subsection{Adjunct ACC-ing subjects \textit{must} move.}
\begin{itemize}
  \item If labelling is required at the CI interface, it must have some semantic potency.
  \item So $\left\{ \text{XP, YP} \right\}$ will be interpreted differently depending on its label
    \begin{itemize}
      \item Criterial: Label($\left\{ \text{XP, YP} \right\}$)= $\langle\text{F,F}\rangle \rightarrow$ Abstraction
	      \begin{itemize}
		\item Including, but not limited to, lambda abstraction \parencite[cp][]{lohndal2011interrogatives}
	      \end{itemize}
      \item Adjunct: Label($\left\{ \text{XP, YP} \right\}$)= $\emptyset \rightarrow$ Conjunction
    \end{itemize}
    \item An unlabelled $\left\{ \text{DP}, \left\{ \text{Prog, YP} \right\} \right\}$ is interpreted as the conjunction of a ProgP predicate and its subject.
	    \begin{itemize}
		    \item This is (likely) a deviant interpretation
	    \end{itemize}
    \item An unlabelled $\left\{ t, \left\{ \text{Prog, YP} \right\} \right\}$, however, does not yield a deviant interpretation.
      \begin{itemize}
	\item This is stipulated for now.
      \end{itemize}
\end{itemize}

\section{``Conclusion''}
\begin{itemize}
	\item Chomsky proposes that for a derivation to converge at CI it must produce a labellable syntactic object.
	\item If Chomsky is right, we need to rethink our conception of the syntax-semantics interface.\\
	  \textbf{My Proposal:} The label of a syntactic object has consequences for that object's interpretation.
	\item I have shown how a puzzling fact about ACC-ing subjects can be straightforwardly explained, given Chomsky's proposal (and my extension).
	\item Several aspects require more explanation/work:
	  \begin{itemize}
	    \item How does the LA ``know'' to skip adjunction structures?
	    \item How are $\left\{ t, \text{XP} \right\}$ structures interpreted?
	    \item I predict that no adjunct phrases should have criterial specifiers.
	  \end{itemize}
\end{itemize}
\printbibliography
\end{document}
