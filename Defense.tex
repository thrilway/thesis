%        File: Defense.tex
%     Created: Wed Apr 10 01:00 PM 2019 E
% Last Change: Wed Apr 10 01:00 PM 2019 E
%
% arara: pdflatex
% arara: biber
% arara: pdflatex
% arara: pdflatex
\documentclass[letterpaper,12pt]{article}

\usepackage{linguex}

\usepackage[
	backend=biber,
	style=authoryear,
	citestyle=authoryear-comp,
	language=british,
	dashed=true
]{biblatex}


\addbibresource{Thesis.bib}
\begin{document}
\begin{itemize}
	\item One of the main questions of linguistics: How much can languages vary?
	\item Empiricist answer: Languages can vary without limit.
		\begin{itemize}
			\item Generative research has refuted this answer.
			\item There are limits on possible languages.
		\end{itemize}
	\item The generativist version of the question: What aspects of grammars can vary?
	\item Various answers to this version of the question:
		\begin{itemize}
			\item Parameter theory
			\item Lexical parameterization hypothesis
			\item Only surface variation
		\end{itemize}
	\item The resultative parameter provides an empirical way of deciding between these options.
		\begin{itemize}
			\item Languages vary with respect to the number of readings assigned to secondary predication structures like \ref{ex:ambiguish}.
			\item English assigns depictive and resultative readings
			\item French assigns only depictive readings.
		\end{itemize}
\end{itemize}
\ex.\label{ex:ambiguish} The chef fried the fish dry.

\begin{itemize}
	\item *Resultative languages express resultatives periphrastically
		\begin{itemize}
			\item This option is also available to resultative languages.
		\end{itemize}
\end{itemize}
\begin{itemize}
	\item This points away from the only-surface-variation hypothesis.
		\begin{itemize}
			\item 
		\end{itemize}
\end{itemize}
\begin{itemize}
	\item \textcite{snyder2001nature} 
\end{itemize}<++>
\end{document}


