%        File: FivePager.tex
%     Created: Thu Mar 24 12:00 PM 2016 E
% Last Change: Thu Mar 24 12:00 PM 2016 E
%
% arara: pdflatex
% arara: biber
% arara: pdflatex
% arara: pdflatex
\documentclass[letterpaper]{article}

\usepackage[margin=1in]{geometry}
\usepackage[backend=biber,style=authoryear-comp,useprefix=false]{biblatex}

\usepackage{stmaryrd}
\usepackage[]{amsmath}
\usepackage{amsfonts}
\usepackage{amssymb}
\usepackage{forest}
\usepackage{tabularx}
\usepackage{linguex}
\usepackage{centernot}

\forestset{tree defaults/.style={for tree={parent anchor=south, child anchor=north},every tree node/.style={align=center,anchor=north},level/.style={sibling distance=50mm/#1},baseline}}

\forestset{en/.style={parent anchor=center, child anchor=center}}
\forestset{em/.style={parent anchor=north west, child anchor=north west}}
\forestset{el/.style={parent anchor=north, child anchor=north}}

\usetikzlibrary{positioning}
\DeclareNameFormat{labelname:poss}{% Based on labelname from biblatex.def
  \ifcase\value{uniquename}%
    \usebibmacro{name:last}{#1}{#3}{#5}{#7}%
  \or
    \ifuseprefix
      {\usebibmacro{name:first-last}{#1}{#4}{#5}{#8}}
      {\usebibmacro{name:first-last}{#1}{#4}{#6}{#8}}%
  \or
    \usebibmacro{name:first-last}{#1}{#3}{#5}{#7}%
  \fi
  \usebibmacro{name:andothers}%
  \ifnumequal{\value{listcount}}{\value{liststop}}{'s}{}}

\DeclareFieldFormat{shorthand:poss}{%
  \ifnameundef{labelname}{#1's}{#1}}

\DeclareFieldFormat{citetitle:poss}{\mkbibemph{#1}'s}

\DeclareFieldFormat{label:poss}{#1's}

\newrobustcmd*{\posscitealias}{%
  \AtNextCite{%
    \DeclareNameAlias{labelname}{labelname:poss}%
    \DeclareFieldAlias{shorthand}{shorthand:poss}%
    \DeclareFieldAlias{citetitle}{citetitle:poss}%
    \DeclareFieldAlias{label}{label:poss}}}

\newrobustcmd*{\posscite}{%
  \posscitealias%
  \textcite}

\newrobustcmd*{\Posscite}{\bibsentence\posscite}

\newrobustcmd*{\posscites}{%
  \posscitealias%
  \textcites}

\newcommand\quelle[1]{{%
  \unskip\nobreak\hfil\penalty50
  \hskip2em\hbox{}\nobreak\hfil#1%
  \parfillskip=0pt \finalhyphendemerits=0 \par}}

\bibliography{Thesis}

\begin{document}
There is a set of constructions found in natural language, which I will refer to as argument sharing constructions, in which multiple predicates in a single clause share a single syntactic argument.
I include in this set, depictives, adjectival resultatives (ARs), directionalized locatives (DLs), and serial verb constructions(SVCs).
\ex.
  \a. Depictives \parencite[French,][]{legendre1997secondary}
    \ag. J'ai connu Marie heureuse.\\
    {I have} known Mary happy.FSg\\
    ``I've known Mary happy''
    \bg. Il est mort jeune.\\
    he is died young\\
    ``He died young.''
    \z.
  \b. Resultatives
    \a. Mary hammered the metal flat.
    \b. We drank the teapot dry.
    \z.
  \b. Directionalized Locatives
    \a. The bottle floated under the bridge
    \b. She kicked the ball between the pylons.
    \z.
  \b. Serial Verb Constructions \parencite[Edo,][]{bakerstewart1999double}
    \ag. \`Oz\'o gh\'a d\`e ìy\'an r\`e.\\
    Ozo FUT buy yam eat\\
    ``Ozo will buy yams and eat them.''
    \z.
  \z.

The latter three constructions are of particular interest in this thesis due to the fact that they seem to be parameterized in natural language.
Most varieties of French, for example, lack all of these constructions. English and German have ARs and DLs but lack SVCs, while Edo has SVCs and ARs \parencite{bakerstewart1999double}.
<++>

In order to explain the parameterization of these constructions, we must first understand their underlying syntactic structure.
The standard analysis of RLs (which \textcite{milway2015generals} extends to DLs) was proposed by \textcite{kratzer_building_2004} and is shown below in \Next.
\ex.
\begin{forest}
  tree defaults
  [VP
    [DP$_i$[the metal,triangle]]
    [,em
      [V\\hammer,align=center]
      [ResP
        [Res]
        [,em
          [$\langle$DP$_i\rangle$]
          [AP[flat,triangle]]
        ]
      ]
    ]
  ] 
\end{forest}

The DP is base-generated as the argument of the predicate \textit{flat} and raises to be interpreted as the theme of the verb \textit{hammer}.
This is problematic if we take seriously \posscite{baker1997thematic} Absolute UTAH which states that a given thematic role is mapped to a particular structural position.

\printbibliography
\end{document}


