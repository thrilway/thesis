%        File: FivePager.tex
%     Created: Thu Mar 24 12:00 PM 2016 E
% Last Change: Thu Mar 24 12:00 PM 2016 E
%
% arara: pdflatex
% arara: biber
% arara: pdflatex
% arara: pdflatex
\documentclass[letterpaper]{article}

\usepackage[margin=1in]{geometry}
\usepackage[backend=biber,style=authoryear,useprefix=false]{biblatex}

\usepackage{stmaryrd}
\usepackage[]{amsmath}
\usepackage{amsfonts}
\usepackage{amssymb}
\usepackage{forest}
\usepackage{tabularx}
\usepackage{linguex}
\usepackage{centernot}

\forestset{tree defaults/.style={for tree={parent anchor=south, child anchor=north},every tree node/.style={align=center,anchor=north},level/.style={sibling distance=50mm/#1},baseline}}

\forestset{en/.style={parent anchor=center, child anchor=center}}
\forestset{em/.style={parent anchor=north west, child anchor=north west}}
\forestset{el/.style={parent anchor=north, child anchor=north}}

\usetikzlibrary{positioning}
\DeclareNameFormat{labelname:poss}{% Based on labelname from biblatex.def
  \ifcase\value{uniquename}%
    \usebibmacro{name:last}{#1}{#3}{#5}{#7}%
  \or
    \ifuseprefix
      {\usebibmacro{name:first-last}{#1}{#4}{#5}{#8}}
      {\usebibmacro{name:first-last}{#1}{#4}{#6}{#8}}%
  \or
    \usebibmacro{name:first-last}{#1}{#3}{#5}{#7}%
  \fi
  \usebibmacro{name:andothers}%
  \ifnumequal{\value{listcount}}{\value{liststop}}{'s}{}}

\DeclareFieldFormat{shorthand:poss}{%
  \ifnameundef{labelname}{#1's}{#1}}

\DeclareFieldFormat{citetitle:poss}{\mkbibemph{#1}'s}

\DeclareFieldFormat{label:poss}{#1's}

\newrobustcmd*{\posscitealias}{%
  \AtNextCite{%
    \DeclareNameAlias{labelname}{labelname:poss}%
    \DeclareFieldAlias{shorthand}{shorthand:poss}%
    \DeclareFieldAlias{citetitle}{citetitle:poss}%
    \DeclareFieldAlias{label}{label:poss}}}

\newrobustcmd*{\posscite}{%
  \posscitealias%
  \textcite}

\newrobustcmd*{\Posscite}{\bibsentence\posscite}

\newrobustcmd*{\posscites}{%
  \posscitealias%
  \textcites}

\newcommand\quelle[1]{{%
  \unskip\nobreak\hfil\penalty50
  \hskip2em\hbox{}\nobreak\hfil#1%
  \parfillskip=0pt \finalhyphendemerits=0 \par}}

\bibliography{Thesis}

\title{Thesis ``Proposal''}
\author{Dan Milway}
\begin{document}
\maketitle

\section{The Phenomenon: Argument Sharing}
There is a set of constructions found in natural language, which I will refer to as argument sharing constructions, in which multiple predicates in a single clause share a single syntactic argument.
I include in this set, depictives, adjectival resultatives (ARs), directionalized locatives (DLs), and serial verb constructions(SVCs).
\ex.
  \a. Depictives \parencite[French,][]{legendre1997secondary}
    \ag. J'ai connu Marie heureuse.\\
    {I have} known Mary happy.FSg\\
    ``I've known Mary happy''
    \bg. Il est mort jeune.\\
    he is died young\\
    ``He died young.''
    \z.
  \b. Resultatives
    \a. Mary hammered the metal flat.
    \b. We drank the teapot dry.
    \z.
  \b. Directionalized Locatives
    \a. The bottle floated under the bridge
    \b. She kicked the ball between the pylons.
    \z.
  \b. Serial Verb Constructions \parencite[Edo,][]{bakerstewart1999double}
    \ag. \`Oz\'o gh\'a d\`e ìy\'an r\`e.\\
    Ozo FUT buy yam eat\\
    ``Ozo will buy yams and eat them.''
    \z.
  \z.

The latter three constructions are of particular interest in this thesis due to the fact that they seem to be parameterized in natural language.
Most varieties of French, for example, lack all of these constructions. English and German have ARs and DLs but lack SVCs, while Edo has SVCs and ARs \parencite{bakerstewart1999double}.

\section{The Analysis: Sideward Antecedence}
First, we must ask how such argument sharing comes about.
For the answer, I look to Theta Theory and, in particular, \posscite{baker1997thematic} UTAH, which posits that thematic relationships in a sentence are represented in structural relationships.
The classic example of this mapping between structure and thematic relations is the hypothesis that the agent role maps to Spec VoiceP \parencite{kratzer_severing_1996}.
Considering ARs and DLs, we can see that the direct objects of these constructions is interpreted as the internal argument of the verb and as an argument of the AdjP and PP, respectively.
\ex.
\a. John has hammered the metal flat.
\a. $\implies$ John has hammered the metal.
\b. $\implies$ The metal is/became flat.
\z.
\b. Mary kicked the ball between the pylons.
\a. $\implies$ Mary kicked the ball.
\b. $\implies$ the ball \{ended up\}/was between the pylons
\z.

Given UTAH, this means that the direct object (or some element that corefers with it) must be in a structural relationship with both the verb and the adjective/PP.
\textcite{kratzer_building_2004} proposes the structure in \Next for ARs, to account for the argument sharing.
The first merge of \textit{the metal} with the AP means it is interpreted as an argument of the adjective, and its remerge in the verbal projection means it is interpreted as the internal argument of the verb.\footnote{
  \posscite{kratzer_building_2004} use of movement in her analysis violates the $\Theta$-criterion which states that each $\Theta$-role must be assigned to exactly one DP and each DP must be assigned exactly one $\Theta$-role.
  Following \textcite{hornstein1999movement}, I dispense with the standard $\Theta$-criterion in favor of the version proposed by \textcite{higginbotham1985semantics}.
}
\ex.
\begin{forest}
  tree defaults
  [VP
    [DP$_i$[the metal,triangle,name=Spec VP]]
    [,em
      [V\\hammer,align=center]
      [ResP
        [Res]
        [,em
          [$\langle$DP$_i\rangle$,name=Spec AP]
          [AP[flat,triangle]]
        ]
      ]
    ]
  ]
  \draw[->,thick] (Spec AP) to[out=west, in=south] (Spec VP);
\end{forest}

This is problematic if we take seriously \posscite{baker1997thematic} Absolute UTAH which states that a given thematic role is mapped to a particular structural position.
In the structure in \Last, \textit{the metal} is interpreted as the theme argument by virtue of it being in Spec VP, while in the simple transitive structure (\textit{hammer the metal}) it is merged in Comp V.
To retain an absolute UTAH, I propose that argument sharing in resultative structures is the result of sideward movement as in \Next below.
\ex.
\begin{forest}
  tree defaults
  [VP
    [VP
      [DP$_i$[the metal,triangle]]
      [V\\hammer,align=center]
    ]
    [ResP
        [Res]
        [,em
          [$\langle$DP$_i\rangle$]
          [AP[flat,triangle]]
        ]
      ]
    ]
\end{forest}

While sideward movement is often taken to be ruled out in GB/Minimalist syntax, \textcite{bobaljikbrown1997interarboreal} and \textcite{nunes2001sideward} argue that movement operations such as the one in \Last are not ruled out by any proposed principle of grammar.
Consider the fragment of the derivation of \Last below where L and M are derived structures and S is the selected item.
\ex.
\a. L = [Res [ [the metal] flat]]\\
M = [hammer]\\
S = [the metal]
\b. L = [Res [ [the metal] flat]]\\
M = [hammer [the metal]]
\c. L = [[$_{VP}$ hammer [the metal]] [$_{ResP}$ Res [ [the metal] flat]]]
\z.

In \Last[a], \textit{the metal} has been copied from L, and then merged with M to get \Last[b].
Finally, L and M are merged to get \Last[c].

At this point I should note that facts about double object constructions lead \textcite{baker1997thematic} to argue that themes occupy Spec V, and goals/locations/etc. occupy Comp V.
In the structures above I assume that themes are associated with Comp V.
This departure from Baker's particular UTAH will require some discussion, but I believe there are good theoretical reasons to accept my version of UTAH.

This sideward movement structure has the additional advantage of being quite similar to the structure proposed by \textcite{bakerstewart1999double} for SVCs, given below.
This analysis differs slightly from my analysis of ARs and DLs, though, in that \textcite{bakerstewart1999double} do not propose that argument sharing occurs via movement.\footnote{
  \posscite{bakerstewart1999double} analysis of Edo resultatives, however, differs form the one I propose.
  They adopt a complex predicate analysis in contrast to the small clause analysis I adopt.
}
\ex.
\ag. \`Oz\'o gh\'a d\`e ìy\'an r\`e.\\
    Ozo FUT buy yam eat\\
    ``Ozo will buy yams and eat them.''
\b.
\begin{forest}
  tree defaults
  [TP
    [DP$_i$[Ozo,triangle,name=Spec T]]
    [,en
      [T\\will,align=center]
      [VoiceP
        [$\langle$DP$_i\rangle$,name=Spec Voice]
        [,en
          [Voice]
          [vP
            [vP
              [v
                [v]
                [V\\cook,align=center,name=v1]
              ]
              [VP
                [DP$_k$[food,triangle]]
                [$\langle$V$\rangle$,name=V1]
              ]
            ]
            [vP
              [v
                [v]
                [V\\eat,align=center,name=v2]
              ]
              [VP
                [pro$_k$]
                [$\langle$V$\rangle$,name=V2]
              ]
            ]
          ]
        ]
      ]
    ]
  ]
  \draw[->,thick] (Spec Voice)to[out=west, in=south east] (Spec T);
  \draw[->,thick] (V1)to[out=south, in=south] (v1);
  \draw[->,thick] (V2)to[out=south, in=south] (v2);
\end{forest}

\section{Generalization: Inflection $\rightarrow$ *Argument Sharing}
Given these analyses of ARs, DLs and SVCs, I am led to the questions of why English allows ARs and DLs but not SVCs and why French lacks all three.
A recurring observation in this domain is that a lack of inflectional morphology on predicates tends to be associated with argument sharing.
\textcite{kratzer_building_2004} points out that French, which lacks ARs, has obligatory agreement morphology on predicative adjectives, while German, which has ARs, has bare predicative adjectives.
\ex.
\ag. La femme est heureuse\\
the woman is happy.FSg\\
``The woman is happy.''
\bg. Die Frau ist gl\"ucklich (*-e)\\
the woman is happy (Agr)\\
``The woman is happy.''

\textcite{bakerstewart1999double}, citing \textcite{dechaine1993diss}, say that languages with SVCs tend to lack inflectional morphology on verbs.
They also note that the past perfective in Edo is the only tense with inflectional morphology on verbs and is incompatible with SVCs.
I propose \parencite[contrary to][]{bakerstewart1999double} that the parametric variation of argument sharing constructions is reducible to the interaction between sideward antecedence relations and overt predicate morphology.

\section{The Account: Linearization?}
Assuming the antecedence relation is an instance of sideward movement,\footnote{
  Other possible analyses include proposing that the ``trace'' is actually a null resumptive pronoun, or the suggestion that the null element has arbitrary reference which is constrained by the fact that a given event has only one theme \parencite{pietroski2005events}.
  These alternatives will have to be investigated.
} why is such movement allowed where it is allowed and disallowed elsewhere?
\textcite{nunes2001sideward} argues that sideward movement is allowed only in those cases in which it forms chains that can be linearized.
If this is correct, than I should be able to show that \Next[a] is linearizable but \Next[b] is not.
\ex. 
\a. Mary [hammered the metal] [flat $\langle$the metal$\rangle$]. 
\b.* Mary cooks the food eats $\langle$the food$\rangle$
\z.

If a linearization-based account is incorrect for argument sharing, I will need to propose an alternative account that is either compatible with Nunes' (2001) proposal or able to supplant it.

\section{A Problem: Directionalized Locatives}
Accounting for directionalized locatives reperesents is another issue that will be addressed by this thesis.
While there are good reasons to believe that DLs are isomorphic with ARs \textcite{milway2015generals}, it is not obvious how the parametric variation of DLs can be explained using the logic proposed above.
Prepositions generally do not show inflectional morphology, yet the presence of ARs (by virtue of the lack of predicative adjective agreement) seems to predict the presence of DLs.

\textcite{milway2015generals} discusses some suggestive facts linking the (un)availability of DLs in English and French to the (im)possibility of P-stranding in those languages.
English allows both DLs and P-stranding, while most dialects  of French allow neither.
Ruth King and Yves Roberge \parencite[p.c. cited in][pp 253--254]{rooryck1996prepositions}, however, report that PEI French patterns with English with respect to DLs and P-stranding, and with other Frenches with respect to ARs.
While this is a long way from an empirical generalization, it does provide a useful starting point for a discussion of DLs.
\section{Theoretical Implications}
This work will assume a version of \posscite{baker1997thematic} Absolute UTAH, which I take to be an empirical generalization.
Being an empirical statement, a goal of our theory of the language faculty ought to be its explanation.
In order to explain UTAH, we must first decide where/when in our computational system it is enforced.
Since it deals with a mapping between grammatical and thematic structure -- the former being syntactic, the latter, semantic -- it seems to be a fact about the syntax-semantics interface.
The current theory of semantic composition, however, is much more flexible than any version of UTAH, which is merely stipulated in the denotations of verbs and other argument introducers.
The closest DP to the verb \textit{hit} is interpreted as a theme by virtue of the fact that the denotation of \textit{hit} includes the \textsc{Theme} predicate, but there is nothing in the theory that prevents the \textit{hit} being bundled with the \textsc{Agent} predicate.
Furthermore, UTAH is expressed in structural terms (specifier, complement, dominance, etc.) which cannot be expressed in the standard compositional system, which only refers to sisterhood relations. 

So, while our current conception of the syntax-semantics interface allows for UTAH, it also allows for virtually every other possible version of UTAH.
A syntax-semantics interface that allowed for only the version of UTAH that describes the empirical facts would be preferable.

\printbibliography
\end{document}


