%        File: GURT.tex
%     Created: Thu Oct 27 08:00 PM 2016 E
% Last Change: Thu Oct 27 08:00 PM 2016 E
%
\documentclass[letterpaper]{article}

\usepackage[
margin=1in
]{geometry}
\usepackage[backend=biber,style=authoryear-comp,useprefix=false]{biblatex}

\usepackage{stmaryrd}
\usepackage[]{amsmath}
\usepackage{amsfonts}
\usepackage{amssymb}
\usepackage{forest}
\usepackage{tabularx}
\usepackage{linguex}
\usepackage{centernot}
\usepackage{todonotes}

\useforestlibrary{linguistics}

\forestset{tree defaults/.style={for tree={parent anchor=south, child anchor=north},every tree node/.style={align=center,anchor=north},level/.style={sibling distance=50mm/#1},baseline}}

\forestset{en/.style={parent anchor=center, child anchor=center}}
\forestset{em/.style={parent anchor=north west, child anchor=north west}}
\forestset{el/.style={parent anchor=north, child anchor=north}}

\usetikzlibrary{positioning}
\usetikzlibrary{calc}
\usetikzlibrary{arrows}
\usetikzlibrary{decorations.markings}

\newcommand\quelle[1]{{%
  \unskip\nobreak\hfil\penalty50
  \hskip2em\hbox{}\nobreak\hfil#1%
  \parfillskip=0pt \finalhyphendemerits=0 \par
}
}

\newcommand{\figex}{\refstepcounter{ExNo}\theExNo\hspace{\Exlabelsep}}

\bibliography{Thesis}
\begin{document}
Secondary predication (SP) displays a puzzling type of variation which this paper refers to as The Resultative Parameter.
Depending on the setting of this parameter, certain SP constructions are able to recieve a resultative interpretation.
For instance English allows resultative interpretations of SP, while Italian does not.
\ex. <+Res+>

Since the interpretation rather than the surface form is paramerized, it is not directly learnable from the primary linguistic data (PLD).

\textcite{snyder2001nature} argues that resultative SPs are only available in languages with productive endocentric compounding.
While this is supported by typological and acquisition evidence, it is not obvious how compounding and resultatives could be related.
\textcite{kratzer_building_2004} proposes a structure for resultatives, reproduced in \Next, and attempts to give an account of the parametric variation, but ultimately falls short for theoretical and empirical reasons.
\ex. <+Kratzer+>

Given these two results, the task of explaining The Resultative Parameter can be reduced to the task of showing that whatever property of a grammar that (dis)allows endocentric compounding also (dis)allows the structure in \Last.

In order to show this, I adopt a Chomsky's (\citeyear{<++>}<++>) label theory, along with a suite of other theoretical assumptions \parencite{<++>}<++>, and propose that only bare categories, in the sense of \textcite{<+Lasnik+>}<++>, can participate in compounding.
In \Last, then, the adjective is a bare aP, and as such the small clause is unlabelable unless the theme DP moves to spec resP.
From this higher position, the theme is available for movement into the VP.
If the aP were not bare, as in Italian, the theme and the aP would agree, and the small clause would therefore be labelable.

\end{document}


