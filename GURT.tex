%        File: GURT.tex
%     Created: Thu Oct 27 08:00 PM 2016 E
% Last Change: Thu Oct 27 08:00 PM 2016 E
%
\documentclass[letterpaper]{article}

\usepackage[
margin=1in
]{geometry}
\usepackage[backend=biber,style=authoryear-comp,useprefix=false]{biblatex}

\usepackage{stmaryrd}
\usepackage[]{amsmath}
\usepackage{amsfonts}
\usepackage{amssymb}
\usepackage{forest}
\usepackage{tabularx}
\usepackage{linguex}
\usepackage{centernot}
\usepackage{todonotes}

\useforestlibrary{linguistics}

\forestset{tree defaults/.style={for tree={parent anchor=south, child anchor=north},every tree node/.style={align=center,anchor=north},level/.style={sibling distance=50mm/#1},baseline}}

\forestset{en/.style={parent anchor=center, child anchor=center}}
\forestset{em/.style={parent anchor=north west, child anchor=north west}}
\forestset{el/.style={parent anchor=north, child anchor=north}}

\usetikzlibrary{positioning}
\usetikzlibrary{calc}
\usetikzlibrary{arrows}
\usetikzlibrary{decorations.markings}

\newcommand\quelle[1]{{%
  \unskip\nobreak\hfil\penalty50
  \hskip2em\hbox{}\nobreak\hfil#1%
  \parfillskip=0pt \finalhyphendemerits=0 \par
}
}

\newcommand{\figex}{\refstepcounter{ExNo}\theExNo\hspace{\Exlabelsep}}

\bibliography{Thesis}
\begin{document}
\textbf{Abstract}

The fact that languages vary with repect to resultative interpretation of secondary predication (SP) presents an acquisiton puzzle.
For instance, English allows resultative SP, while Italian does not.
\ex.
\ag. l' ho mangiato crudo.\\
3sg \textsc{aux}.1sg eaten raw\\
``I ate it raw'' (depictive)
\bg.* l' ho bevuto vuoto.\\
3sg \textsc{aux}.1sg drank empty\\
``I drank it dry'' (resultative)

Since the interpretation rather than the surface form is parameterized, it is not directly learnable from the primary linguistic data.
This paper shows how the variation can be acquired indirectly.

\textcite{snyder2001nature} gives typological and child language evidence that resultative SPs are only available in languages with productive endocentric compounding, but does not explain how the two phenomena are linked.
\textcite{kratzer_building_2004} proposes a structure for resultatives, a modified version of which is given in \Next, and attempts to give an account of the parametric variation, but falls short for theoretical and empirical reasons.
\ex. 
\begin{forest}
  nice empty nodes,sn edges,baseline
    [VP
	    [{the shirt},name=V theme] 
	    [
		    [dry] 
		    [resP 
			    [res$^\circ$] 
			    [SC
				    [{$\langle\text{the shirt}\rangle$},name=sc theme]
				    [aP
					    [a$^\circ$]
					    [\textsc{wrinkled}]
				    ]
			    ]
		    ]
	    ]
    ]
    \draw[->] (sc theme) to[out=south west, in=south] (V theme);
\end{forest}

Given these two results, the task of explaining The Resultative Parameter can be reduced to the task of showing that whatever property of a grammar that (dis)allows endocentric compounding also (dis)allows the structure in \Last.

Adopting Chomsky's (\citeyear{chomsky2013problems,chomsky2015problems}) label theory, along with a suite of other theoretical assumptions \parencite{nunes2001sideward,chametzky1996theory,hornstein1999movement}, this paper proposes that only bare categories, in the sense of \textcite{lasnik1999verbal}, can participate in compounding.
An SC with a bare aP (as in English) is unlabelable, which triggers movement of the theme, while one with an inflected aP (as in Italian) is labelable.
Since the SC is labelable, the DP remains \textit{in situ} and cannot be theta-marked by the verb.
Such an account has the additional benefit of predicting, correctly, that SP is universally allowed in all languages.

\textbf{Presentation Summary}


\end{document}


