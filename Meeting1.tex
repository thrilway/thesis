%        File: Meeting1.tex
%     Created: Tue Feb 23 03:00 PM 2016 E
% Last Change: Tue Feb 23 03:00 PM 2016 E
%
% arara: pdflatex: {options: "-file-line-error-style"}
\documentclass[letterpaper]{article}

\usepackage[margin=1in]{geometry}
\usepackage[backend=biber,style=authoryear-comp,useprefix=false]{biblatex}
\usepackage{stmaryrd}
\usepackage[]{amsmath}
\usepackage{amsfonts}
\usepackage{amssymb}
\usepackage{forest}
\usepackage{tabularx}
\usepackage{linguex}

\forestset{tree defaults/.style={for tree={parent anchor=south, child anchor=north},every tree node/.style={align=center,anchor=north},level/.style={sibling distance=50mm/#1},baseline}}

\forestset{en/.style={parent anchor=center, child anchor=center}}
\forestset{em/.style={parent anchor=north west, child anchor=north west}}
\forestset{el/.style={parent anchor=north, child anchor=north}}

\usetikzlibrary{positioning}
\DeclareNameFormat{labelname:poss}{% Based on labelname from biblatex.def
  \ifcase\value{uniquename}%
    \usebibmacro{name:last}{#1}{#3}{#5}{#7}%
  \or
    \ifuseprefix
      {\usebibmacro{name:first-last}{#1}{#4}{#5}{#8}}
      {\usebibmacro{name:first-last}{#1}{#4}{#6}{#8}}%
  \or
    \usebibmacro{name:first-last}{#1}{#3}{#5}{#7}%
  \fi
  \usebibmacro{name:andothers}%
  \ifnumequal{\value{listcount}}{\value{liststop}}{'s}{}}

\DeclareFieldFormat{shorthand:poss}{%
  \ifnameundef{labelname}{#1's}{#1}}

\DeclareFieldFormat{citetitle:poss}{\mkbibemph{#1}'s}

\DeclareFieldFormat{label:poss}{#1's}

\newrobustcmd*{\posscitealias}{%
  \AtNextCite{%
    \DeclareNameAlias{labelname}{labelname:poss}%
    \DeclareFieldAlias{shorthand}{shorthand:poss}%
    \DeclareFieldAlias{citetitle}{citetitle:poss}%
    \DeclareFieldAlias{label}{label:poss}}}

\newrobustcmd*{\posscite}{%
  \posscitealias%
  \textcite}

\newrobustcmd*{\Posscite}{\bibsentence\posscite}

\newrobustcmd*{\posscites}{%
  \posscitealias%
  \textcites}

\newcommand\quelle[1]{{%
  \unskip\nobreak\hfil\penalty50
  \hskip2em\hbox{}\nobreak\hfil#1%
  \parfillskip=0pt \finalhyphendemerits=0 \par}}

\bibliography{Thesis}

\begin{document}
\section{Idea 1: Directed Motion}
\subsection{The puzzle}
\ex. The bottle floated under the bridge.

\exg. La bouteille a flott\'e sous le pont.\\
the bottle has floated under the bridge\\
``The bottle floated under the bridge.''

\begin{itemize}
  \item \LLast is ambiguous:
    \begin{itemize}
      \item = The bottle was under the bridge, floating.      (Located Motion)
      \item = The bottle went under the bridge in a floating. (Directed Motion)
    \end{itemize}
  \item \Last is unambiguous:
    \begin{itemize}
      \item = The bottle was under the bridge, floating.           (Located Motion)
      \item $\neq$ The bottle went under the bridge in a floating. (Directed Motion)
    \end{itemize}
\end{itemize}
\ex.[\textbf{Question: }] How is the (lack of) ambiguity acquirable from PLD?

\begin{itemize}
  \item Most analyses of this variation say that English bundles \textsc{Path} and \textsc{Manner} in motion verbs, while French does not.
  \item These analyses might be descriptively accurate, but do not explain how the variation is acquired.
\end{itemize}

\subsection{Results of my 1st GP}
\begin{itemize}
  \item Directed motion interpretations map to a structure isometric to adjectival resultatives.
\end{itemize}
\ex. 
\begin{forest}
  tree defaults
  [,en
    [DP$_i$[The bottle,triangle]]
    [VP
      [V\\float,align=center]
      [resP
	[res]
	[,em
	  [$\langle DP_i\rangle$]
	  [PP[under the bridge,triangle]]
	]
      ]
    ]
  ]
\end{forest}

\ex. 
\begin{forest}
  tree defaults
  [,en
    [DP$_i$[the metal,triangle]]
    [VP
      [V\\hammer,align=center]
      [resP
	[res]
	[,em
	  [$\langle DP_i\rangle$]
	  [AP[flat,triangle]]
	]
      ]
    ]
  ]
\end{forest}

\begin{itemize}
  \item The DP in \LLast and \Last must move to be interpreted as the theme of \textit{float}/\textit{hammer}.
  \item French also lacks adjectival resultatives.
  \item My Proposal: Raising a DP out of a resP is barred in French.
  \item Overt correlates to this ``parameter'':
    \begin{itemize}
      \item predicative adjectival agreement $\rightarrow$ *adjectival resultatives
      \item P-stranding $\rightarrow$ directed motion
    \end{itemize}
\end{itemize}
\begin{tabularx}{.75\textwidth}{lllll}
    & Pred Agr & resultatives & P-Stranding & Dir Motion\\
    \hline
    English & -- & yes & yes & yes\\
    German & no & yes & no\footnote{There are other peculiarities in the German P system though} & yes\\
    Std French & yes & no & no & no\\
    PEI French\footnote{This is on the basis of a PC citation to Yves Roberge and Ruth King in the footnote of a paper by Johan Rooryk.} & yes & no & yes & yes\\
    \hline
  \end{tabularx}

\begin{itemize}
  \item My task would be to argue for a causal link in these (or maybe other) correlations.
\end{itemize}

\section{``Idea'' 2:}
\begin{itemize}
  \item Pro-form replacement tests probe two types of knowledege:
    \begin{itemize}
      \item Constituency
      \item Category
    \end{itemize}
\end{itemize}
\ex. The woman in the red coat laughed.
\a. $\not\rightarrow$ *She in the red coat laughed
\b. $\not\rightarrow$ He laughed.

\begin{itemize}
  \item Some component of our grammar says that replacing a DP in a structure with an appropriate pronoun preserves that structure and its (un)grammaticality.
  \item The fact that (complex) DPs are Islands, follows from this.
\end{itemize}
\ex. 
\a. They hired someone who speaks a Balkan language.
\b. The hired them.

\ex.
\a.* What Balkan language did they hire someone who speaks $t_{wh}$?
\b.* What Balkan language did they hire them?

\begin{itemize}
  \item Replacing the complex DP in \Last[a] with \textit{them} eliminates the \textit{wh}-trace.
  \item Part of or grammar says that eliminating an adjunct from a structure preserves that structure and its (un)grammaticality.
  \item The Islandhood of Adjuncts follows from this.
\end{itemize}
\ex.
\a. Mary met the boy after Sue saw the girl.
\b. Mary met the boy.

\ex.
\a.* Who did Mary meet the boy after Sue saw $t_{wh}$?
\b.* Who did Mary meet the boy?

\begin{itemize}
  \item Eliminating the adjunct eliminates the \textit{wh}-trace.
  \item Parasitic gaps are expected to be allowed. 
\end{itemize}
\ex.
\a. Who did Mary meet $t_{wh}$ after Sue saw $t_{wh}$?
\b. Who did Mary meet $t_{wh}$?

\begin{itemize}
  \item If this logic could be extended to all (strong) islands, then explaining islands would be reduced to explaining pronoun replacement and adjunct dropping.
\end{itemize}
\subsection{Background: Bare Grammar (Keenan \& Stabler 2003)}
\ex.
\a. Sue laughed.
\b. Mary cried.

\begin{itemize}
  \item Standard GG asks what are the structures of the sentences in \Last, and how is knowledge of those structures instantiated in the mind.
  \item Bare Grammar asks how are the structures of the sentences in \Last related, and how is knowledge of that relation instantiated in the mind.
  \item K\&S say that \Last[a] and \Last[b] have the same structure and our knowledge of that is based on a mapping function $h$.
  \item $h$ maps \textit{Sue} to \textit{Mary} and \textit{laugh} to \textit{cry}
  \item This mapping function seems to be related to our notion of lexical categories
    \begin{itemize}
      \item $h(\alpha) = \beta$ iff $\alpha$ and $\beta$ belong to the same lexical category
      \item Functional items do not map to distinct items
    \end{itemize}
\end{itemize}
\end{document}


