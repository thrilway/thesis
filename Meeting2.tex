%        File: Meeting2.tex
%     Created: Wed Mar 09 10:00 AM 2016 E
% Last Change: Wed Mar 09 10:00 AM 2016 E
%
% arara: pdflatex
% arara: biber
% arara: pdflatex
% arara: pdflatex

\documentclass[letterpaper]{article}

\usepackage[margin=1in]{geometry}
\usepackage[backend=biber,style=authoryear-comp,useprefix=false]{biblatex}
\usepackage{stmaryrd}
\usepackage[]{amsmath}
\usepackage{amsfonts}
\usepackage{amssymb}
\usepackage{forest}
\usepackage{tabularx}
\usepackage{linguex}
\usepackage{centernot}

\forestset{tree defaults/.style={for tree={parent anchor=south, child anchor=north},every tree node/.style={align=center,anchor=north},level/.style={sibling distance=50mm/#1},baseline}}

\forestset{en/.style={parent anchor=center, child anchor=center}}
\forestset{em/.style={parent anchor=north west, child anchor=north west}}
\forestset{el/.style={parent anchor=north, child anchor=north}}

\usetikzlibrary{positioning}
\DeclareNameFormat{labelname:poss}{% Based on labelname from biblatex.def
  \ifcase\value{uniquename}%
    \usebibmacro{name:last}{#1}{#3}{#5}{#7}%
  \or
    \ifuseprefix
      {\usebibmacro{name:first-last}{#1}{#4}{#5}{#8}}
      {\usebibmacro{name:first-last}{#1}{#4}{#6}{#8}}%
  \or
    \usebibmacro{name:first-last}{#1}{#3}{#5}{#7}%
  \fi
  \usebibmacro{name:andothers}%
  \ifnumequal{\value{listcount}}{\value{liststop}}{'s}{}}

\DeclareFieldFormat{shorthand:poss}{%
  \ifnameundef{labelname}{#1's}{#1}}

\DeclareFieldFormat{citetitle:poss}{\mkbibemph{#1}'s}

\DeclareFieldFormat{label:poss}{#1's}

\newrobustcmd*{\posscitealias}{%
  \AtNextCite{%
    \DeclareNameAlias{labelname}{labelname:poss}%
    \DeclareFieldAlias{shorthand}{shorthand:poss}%
    \DeclareFieldAlias{citetitle}{citetitle:poss}%
    \DeclareFieldAlias{label}{label:poss}}}

\newrobustcmd*{\posscite}{%
  \posscitealias%
  \textcite}

\newrobustcmd*{\Posscite}{\bibsentence\posscite}

\newrobustcmd*{\posscites}{%
  \posscitealias%
  \textcites}

\newcommand\quelle[1]{{%
  \unskip\nobreak\hfil\penalty50
  \hskip2em\hbox{}\nobreak\hfil#1%
  \parfillskip=0pt \finalhyphendemerits=0 \par}}

\bibliography{Thesis}
\title{Secondary Predication and Sideward Movement}
\author{Dan Milway}

\begin{document}
\maketitle
\section{Assumptions}
\begin{enumerate}
  \item The mapping of structure to thematic roles is transparent and regular.
    \begin{itemize}
      \item Comp V $=$ \textsc{Internal}
      \item Spec $v$ $=$ \textsc{External}
    \end{itemize}
  \item Semantically, thematic roles are conjoined predicates of events.
  \item Neo-Davidsonian explanation of entailment patterns
\end{enumerate}
\ex.
\begin{tabular}[t]{rcl}
  John kissed Bill & $\implies$ & Bill was kissed\\
  $\exists e [kiss(e) \wedge kisser(e,\mathbf{j}) \wedge kissee(e, \mathbf{b})]$ & $\implies$ &$\exists e [kiss(e) \wedge kissee(e, \mathbf{b})]$
\end{tabular}

\section{Subjects of secondary predicates}
\begin{itemize}
  \item They tend to be interpreted as arguments of the verb \parencite{kratzer_building_2004}
\end{itemize}
\ex.
\a. Mary hammered the metal flat $\implies$ Mary hammered the metal.
\b. Bill left the room angry $\implies$ Bill left the room.
\c. Meryl saw Beryl feed the animals. $\implies$ Meryl saw Beryl.
\d. John kicked the ball under the table $\implies$ John Kicked the ball.

\begin{itemize}
  \item By assumption 3, these entailment patterns are due to Conjunction of event predicates.
\end{itemize}
\ex. 
\begin{tabular}[t]{rcl}
  Mary hammered the metal flat. & $\implies$ & Mary hammered the metal.\\
  $\exists e,s[h(e)\wedge \textsc{Ag}(e,\mathbf{m}) \wedge \textsc{Th}(e, \mathbf{tm})\wedge$ & $\implies$ & $\exists e,s[h(e)\wedge \textsc{Ag}(e,\mathbf{m}) \wedge \textsc{Th}(e, \mathbf{tm})]$\\
$\textsc{Cause}(e,s) \wedge flat(s, \mathbf{tm})]$ & & \\
\end{tabular}

\begin{itemize}
  \item By Assumption 1, in order for \textit{the metal} to be interpreted as the Theme of \textit{hammer} it must be in Comp V.
  \item According to the standard analysis, \textit{the metal} is in Spec V.
\end{itemize}
\ex. Standard\\
\begin{forest}
  tree defaults
  [VP
    [DP[the metal,triangle]]
    [,en
      [V\\hammer,align=center]
      [ResP
	[Res]
	[SC
	  [$\langle$DP$\rangle$]
	  [AP[flat,triangle]]
	]
      ]
    ]
  ]
\end{forest}

\ex. Predicted\\
\begin{forest}
  tree defaults
  [VP
    [VP
      [V\\hammer,align=center]
      [DP[the metal,triangle]]
    ]
    [ResP
      [Res]
      [SC
	[$\langle$DP$\rangle$]
	[AP[flat,triangle]]
      ]
    ]
  ]
\end{forest}

\begin{itemize}
  \item Problem: This requires sideward movement
\end{itemize}

\section{How to salvage the standard analysis}
\begin{itemize}
  \item Eliminate Assumption 1
  \item Event identification \parencite{kratzer_severing_1996}
\end{itemize}
\ex. Event Identification\\
\begin{tabular}[t]{cclc}
  $f$ & $g$ & $\rightarrow$ & $\lambda x \lambda e [f(x)(e) \wedge g(e)]$\\
  $\langle e, \langle s,t\rangle\rangle$ & $\langle s,t\rangle$ & $\rightarrow$ &$\langle e, \langle s,t\rangle\rangle$
\end{tabular}

\ex. $\llbracket$hammer$\rrbracket = \lambda x\lambda e[h(e) \wedge \textsc{Th}(e,x)]$

\ex. $\llbracket$ResP$\rrbracket = \lambda e \exists s[\textsc{Cause}(e,s) \wedge \Phi(s)]$

\begin{itemize}
  \item Combining V + ResP by EI allows Spec V to be interpreted as the internal argument.
\end{itemize}

\section{Is the standard account worth saving?}
\begin{itemize}
  \item Can we sever other dyadic functions severed from their internal arguments?
\end{itemize}
\ex. F(Int)(Ext)
\a. V(Theme)(Agent)
\b. P(Ground)(Figure)
\c. Q(Restrictor)(Nuclear Scope)

\begin{itemize}
  \item Internal arguments seem to always be very local to their function.
\end{itemize}

\section{Issues/Questions}
\begin{itemize}
  \item Resultatives with reflexives don't match the entailment pattern above.
\end{itemize}
\ex. Mary ran herself tired $\centernot\implies$ Mary ran herself.

\begin{itemize}
  \item What are the restrictions on sideward movement?
  \item If our compositional system allows the standard account, why not use it?
    \begin{itemize}
      \item If we want to bar the standard account, should we adopt a different compositional semantics? (\textit{e.g.},  conjunctivist semantics \parencite{pietroski2011minimal})
    \end{itemize}
\end{itemize}
\printbibliography
\end{document}
