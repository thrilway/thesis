%        File: AC2.tex
%     Created: Thu Jan 26 04:00 PM 2017 E
% Last Change: Thu Jan 26 04:00 PM 2017 E
%
% arara: pdflatex: {options: "-draftmode"}
% arara: biber
% arara: pdflatex: {options: "-draftmode"}
% arara: pdflatex: {options: "-file-line-error-style"}
\documentclass[Proposal]{subfiles}

\begin{document}
\begin{frame}
  {What can this version of Label Theory get us?}
  \begin{itemize}
    \item \textcite{cinque1996pseudo}: Direct perception reports with ACC-ing clauses (ACs) are ambiguous.
  \end{itemize}
  \pause
  {\rm I saw the boy running.}\\
  \begin{columns}
    \begin{column}
      [T]{0.5\textwidth}
  \onslide<3->{
    Argument AC\\
    {\small 
      \begin{forest}
	nice empty nodes,sn edges,baseline,for tree={
	  calign=fixed edge angles,
  calign primary angle=-30,calign secondary angle=70}
	[VP
	  [V\\see,align=center]
	  [ProgP
	    [DP[the boy,roof]]
	    [Prog$^\prime$[running,roof]]
	  ]
	]
      \end{forest}
    }

  }
    \end{column}
    \begin{column}
      [T]{0.5\textwidth}
      \onslide<4->{
	Adjunct AC\\
    {\small
      \begin{forest}
	nice empty nodes,sn edges,baseline,for tree={
	  calign=fixed edge angles,
	calign primary angle=-30,calign secondary angle=70}
	[VP
	  [VP
	    [V\\see,align=center]
	    [DP[the boy,roof,name=theme]]
	  ]
	  [ProgP
	    [DP[the boy,roof,name=ac subj]]
	    [Prog$^\prime$[running,roof]]
	  ]
	]
      \draw[->] (ac subj) to[out=south, in=south] (theme);
      \end{forest}
    }

  }
    \end{column}
  \end{columns}
\end{frame}
\begin{frame}
  {What can this version of Label Theory get us?}
  \ex. 
  \a. {\rm They heard the boy being slandered.}\\
  \onslide<2->{
  ($\centernot\rightarrow$ {\rm They heard the boy})}
  \b.* {\rm The boy was heard being slandered.}\\
  \onslide<2->{
  ($\rightarrow$ {\rm They heard the boy})}\\
  \onslide<3->{
  (\textit{cf.} {\rm The boy was heard to be slandered})}
  
  \onslide<3->{
  \begin{itemize}
    \item Argument ACs: Subjects cannot move.
\end{itemize}}
\end{frame}
\begin{frame}
  {What can this version of Label Theory get us?}
  \begin{itemize}
    \item Argument ACs: Subjects cannot move.
  \end{itemize}
  {\rm *the boy was heard being slandered.}\\
  {\tiny
  \begin{forest}
    nice empty nodes,sn edges,baseline,for tree={
      calign=fixed edge angles,
    calign primary angle=-30,calign secondary angle=70}
    [TP,name=tp
      [DP[the boy,roof,name=subj]]
      [
	[T]
	[\ldots
	  [,phantom]
	  [VP
	    [V\\hear,align=center]
	    [ProgP
	      [DP[the boy,roof,name=ac subj]]
	      [Prog$^\prime$[being\\slandered,align=center,roof]]
	    ]
	  ]
	]
      ]
    ]
    \draw[->] (ac subj) to[out=south west, in=south] (subj);
\end{forest}}
\end{frame}
\begin{frame}
  {What can this version of Label Theory get us?}
  \begin{itemize}
    \item Argument ACs: Subjects cannot move
  \end{itemize}
  {\rm *the boy was heard being slandered.}\\
  {\tiny
  \begin{forest}
    nice empty nodes,sn edges,baseline,for tree={
      calign=fixed edge angles,
    calign primary angle=-30,calign secondary angle=70}
    [TP,tikz={\node [draw,red,inner sep=0,cross out,thick,fit to=tree]{};}name=tp
      [DP[the boy,roof,name=subj]]
      [
	[T]
	[\ldots
	  [,phantom]
	  [VP
	    [V\\hear,align=center]
	    [ProgP
	      [DP[the boy,roof,name=ac subj]]
	      [Prog$^\prime$[being\\slandered,align=center,roof]]
	    ]
	  ]
	]
      ]
    ]
    \draw[->] (ac subj) to[out=south west, in=south] (subj);
\end{forest}}
\end{frame}
\begin{frame}
  {What can this version of Label Theory get us?}
  \begin{itemize}
    \item Argument ACs: Subjects cannot move.
  \end{itemize}
  {\rm *the boy was heard being slandered.}\\
  {\tiny
  \begin{forest}
    nice empty nodes,sn edges,baseline,for tree={
      calign=fixed edge angles,
    calign primary angle=-30,calign secondary angle=70}
    [TP,name=tp
      [DP[the boy,roof,name=subj]]
      [
	[T]
	[\ldots
	  [,phantom]
	  [VP
	    [VP
	      [V\\hear,align=center]
	      [DP[the boy,roof,name=theme]]
	    ]
	    [ProgP
	      [DP[the boy,roof,name=ac subj]]
	      [Prog$^\prime$[being\\slandered,align=center,roof]]
	    ]
	  ]
	]
      ]
    ]
    \draw[->] (ac subj) to[out=south, in=south] (theme);
    \draw[->] (theme) to[out=south west, in=south] (subj);
\end{forest}}
\end{frame}
\begin{frame}
  {What can this version of Label Theory get us?}
  \begin{itemize}
    \item Adjunct ACs: Subjects must move.
  \end{itemize}
  \pause
  {\rm *I saw the girl the boy running}
  \pause
  \begin{columns}
    \begin{column}
      [T]{0.5\textwidth}
	  {\small
	      \begin{forest}
		nice empty nodes,sn edges,baseline,for tree={
		  calign=fixed edge angles,
		calign primary angle=-30,calign secondary angle=70}
		[VP
		  [VP
		    [V\\see,align=center]
		    [DP[the girl,roof,name=theme]]
		  ]
		  [ProgP
		    [DP[the boy,roof,name=ac subj]]
		    [Prog$^\prime$[running,roof]]
		  ]
		]
	      \end{forest}
	    }
    \end{column}
    \pause
    \begin{column}
      [T]{0.5\textwidth}
      \begin{itemize}
	\item We just saw that Prog can license an overt subject.
      \end{itemize}
    \end{column}
  \end{columns}
\end{frame}
\begin{frame}
  {What can this version of Label Theory get us?}
  \begin{itemize}
    \item Whether an AC ``licenses'' an overt subject depends on whether it is an argument or an adjunct.
      \pause
    \item Standard Minimalist syntax is ill-equipped to account for this
      \pause
      \begin{itemize}
	\item If a DP is licensed at some point in a derivation, it stays licensed
      \end{itemize}
      \pause
    \item In (modified) Label Theory, however \ldots  
  \end{itemize}
\end{frame}
\begin{frame}
  {What can this version of Label Theory get us?}
  \begin{columns}
    \begin{column}
      [T]{0.5\textwidth}
      \begin{itemize}[<+->]
	\item Argument ACs license subjects in order to be labelled
	  \begin{itemize}
	    \item Prog is a phase head. \parencite{harwood2015being}
	    \item It has $\varphi$-features.
	    \item Label($\beta$) = $\langle\varphi,\varphi\rangle$
	    \item $\beta$ gets an Operator-Variable intepretation
	  \end{itemize}
	\item If DP has moved ($\therefore$ is invisible) Prog$_\varphi$ can't label.
      \end{itemize}
    \end{column}
    \begin{column}
      [T]{0.5\textwidth}
      \onslide<1->
      {\small 
      \begin{forest}
	nice empty nodes,sn edges,baseline,for tree={
	  calign=fixed edge angles,
  calign primary angle=-30,calign secondary angle=70}
	[$\gamma$
	  [V\\see,align=center]
	  [$\beta$
	    [DP$_\varphi$[the boy,roof]]
	    [$\alpha$
	      [Prog$_\varphi$]
	      [\ldots]
	    ]
	  ]
	]
      \end{forest}
    }
    \end{column}
  \end{columns}
\end{frame}
\begin{frame}
  {What can this version of Label Theory get us?}
  \begin{columns}
    \begin{column}
      [T]{0.5\textwidth}
      \begin{itemize}[<+->]
	\item Adjunct ACs don't license subjects for interpretive reasons
	  \begin{itemize}
	    \item $\beta$ is ignored by LA.
	    \item It gets a conjunctive interpretation.
	    \item $\lambda x [\text{the\_boy}(x) \& \text{running}(x)]$
	  \end{itemize}
      \end{itemize}
    \end{column}
    \begin{column}
      [T]{0.5\textwidth}
      {\small 
      \begin{forest}
	nice empty nodes,sn edges,baseline,for tree={
	  calign=fixed edge angles,
  calign primary angle=-30,calign secondary angle=70}
	[$\delta$
	  [VP[saw\\the girl,roof,align=center]]
	  [$\beta$
	    [DP$_\varphi$[the boy,roof]]
	    [$\alpha$
	      [Prog$_\varphi$]
	      [\ldots]
	    ]
	  ]
	]
      \end{forest}
    }
    \end{column}
  \end{columns}
\end{frame}
\end{document}
