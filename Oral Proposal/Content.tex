%        File: Content.tex
%     Created: Wed Feb 01 01:00 PM 2017 E
% Last Change: Wed Feb 01 01:00 PM 2017 E
%
% arara: pdflatex: {options: "-draftmode"}
% arara: biber
% arara: pdflatex: {options: "-draftmode"}
% arara: pdflatex: {options: "-file-line-error-style"}
\documentclass[Proposal]{subfiles}

\begin{document}
\frame[plain]{\titlepage}
\begin{frame}[plain]
	\frametitle{The plan for my proposal/thesis}
	\tableofcontents
\end{frame}
\section[What is the resultative parameter?]{What is the resultative parameter?}
\subfile{Intro}
\section{How will I explain it?}
\begin{frame}
  \frametitle{Ingredients of an explanation}
  \begin{enumerate}
    \item A structural analysis of resultatives.
    \item The surface phenomena associated with \{*\}Resultatives.
    \item A way of linking the first two ingredients.
  \end{enumerate}
\end{frame}
\begin{frame}
  \frametitle{Ingredients of an explanation}
  \begin{enumerate}
    \setcounter{enumi}{0}
    \item A structural analysis of resultatives.
  \end{enumerate}
  {\rm Natalie hammered the metal flat.}\\
  {\footnotesize
  \begin{forest}
    nice empty nodes,sn edges,baseline,for tree={
    calign=fixed edge angles,
  calign primary angle=-30,calign secondary angle=70}
    [AgrOP
      [DP,[{\rm the metal},roof,name=obj]]
      [
	[AgrO]
	[VP
	  [VP,calign=center
	    [{\rm hammer}]
	    [$\langle$DP$\rangle$,name=theme]
	  ]
	  [resP
	    [$\langle$DP$\rangle$,name=spec res]
	    [
	      [res]
	      [SC
		[$\langle$DP$\rangle$,name=SC theme]
		[{\rm flat}]
	      ]
	    ]
	  ]
	]
      ]
    ]
    \draw[->] (SC theme) to[out=west, in=south] (spec res);
    \draw[->] (spec res) to[out=south, in=south] (theme);
    \draw[->] (theme) to[out=south west, in=south] (obj);
  \end{forest}}
\end{frame}
\begin{frame}
  \frametitle{Ingredients of an explanation}
  \begin{enumerate}
    \setcounter{enumi}{1}      
    \item The surface phenomena associated with \{*\}Resultatives.
      \begin{itemize}
	\item What learnable pattern correlates with resultatives?
        \item Resultatives are strongly correlated with Productive Bare-Stem Compounding. \parencite[][and following]{snyder1995language}
	\item I propose that Bare-Stem Compounding is allowed iff a language's lexicon has Bare Stems.
      \end{itemize}
  \end{enumerate}
\end{frame}
\begin{frame}
  \frametitle{Ingredients of an explanation}
  \begin{enumerate}
    \setcounter{enumi}{2}  
    \item A way of linking the first two ingredients.
      \begin{itemize}
	\item A modified version of Chomsky's (\citeyear{chomsky2013problems,chomsky2015problems}) Label Theory
	\item Resultative SP is derivable only if the secondary predicate is instantiated by a bare stem.
      \end{itemize}
  \end{enumerate}
\end{frame}
\section{What are resultatives?}
\begin{frame}
  \frametitle{Interpretive properties}
  \ex.
  \a. {\rm Natalie hammered the metal flat.}
  \b. There was a hammering event $e$, Natalie was the agent of $e$, and \alert<2|handout:0>{the metal} was the theme of $e$.\\
  $e$ \alert<3|handout:0>{caused} a flatness state $s$, and \alert<2|handout:0>{the metal} was the theme of $s$.

  \pause
  \begin{enumerate}
    \item \alert<2|handout:0>{Argument Sharing}
    \item \alert<3|handout:0>{Causativity}
  \end{enumerate}
\end{frame}
\begin{frame}
  \frametitle{What is Argument Sharing?}
  When a single object/element/phrase/symbol/\textit{etc.} is an argument of multiple predicates.
\end{frame}
\begin{frame}
  \frametitle{Non-resultative Argument Sharing}
  \pause
  \ex. \textbf{Control}\\
    {\rm Alice wants to win.}\\
    Alice$_i$ wants [$ec_i$ to win]

  \pause
\ex. \textbf{Parasitic Gaps}\\
    {\rm Who did you discuss without meeting?}\\
    Who$_i$ did you [[discuss $ec_i$] [without meeting $ec_i$]]?
    
    \pause
    \ex. \textbf{Depictives}\\
    {\rm Monica left angry.}\\
    Monica$_i$ left [$ec_i$ angry].

\end{frame}

%\begin{frame}
%  {What does Argument Sharing look like?}
%  \begin{itemize}
%    \item Where are the shared arguments?
%    \item What are those $ec$s?
%  \end{itemize}
%  
%\end{frame}
%\subfile{UTAH}
%\subfile{MTC}
%\subfile{Causativity}
\begin{frame}
  {What is causativity?}
  
  \ex.\label{inch} {\rm the toast burned.}

  \ex.\label{caus} {\rm Paul burned the toast.}

  \begin{itemize}
    \item \ref{caus} is a causative counterpart to \ref{inch}
      \pause
    \item For \textcite{pietroski2003small}, \ref{caus} describes an eventuality initiated by some action of Paul's and terminating in the eventuality described by \ref{inch}.
      \begin{itemize}
	\item This is quite similar (maybe identical) to Kratzer's (\citeyear{kratzer_building_2004}) discussion of resultatives.
      \end{itemize}
      \pause
    \item $\llbracket res \rrbracket = \lambda e_s \lambda f_s . \textsc{cause}(f)(e)$
  \end{itemize}
\end{frame}
\subfile{Struct}
%\subsection{Deriving Resultatives}
%\begin{frame}
%  \frametitle{Deriving Resultatives}
%  \ex.{\rm Natalie hammered the metal flat.}
%
%  \begin{columns}
%    \begin{column}[T]{0.5\textwidth}
%      \begin{block}
%	{Build the adjunct}
%	\begin{itemize}
%	  \item<2-> Build the Small Clause
%	  \item<3-> Merge(\textit{res}$^\circ$, $\alpha$)
%	  \item<4-> Copy(DP) + Merge(DP, $\beta$)
%	  \item<5-> Copy(DP)
%	\end{itemize}
%      \end{block}
%    \only<5->{
%      \begin{forest}
%	[DP[{\rm the metal},roof]]
%      \end{forest}}
%    \end{column}
%    \begin{column}[T]{0.5\textwidth}
%      {\small
%      \begin{forest}
%	nice empty nodes,sn edges,baseline
%	[$\gamma$,visible on=<4-5>
%	  [DP,visible on=<4-5> [{\rm the metal},roof,visible on=<4-5>]]
%	  [$\beta$,visible on=<4-5>
%	    [\textit{res},visible on=<3-5>]
%	    [$\alpha$,visible on=<3-5>
%	      [DP [{\rm the metal},roof,visible on=<2-5>]]
%	      [{\rm flat},visible on=<2-5>]
%	    ]
%	  ]
%	]
%      \end{forest}
%    }
%    \end{column}
%  \end{columns}
%\end{frame}
%\begin{frame}
%  \frametitle{Deriving Resultatives}
%  \ex.{\rm Natalie hammered the metal flat.}
%
%  \begin{columns}
%    \begin{column}[T]{0.5\textwidth}
%      \begin{block}
%	{Build the VP}
%	\begin{itemize}
%	  \item<2-> Merge({\rm hammer}, DP) 
%	  \item<3-> Merge($\delta$, $\gamma$)
%	  \item<4-> \dots
%	\end{itemize}
%      \end{block}
%    \only<1>{
%      \begin{forest}
%	[DP[{\rm the metal},roof]]
%      \end{forest}}
%    \end{column}
%    \begin{column}[T]{0.5\textwidth}
%    {\small
%      \begin{forest}
%	nice empty nodes,sn edges,baseline
%	[$\zeta$,visible on=<3->
%	  [$\delta$,visible on=<3->
%	    [{\rm hammer},visible on=<2->]
%	    [DP,visible on=<2-> [{\rm the metal},visible on=<2->]]
%	  ]
%	  [$\gamma$,visible on=<3->]
%	]
%      \end{forest}
%    }
%    \end{column}
%  \end{columns}
%\end{frame}
%
\begin{frame}
  {Summary}
  \begin{itemize}
    \item Resultatives involve a resP adjoined to a VP.
    \item The theme DP originates in the resP and moves into the VP.
  \end{itemize}
\end{frame}
\section{Compounding and resultatives}
\subfile{TCP}
\begin{frame}
  {Summary}

  \begin{itemize}
    \item Resultatives are strongly correlated with Bare Stem Compounding
    \item Bare Stems require $cat_\emptyset$ heads.
    \item Only languages with $cat_\emptyset$ heads allow Resultatives
  \end{itemize}
\end{frame}
\section{Label theory}
\subfile{Labels}
\section{Deriving the resultative parameter}
\subfile{PutTogether}
\section{Conclusion and next steps}
\subfile{Conclusion}
\begin{frame}<handout:0>[plain]
  \begin{center}
    {\Huge Thank you}
  \end{center}
\end{frame}
\begin{frame}[t,allowframebreaks]
  \frametitle{References}
  \printbibliography
\end{frame}
\end{document}


