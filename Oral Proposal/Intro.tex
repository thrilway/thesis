%        File: Intro.tex
%     Created: Thu Jan 12 04:00 PM 2017 E
% Last Change: Thu Jan 12 04:00 PM 2017 E
%
% arara: pdflatex: {options: "-draftmode"}
% arara: biber
% arara: pdflatex: {options: "-draftmode"}
% arara: pdflatex: {options: "-file-line-error-style"}
\documentclass[Proposal]{subfiles}

\begin{document}
\begin{frame}
  \frametitle{What is the resultative parameter?}
  \begin{itemize}
    \item Resultative interpretation of secondary predicates
      \begin{itemize}
	\item<1-|handout:1-> Some languages allow it (\textit{e.g.} Germanic languages)
	\item<4-|handout:3-> Some disallow it (\textit{e.g.} Romance Languages)
      \end{itemize}
  \end{itemize}
  \begin{overprint}
    \onslide<2|handout:1>
    \ex. English
    \a. I ate the fish raw. (depictive)
    \b. I hammered the metal flat. (resultative)
    \z.

    \onslide<3|handout:2>
    \ex.German 
    \ag. Er i\ss{}t das Fleisch roh.\\
    He eats the meat raw\\
    ``He's eating the meat raw.'' (depictive)\\
    \parencite{muller2004analysis}
    \bg. Wir haben die teekanne leer getrunken.\\
    we have the teapot empty drink.\textsc{part}\\
    ``We drank the teapot dry.''(resultative)\\
    \parencite{kratzer_building_2004}
    \z.

    \onslide<5|handout:3>
    \ex. French
    \ag. Pierre mange la viande crue.\\
    Pierre eat.3sg the.fem meat raw.fem\\
    ``Pierre ate the meat raw'' (depictive)\\
    \parencite{legendre1997secondary}
    \bg.* Il a march\'e les jambes raides.\\
    He has walked the legs stiff\\
    ``He walked his legs off.'' (resultative)\\
    \parencite{washio1997resultatives}
    \z.

    \onslide<6|handout:4>
    \ex. Italian
    \ag. l' ho mangiato crudo.\\
    3sg \textsc{aux}.1sg eaten raw\\
    ``I ate it raw'' (depictive)
    \bg.* l' ho bevuto vuoto.\\
    3sg \textsc{aux}.1sg drank empty\\
    ``I drank it dry'' (resultative)
    \z.

  \end{overprint}
\end{frame}
\begin{frame}
  \frametitle{Why does it need to be explained?}
  \begin{itemize}
    \item It's not directly learnable from the PLD.
      \begin{itemize}
	\item<3-|handout:2-> Compare this to V-to-T movement.
      \end{itemize}
  \end{itemize}
  \begin{overprint}
    \onslide<2|handout:1>
    \ex. English-type (SP)
    \a. \textsc{Subj} V \textsc{Obj} Adj (depictive)
    \b. \textsc{Subj} V \textsc{Obj} Adj (resultative)

    \ex. French-type (SP)
    \a. \textsc{Subj} V \textsc{Obj} Adj (depictive)
    \b.* \textsc{Subj} V \textsc{Obj} Adj (resultative)

    \onslide<3|handout:2>
    \ex. English-type (Y/N)
    \a. \textsc{do Subj} V \textsc{Obj}? 
    \b.* V \textsc{Subj} \textsc{Obj}?

    \ex. German-type (Y/N)
    \a.* \textsc{do Subj} V \textsc{Obj}? 
    \b. V \textsc{Subj} \textsc{Obj}?

    \onslide<4|handout:3>
    \begin{figure}[h]
      \centering
      \begin{tikzpicture}
	\node (param) at (0,0) {\{*\}V-to-T};
	\node (PLD) at (4,0) {PLD};
	\node (UG) at (0,-2) {UG};
	\draw[->] (PLD) -- (param);
	\draw[->] (UG) -- (param);
      \end{tikzpicture}
      \caption{Acquiring V-to-T}
      \label{fig:VtoTAcq}
    \end{figure}
    \onslide<5|handout:4>
    \begin{figure}[h]
      \centering
      \begin{tikzpicture}
	\node (param) at (0,0) {\{*\}Resultatives};
	\node[cloud,draw] (cloud) at (3,0) {????};
	\node (PLD) at (5,0) {PLD};
	\node (UG) at (1.5,-2) {UG};
	\draw (PLD) -- (cloud);
	\draw[->] (cloud) -- (param);
	\draw[->] (UG) -- (param);
	\draw[->,dashed] (UG) -- (cloud);
      \end{tikzpicture}
      \caption{Acquiring Resultatives}
      \label{fig:resAcq}
    \end{figure}
  \end{overprint}
\end{frame}
\begin{frame}
  \begin{block}
    {My Thesis Project:}
    \begin{itemize}
      \item Explaining how a semantic ``parameter setting'' can be acquired from surface phenomena in the PLD.
    \end{itemize}
    \onslide<2>{
      In other words:
    \begin{itemize}
      \item Figuring out what's behind those question marks.
    \end{itemize}}
  \end{block}
  \onslide<2>{
      \centering
      \begin{tikzpicture}
	\node (param) at (0,0) {\{*\}Resultatives};
	\node[cloud,draw] (cloud) at (3,0) {????};
	\node (PLD) at (5,0) {PLD};
	\node (UG) at (1.5,-2) {UG};
	\draw (PLD) -- (cloud);
	\draw[->] (cloud) -- (param);
	\draw[->] (UG) -- (param);
	\draw[->,dashed] (UG) -- (cloud);
      \end{tikzpicture}
  }
\end{frame}
\end{document}


