% arara: pdflatex: {options: "-draftmode"}
% arara: biber
% arara: pdflatex: {options: "-draftmode"}
% arara: pdflatex: {options: "-file-line-error-style"}
\documentclass[Proposal]{subfiles}

\begin{document}
\begin{frame}
  {Label Theory}
  {\textcite{chomsky2013problems,chomsky2015problems}}
  \begin{itemize}
    \item UG is reducible to simplest merge
  \end{itemize}
  \ex. Merge($\alpha,\beta$) = $\left\{ \alpha,\beta \right\}$

  \pause
  \begin{itemize}
    \pause
    \item Accounts for the fundamental properties of language (\textit{e.g.}, structure-dependence of rules, displacement)
      \pause
      \begin{itemize}
	\item Except for Projection/Labelling
      \end{itemize}
    \pause
    \item Chomsky's proposal: Labels are assigned at the CI interface by a Labelling Algorithm (LA)
  \end{itemize}
\end{frame}
\begin{frame}
  {The Labelling Algorithm}
  \begin{itemize}
    \item LA is a special instance of Minimal Search.
      \pause
      \begin{itemize}
	\item Picks out the most prominent item as a syntactic object's (SO) label
      \end{itemize}
      \pause
    \item There are three relevant classes of SOs for LA:
  \end{itemize}
  \begin{overprint}
    \onslide<3|handout:1>
    \begin{block}
      {(i) Head-Phrase Structures}
      \begin{itemize}[<+->]
        \item Label($\left\{ \text{X, YP} \right\}$) = X
      \end{itemize}
    \end{block}
    \onslide<4-5|handout:2>
    \begin{block}
      {(ii) Head-Head Structures}
      \begin{itemize}[<+->]
        \item Label($\left\{ \text{X, }\textsc{root} \right\}$) = X
          \begin{itemize}
            \item Roots cannot label
          \end{itemize}
        \item Undefined otherwise
      \end{itemize}
    \end{block}
    \onslide<6-8|handout:3>
    \begin{block}
      {(iii) Phrase-Phrase Structures}
      \begin{itemize}[<+->]
        \item Label($\left\{ \text{XP}, \langle\text{YP}\rangle \right\}$) = Label(XP)
          \begin{itemize}
            \item Lower copies are invisible to LA.
          \end{itemize}
        \item Label($\left\{ \text{XP}_F, \text{YP}_F \right\}$)= $\langle\text{F,F}\rangle$
          \begin{itemize}
            \item Iff XP and YP agree for some feature F
          \end{itemize}
        \item Undefined otherwise
      \end{itemize}
    \end{block}
  \end{overprint}
\end{frame}
\begin{frame}
  {The Labelling Algorithm}

  \begin{block}
    {Not everything can label}
    \begin{itemize}[<+->]
      \item Some heads are ``too weak'' to label:
	\begin{itemize}
	  \item Roots
	  \item Heads with only one $\varphi$-set
	    \begin{itemize}
	      \item English T$_\varphi$ needs a subject for labelling.
	    \end{itemize}
	\end{itemize}
    \end{itemize}
  \end{block}
  \begin{itemize}[<+->]
    \item T$_{\langle \varphi,\varphi\rangle}$ (\textit{e.g.,} Italian T) can label on its own.
    \item T$_\emptyset$ (\textit{e.g.,} English Non-finite T) can label on its own.
  \end{itemize}
\end{frame}
\begin{frame}
  {My extensions to Label Theory}
  \begin{block}
    {Labels determine composition}
    \begin{itemize}
      \item The CI interface interprets pairs of SOs and their labels: $\langle \text{L, SO} \rangle$
      \item<2-> If Label(SO) is a head, then SO composes by function application
	\begin{itemize}
	  \item<3-> $\llbracket\langle \text{X}, \left\{ \text{X, YP} \right\}\rangle\rrbracket$ = X(YP)
	\end{itemize}
      \item<4-> If Label(SO) = $\langle\text{F,F}\rangle$, then SO is interpreted as an Operator-variable structure
	\begin{itemize}
	  \item<5-> $\llbracket\langle \langle\text{Q,Q} \rangle, \left\{ \text{DP}_Q, \text{CP}_Q \right\}\rangle\rrbracket$ = $(\text{Wh}x)(\dots x \dots)$
	\end{itemize}
    \end{itemize}
  \end{block}
\end{frame}
\begin{frame}
  {My extensions to Label Theory}
  \begin{block}
    {Adjunction structures are unlabelled}
    \begin{itemize}
      \item Adjuncts are ignored by LA.
	\begin{itemize}
	  \item Label($\left\{ \text{XP, ZP} \right\}$) = $\emptyset$ if ZP is an adjunct
	  \item Since ZP is ignored by LA, it is internally unlabelled. (Label(ZP)=$\emptyset$)
	\end{itemize}
	\pause
      \item Unlabelled SOs compose by conjunction.
	\begin{itemize}
	  \item $\llbracket\langle \emptyset, \left\{ \text{XP, ZP} \right\}\rangle\rrbracket$ = XP \& ZP 
	\end{itemize}
    \end{itemize}
  \end{block}
\end{frame}
\subfile{AC2}
\end{document}
