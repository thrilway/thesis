% arara: pdflatex: {options: "-draftmode"}
% arara: biber
% arara: pdflatex: {options: "-draftmode"}
% arara: pdflatex: {options: "-file-line-error-style"}
\documentclass[Proposal]{subfiles}

\begin{document}
\begin{frame}
  {Label Theory}
  {\textcite{chomsky2013problems,chomsky2015problems}}
  \begin{itemize}
    \item UG is reducible to simplest merge
  \end{itemize}
  \ex. Merge($\alpha,\beta$) = $\left\{ \alpha,\beta \right\}$

  \begin{itemize}
    \item Accounts for the fundamental properties of language (\textit{e.g.}, structure-dependence of rules, displacement)
      \begin{itemize}
	\item Except for Projection/Labeling
      \end{itemize}
    \item Chomsky's proposal: Labels are assigned at the CI interface by a Labeling Algorithm (LA)
  \end{itemize}
\end{frame}
\begin{frame}
  {The Labeling Algorithm}
  \begin{itemize}
    \item LA is a special instance of Minimal Search.
      \begin{itemize}
	\item Picks out the most prominent item as a syntactic object's label
      \end{itemize}
      \pause
    \item There are three relevant classes of syntactic objects for LA:
  \end{itemize}
  \begin{overprint}
    \onslide<3>
    \begin{block}
      {(i) Head-Phrase Structures}
      \begin{itemize}
	\item Label($\left\{ \text{X, YP} \right\}$) = X
      \end{itemize}
    \end{block}
    \onslide<4-5>
    \begin{block}
      {(ii) Head-Head Structures}
      \begin{itemize}
	\item<4-5> Label($\left\{ \text{X, }\textsc{root} \right\}$) = X
	  \begin{itemize}
	    \item<4-5> Roots cannot label
	  \end{itemize}
	\item<5> Undefined otherwise
      \end{itemize}
    \end{block}
    \onslide<6-8>
    \begin{block}
      {(iii) Phrase-Phrase Structures}
      \begin{itemize}
	\item<6-8> Label($\left\{ \text{XP}, \langle\text{YP}\rangle \right\}$) = Label(XP)
	  \begin{itemize}
	    \item<6-8> Lower copies are invisible to LA.
	  \end{itemize}
	\item<7-8> Label($\left\{ \text{XP}_F, \text{YP}_F \right\}$)= $\langle\text{F,F}\rangle$
	  \begin{itemize}
	    \item<7-8> Iff XP and YP agree for some feature F
	  \end{itemize}
	\item <8> Undefined otherwise
      \end{itemize}
    \end{block}
  \end{overprint}
\end{frame}
\begin{frame}
  {My extensions to Label Theory}
  \begin{block}
    {Labels determine composition}
    \begin{itemize}
      \item<2-> If Label(SO) $\in$ SO, then SO composes by function application
	\begin{itemize}
	  \item<3-> $\llbracket\left\{ \text{X, YP} \right\}\rrbracket$ = X(YP)
	\end{itemize}
      \item<4-> If Label(SO) = $\langle\text{F,F}\rangle$, then SO is interpreted as an Operator-variable structure
	\begin{itemize}
	  \item<5-> $\llbracket\left\{ \text{DP}_Q, \text{CP}_Q \right\}\rrbracket$ = $(\text{Wh}x)(\dots x \dots)$
	\end{itemize}
    \end{itemize}
  \end{block}
\end{frame}
\begin{frame}
  {My extensions to Label Theory}
  \begin{block}
    {Adjunction structures are unlabeled}
    \begin{itemize}
      \item Adjuncts are ignored by LA.
	\begin{itemize}
	  \item Label($\left\{ \text{XP, ZP} \right\}$) = $\emptyset$ if ZP is an adjunct
	  \item Since ZP is ignored by LA, it is internally unlabelled. (Label(ZP)=$\emptyset$)
	\end{itemize}
	\pause
      \item Unlabeled SOs compose by conjuction.
	\begin{itemize}
	  \item $\llbracket\left\{ \text{XP, ZP} \right\}\rrbracket$ = XP \& ZP (if $\left\{ \text{XP, ZP} \right\}$ is unlabelled)
	\end{itemize}
    \end{itemize}
  \end{block}
\end{frame}
\begin{frame}
  {What can this version of Label Theory get us?}
  \begin{block}
    {Subjects of ACC-ing clauses and pseudo-relatives}
    \begin{itemize}
      \item<2-> \textcite{cinque1996pseudo} shows that the position of an ACC-ing/pseudo-relative clause affects the behaviour of its subject.
    \end{itemize}
    \only<2->{
    \ex. \textbf{Italian pseudo-relative (PR)}\\
    {\rm Mario che correva a tutti velocit\`a}

    \ex. \textbf{English ACC-ing clause (AC)}\\
    {\rm Mario running at full speed}

  }
    \begin{itemize}
      \item<3-> If the AC/PR is a complement of V, the subject cannot move.
      \item<3-> If the AC/PR is a VP adjunct, the subject must move.
    \end{itemize}
  \end{block}
\end{frame}
\begin{frame}
  {What can this version of Label Theory get us?}

    \begin{columns}
    \begin{column}[T]{0.6\textwidth}
      {\rm *Mario$_i$ was [$_{VP}$ [seen[$t_i$ running]]]}
      \begin{block}
	{Complement ACs}
	\begin{itemize}
	  \item Subject is frozen in the AC
	  \item Prog$^\circ$ is too weak to label $\zeta$
	    \begin{itemize}
	      \item Compare Chomsky's (2015) discussion of EPP
	    \end{itemize}
	  \item If {\rm Mario} were \textit{in situ}, Label($\zeta$) = $\langle\text{F,F}\rangle$
	  \item Lower copies are invisible, so $\zeta$ is unabelable.
	  \item The derivation crashes at CI
	\end{itemize}
      \end{block}
    \end{column}
    \begin{column}[T]{0.4\textwidth}
	{\small
	  \begin{forest}
	    nice empty nodes,sn edges,baseline
	    [$\alpha$
	      [{\rm Mario}]
	      [$\beta$
		[T]
		[$\gamma$
		  [Voice$_{pass}$]
		  [$\delta$
		    [{\rm see}]
		    [$\zeta$
		      [$\langle${\rm Mario}$\rangle$]
		      [Prog]
		    ]
		  ]
		]
	      ]
	    ]
	  \end{forest}
	}
    \end{column}
  \end{columns}
\end{frame}
\begin{frame}
  {What can this version of Label Theory get us?}
  \begin{columns}
    \begin{column}[T]{0.55\textwidth}
      {\rm *I [ [$_{VP}$ saw Bill] [Mario running]]}
      \begin{block}
	{Adjunct ACs}
	\begin{itemize}
	  \item Subjects must move to theme position
	  \item $\eta$ is adjoined, therefore ignored by LA
	  \item $\eta$ is interpreted as the conjunction of {\rm Mario} and {\rm running}
	    \begin{itemize}
	      \item This is an ill-formed interpretation
	    \end{itemize}
	\end{itemize}
      \end{block}
    \end{column}
    \begin{column}[T]{0.45\textwidth}
      {\small
	  \begin{forest}
	    nice empty nodes,sn edges,baseline
	    [$\alpha$
	      [{\rm I}]
	      [$\beta$
		[Voice]
		[$\gamma$
		  [$\zeta$
		    [{\rm see}]
		    [{\rm Bill}]
		  ]
		  [$\eta$
		    [{\rm Mario}]
		    [Prog]
		  ]
		]
	      ]
	    ]
	  \end{forest}
	}
    \end{column}
  \end{columns}
\end{frame}
\begin{frame}
  {What can this version of Label Theory get us?}
    
\end{frame}
\end{document}
