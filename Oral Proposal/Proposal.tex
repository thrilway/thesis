% arara: pdflatex
% arara: biber
% arara: pdflatex
% arara: pdflatex

\documentclass[
  %handout
]{beamer}
\usetheme{CambridgeUS}
\usecolortheme{crane}
\usepackage[
  backend=biber,
  style=authoryear-comp,
  useprefix=false,
  sorting=ynt
]{biblatex}

\usepackage{subfiles}
\usepackage{stmaryrd}
\usepackage[]{amsmath}
\usepackage{amsfonts}
\usepackage{amssymb}
\usepackage{mathtools}
\usepackage{forest}
\usepackage{tabularx}
\usepackage{linguex}
\usepackage{centernot}
\usepackage{todonotes}
\usepackage[normalem]{ulem}
\useforestlibrary{linguistics}

\tikzset{
    invisible/.style={opacity=0,text opacity=0},
    visible on/.style={alt=#1{}{invisible}},
    alt/.code args={<#1>#2#3}{%
      \alt<#1>{\pgfkeysalso{#2}}{\pgfkeysalso{#3}} % \pgfkeysalso doesn't change the path
    },
}
%\forestset{
%  visible on/.style={
%    for current and ancestors={
%      /tikz/visible on={#1},
%      %edge={/tikz/invisible}
%      %edge={/tikz/visible on={#1}}
%    },
%    for descendants={
%      %/tikz/visible on={#1},
%      %edge={/tikz/invisible}
%      edge={/tikz/visible on={#1}}
%    }
%  }
%  }

\forestset{%
  declare toks={no node before}{1},
  declare toks={no edge before}{1},
    not before/.style={
    no node before=#1,
    for children={
      no edge before=#1,
    }
  },
  bottom up/.style={% based on Qrrbrbirlbel's answer at http://tex.stackexchange.com/a/112895/
      /tikz/visible on=<\forestoption{no node before}->,
      /tikz/every label/.append style={visible on=<\forestoption{no node before}->},
      /tikz/every edge label/.append style={visible on=<\forestoption{no edge before}->},
      edge={/tikz/visible on=<\forestoption{no edge before}->},
  }
}
\forestset{tree defaults/.style={for tree={parent anchor=south, child anchor=north},every tree node/.style={align=center,anchor=north},level/.style={sibling distance=50mm/#1},baseline}}

\forestset{en/.style={parent anchor=center, child anchor=center}}
\forestset{em/.style={parent anchor=north west, child anchor=north west}}
\forestset{el/.style={parent anchor=north, child anchor=north}}

\usetikzlibrary{positioning}
\usetikzlibrary{calc}
\usetikzlibrary{arrows}
\usetikzlibrary{decorations.markings}
%\DeclareNameFormat{labelname:poss}{% Based on labelname from biblatex.def
%  \ifcase\value{uniquename}%
%  \usebibmacro{name:last}{#1}{#3}{#5}{#7}%
%  \or
%  \ifuseprefix
%  {\usebibmacro{name:first-last}{#1}{#4}{#5}{#8}}
%  {\usebibmacro{name:first-last}{#1}{#4}{#6}{#8}}%
%  \or
%  \usebibmacro{name:first-last}{#1}{#3}{#5}{#7}%
%  \fi
%  \usebibmacro{name:andothers}%
%  \ifnumequal{\value{listcount}}{\value{liststop}}{'s}{}
%}
%
%\DeclareFieldFormat{shorthand:poss}{%
%  \ifnameundef{labelname}{#1's}{#1}
%}
%
%\DeclareFieldFormat{citetitle:poss}{\mkbibemph{#1}'s}
%
%\DeclareFieldFormat{label:poss}{#1's}
%
%\newrobustcmd*{\posscitealias}{%
%  \AtNextCite{%
%    \DeclareNameAlias{labelname}{labelname:poss}%
%    \DeclareFieldAlias{shorthand}{shorthand:poss}%
%    \DeclareFieldAlias{citetitle}{citetitle:poss}%
%    \DeclareFieldAlias{label}{label:poss}
%  }
%}
%
%\newrobustcmd*{\posscite}{%
%  \posscitealias%
%  \textcite
%}
%
%\newrobustcmd*{\Posscite}{\bibsentence\posscite}
%
%\newrobustcmd*{\posscites}{%
%  \posscitealias%
%  \textcites
%}

\newcommand\quelle[1]{{%
  \unskip\nobreak\hfil\penalty50
  \hskip2em\hbox{}\nobreak\hfil#1%
  \parfillskip=0pt \finalhyphendemerits=0 \par
}
}

\newcommand{\figex}{\refstepcounter{ExNo}\theExNo\hspace{\Exlabelsep}}

\renewcommand<>{\sout}[1]{
  \only#2{\beameroriginal{\sout}{#1}}
  \invisible#2{#1}
}

\newcounter{DerivStep}

\newcommand{\hxp}{$\left\{ \text{X, YP} \right\}$}
\newcommand{\hh}{$\left\{ \text{X, Y} \right\}$}
\newcommand{\xpyp}{$\left\{ \text{XP, YP} \right\}$}
\bibliography{Thesis}
%\AtBeginSection[]
%{
%  \begin{frame}
%    \frametitle{Table of Contents}
%    \tableofcontents[currentsection]
%  \end{frame}
%}
\title{Explaining the Resultative Parameter}
\subtitle{Thesis Proposal}
\author{Dan Milway}
\AtBeginSection[]{
  \begin{frame}
  \vfill
  \centering
  \begin{beamercolorbox}[sep=8pt,center,shadow=true,rounded=true]{title}
    \usebeamerfont{title}\insertsectionhead\par%
  \end{beamercolorbox}
  \vfill
  \end{frame}
}

\begin{document}
\section{}
\frame[plain]{\titlepage}
\section[What is the Resultative Parameter?]{What is the Resultative Parameter?}
\subfile{Intro}
\section{How will I explain it?}
\begin{frame}
  \frametitle{Ingredients of an explanation}
  \begin{enumerate}
    \item A structural analysis of resultatives.
    \item The surface phenomena associated with \{*\}Resultatives.
    \item A way of linking the first two ingredients.
  \end{enumerate}
\end{frame}
\begin{frame}
  \frametitle{Ingredients of an explanation}
  \begin{enumerate}
    \setcounter{enumi}{0}
    \item A structural analysis of resultatives.
  \end{enumerate}
  {\rm Natalie hammered the metal flat.}\\
  {\footnotesize
  \begin{forest}
    nice empty nodes,sn edges,baseline,for tree={
    calign=fixed edge angles,
  calign primary angle=-30,calign secondary angle=70}
    [AgrOP
      [DP,[{\rm the metal},roof,name=obj]]
      [
	[AgrO]
	[VP
	  [VP,calign=center
	    [{\rm hammer}]
	    [$\langle$DP$\rangle$,name=theme]
	  ]
	  [resP
	    [$\langle$DP$\rangle$,name=spec res]
	    [
	      [res]
	      [SC
		[$\langle$DP$\rangle$,name=SC theme]
		[{\rm flat}]
	      ]
	    ]
	  ]
	]
      ]
    ]
    \draw[->] (SC theme) to[out=west, in=south] (spec res);
    \draw[->] (spec res) to[out=south, in=south] (theme);
    \draw[->] (theme) to[out=south west, in=south] (obj);
  \end{forest}}
\end{frame}
\begin{frame}
  \frametitle{Ingredients of an explanation}
  \begin{enumerate}
    \setcounter{enumi}{1}      
    \item The surface phenomena associated with \{*\}Resultatives.
      \begin{itemize}
	\item What learnable pattern correlates with resultatives?
        \item Resultatives are strongly correlated with Productive Bare-Stem Compounding. \parencite[][and following]{snyder1995language}
	\item I propose that Bare-Stem Compounding is allowed iff a language's lexicon has Bare Stems.
      \end{itemize}
  \end{enumerate}
\end{frame}
\begin{frame}
  \frametitle{Ingredients of an explanation}
  \begin{enumerate}
    \setcounter{enumi}{2}  
    \item A way of linking the first two ingredients.
      \begin{itemize}
	\item A modified version of Chomsky's (\citeyear{chomsky2013problems,chomsky2015problems}) Label Theory
	\item Resultative SP is derivable only if the secondary predicate is instantiated by a bare stem.
      \end{itemize}
  \end{enumerate}
\end{frame}
\section{What are resultatives?}
\begin{frame}
  \frametitle{Interpretive properties}
  \ex.
  \a. {\rm Natalie hammered the metal flat.}
  \b. There was a hammering event $e$, Natalie was the agent of $e$, and \alert<2>{the metal} was the theme of $e$.\\
  $e$ \alert<3>{caused} a flatness state $s$, and \alert<2>{the metal} was the theme of $s$.

  \pause
  \begin{enumerate}
    \item \alert<2>{Argument Sharing}
    \item \alert<3>{Causativity}
  \end{enumerate}
\end{frame}
\subsection{Argument Sharing}
\begin{frame}
  \frametitle{What is Argument Sharing?}
  When a single object/element/phrase/symbol/\textit{etc.} is an argument of multiple predicates.
\end{frame}
\begin{frame}
  \frametitle{Non-resultative Argument Sharing}
  \pause
  \ex. \textbf{Control}\\
    {\rm Alice wants to win.}\\
    Alice$_i$ wants [$ec_i$ to win]

  \pause
\ex. \textbf{Parasitic Gaps}\\
    {\rm Who did you discuss without meeting?}\\
    Who$_i$ did you [[discuss $ec_i$] [without meeting $ec_i$]]?
    
    \pause
    \ex. \textbf{Depictives}\\
    {\rm Monica left angry.}\\
    Monica$_i$ left [$ec_i$ angry].

\end{frame}
\begin{frame}
  {What does Argument Sharing look like?}
  \begin{itemize}
    \item Where are the shared arguments?
    \item What are those $ec$s?
  \end{itemize}
  
\end{frame}
\subfile{UTAH}
\subfile{MTC}
\subsubsection{Sideward Movement}
\subfile{SWM}
\subsection{Causativity}
\subfile{Causativity}
\subsection{Deriving Resultatives}
\begin{frame}
  \frametitle{Deriving Resultatives}
  \ex.{\rm Natalie hammered the metal flat.}

  \begin{columns}
    \begin{column}[T]{0.5\textwidth}
      \begin{block}
	{Build the adjunct}
	\begin{itemize}
	  \item<2-> Build the Small Clause
	  \item<3-> Merge(\textit{res}$^\circ$, $\alpha$)
	  \item<4-> Copy(DP) + Merge(DP, $\beta$)
	  \item<5-> Copy(DP)
	\end{itemize}
      \end{block}
    \only<5->{
      \begin{forest}
	[DP[{\rm the metal},roof]]
      \end{forest}}
    \end{column}
    \begin{column}[T]{0.5\textwidth}
      {\small
      \begin{forest}
	nice empty nodes,sn edges,baseline
	[$\gamma$,visible on=<4-5>
	  [DP,visible on=<4-5> [{\rm the metal},roof,visible on=<4-5>]]
	  [$\beta$,visible on=<4-5>
	    [\textit{res},visible on=<3-5>]
	    [$\alpha$,visible on=<3-5>
	      [DP [{\rm the metal},roof,visible on=<2-5>]]
	      [{\rm flat},visible on=<2-5>]
	    ]
	  ]
	]
      \end{forest}
    }
    \end{column}
  \end{columns}
\end{frame}
\begin{frame}
  \frametitle{Deriving Resultatives}
  \ex.{\rm Natalie hammered the metal flat.}

  \begin{columns}
    \begin{column}[T]{0.5\textwidth}
      \begin{block}
	{Build the VP}
	\begin{itemize}
	  \item<2-> Merge({\rm hammer}, DP) 
	  \item<3-> Merge($\delta$, $\gamma$)
	  \item<4-> \dots
	\end{itemize}
      \end{block}
    \only<1>{
      \begin{forest}
	[DP[{\rm the metal},roof]]
      \end{forest}}
    \end{column}
    \begin{column}[T]{0.5\textwidth}
    {\small
      \begin{forest}
	nice empty nodes,sn edges,baseline
	[$\zeta$,visible on=<3->
	  [$\delta$,visible on=<3->
	    [{\rm hammer},visible on=<2->]
	    [DP,visible on=<2-> [{\rm the metal},visible on=<2->]]
	  ]
	  [$\gamma$,visible on=<3->]
	]
      \end{forest}
    }
    \end{column}
  \end{columns}
\end{frame}

\section{Compounding and Resultatives}
\subfile{TCP}
\section{Label Theory}
\subfile{Labels}
\section{Deriving the Resultative Parameter}
\subfile{PutTogether}
\end{document}
