%        File: Struct.tex
%     Created: Thu Jan 19 10:00 AM 2017 E
% Last Change: Thu Jan 19 10:00 AM 2017 E
%
% arara: pdflatex: {options: "-draftmode"}
% arara: biber
% arara: pdflatex: {options: "-draftmode"}
% arara: pdflatex: {options: "-file-line-error-style"}
\documentclass[Proposal]{subfiles}

\begin{document}
\begin{frame}
  \begin{block}
    {Two types of analyses for SP}
    \begin{enumerate}
      \item<2-> \alert<2>{Small Clause} \parencite[\textit{e.g.,}][]{kratzer_building_2004}
      \item<3-> \alert<3>{Complex Predicate} \parencite[\textit{e.g.,}][]{snyder1995language}
    \end{enumerate}
  \end{block}
  \begin{columns}
    \begin{column}
      [T]{0.5\textwidth}
      \onslide<2->{
      \begin{center}
      {\small
      \begin{forest}
	nice empty nodes,sn edges,baseline
	[VP
	  [DP$_i$[{\rm the fish},roof]]
	  [
	    [V\\{\rm eat},align=center]
	    [SC
	      [$ec_i$]
	      [Adj\\{\rm raw},align=center]
	    ]
	  ]
	]
      \end{forest}}	
    \end{center}}
    \end{column}
    \begin{column}
      [T]{0.5\textwidth}
      \onslide<3->{
      \begin{center}
      {\small
      \begin{forest}
	nice empty nodes,sn edges,baseline
	[VP
	  [DP$_i$[{\rm the fish},roof]]
	    [V
	      [V\\{\rm eat},align=center]
	      [Adj\\{\rm raw},align=center]
	    ]
	  ]
      \end{forest}
      }	
      \end{center}
    }
    \end{column}
  \end{columns}
  \onslide<4>{How does each account for the properties of resultatives?}
\end{frame}
\begin{frame}
  {Small Clause}
  \begin{columns}
    \begin{column}
      [T]{0.5\textwidth}
      \begin{block}
	{Argument Sharing}
	\begin{itemize}
	  \item The shared argument is local to both predicates.
	  \item Argument sharing happens by $\Theta$-role assignment.
	\end{itemize}
      \end{block}
      \begin{block}
	{Causativity}
	\begin{itemize}
	  \item A functional head merged with SC
	\end{itemize}
      \end{block}
    \end{column}
    \begin{column}
      [T]{0.5\textwidth}
      {\small
      \begin{forest}
	nice empty nodes,sn edges,baseline
	[VP
	  [DP$_i$[{\rm the metal},roof]]
	  [
	    [V\\{\rm hammer},align=center]
	    [resP
	      [res]
	      [SC
		  [$ec_i$]
		  [Adj\\{\rm flat},align=center]
		]
	      ]
	    ]
	  ]
      \end{forest}}	
    \end{column}
  \end{columns}
\end{frame}
\begin{frame}
  {Complex Predicate}
  {Version 1}
  \begin{columns}
    \begin{column}
      [T]{0.5\textwidth}
      \begin{block}
	{Argument Sharing}
	\begin{itemize}
	  \item A functional head merged with SP
	\end{itemize}
      \end{block}
      \begin{block}
	{Causativity}
	\begin{itemize}
	  \item A functional head merged with SP
	\end{itemize}
      \end{block}
    \end{column}
    \begin{column}
      [T]{0.5\textwidth}
      {\small
      \begin{forest}
	nice empty nodes,sn edges,baseline
	[VP
	  [DP$_i$[{\rm the metal},roof]]
	  [V
	      [V\\{\rm hammer},align=center]
	      [
		[res+]
		[Adj\\{\rm flat},align=center]
	      ]
	    ]
	  ]
      \end{forest}}	
    \end{column}
  \end{columns}
\end{frame}
\begin{frame}
  {Complex Predicate}
  {Version 2}
  \begin{columns}
    \begin{column}
      [T]{0.5\textwidth}
      \begin{block}
	{Argument Sharing}
	\begin{itemize}
	  \item A special mode of composing V+Adj
	\end{itemize}
      \end{block}
      \begin{block}
	{Causativity}
	\begin{itemize}
	  \item A special mode of composing V+Adj
	\end{itemize}
      \end{block}
    \end{column}
    \begin{column}
      [T]{0.5\textwidth}
      {\small
      \begin{forest}
	nice empty nodes,sn edges,baseline
	[VP
	  [DP$_i$[{\rm the metal},roof]]
	  [V
	      [V\\{\rm hammer},align=center]
	      [Adj\\{\rm flat},align=center]
	    ]
	  ]
      \end{forest}}	
    \end{column}
  \end{columns}
\end{frame}
\begin{frame}
  
  \begin{itemize}
    \item I will assume a Small Clause Structure
    \item Its assumptions are well-founded:
      \begin{itemize}
	\item $\Theta$-role assignment is local.
	\item An element like \textit{res} is needed anyway
	  \begin{itemize}
	    \item {\rm flat}+res $\rightarrow$ {\rm flatten}
	  \end{itemize}
      \end{itemize}
    \item Complex Predicate analyses require ad-hoc stipulations
      \begin{itemize}
	\item $\Theta$-role assignment at a distance.
	\item Both \textit{res+} and the special composition are too specific to resultatives.
      \end{itemize}
  \end{itemize}
\end{frame}
\begin{frame}
  {\textcite{kratzer_building_2004}}

  {\rm hammer the metal flat}\\
  \begin{forest}
    nice empty nodes,sn edges,baseline,for tree={
    calign=fixed edge angles,
  calign primary angle=-30,calign secondary angle=70}
    [VP
      [DP$_i$[{\rm the metal},roof]]
      [
	[V\\{\rm hammer},align=center]
	[resP
	  [res]
	  [AdjP
	    [$\langle\text{DP}_i\rangle$]
	    [Adj\\{\rm flat},align=center]
	  ]
	]
      ]
    ]
  \end{forest}

\end{frame}
\begin{frame}
  {\textcite{kratzer_building_2004}}
  {Modifications}

  \begin{itemize}
    \item For Kratzer resP is of type $\langle s, t\rangle$
      \begin{itemize}
	\item $\lambda e_s \exists s_s [ \textsc{cause}(e, s) \& \text{flat}(s) \& \textsc{theme}(s, \textbf{the\_metal})]$
      \end{itemize}
    \item V is of type $\langle e, \langle s,t\rangle\rangle$
      \begin{itemize}
	\item $\lambda x_e \lambda e_s [\text{hammering}(e) \& \textsc{theme}(e)(x)]$
      \end{itemize}
    \item A new composition rule is needed: 
  \end{itemize}
  \ex. Event Identification \parencite{kratzer_severing_1996}\\
  \begin{tabular}[t]{cccc}
    $f$ & $g$ & $\rightarrow$ & $h$\\
    $\langle e, \langle s,t\rangle\rangle$ & $\langle s, t\rangle$ & & $\langle e, \langle s,t\rangle\rangle$\\
    & & & $\lambda x_e \lambda e_s[f(x)(e) \& g(e)]$\\
  \end{tabular}

\end{frame}
\begin{frame}
  {\textcite{kratzer_building_2004}}
  {Modifications}

  \begin{itemize}
    \item If resP adjoins to VP\ldots
  \end{itemize}
    \only<1>{
	\begin{forest}
      nice empty nodes,sn edges,baseline,for tree={
    calign=fixed edge angles,
  calign primary angle=-30,calign secondary angle=70}
      [VP
	[VP
	  [DP$_i$[{\rm the metal},roof]]
	  [V\\{\rm hammer},align=center]
	]
	[resP
	  [res]
	  [AdjP
	    [$\langle\text{DP}_i\rangle$]
	    [Adj\\{\rm flat},align=center]
	  ]
	]
      ]
    \end{forest}
  
  }
  \onslide<2->{
    \begin{itemize}
      \item<2-> resP and VP are both of type $\langle s,t\rangle$
	\begin{itemize}
	  \item<2-> $\lambda e_s \exists s_s [ \textsc{cause}(e, s) \& \text{flat}(s) \& \textsc{theme}(s, \textbf{the\_metal})]$
	  \item<2-> $\lambda e_s [\text{hammering}(e) \& \textsc{theme}(e)(\textbf{the\_metal})]$
	\end{itemize}
      \item<3-> They can combine by a simple conjunction operation.
      \item<3-> Predicate Modification in \textcite{heimkratzer1998semantics}.
	\begin{itemize}
	  \item<4-> $\llbracket \left[ \text{VP resP} \right]\rrbracket = \lambda e_s [\llbracket\text{VP}\rrbracket(e) \& \llbracket \text{resP}\rrbracket(e)]$
	\end{itemize}
      \item<5-> No need to complicate our compositional semantics.
      \item<6-> But\ldots
    \end{itemize}
  }
\end{frame}
\begin{frame}
  {\textcite{kratzer_building_2004}}
  {Modifications}

  \begin{itemize}
    \item The movement proposed is sideward. 
  \end{itemize}{\small
	\begin{forest}
      nice empty nodes,sn edges,baseline,for tree={
    calign=fixed edge angles,
  calign primary angle=-30,calign secondary angle=70}
      [VP
	[VP
	  [DP$_i$[{\rm the metal},roof]]
	  [V\\{\rm hammer},align=center]
	]
	[resP
	  [res]
	  [AdjP
	    [$\langle\text{DP}_i\rangle$]
	    [Adj\\{\rm flat},align=center]
	  ]
	]
      ]
    \end{forest}
  }
  \begin{itemize}
    \item<2-> Standard Minimalist/GB: Movement is always upward.
      \begin{itemize}
	\item<3-> Sideward movement (SWM) is impossible
      \end{itemize}
    \item<4-> \textcite{nunes2001sideward}: SWM follows from (a version of) the copy theory of movement.
  \end{itemize}
\end{frame}
\subsubsection{Sideward Movement}
\subfile{SWM}
\subsubsection{Two further assumptions}
\begin{frame}
  
\end{frame}<++>
\end{document}


