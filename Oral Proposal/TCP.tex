%        File: TCP.tex
%     Created: Fri Jan 13 01:00 PM 2017 E
% Last Change: Fri Jan 13 01:00 PM 2017 E
%
% arara: pdflatex: {options: "-draftmode"}
% arara: biber
% arara: pdflatex: {options: "-draftmode"}
% arara: pdflatex: {options: "-file-line-error-style"}
\documentclass[Proposal]{subfiles}

\begin{document}
\begin{frame}
  \frametitle{Compounding and Resultatives}
  \begin{itemize}
    \item \textcite{snyder1995language} presents a crosslinguistic generalization:
  \end{itemize}
  \ex. A language allows \textcolor<2>{red}{complex predicates} if and only if it freely allows \textcolor<2>{green}{open class, ordinary non-affixal lexical items to function as affixes}. \parencite[35]{snyder1995language}

  \begin{itemize}
    \item<2> \textcolor{red}{Resultatives} $\leftrightarrow$ \textcolor{green}{N-N Compounding}
  \end{itemize}
\end{frame}
\begin{frame}
  \frametitle{Compounding and Resultatives}
  \begin{center}
  {\small
  \begin{tabular}[t]{lcc}
    \textbf{Language} & \textbf{Resultatives} & \textbf{N-N Compounding}\\
    \hline
    English & Yes & Yes\\
    Dutch & Yes & Yes\\
    German & Yes & Yes\\
    Khmer & Yes & Yes\\
    Hungarian & Yes & Yes\\
    & & \\
    French & No & No\\
    Spanish & No & No\\
    Russian & No & No\\
    Serbo-Croatian & No & No\\
    Japanese & No & No\\
    ASL & No & No\\
    Mandarin & No & No\\
    Modern Hebrew & No & No\\
    Palestinian Arabic & No & No\\
  \end{tabular}}
\end{center}
  \quelle{\parencite[31]{snyder1995language}}
\end{frame}
\end{document}


