%        File: TCP.tex
%     Created: Fri Jan 13 01:00 PM 2017 E
% Last Change: Fri Jan 13 01:00 PM 2017 E
%
% arara: pdflatex: {options: "-draftmode"}
% arara: biber
% arara: pdflatex: {options: "-draftmode"}
% arara: pdflatex: {options: "-file-line-error-style"}
\documentclass[Proposal]{subfiles}

\begin{document}
\begin{frame}
  \frametitle{Compounding and Resultatives}
  \begin{itemize}
    \item \textcite{snyder1995language} presents a crosslinguistic generalization:
  \end{itemize}
  \ex. A language allows \textcolor<2>{red}{complex predicates} if and only if it freely allows \textcolor<2>{green}{open class, ordinary non-affixal lexical items to function as affixes}. \parencite[35]{snyder1995language}

  \begin{itemize}
    \item<2> \textcolor{red}{Resultatives} $\leftrightarrow$ \textcolor{green}{N-N Compounding}
  \end{itemize}
\end{frame}
\begin{frame}
  \frametitle{Compounding and Resultatives}
  \begin{center}
  {\small
  \begin{tabular}[t]{lcc}
    \textbf{Language} & \textbf{Resultatives} & \textbf{N-N Compounding}\\
    \hline
    English & Yes & Yes\\
    Dutch & Yes & Yes\\
    German & Yes & Yes\\
    Khmer & Yes & Yes\\
    Hungarian & Yes & Yes\\
    & & \\
    French & No & No\\
    Spanish & No & No\\
    Russian & No & No\\
    Serbo-Croatian & No & No\\
    Japanese & No & No\\
    ASL & No & No\\
    Mandarin & No & No\\
    Modern Hebrew & No & No\\
    Palestinian Arabic & No & No\\
  \end{tabular}}
\end{center}
  \quelle{\parencite[31]{snyder1995language}}
\end{frame}
\begin{frame}
  \frametitle{Compounding and Resultatives}
  \begin{itemize}
    \item \textcite{snyder2016compound} refines his generalization to:
      \begin{itemize}
	\item Resultatives $\leftrightarrow$ Creative bare-stem compounding (BSC)
      \end{itemize}
    \item A bare stems is any form that:
      \begin{itemize}
	\item ``could be used as an independent word''
	\item ``is the  form that inflectional morphology would combine with''
	\item ``does not yet bear any inflection'' \hfill\parencite{snyder2016compound}
      \end{itemize}
  \end{itemize}
  \ex. 
  \a. {\rm book shelf}
  \b.* {\rm books shelf}
  \z.

  \begin{itemize}
    \item Compounding is creative if:
      \begin{itemize}
	\item ``it is available for automatic, impromptu use whenever a new word is needed to fit the occasion.'' \parencite{snyder2016compound}
      \end{itemize}
  \end{itemize}
\end{frame}
\begin{frame}
  \frametitle{Compounding and Resultatives}
  \begin{itemize}
    \item Bare-stem compounding excludes:
  \end{itemize}
  \begin{overprint}
    \onslide<2|handout:1>
    \textbf{Synthetic -ER compounds}
    \exg. lave -vaisselle \parencite[French,][]{snyder2016compound}\\
    wash -dishware\\
    ``dishwasher'' 

    \onslide<3|handout:2>
    \textbf{Compound phrases}
    \exg. boite aux verres \parencite[French,][]{snyder2016compound}\\
    can for.the worms\\
    ``worm can''

    \onslide<4|handout:3>
    \textbf{Construct state expressions}
    \exg. kufsat tulaAim \parencite[Hebrew,][]{snyder2016compound}\\
    can.of worm\\
    ``worm can''

  \end{overprint}
\end{frame}
\begin{frame}
  {Bare-stem compounding}
  {The lexical basis}

  \begin{itemize}
    \item \sout<3->{How is the bare-stem compounding parameter represented in the grammar?}
  \end{itemize}
  \pause
  \ex. \textbf{The Borer-Chomsky Conjecture} \parencite{baker2008microparameter}\\
  All parameters of variation are attributable to differences in the features of particular items (e.g., the functional heads) in the lexicon.

  \begin{itemize}
      \pause
    \item What lexical properties are responsible for the bare-stem compounding parameter?
  \end{itemize}
\end{frame}
\begin{frame}
  {Bare-stem compounding}
  {The lexical basis}
  \begin{itemize}
    \item<1-> Bare stems are necessary for BSC.
    \item<2-> Bare stems are categories without inherent $\varphi$-features. \parencite[see also][]{lasnik1999verbal}
    \item<3-> Syntax represents stems as (minimally) roots with category-determining heads.
      \begin{itemize}
	\item<3-> Nouns = [$n$ \textsc{root}]
      \end{itemize}
    \item<4-> A language generates bare stems iff its lexicon contains category-determining heads without $\varphi$-features ($cat_\emptyset$).
  \end{itemize}
  \onslide<5->{
  \ex. 
  \a. $n_\emptyset \in \text{LEX}_\text{ENG}$
  \b. $n_\emptyset \centernot\in \text{LEX}_\text{FRE}$

}
\end{frame}
\end{document}


