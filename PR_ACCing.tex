%        File: PR_ACCing.tex
%     Created: Fri Jul 29 11:00 AM 2016 E
% Last Change: Fri Jul 29 11:00 AM 2016 E
%
% arara: pdflatex
% arara: biber
% arara: pdflatex
% arara: pdflatex
\documentclass[letterpaper]{article}

\usepackage[margin=1in]{geometry}
\usepackage[backend=biber,style=authoryear-comp,useprefix=false]{biblatex}

\usepackage{stmaryrd}
\usepackage[]{amsmath}
\usepackage{amsfonts}
\usepackage{amssymb}
\usepackage{forest}
\usepackage{tabularx}
\usepackage{linguex}
\usepackage{centernot}
\usepackage{subscript}

\useforestlibrary{linguistics}

\forestset{tree defaults/.style={for tree={parent anchor=south, child anchor=north},every tree node/.style={align=center,anchor=north},level/.style={sibling distance=50mm/#1},baseline}}

\forestset{en/.style={parent anchor=center, child anchor=center}}
\forestset{em/.style={parent anchor=north west, child anchor=north west}}
\forestset{el/.style={parent anchor=north, child anchor=north}}

\usetikzlibrary{positioning,arrows,automata}
\DeclareNameFormat{labelname:poss}{% Based on labelname from biblatex.def
  \ifcase\value{uniquename}%
  \usebibmacro{name:last}{#1}{#3}{#5}{#7}%
  \or
  \ifuseprefix
  {\usebibmacro{name:first-last}{#1}{#4}{#5}{#8}}
  {\usebibmacro{name:first-last}{#1}{#4}{#6}{#8}}%
  \or
  \usebibmacro{name:first-last}{#1}{#3}{#5}{#7}%
  \fi
  \usebibmacro{name:andothers}%
  \ifnumequal{\value{listcount}}{\value{liststop}}{'s}{}
}

\DeclareFieldFormat{shorthand:poss}{%
  \ifnameundef{labelname}{#1's}{#1}
}

\DeclareFieldFormat{citetitle:poss}{\mkbibemph{#1}'s}

\DeclareFieldFormat{label:poss}{#1's}

\newrobustcmd*{\posscitealias}{%
  \AtNextCite{%
    \DeclareNameAlias{labelname}{labelname:poss}%
    \DeclareFieldAlias{shorthand}{shorthand:poss}%
    \DeclareFieldAlias{citetitle}{citetitle:poss}%
    \DeclareFieldAlias{label}{label:poss}
  }
}

\newrobustcmd*{\posscite}{%
  \posscitealias%
  \textcite
}

\newrobustcmd*{\Posscite}{\bibsentence\posscite}

\newrobustcmd*{\posscites}{%
  \posscitealias%
  \textcites
}

\newcommand\quelle[1]{{%
  \unskip\nobreak\hfil\penalty50
  \hskip2em\hbox{}\nobreak\hfil#1%
  \parfillskip=0pt \finalhyphendemerits=0 \par
}
}

\bibliography{Thesis}

\begin{document}
In this section I discuss a pair of constructions that are superficially similar to argument sharing SCs: pseudo-relatives (PRs) and ACC-ing clauses (ACs).
The subjects of PRs and ACs show a strange pattern of extraction wherein when the clause is merged in Theme position, its subject cannot raise, but when the clause is merged as an adjunct, its subject must move.

\ex.
\a. Mary saw Bill crying.\\
\begin{forest}
  nice empty nodes,sn edges
  [TP
    [Mary]
    [
      [+pst]
      [\dots
	[,no edge]
	[VP
	  [see]
	  [ProgP
	    [Bill]
	    [[crying,roof]]
	  ]
	  ]
	]
      ]
    ]
\end{forest}
\b. Bill was seen crying.\\
\begin{forest}
  nice empty nodes,sn edges
  [TP
  [Bill,name=fin subj]
  [
    [was]
    [\dots
      [,no edge]
      [VP
	[VP
	  [see]
	  [$\langle\text{Bill}\rangle$,name=theme]
	]
	[ProgP
	  [$\langle\text{Bill}\rangle$,name=prog subj]
	  [
	    [crying,roof]
	  ]
	]
      ]
    ]
  ]
]
\draw [->] (prog subj.south) --++ (0em,-4ex) -|(fin subj);
\draw [->] (theme.south) --++ (0em,-5ex) -|(fin subj);
\end{forest}
\z.

First I will present the evidence for such a claim, followed by a labeling explanation for the facts.

\subsection{ACC-ing Clauses}
In this section I discuss English ACC-ing clauses, arguing that the sentences in \Next have distinct structures.
In Transformational Grammar terms, \Next[b] is not the result of passivizing \Next[a].
\ex.
\a. Mary saw Bill crying.
\b. Bill was seen crying (by Mary).

In \Last[a], the AC \textit{Bill crying} is merged as the Theme of the verb, and \textit{Bill} does not move out of the AC.
In \Last[b], the AC is adjoined to the VP, the AC subject moves sidewards to be the Theme of the verb, and raises to become subject of the sentence.

I will first compare active and passive perception reports with ACs to see whether the AC subjects are interpreted as Themes of the higher verb.
To test for this I will <++>

\subsection{Labeling}
In this section I will discuss the two current label theories.

Both theories start with what <+chomsky+> calls ``Darwin's problem,'' that is the question of how human language emerged as suddenly as it did.
The general answer that has drives both theories is that the language-specific component of the human language faculty must be quite simple.
This means, that the narrow syntax consist only of the operations necessary and sufficient to generate an unbounded set of structured expressions.
Everything else that has commonly been associated with ``syntax'' is pushed to the interfaces.

Chomsky identifies the narrow syntax with Merge, as defined below.
Since Merge takes syntactic objects as its input and returns a syntactic object as its output, it is able to construct structured expressions.
\ex. For all syntactic objects $\alpha,\beta$, Merge($\alpha, \beta$) = $\left\{ \alpha, \beta \right\}$

Since Labels are not necessary, they are not part of the narrow syntax, and therefore must be part of one of the interfaces.
The results of research in morphology and phonology suggests that labels do not play a role at the sensorimotor interface, so labels must be required for interpretation at the CI interface.
This being the case, labels must be assigned upon transfer at the phase level.

Hornstein argues that merely combining two objects to form a new object is far too general an operation to be the language-specific operation, and proposes labelling as the language-specific operation.
So, for Hornstein, tha narrow syntax consists of three operations: Select, Union, and Label.
Select is a function from atoms to their singleton sets, Union is set-union, and Label is a function from sets to atoms.
Select and Union are domain general, but not sufficient to form the basis of syntax
Label, for Hornstein, is crucial for the narrow syntax, as it allows the output of a syntactic cycle to be the input of the next cycle.
That is, the addition of Label to the computation gives us recursion.

Hornstein further argues that Chomsky's conjecture that labels are required by the CI interface is unmotivated.
A standard Montagovian theory of composition requires only lexically stored types for proper interpretation, and a neo-Davidsonian theory needs only the merge order of arguments.
The other side of this argument, however, is that if labelling is required at the CI interface then we must rethink our conception of the CI interface.

\subsection{The Analysis}

\subsubsection{Argument ACC-ing}
First, let's consider the case of argument ACs.
As discussed above, the subject of an argument AC cannot move out of the AC, meaning the the structure of the passive in \Next is unavailable.
\ex.* Barbara$_i$ was seen [$_\text{VP}$ $t_{see}$ [$_\text{ProgP}$ $t_i$  solving the theorem]].

If we assume Prog$^0$ is a phase head \parencite{harwood2015being}, then let's consider <++>

The ill-formedness of \Last is, by all appearances, an indication of \textit{criterial freezing}, which \textcite{chomsky2015problems} gives a straightforward labelling account.
The account, adapted for the current discussion, says that Prog$^0$ is too weak to be the label of a phrase on its own.
When a DP is (internally) merged in Spec Prog, the resulting object is labelled $\langle\text{F,F}\rangle$ where F is some feature for which D and Prog$^0$ agree.
If the DP is raised, its lower copy is invisible to the labelling algorithm, and thus, the ProgP will be unlabellable, leading to a crash at the CI interface.

\subsubsection{Adjunct ACC-ing}
When ACs are adjuncts, the AC subject must move.
If we adopt Hornstein's proposal that host-adjunct structures are unlabelled, along with Chomsky's conception of labelling it follows that AC subjects are able to move.
It does not straightforwardly follow that adjunct AC subjects must move, though.
This will require a further ad hoc assumption about the interpretive utility of labelling.

Consider the adjunction structure [XP, YP] where XP is the host and YP is the adjunct.
By assumption, this structure is unlabelled, meaning the labelling algorithm skips it when labelling the larger structure.
This means that previously unlabelled structures contained in YP remain unlabelled.
If we apply these assumptions to adjunct ACs, we can see why movement from Spec Prog is licit.
Consider the structure in \Next.
\ex. [Barbara$_i$ was seen [ [$_\text{VP}$ $t_{see}$ $t_i$] [ $t_i$ solving the theorem]].]

The complement of Prog is constructed as always, transferred, and labelled, leaving Prog$^0$ and the subject in the computation.
The subject moves sideward into theme position, and ProgP is adjoined to the VP.
The AC subject then moves to Spec TP.
The Complementizer is merged, and the sentence is transferred and labelled.
Since lower copies are invisible to labelling, we would expect the ProgP to be unlabellable, which is true but irrelevant since the ProgP is ignored by the labelling algorithm.
Adjunction, then, obviates criterial freezing in this case.

Having explained why AC subjects are able to move out of adjuncts, we must explain why such a movement must occur.
Such an explanation requires an extension of Chomsky's labelling theory.
Recall that labels are required by the CI interface in Chomsky's theory, it follows from this that the label of a syntactic object affects how the object is interpreted.
We can see how this might work by comparing spec XP and adjunct-host structures.
Both are instances of $\left\{ \text{XP, YP} \right\}$ objects, but have different interpretations and different labels.
If, say, YP is the specifier of XP, then the resulting structure is labelled $\langle\text{F,F}\rangle$, is interpreted using predicate abstraction or some similar interpretive rule.
If YP is adjoined to XP, then the resulting object is unlabelled and interpreted by simple conjunction.

With that in mind, let's consider how an adjunct AC would be interpreted with its subjec in situ, versus the same case for an argument AC.
The structures in question, are given below in \Next.
\ex.
\a. Adjunct AC\\
{[$_\alpha$ VP [$_\beta$ DP [ Prog$^0$ XP ]]]}
\b. Argument AC\\
{[$_\alpha$ V [$_\beta$ DP [ Prog$^0$ XP ]]]}

Starting with the argument AC, the entire object $\alpha$ is labelled when it is transferred to CI, and $\beta$ is labelled $\langle\text{F,F}\rangle$.
<+SubjInterpretation+>

In the case of the Adjunct AC, $\alpha$ is not labelled, nor is $\beta$.
This means that $\alpha$ is interpreted as the conjunction of the VP and $\beta$, which is the proper interpretation.
This also means that $\beta$ is interpreted as the conjunction of th DP and the ProgP, which is a nonsensical interpretation.

\end{document}


