%        File: PR_ACCing.tex
%     Created: Fri Jul 29 11:00 AM 2016 E
% Last Change: Fri Jul 29 11:00 AM 2016 E
%
% arara: pdflatex
% arara: biber
% arara: pdflatex
% arara: pdflatex
\documentclass[letterpaper]{article}

\usepackage[margin=1in]{geometry}
\usepackage[backend=biber,style=authoryear-comp,useprefix=false]{biblatex}

\usepackage{stmaryrd}
\usepackage[]{amsmath}
\usepackage{amsfonts}
\usepackage{amssymb}
\usepackage{forest}
\usepackage{tabularx}
\usepackage{linguex}
\usepackage{centernot}
\usepackage{subscript}

\useforestlibrary{linguistics}

\forestset{tree defaults/.style={for tree={parent anchor=south, child anchor=north},every tree node/.style={align=center,anchor=north},level/.style={sibling distance=50mm/#1},baseline}}

\forestset{en/.style={parent anchor=center, child anchor=center}}
\forestset{em/.style={parent anchor=north west, child anchor=north west}}
\forestset{el/.style={parent anchor=north, child anchor=north}}

\usetikzlibrary{positioning}
\DeclareNameFormat{labelname:poss}{% Based on labelname from biblatex.def
  \ifcase\value{uniquename}%
  \usebibmacro{name:last}{#1}{#3}{#5}{#7}%
  \or
  \ifuseprefix
  {\usebibmacro{name:first-last}{#1}{#4}{#5}{#8}}
  {\usebibmacro{name:first-last}{#1}{#4}{#6}{#8}}%
  \or
  \usebibmacro{name:first-last}{#1}{#3}{#5}{#7}%
  \fi
  \usebibmacro{name:andothers}%
  \ifnumequal{\value{listcount}}{\value{liststop}}{'s}{}
}

\DeclareFieldFormat{shorthand:poss}{%
  \ifnameundef{labelname}{#1's}{#1}
}

\DeclareFieldFormat{citetitle:poss}{\mkbibemph{#1}'s}

\DeclareFieldFormat{label:poss}{#1's}

\newrobustcmd*{\posscitealias}{%
  \AtNextCite{%
    \DeclareNameAlias{labelname}{labelname:poss}%
    \DeclareFieldAlias{shorthand}{shorthand:poss}%
    \DeclareFieldAlias{citetitle}{citetitle:poss}%
    \DeclareFieldAlias{label}{label:poss}
  }
}

\newrobustcmd*{\posscite}{%
  \posscitealias%
  \textcite
}

\newrobustcmd*{\Posscite}{\bibsentence\posscite}

\newrobustcmd*{\posscites}{%
  \posscitealias%
  \textcites
}

\newcommand\quelle[1]{{%
  \unskip\nobreak\hfil\penalty50
  \hskip2em\hbox{}\nobreak\hfil#1%
  \parfillskip=0pt \finalhyphendemerits=0 \par
}
}

\bibliography{Thesis}

\begin{document}
In this section I discuss a pair of constructions that are superficially similar to argument sharing SCs: pseudo-relatives (PRs) and ACC-ing clauses (ACs).
The subjects of PRs and ACs show a strange pattern of extraction wherein when the clause is merged in Theme position, its subject cannot raise, but when the clause is merged as an adjunct, its subject must move.

\ex.
\a. Mary saw Bill crying.\\
\begin{forest}
  nice empty nodes,sn edges
  [TP
    [Mary]
    [
      [+pst]
      [\dots
	[,no edge]
	[VP
	  [see]
	  [ProgP
	    [Bill]
	    [[crying,roof]]
	  ]
	  ]
	]
      ]
    ]
\end{forest}
\b. Bill was seen crying.\\
\begin{forest}
  nice empty nodes,sn edges
  [TP
  [Bill,name=fin subj]
  [
    [was]
    [\dots
      [,no edge]
      [VP
	[VP
	  [see]
	  [$\langle\text{Bill}\rangle$,name=theme]
	]
	[ProgP
	  [$\langle\text{Bill}\rangle$,name=prog subj]
	  [
	    [crying,roof]
	  ]
	]
      ]
    ]
  ]
]
\draw [->] (prog subj.south) --++ (0em,-4ex) -|(fin subj);
\draw [->] (theme.south) --++ (0em,-5ex) -|(fin subj);
\end{forest}
\z.

First I will present the evidence for such a claim, followed by a labeling explanation for the facts.

\subsection{ACC-ing Clauses}
In this section I discuss English ACC-ing clauses, arguing that the sentences in \Next have distinct structures.
In Transformational Grammar terms, \Next[b] is not the result of passivizing \Next[a].
\ex.
\a. Mary saw Bill crying.
\b. Bill was seen crying (by Mary).

In \Last[a], the AC \textit{Bill crying} is merged as the Theme of the verb, and \textit{Bill} does not move out of the AC.
In \Last[b], the AC is adjoined to the VP, the AC subject moves sidewards to be the Theme of the verb, and raises to become subject of the sentence.

I will first compare active and passive perception reports with ACs to see whether the AC subjects are interpreted as Themes of the higher verb.
To test for this I will <++>

\subsection{Labeling}
In this section I will discuss the two current label theories.

Both theories start with what <+chomsky+> calls ``Darwin's problem,'' that is the question of how human language emerged as suddenly as it did.
The general answer that has drives both theories is that the language-specific component of the human language faculty must be quite simple.
This means, that the narrow syntax consist only of the operations necessary and sufficient to generate an unbounded set of structured expressions.
Everything else that has commonly been associated with ``syntax'' is pushed to the interfaces.

Chomsky identifies the narrow syntax with Merge, as defined below.
Since Merge takes syntactic objects as its input and returns a syntactic object as its output, it is able to construct structured expressions.
\ex. For all syntactic objects $\alpha,\beta$, Merge($\alpha, \beta$) = $\left\{ \alpha, \beta \right\}$

Since Labels are not necessary, they are not part of the narrow syntax, and therefore must be part of one of the interfaces.
The results of research in morphology and phonology suggests that labels do not play a role at the sensorimotor interface, so labels must be required for interpretation at the CI interface.
This being the case, labels must be assigned upon transfer at the phase level.

Hornstein argues that merely combining two objects to form a new object is far too general an operation to be the language-specific operation, and proposes labelling as the language-specific operation.
So, for Hornstein, tha narrow syntax consists of three operations: Select, Set-union, and Label.
\ex.
\a. For all atoms $\alpha$, Select($\alpha$) = $\left\{ \alpha \right\}$.
\b. For all sets $\alpha, \beta$, $$<++>
\end{document}


