%        File: Proposal.tex
%     Created: Thu Aug 25 12:00 PM 2016 E
% Last Change: Thu Aug 25 12:00 PM 2016 E
%
% arara: pdflatex
% arara: biber
% arara: pdflatex
% arara: pdflatex
\documentclass[letterpaper,12pt]{article}

\usepackage[
%margin=1in
]{geometry}
\usepackage[backend=biber,style=authoryear-comp,useprefix=false]{biblatex}

\usepackage{stmaryrd}
\usepackage[]{amsmath}
\usepackage{amsfonts}
\usepackage{amssymb}
\usepackage{forest}
\usepackage{tabularx}
\usepackage{linguex}
\usepackage{centernot}
\usepackage{todonotes}

\useforestlibrary{linguistics}

\forestset{tree defaults/.style={for tree={parent anchor=south, child anchor=north},every tree node/.style={align=center,anchor=north},level/.style={sibling distance=50mm/#1},baseline}}

\forestset{en/.style={parent anchor=center, child anchor=center}}
\forestset{em/.style={parent anchor=north west, child anchor=north west}}
\forestset{el/.style={parent anchor=north, child anchor=north}}

\usetikzlibrary{positioning}
\usetikzlibrary{calc}
\usetikzlibrary{arrows}
\usetikzlibrary{decorations.markings}
%\DeclareNameFormat{labelname:poss}{% Based on labelname from biblatex.def
%  \ifcase\value{uniquename}%
%  \usebibmacro{name:last}{#1}{#3}{#5}{#7}%
%  \or
%  \ifuseprefix
%  {\usebibmacro{name:first-last}{#1}{#4}{#5}{#8}}
%  {\usebibmacro{name:first-last}{#1}{#4}{#6}{#8}}%
%  \or
%  \usebibmacro{name:first-last}{#1}{#3}{#5}{#7}%
%  \fi
%  \usebibmacro{name:andothers}%
%  \ifnumequal{\value{listcount}}{\value{liststop}}{'s}{}
%}
%
%\DeclareFieldFormat{shorthand:poss}{%
%  \ifnameundef{labelname}{#1's}{#1}
%}
%
%\DeclareFieldFormat{citetitle:poss}{\mkbibemph{#1}'s}
%
%\DeclareFieldFormat{label:poss}{#1's}
%
%\newrobustcmd*{\posscitealias}{%
%  \AtNextCite{%
%    \DeclareNameAlias{labelname}{labelname:poss}%
%    \DeclareFieldAlias{shorthand}{shorthand:poss}%
%    \DeclareFieldAlias{citetitle}{citetitle:poss}%
%    \DeclareFieldAlias{label}{label:poss}
%  }
%}
%
%\newrobustcmd*{\posscite}{%
%  \posscitealias%
%  \textcite
%}
%
%\newrobustcmd*{\Posscite}{\bibsentence\posscite}
%
%\newrobustcmd*{\posscites}{%
%  \posscitealias%
%  \textcites
%}

\newcommand\quelle[1]{{%
  \unskip\nobreak\hfil\penalty50
  \hskip2em\hbox{}\nobreak\hfil#1%
  \parfillskip=0pt \finalhyphendemerits=0 \par
}
}

\newcommand{\figex}{\refstepcounter{ExNo}\theExNo\hspace{\Exlabelsep}}

\bibliography{Thesis}
\linespread{1.3}
\begin{document}

\section{Introduction}

<+BigQuestion+>

\subsection{Previous Answers}
\subsubsection{Harley, Ramchand, Folli, et al.}
\begin{itemize}
  \item \textsc{Manner, Path, Place, Result}, \textit{etc.} features are lexicalized on different heads in different languages
  \item Basically a reiteration of Talmy
  \item May be true but not an account
\end{itemize}
\subsubsection{Snyder, Beck, Kratzer}
\begin{itemize}
  \item Compounding parameter determines V-framed/S-framed
  \item Similar in spirit to what I will propose
  \item Compounding parameter is likely unformulable in current theory
\end{itemize}

\section{Sharpening the Question}
A more precise definition of resultatives is needed to proceed in this study.
Resultative clauses are clauses that contain two distinct predicates (Pred1$\neq$Pred2) which share an argument, in which the primary predicate is construed as the cause of the secondary.
Each of these properties that define resultatives (argument sharing and ``causativity'') is quite common and neither is is sufficient for resultatives.
Depictives, which seem to occur in all of the worlds languages, such as those seen in \Next show argument sharing without ``causativity''.
\ex. \textbf{Depictives}
\a. Mary left angry. (Mary was angry.)
\b. Bill ate the fish raw. (The fish was raw.)
\b. Jamie swam the race naked. (Jamie was naked.)
\b. <+DepictivesInOtherLangs+>
\z.

So, the clause \textit{Mary left angry} means that the two eventualities, the event $e$ of Mary leaving and the state $s$ of Mary being angry, are stand in either an identity ($e=s$) or containment ($e\leq s$) relation rather than a causal relation.

``Causativity'', broadly construed, is even more prevalent in language.
Every instance of a sentence with an agent encodes ``causativity''.
For example, the clause \textit{John ironed the shirt} means that John acted in such a way as to cause the shirt to be ironed.

English-type languages, then, are those which generate clauses in which a single argument is shared between two predicates which are in a ``causative'' relation.
French-type languages, on the other hand, are those which do not generate clauses with both properties.
This means that an explanation of the parametric split between English- and French-type languages needs three components.
First, it requires a theory of argument sharing.
Second, must include a theory of how relations between predicates are established.
Finally, it must show that, for a given language, there is some component of the primary lingustic data which determines whether the two phenomena are compatible.

\subsection{What is Argument Sharing}
First, Lets consider cases of argument sharing, which, semantically speaking, is a situation in which one entity is interpreted as being a participant in multiple distinct events expressed by a single utterence.
A familiar type of argument sharing is control sentences such as \Next, in which Alice is both the holder of the wanting attitude and the agent of the non-actual winning event.
\ex.\label{ex:Control} Alice$_i$ wants $ec_i$ to win.

Other instances of argument sharing are seen in parasitic gaps, and depictives as in \Next and \NNext, respectively.

\ex. Who$_i$ did you discuss $ec_i$ without meeting $ec_i$?

\ex. Monica left angry.

The notion of $\Theta$-roles is the syntactic analogue to the predicate-argument relation in semantics.
If an entity is interpreted as the argument of a predicate, then the phrase that encodes that entity is $\Theta$-marked by the head/phrase that encodes the predicate.
Following a basic minimalist assumption -- TRAP \parencite{hornsteinetal2005understanding} -- $\Theta$-roles are assigned upon merge, and, following a more controversial assumption -- UTAH \parencite{baker1988incorporation}-- there is a mapping between $\Theta$-roles and syntactic positions.
This means that, for example, a DP is interpreted as the Theme of a verb iff it is merged as the complement of that verb.

I will be making two non-standard assumptions regarding $\Theta$-roles which, being non-standard, require justification.
First, I assume that DPs can receive multiple $\Theta$-roles, or in other words, I do not assume the $\Theta$ criterion.
There are two justifications for this assumption, one metatheoretical and the other theoretical.
The metatheoretical argument is based on the method of theory building prescribed by the minimalist program.
Theories are built by identifying virtual conceptual necessities (VCNs) and assuming those as axioms of the theory.
Any proposed principles or theoretical constructs are admited to the theory in one of two ways.
Either they are shown to follow logically from VCNs or they are argued to be VCNs.
In current syntactic theory, the set of VCNs is restricted to a lexicon, a structure building operation (merge) and the interfaces with other cognitive modules (sensorymotor and conceptual-intentional).
Since the $\Theta$-criterion is not a member of this set, its existence, rather than its nonexistence, must be argued for.
In other words, the burden of proof is on those who would propose the $\Theta$-criterion rather than those who reject it.

The $\Theta$ criterion was a constraint in GB syntax that held a D-Structure. 
It stated that every $\Theta$-role must be assigned to exactly one argument.
Since D-Structure has been eliminated from syntactic theory, if the $\Theta$-criterion does hold, it must hold at one of the interfaces, and since $\Theta$-roles are essentially semantic it must hold at the CI-interface.
<+FinishThisThought+>

The second non-standard assumption is in the particular version of UTAH I use.
\textcite{bakerXXXX} argues for the following mapping of $\Theta$-roles to structural position.
Agents are asssociated with Spec Voice, Themes with Spec V, and Goals with Comp V as in \Next, below.
\begin{figure}[h]
  \refstepcounter{ExNo}\theExNo\hspace{\Exlabelsep}\label{fig:BakerUTAH}
  {\small
\begin{forest}
  nice empty nodes,sn edges,baseline
  [VoiceP
    [\textsc{Agent}]
    [
      [Voice]
      [VP
	[\textsc{Theme}]
	[
	  [V]
	  [\textsc{Goal}]
	]
      ]
    ]
  ]
\end{forest}}
\end{figure}

<+FinishThisThought+>

This leads to a conception of argument sharing, according to which, DPs are shared arguments iff they are merged in two $\Theta$ positions in the course of a single derivation.
For ordinary control sentences, this is trivial to represent.
Consider \ref{ex:Control}, above, in which the subject \textit{Alice} bears two $\Theta$-roles: External argument of \textit{want} and external argument of \textit{win}.
Assuming external $\Theta$-roles are assigned to DPs merged in Spec-Voice, this means that \textit{Alice} was merged in two distinct Voice projections.
This can be attained by a derivation including only standard upward movement operations as shown below in \Next.
\begin{figure}[h]
  \figex
  {\small 
    \begin{forest}
  nice empty nodes,sn edges,baseline
  [VoiceP
    [Alice,name=wanter]
    [
      [Voice,name=want voice]
      [VP
	[want]
	[TP
	  [{$\langle\text{Alice}\rangle$},name=to subj]
	  [
	    [to]
	    [VoiceP
	      [{$\langle\text{Alice}\rangle$},name=winner]
	      [
		[Voice,name=win voice]
		[win]
	      ]
	    ]
	  ]
	]
      ]
    ]
  ]
  \draw [->,thick] (winner) to[out=south west, in=south] (to subj);
  \draw [->,thick] (to subj) to[out=south west, in=south] (wanter);
  %\draw [->,dashed] (win voice) to[out=west, in=south] node[below]  {\small $\Theta$} (winner);
\end{forest}}
\end{figure}

Other instances of argument sharing, however, require movement into internal argument positions which, assuming internal $\Theta$-roles are assigned in Comp V, requires sideward movement.
Consider the depictive sentence \textit{I ate the meat raw}, represented below in \Next.
\begin{figure}[h]
  \figex
  {\small
\begin{forest}
  nice empty nodes,sn edges,baseline
  [VoiceP
    [I]
    [
      [Voice]
      [VP
	[VP
	  [eat]
	  [DP[the meat,roof]]
	]
	[SC
	  [DP,[{$\langle\text{the meat}\rangle$},roof]]
	  [raw]
	]
      ]
    ]
  ]
\end{forest}}
\end{figure}
While sideward movement is not usually assumed to be allowed in merge-based derivations, certain interpretations of minimalist syntactic theory do allow it \parencite{nunes2001sideward,hornstein2009theory}.
I adapt these interpretations slightly to allow for the sideward movement necessary for movement to theme position.

Following \textcite{hornstein2009theory,nunes2001sideward}, I assume that in addition to linguistically proprietary operations, the faculty of language also uses domain general operations, specifically, a copying operation.
A copying operation, along with the necessary assumption that subtrees are derived separately before being merged together, gives us sideward movement.
To see how this works, consider the derivation of \Last (given in \Next, below).
Starting with the DP \textit{the meat} preconstructed, we build the small clause (a-c).
We then select the verb from the lexical array (c), copy the DP from the small clause (d), and merge the two to form the VP (e).
Finally, we merge the VP and the small clause (f) and we are left with a sideward movement structure.
\begin{table}
  \figex \textbf{Deriving \LLast}\\
  {\small
\begin{tabular}[t]{rll}
  & Lexical Array & Workspace\\
  (a) & $\left\{ \textit{raw, eat, \dots} \right\}$ & $\left\{ \left\{ \textit{the, meat} \right\} \right\}$\\
  \multicolumn{3}{l}{Select(raw)}\\
  (b) & $\left\{ \textit{eat, \dots} \right\}$ & $\left\{ \textit{raw}, \left\{\textit{the, meat}\right\} \right\}$ \\
  \multicolumn{3}{l}{Merge(raw, $\left\{ \textit{the, meat} \right\}$)}\\
  (c) & $\left\{  \textit{eat, \dots}\right\}$ & $\left\{ \left\{ \textit{raw}, \left\{\textit{the, meat}\right\} \right\} \right\}$ \\
  \multicolumn{3}{l}{Select(eat)}\\
  (d) & $\left\{ \textit{\dots} \right\}$ & $\left\{ \textit{eat}, \left\{ \textit{raw}, \left\{\textit{the, meat}\right\} \right\} \right\}$ \\
  \multicolumn{3}{l}{Copy($\left\{ \textit{the, meat} \right\}$)}\\
  (e) & $\left\{ \textit{\dots} \right\}$ &$\left\{ \textit{eat}, \left\{\textit{the, meat}\right\}, \left\{ \textit{raw}, \left\{\textit{the, meat}\right\} \right\} \right\}$ \\
  \multicolumn{3}{l}{Merge(eat,  $\left\{ \textit{the, meat} \right\}$)}\\
  (f) & $\left\{ \textit{\dots} \right\}$ &$\left\{ \left\{\textit{eat}, \left\{\textit{the, meat}\right\}\right\}, \left\{ \textit{raw}, \left\{\textit{the, meat}\right\} \right\} \right\}$ \\
  \multicolumn{3}{l}{Merge($\left\{\textit{eat}, \left\{\textit{the, meat}\right\}\right\}$, $\left\{ \textit{raw}, \left\{\textit{the, meat}\right\} \right\}$)}\\
  (g) & $\left\{ \ldots \right\}$ & $\left\{  \left\{ \left\{\textit{eat}, \left\{\textit{the, meat}\right\}\right\}, \left\{ \textit{raw}, \left\{\textit{the, meat}\right\} \right\} \right\}\right\}$ \\
  \multicolumn{3}{l}{\dots}
\end{tabular}}
\end{table}
As discussed by \textcite{nunes2001sideward}, the derivation in \Last and the structure it derives in \LLast would lead to a crash at PF due to a failure of copy deletion.
Assuming that, all else being equal, the higher copy of a syntactic object is pronounced and lower copies are deleted, a sideward movement structure ought to be unpronounceable.
If sideward movement is followed by movement to a position that c-commands both copies, then the copy deletion issues evaporate, and the higher copy is pronounced.

Returning to the specific example of object-oriented depictives, the preceding discussion means that the structure in \LLast cannot be the final structure of the sentence given, as such a structure is unpronounceable.
If we assume, following \textcite{lasnik1999minimalist}, that grammatical objects raise to Spec AgrO\footnote{It seems unlikely to me that there is a specialised grammatical category whose only property is that is licenses Object DPs. As such I assume AgrO to be some meaningful category, but I take no stance on what that category might be.} for abstract Case\footnote{I take abstract Case to be a phenomenon in need of explanation, rather than an explanation for a phenomenon.} licensing, then our sideward moved DP must further raise to and the issue dissolves.
\begin{figure}[h]
\figex
{\small
\begin{forest}
  nice empty nodes,sn edges,baseline
  [AgrOP
    [DP]
    [
      [AgrO]
      [VP
	[VP
	  [eat]
	  [{$\langle\text{DP}\rangle$}]
	]
	[SC
	  [{$\langle\text{DP}\rangle$}]
	  [raw]
	]
      ]
    ]
  ]
\end{forest}}
\end{figure}
To summarize, argument sharing is represented in the syntax by movement from one $\Theta$ position to another.
In canonical control constructions, this is a trivial upward movement operation.
In other argument sharing constructions, however, sideward movement is necessary which requires a further upward movement to a position that c-commands all other instances of the moved element.


\subsection{What is ``Causativity''}
Causativity as a phenomenon is frequently discussed using causative-inchoative verbs like \textit{burn}.
\ex.\label{ex:caus-inch}
\a.\label{ex:burn-tr} Paul burned the toast.
\b.\label{ex:burn-intr} The toast burned.
\z.


The principal debate regarding causativity is about whether \Last[a] is, in some sense, derived from \Last[b].
In the earliest days of this debate, those who argued for the affirmative (McCawley, Lakoff, et al.)\todo[inline]{\small I need to find and read some representative Generetive Semantics literature} proposed \Last[a] meant and was derived from \Next.
\ex.\label{ex:cause-to-burn} Paul caused the toast to burn.

\todo[inline]{Give arguments for this claim}

\textcite{fodor1970three}, however, argues against this claim.
\todo[inline]{Give Fodor's arguments}

\textcite{pietroski2003small} synthesizes these two approaches and proposes that \ref{ex:burn-tr} is ``derived'' from \ref{ex:burn-intr} does not mean \ref{ex:cause-to-burn}.

Pietroski starts from a neo-davidsonian theory of verbal semantics, according to which verbs encode conjoined predicates of events.
So, an expression like the inchoative in \ref{ex:burn-intr} describes an event which is a burning event and which has \textit{the toast} as a theme.
This is given formally, abstracting away from tense etc., in \Next[b] below.
\ex.
\a. The toast burned.
\b. $\exists e [\text{Burning}(e) \& \text{Theme}(e, \mathbf{the\_toast})]$

The causative version in \ref{ex:burn-tr}, then, describes a complex event, which has Paul as an agent and which \textit{terminates} in a burning event.
\ex.
\a. Paul burned the toast.
\b. $\exists e,f [\text{Agent}(e, \mathbf{paul}) \& \text{terminates-in}(e,f) \& \text{Burning}(f) \& \text{Theme}(f, \mathbf{the\_toast})]$

\begin{itemize}
  \item Syntax of ``causativity''
    \begin{itemize}
      \item Generally in the verbal domain $\{\text{H, XP}\} \rightarrow $Causative
    \end{itemize}
\end{itemize}

(The above will largely be an excercise in making my assumptions explicit)

\section{Enter Resultatives}
\begin{itemize}
  \item Structural description: \textcite{kratzer_building_2004} + Sideward movement
\end{itemize}
\begin{figure}[h]
\figex Kratzer\\
\begin{forest}
  nice empty nodes,sn edges,baseline
  [vP
    [v] 
    [
      [DP[the metal,roof]] 
      [VP
	[hammer] 
	[resP 
	  [res] 
	  [SC
	    [{$\langle\text{the metal}\rangle$}]
	    [flat]
	  ]
	]
      ]
    ]
  ]
\end{forest}
\end{figure}
\begin{figure}[h]
\figex Kratzer + Sideward movement\\
\begin{forest}
  nice empty nodes,sn edges,baseline
  [vP
    [v]
    [
      [VP
	[hammer]
	[DP[the metal,roof]]
      ]
      [resP
	[res]
	[SC
	  [DP[the metal,roof]]
	  [flat]
	]
      ]
    ]
  ]
\end{forest}
\end{figure}
\begin{itemize}
  \item How does \Last[b] get a ``causative''  interpretation?
  \item How is \Last[b] ruled out in French-like languages?
\end{itemize}

\section{Appendices}
\begin{itemize}
  \item 1st Generals Paper.
  \item Dog Days 2016 handout.
  \item Forum Paper? (perhaps tangientially relevent)
\end{itemize}
\printbibliography
\end{document}
