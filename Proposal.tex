%        File: Proposal.tex
%     Created: Thu Aug 25 12:00 PM 2016 E
% Last Change: Thu Aug 25 12:00 PM 2016 E
%
% arara: pdflatex
% arara: biber
% arara: pdflatex
% arara: pdflatex
\documentclass[letterpaper,12pt]{article}

\usepackage[margin=1in]{geometry}
\usepackage[backend=biber,style=authoryear-comp,useprefix=false]{biblatex}

\usepackage{stmaryrd}
\usepackage[]{amsmath}
\usepackage{amsfonts}
\usepackage{amssymb}
\usepackage{forest}
\usepackage{tabularx}
\usepackage{linguex}
\usepackage{centernot}


\useforestlibrary{linguistics}

\forestset{tree defaults/.style={for tree={parent anchor=south, child anchor=north},every tree node/.style={align=center,anchor=north},level/.style={sibling distance=50mm/#1},baseline}}

\forestset{en/.style={parent anchor=center, child anchor=center}}
\forestset{em/.style={parent anchor=north west, child anchor=north west}}
\forestset{el/.style={parent anchor=north, child anchor=north}}

\usetikzlibrary{positioning}
\usetikzlibrary{calc}
\usetikzlibrary{arrows}
\usetikzlibrary{decorations.markings}
\DeclareNameFormat{labelname:poss}{% Based on labelname from biblatex.def
  \ifcase\value{uniquename}%
  \usebibmacro{name:last}{#1}{#3}{#5}{#7}%
  \or
  \ifuseprefix
  {\usebibmacro{name:first-last}{#1}{#4}{#5}{#8}}
  {\usebibmacro{name:first-last}{#1}{#4}{#6}{#8}}%
  \or
  \usebibmacro{name:first-last}{#1}{#3}{#5}{#7}%
  \fi
  \usebibmacro{name:andothers}%
  \ifnumequal{\value{listcount}}{\value{liststop}}{'s}{}
}

\DeclareFieldFormat{shorthand:poss}{%
  \ifnameundef{labelname}{#1's}{#1}
}

\DeclareFieldFormat{citetitle:poss}{\mkbibemph{#1}'s}

\DeclareFieldFormat{label:poss}{#1's}

\newrobustcmd*{\posscitealias}{%
  \AtNextCite{%
    \DeclareNameAlias{labelname}{labelname:poss}%
    \DeclareFieldAlias{shorthand}{shorthand:poss}%
    \DeclareFieldAlias{citetitle}{citetitle:poss}%
    \DeclareFieldAlias{label}{label:poss}
  }
}

\newrobustcmd*{\posscite}{%
  \posscitealias%
  \textcite
}

\newrobustcmd*{\Posscite}{\bibsentence\posscite}

\newrobustcmd*{\posscites}{%
  \posscitealias%
  \textcites
}

\newcommand\quelle[1]{{%
  \unskip\nobreak\hfil\penalty50
  \hskip2em\hbox{}\nobreak\hfil#1%
  \parfillskip=0pt \finalhyphendemerits=0 \par
}
}

\bibliography{Thesis}

\begin{document}

\section{Introduction}

<+BigQuestion+>

\subsection{Previous Answers}
\subsubsection{Harley, Ramchand, Folli, et al.}
\begin{itemize}
  \item \textsc{Manner, Path, Place, Result}, \textit{etc.} features are lexicalized on different heads in different languages
  \item Basically a reiteration of Talmy
  \item May be true but not an account
\end{itemize}
\subsubsection{Snyder, Beck, Kratzer}
\begin{itemize}
  \item Compounding parameter determines V-framed/S-framed
  \item Similar in spirit to what I will propose
  \item Compounding parameter is likely unformulable in current theory
\end{itemize}

\section{Sharpening the Question}
A more precise definition of resultatives is needed to proceed in this study.
Resultative clauses are clauses that contain two distinct predicates (Pred1$\neq$Pred2) which share an argument, in which the primary predicate is construed as the cause of the secondary.
Each of these properties that define resultatives (argument sharing and ``causativity'') is quite common and neither is is sufficient for resultatives.
Depictives, which seem to occur in all of the worlds languages, such as those seen in \Next show argument sharing without ``causativity''.
\ex. \textbf{Depictives}
\a. Mary left angry. (Mary was angry.)
\b. Bill ate the fish raw. (The fish was raw.)
\b. Jamie swam the race naked. (Jamie was naked.)
\b. <+DepictivesInOtherLangs+>
\z.

So, the clause \textit{Mary left angry} means that the two eventualities, the event $e$ of Mary leaving and the state $s$ of Mary being angry, are stand in either an identity ($e=s$) or containment ($e\leq s$) relation rather than a causal relation.

``Causativity'', broadly construed, is even more prevalent in language.
Every instance of a sentence with an agent encodes ``causativity''.
For example, the clause \textit{John ironed the shirt} means that John acted in such a way as to cause the shirt to be ironed.

English-type languages, then, are those which generate clauses in which a single argument is shared between two predicates which are in a ``causative'' relation.
French-type languages, on the other hand, are those which do not generate clauses with both properties.
This means that an explanation of the parametric split between English- and French-type languages needs three components.
First, it requires a theory of argument sharing.
Second, must include a theory of how relations between predicates are established.
Finally, it must show that, for a given language, there is some component of the primary lingustic data which determines whether the two phenomena are compatible.

\subsection{What is Argument Sharing}
First, Lets consider cases of argument sharing, which, semantically speaking, is a situation in which one entity is interpreted as being a participant in multiple distinct events expressed by a single utterence.
A familiar type of argument sharing is control sentences such as \Next, in which Alice is both the holder of the wanting attitude and the agent of the non-actual winning event.
\ex.\label{ex:Control} Alice$_i$ wants $ec_i$ to win.

Other instances of argument sharing are seen in parasitic gaps, and depictives as in \Next and \NNext, respectively.

\ex. Who$_i$ did you discuss $ec_i$ without meeting $ec_i$?

\ex. Monica left angry.

The notion of $\Theta$-roles is the syntactic analogue to the predicate-argument relation in semantics.
If an entity is interpreted as the argument of a predicate, then the phrase that encodes that entity is $\Theta$-marked by the head/phrase that encodes the predicate.
Following a basic minimalist assumption -- TRAP \parencite{hornsteinetal2005understanding} -- $\Theta$-roles are assigned upon merge, and, following a more controversial assumption -- UTAH \parencite{baker1988incorporation}-- there is a mapping between $\Theta$-roles and syntactic positions.
This means that, for example, a DP is interpreted as the Theme of a verb iff it is merged as the complement of that verb.

This leads to a conception of argument sharing, according to which, DPs are shared arguments iff they are merged in two $\Theta$ positions in the course of a single derivation.
For ordinary control sentences, this is trivial to represent.
Consider \ref{ex:Control}, above, in which the subject \textit{Alice} bears two $\Theta$-roles: External argument of \textit{want} and external argument of \textit{win}.
Assuming external $\Theta$-roles are assigned to DPs merged in Spec-Voice, this means that \textit{Alice} was merged in two distinct Voice projections.
This can be attained by a derivation including only standard upward movement operations as shown below in \Next.
\ex. 
\begin{forest}
  nice empty nodes,sn edges,baseline
  [VoiceP
    [Alice,name=wanter]
    [
      [Voice,name=want voice]
      [VP
	[want]
	[TP
	  [{$\langle\text{Alice}\rangle$},name=to subj]
	  [
	    [to]
	    [VoiceP
	      [{$\langle\text{Alice}\rangle$},name=winner]
	      [
		[Voice,name=win voice]
		[win]
	      ]
	    ]
	  ]
	]
      ]
    ]
  ]
  \draw [->,thick] (winner) to[out=south west, in=south] (to subj);
  \draw [->,thick] (to subj) to[out=south west, in=south] (wanter);
  %\draw [->,dashed] (win voice) to[out=west, in=south] node[below]  {\small $\Theta$} (winner);
\end{forest}

Other instances of argument sharing, however, require movement into internal argument positions which, assuming internal $\Theta$-roles are assigned in Comp V, requires sideward movement.
Consider the depictive sentence \textit{I ate the meat raw}, represented below in \Next.
\ex.
\begin{forest}
  nice empty nodes,sn edges,baseline
  [VoiceP
    [I]
    [
      [Voice]
      [
	[VP
	  [eat]
	  [DP[the meat,roof]]
	]
	[SC
	  [DP,[{$\langle\text{the meat}\rangle$},roof]]
	  [raw]
	]
      ]
    ]
  ]
\end{forest}

While sideward movement is not usually assumed to be allowed in merge-based derivations, certain interpretations of minimalist syntactic theory do allow it \parencite{nunes2001sideward,hornstein2009theory}.
I adapt these interpretations slightly to allow for the sideward movement necessary for movement to theme position.

Following \textcite{hornstein2009theory,nunes2001sideward}, I assume that in addition to linguistically proprietary operations, the faculty of language also uses domain general operations, specifically, a copying operation.

\ex. Deriving \Last.
\begin{tabular}[t]{rlll}
  & Lexical Array & Workspace & \\
  \cline{1-3}
  1 & $\left\{ \text{raw, eat, Voice, 1sg} \right\}$ & $\left\{ \left\{ the, meat \right\} \right\}$ & Select(raw)\\
\end{tabular}

\subsection{What is ``Causativity''}
\begin{itemize}
  \item Semantics of ``causativity''
    \begin{itemize}
      \item \textcite{pietroski2005events}: causativity without a Cause predicate
    \end{itemize}
  \item Syntax of ``causativity''
    \begin{itemize}
      \item Generally in the verbal domain $\{\text{H, XP}\} \rightarrow $Causative
    \end{itemize}
\end{itemize}

(The above will largely be an excercise in making my assumptions explicit)

\section{Enter Resultatives}
\begin{itemize}
  \item Structural description: \textcite{kratzer_building_2004} + Sideward movement
\end{itemize}
\ex. 
\a. Kratzer\\
\begin{forest}
  nice empty nodes,sn edges,baseline
  [vP
    [v] 
    [
      [DP[the metal,roof]] 
      [VP
	[hammer] 
	[resP 
	  [res] 
	  [SC
	    [{$\langle\text{the metal}\rangle$}]
	    [flat]
	  ]
	]
      ]
    ]
  ]
\end{forest}
\b. Kratzer + Sideward movement\\
\begin{forest}
  nice empty nodes,sn edges,baseline
  [vP
    [v]
    [
      [VP
	[hammer]
	[DP[the metal,roof]]
      ]
      [resP
	[res]
	[SC
	  [DP[the metal,roof]]
	  [flat]
	]
      ]
    ]
  ]
\end{forest}
\z.

\begin{itemize}
  \item How does \Last[b] get a ``causative''  interpretation?
  \item How is \Last[b] ruled out in French-like languages?
\end{itemize}

\section{Appendices}
\begin{itemize}
  \item 1st Generals Paper.
  \item Dog Days 2016 handout.
  \item Forum Paper? (perhaps tangientially relevent)
\end{itemize}
\end{document}
