%        File: Proposal.tex
%     Created: Thu Aug 25 12:00 PM 2016 E
% Last Change: Thu Aug 25 12:00 PM 2016 E
%
% arara: pdflatex
% arara: biber
% arara: pdflatex
% arara: pdflatex
\documentclass[letterpaper]{article}

\usepackage[margin=1in]{geometry}
\usepackage[backend=biber,style=authoryear-comp,useprefix=false]{biblatex}

\usepackage{stmaryrd}
\usepackage[]{amsmath}
\usepackage{amsfonts}
\usepackage{amssymb}
\usepackage{forest}
\usepackage{tabularx}
\usepackage{linguex}
\usepackage{centernot}


\forestset{tree defaults/.style={for tree={parent anchor=south, child anchor=north},every tree node/.style={align=center,anchor=north},level/.style={sibling distance=50mm/#1},baseline}}

\forestset{en/.style={parent anchor=center, child anchor=center}}
\forestset{em/.style={parent anchor=north west, child anchor=north west}}
\forestset{el/.style={parent anchor=north, child anchor=north}}

\usetikzlibrary{positioning}
\DeclareNameFormat{labelname:poss}{% Based on labelname from biblatex.def
  \ifcase\value{uniquename}%
  \usebibmacro{name:last}{#1}{#3}{#5}{#7}%
  \or
  \ifuseprefix
  {\usebibmacro{name:first-last}{#1}{#4}{#5}{#8}}
  {\usebibmacro{name:first-last}{#1}{#4}{#6}{#8}}%
  \or
  \usebibmacro{name:first-last}{#1}{#3}{#5}{#7}%
  \fi
  \usebibmacro{name:andothers}%
  \ifnumequal{\value{listcount}}{\value{liststop}}{'s}{}
}

\DeclareFieldFormat{shorthand:poss}{%
  \ifnameundef{labelname}{#1's}{#1}
}

\DeclareFieldFormat{citetitle:poss}{\mkbibemph{#1}'s}

\DeclareFieldFormat{label:poss}{#1's}

\newrobustcmd*{\posscitealias}{%
  \AtNextCite{%
    \DeclareNameAlias{labelname}{labelname:poss}%
    \DeclareFieldAlias{shorthand}{shorthand:poss}%
    \DeclareFieldAlias{citetitle}{citetitle:poss}%
    \DeclareFieldAlias{label}{label:poss}
  }
}

\newrobustcmd*{\posscite}{%
  \posscitealias%
  \textcite
}

\newrobustcmd*{\Posscite}{\bibsentence\posscite}

\newrobustcmd*{\posscites}{%
  \posscitealias%
  \textcites
}

\newcommand\quelle[1]{{%
  \unskip\nobreak\hfil\penalty50
  \hskip2em\hbox{}\nobreak\hfil#1%
  \parfillskip=0pt \finalhyphendemerits=0 \par
}
}

\bibliography{Thesis}

\begin{document}
<++>

A more precise definition of resultatives is needed to proceed in this study.
Resultative clauses are clauses that contain two distinct predicates (Pred1$\neq$Pred2) which share an argument, in which the primary predicate is construed as the cause of the secondary.
Each of these properties that define resultatives (argument sharing and causativity) is quite common and neither is is sufficient for resultatives.
Depictives, which seem to occur in all of the worlds languages, such as those seen in \Next show argument sharing without causativity.
\ex. \textbf{Depictives}
\a. Mary left angry. (Mary was angry.)
\b. Bill ate the fish raw. (The fish was raw.)
\b. Jamie swam the race naked. (Jamie was naked.)
\b. <+DepictivesInOtherLangs+>
\z.

So, the clause \textit{Mary left angry} means that the two eventualities, the event $e$ of Mary leaving and the state $s$ of Mary being angry, are stand in either an identity ($e=s$) or containment ($e\leq s$) relation rather than a causal relation.

Causativity, broadly construed, is even more prevalent in language.
Every instance of a sentence with an agent encodes causativity.
For example, the clause \textit{John ironed the shirt} means that John acted in such a way as to cause the shirt to be ironed.
\end{document}


