%        File: ACCing.tex
%     Created: Tue Feb 21 02:00 PM 2017 E
% Last Change: Tue Feb 21 02:00 PM 2017 E
%
% arara: pdflatex: {options: "-draftmode"}
% arara: biber
% arara: pdflatex: {options: "-draftmode"}
% arara: pdflatex: {options: "-file-line-error-style"}
\documentclass[MilwayThesis]{subfiles}

\begin{document}
\textcite{cinque1996pseudo} discusses pseudo-relatives (PRs) and ACC-ing clauses (ACs) under direct preception verbs and argues that they are three-ways ambiguous.
\ex. 
\a. Ho visto Mario che correva a tutta velocit\'a. (Italian) 
\b. J'ai vu Mario qui courrait \'a tout vitesse. (French)
\c. I saw Mario running at full speed.

According to Cinque, \Last[a] would have three distinct structures 
\ex.
\a. Ho [visto [$_\text{NP}$ Mario [$_\text{CP}$] che correva \ldots ]]
\b. Ho [visto [$_\text{CP}$ Mario [$_{\text{C}^\prime}$ che [$_\text{IP}$ correva \ldots ]]]]
\c. Ho [[visto Mario] [$_\text{CP}$ \textit{ec} che correva \ldots]]

\end{document}


