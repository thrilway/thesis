%        File: LitReview.tex
%     Created: Thu Oct 05 09:00 AM 2017 E
% Last Change: Thu Oct 05 09:00 AM 2017 E
%
% arara: pdflatex: {options: "-draftmode"}
% arara: biber
% arara: pdflatex: {options: "-draftmode"}
% arara: pdflatex: {options: "-file-line-error-style"}
\documentclass[MilwayThesis]{subfiles}
\begin{document}
In this section I will review several previous analyses of adjectival resultatives and the parametric variation associated with them.
I will evaluate the anlyses against two desiderata.
First, I will evaluate whether the variation, as analyzed, is learnable.
Second, I will evaluate whether the analysis comports with the theoretical principles of the minimalist program.
Before reviewing the analyses, however, I will make these desiderata explicit and justify them.

\section{Desiderata for an analysis of resultatives}

\subsection{Desideratum 1: Learnability}
Most of the analyses of the structure of adjectival resultatives are packaged with an account of the associated parametric variation.
While few address directly address the acquisition of parametric variation, any analysis of variation makes implicit claims about acquistion.
Generally, when discussing parametric variation, the claims about acquisition can justifiably be left implict, as the acquisition task is trivial.
The nature of adjectival resultatives, however, is such that we must make those acquisition claims explicit.
To explain why, I will be comparing the resultative parameter to the V-to-T parameter.

Analyses of the V-to-T parameter do not need to address learnability because the ``parameter setting'' is directly learnable from the primary linguistic data.
It is directly learnable because the overt forms of (\textit{e.g.}) polar questions differs depending on the parameter setting.
In a language with V-to-T movement such as German, the language learner will observe that lexical verbs undergo inversion for polar questions, while in a language without V-to-T movement such as English, the learner will observe that lexical verbs do not invert for questions.
\ex. V-to-T movement (German)
\ag. Trinken sie Kaffee?\\
Drink.3plPres they Coffee\\
``Do they drink coffee?''
\b.* \textsc{do} sie Kaffee trinken?

\ex. *V-to-T Movement (English)
\a.* Drink they coffee?
\b. Do they drink coffee?

The form of polar questions, then, can be positive evidence for a parameter setting.
Since we can find direct positive evidence for a parameter setting, the task of the analyst/theoretician, then is merely to formalize the parameter is a way that is consistent with the broader theory.
\textcite{chomsky1995minimalist}, for instance, formalizes the V-to-T parameter in terms of feature strength, while \textcite{lasnik1999verbal} formalizes it in terms of the presence/absence of inflectional features on lexical verbs.
Neither, however, needs to explicitly describe how their parameter is set, but can assume that a certain setting is the default, and the other setting can be deduced from (\textit{e.g.}) the form of polar questions in the language.

Resultatives, on the other hand, are not directly learnable for two reasons
The first reason is that, on the surface, resultatives, which are parameterized, are indistinguishable from depictives, which appear to be universal.
The two construction types are indistinguishable in the sense that both correspond to the string template in \Next (modulo independent word order variation).
\ex. \textsc{Subj} V \textsc{Obj} Adj.

This indistinguishability is evident in the fact that one can construct examples which are truly ambiguous between resultative and depictive readings, as in \Next.
\ex. 
\a. He fried the fish dry.
\a. $\approx$ He fried the fish once it was dry. (\textbf{Depictive})
\b. $\approx$ He fried the fish until it was dry. (\textbf{Resultative})
\z.
\b. She painted the barn red.
\a. $\approx$ The barn is red in her painting. (\textbf{Depictive})
\b. $\approx$ She applied a coat of red paint to the door. (\textbf{Resultative})
\z.

Assuming a child acquiring either French or English encounters sentences with the form of \LLast in their PLD, there is no obvious way for the child to determine whether a given secondary predicate is to be interpreted depictively or resultatively.

An empiricist might object, arguing that the ambiguous examples above are highly constructed, and would easily by disambiguated in context.
They would insist that the learner would infer a positive setting of the resultative parameter from the use of a secondary predication construction in the presence of a resultative event.
So, an English learner, but not a French learner, might be exposed to the context-sentence pairing in \Next.
\ex. 
\a.[\textbf{Context:} ] A woman is methodically hammering a lump of metal.
A parent draws their child's attention to the hammering event and utters:
\b.[\textbf{A:}] She's hammering the metal flat.

Even this, however, is not fully unambiguous.
While it certainly couldn't be interpreted as a depictive, \textit{flat} could be interpreted as a manner adverb, modifying \textit{hammering}.
Such unavoidable ambiguity would make it difficult to employ any sort of semantic bootstrapping in the acquisition of the resultative parameter.

The second problem comes from the fact that both English-type and French-type languages can express resultative semantics periphrastically as in \Next.
\ex. periphrastic resultatives
\ag. Elle a aplati le m\'etal en martelant\\
She has flattened the metal in hammering\\
\b. She flattened the metal by hammering.


The resultative parameter, then, is a one-or-both parameter, unlike other parameters which are either/or choices.
The V-to-T parameter, in contrast, is an either/or choice which can be made on the basis of the PLD.

\end{document}


