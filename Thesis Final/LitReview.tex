%        File: LitReview.tex
%     Created: Thu Oct 05 09:00 AM 2017 E
% Last Change: Thu Oct 05 09:00 AM 2017 E
%
% arara: pdflatex: {options: "-draftmode"}
% arara: biber
% arara: pdflatex: {options: "-draftmode"}
% arara: pdflatex: {options: "-file-line-error-style"}
\documentclass[MilwayThesis]{subfiles}
\begin{document}
In this chapter I will review several previous analyses of adjectival resultatives and the parametric variation associated with them.
I will evaluate the anlyses against two desiderata.
First, I will evaluate whether the variation, as analyzed, is learnable.
Second, I will evaluate whether the analysis comports with the theoretical principles of the minimalist program.
Before reviewing the analyses, however, I will make these desiderata explicit and justify them.

\section{Desiderata for an analysis of resultatives}

\subsection{Desideratum 1: Learnability}
Most of the analyses of the structure of adjectival resultatives are packaged with an account of the associated parametric variation.
While few address directly address the acquisition of parametric variation, any analysis of variation makes implicit claims about acquistion.
Generally, when discussing parametric variation, the claims about acquisition can justifiably be left implict, as the acquisition task is trivial.
The nature of adjectival resultatives, however, is such that we must make those acquisition claims explicit.
To explain why, I will be comparing the resultative parameter to the V-to-T parameter.

Analyses of the V-to-T parameter do not need to address learnability because the ``parameter setting'' is directly learnable from the primary linguistic data.
It is directly learnable because the overt forms of (\textit{e.g.}) polar questions differs depending on the parameter setting.
In a language with V-to-T movement such as German, the language learner will observe that lexical verbs undergo inversion for polar questions, while in a language without V-to-T movement such as English, the learner will observe that lexical verbs do not invert for questions.
\ex. V-to-T movement (German)
\ag. Trinken sie Kaffee?\\
Drink.3plPres they Coffee\\
``Do they drink coffee?''
\b.* \textsc{do} sie Kaffee trinken?

\ex. *V-to-T Movement (English)
\a.* Drink they coffee?
\b. Do they drink coffee?

The form of polar questions, then, can be positive evidence for a parameter setting.
Since we can find direct positive evidence for a parameter setting, the task of the analyst/theoretician, then is merely to formalize the parameter is a way that is consistent with the broader theory.
\textcite{chomsky1995minimalist}, for instance, formalizes the V-to-T parameter in terms of feature strength, while \textcite{lasnik1999verbal} formalizes it in terms of the presence/absence of inflectional features on lexical verbs.
Neither, however, needs to explicitly describe how their parameter is set, but can assume that a certain setting is the default, and the other setting can be deduced from (\textit{e.g.}) the form of polar questions in the language.

Resultatives, on the other hand, are not directly learnable for two reasons
The first reason is that, on the surface, resultatives, which are parameterized, are indistinguishable from depictives, which appear to be universal.
The two construction types are indistinguishable in the sense that both correspond to the string template in \Next (modulo independent word order variation).
\ex. \textsc{Subj} V \textsc{Obj} Adj.

This indistinguishability is evident in the fact that one can construct examples which are truly ambiguous between resultative and depictive readings, as in \Next.
\ex. 
\a. He fried the fish dry.
\a. $\approx$ He fried the fish once it was dry. (\textbf{Depictive})
\b. $\approx$ He fried the fish until it was dry. (\textbf{Resultative})
\z.
\b. She painted the barn red.
\a. $\approx$ The barn is red in her painting. (\textbf{Depictive})
\b. $\approx$ She applied a coat of red paint to the door. (\textbf{Resultative})
\z.

Assuming a child acquiring either French or English encounters sentences with the form of \LLast in their PLD, there is no obvious way for the child to determine whether a given secondary predicate is to be interpreted depictively or resultatively.

An empiricist might object, arguing that the ambiguous examples above are highly constructed, and would easily by disambiguated in context.
They would insist that the learner would infer a positive setting of the resultative parameter from the use of a secondary predication construction in the presence of a resultative event.
So, an English learner, but not a French learner, might be exposed to the context-sentence pairing in \Next.
\ex. 
\a.[\textbf{Context:} ] A woman is methodically hammering a lump of metal.
A parent draws their child's attention to the hammering event and utters:
\b.[\textbf{A:}] She's hammering the metal flat.

Even this, however, is not fully unambiguous.
While it certainly couldn't be interpreted as a depictive, \textit{flat} could be interpreted as a manner adverb, modifying \textit{hammering}.
Such unavoidable ambiguity would make it difficult to employ any sort of semantic bootstrapping in the acquisition of the resultative parameter.

The second problem comes from the fact that both English-type and French-type languages can express resultative semantics periphrastically as in \Next.
\ex. periphrastic resultatives
\ag. Elle a aplati le m\'etal en le martelant\\
She has flattened the metal in the hammering\\
\b. She flattened the metal by hammering.


The resultative parameter, then, is a one-or-both parameter, unlike other parameters which are either/or choices.
The V-to-T parameter, in contrast, is an either/or choice which can be made on the basis of the PLD.

Since there is no direct way to determine from the primary linguistic data whether a language allows or disallows resultatives, we can reject any analysis of resultatives that requires the parameter to be directly set.
Rather the setting of the resultative parameter must be determined indirectly.

\subsection{Desideratum 2: Theoretical consistency}
The second quality an analysis of resultatives should have is consistency with the broader theory of grammar.
While this is a desideratum of all analyses of grammatical phenomena, I will outline the subset of minimalist hypotheses that I will use to evaluate proposed analyses of adjectival resultatives.
First, I will review what is called the Uniformity of Theta Hypothesis (UTAH) which gives us a baseline for our theory of $\Theta$-role assignments.
Second, I will look at the Borer-Chomsky Conjecture, which is the minimalist hypothesis which will guide our theory of variation.
Each of these will be discussed with reference to a particular property of adjectival resultatives.

The canonical formulation of UTAH is that of \textcite{baker1988incorporation} working in pre-minimalist Principles and Parameters framework, given below in \Next.
\ex. \textbf{The Uniformity of Theta Hypothesis (UTAH)}\\
Identical thematic relationships between items are represented by identical structural relationships between those items at the level of D-structure. \parencite[46]{baker1988incorporation}

While I will be assuming something of this flavour in my discussion, the reference to D-structure, which was eliminated in the minimalist program, renders Baker's hypothesis unusable as it stands.
<+MoreDiscussion+>

I propose the following more precise formulation of UTAH:
\ex. \textbf{Minimal UTAH}\\
If a thematic relation $f$ holds between items X and Y in distinct expressions S$_1$ and S$_2$, then there is a structural relation $g$ that also holds between X and Y in S$_1$ and S$_2$.

To understand what this means, consider the following sentence pairs:
\ex. \label{ex:broke-bottle}
\a. Sara broke the bottle.
\b. The bottle broke.

\ex. \label{ex:katie-ran}
\a. Katie ran.
\b. Katie ran a kilometre.

The pair in \ref{ex:broke-bottle} represents a single thematic relation between \textit{broke} and \textit{the bottle}.
UTAH says that there must be a structural relation that holds between \textit{broke} and \textit{the bottle} in both sentences.
Similarly, in \ref{ex:katie-ran}, there is a single thematic relation between \textit{Katie} and \textit{ran}, which means there must be a single corresponding structural relation between the two.
Any structural analysis of these sentences that violates Minimal UTAH, then, can be rejected on theoretical grounds.

Consider, for instance, three possible structural analyses of the pair in \ref{ex:broke-bottle}.
In the first analysis, represented in \Next, \textit{the bottle} is base generated in its surface position, which is different in the two sentences.
\ex. \textbf{The surface analysis}
\a. \textit{Sara broke the bottle.}\\
\begin{forest}
  nice empty nodes,sn edges,baseline
  [TP
    [DP[Sara,roof]]
    [
      [T]
      [VP
        [V\\broke,align=center]
        [DP[the bottle,roof]]
      ]
    ]
  ]
\end{forest}
\b. \textit{The bottle broke.}\\
\begin{forest}
  nice empty nodes,sn edges,baseline
  [TP
    [DP[the bottle,roof,name=subj]]
    [
      [T]
      [VP[broke,roof]]
    ]
  ]
\end{forest}

In this analysis, there is no single structural relations between \textit{broke} and \textit{the bottle}, corresponding to the single thematic relation.
This analysis, then, violates Minimal UTAH, and can therefore be rejected.

In the second analysis, represented in \Next, \textit{the bottle} originates in [Comp V] and moves to subject position in the intransitive situation.
\ex. \textbf{The movement analysis}
\a. \textit{Sara broke the bottle.}\\
\begin{forest}
  nice empty nodes,sn edges,baseline
  [TP
    [DP[Sara,roof]]
    [
      [T]
      [VP
        [V\\broke,align=center]
        [DP[the metal,roof]]
      ]
    ]
  ]
\end{forest}
\b. \textit{The bottle broke.}\\
\begin{forest}
  nice empty nodes,sn edges,baseline
  [TP
    [DP[the bottle,roof,name=subj]]
    [
      [T]
      [VP
        [V\\broke,align=center]
        [DP[the bottle,roof,name=theme]]
      ]
    ]
  ]
  \draw[->] (theme) to[out=south west, in=south] (subj);
\end{forest}

In this analysis the single thematic relation between \textit{broke} and \textit{the bottle} corresponds to a single structural relation between the two, meaning it complies with Minimal UTAH.

While UTAH considerations clearly lead us to prefer the movement analysis over the surface analysis, such considerations are far from decisive evidence in favour of the movement analysis, as it is not the only possible analysis that complies with UTAH.
The third analysis, is such an analysis.

\end{document}


