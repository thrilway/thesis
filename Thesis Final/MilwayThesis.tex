%        File: Milway Thesis.tex
%     Created: Mon Jan 02 03:00 PM 2017 E
% Last Change: Mon Jan 02 03:00 PM 2017 E
%
% arara: pdflatex
% arara: biber
% arara: pdflatex
% arara: pdflatex
\documentclass[
	draft
	]{ut-thesis}


%\usepackage[margin=1in]{geometry}
\usepackage[british]{babel}
\usepackage[]{csquotes}
\usepackage[backend=biber,style=apa,language=british]{biblatex}
\DeclareLanguageMapping{british}{british-apa}
\usepackage[final]{graphicx}

\usepackage{stmaryrd}
\usepackage[]{amsmath}

\usepackage{amsfonts}
\usepackage{amssymb}
\usepackage{amsthm}
\usepackage{mathtools}
\theoremstyle{definition}
\newtheorem{defn}{Definition}

\usepackage{tikz}
\usepackage{forest}
\usepackage{tabularx}
\usepackage{linguex}
\usepackage{centernot}
\usepackage{subfiles}
\usepackage{multirow}
\usepackage[normalem]{ulem}
\usepackage{xcolor}
\usepackage[]{hyperref}
%\listfiles

\usetikzlibrary{positioning,arrows}
\useforestlibrary{linguistics}
%\DeclareNameFormat{labelname:poss}{% Based on labelname from biblatex.def
%  \ifcase\value{uniquename}%
%  \usebibmacro{name:last}{#1}{#3}{#5}{#7}%
%  \or
%  \ifuseprefix
%  {\usebibmacro{name:first-last}{#1}{#4}{#5}{#8}}
%  {\usebibmacro{name:first-last}{#1}{#4}{#6}{#8}}%
%  \or
%  \usebibmacro{name:first-last}{#1}{#3}{#5}{#7}%
%  \fi
%  \usebibmacro{name:andothers}%
%  \ifnumequal{\value{listcount}}{\value{liststop}}{'s}{}
%}
%
%\DeclareFieldFormat{shorthand:poss}{%
%  \ifnameundef{labelname}{#1's}{#1}
%}
%
%\DeclareFieldFormat{citetitle:poss}{\mkbibemph{#1}'s}
%
%\DeclareFieldFormat{label:poss}{#1's}
%
%\newrobustcmd*{\posscitealias}{%
%  \AtNextCite{%
%    \DeclareNameAlias{labelname}{labelname:poss}%
%    \DeclareFieldAlias{shorthand}{shorthand:poss}%
%    \DeclareFieldAlias{citetitle}{citetitle:poss}%
%    \DeclareFieldAlias{label}{label:poss}
%  }
%}
%
%\newrobustcmd*{\posscite}{%
%  \posscitealias%
%  \textcite
%}
%
%\newrobustcmd*{\Posscite}{\bibsentence\posscite}
%
%\newrobustcmd*{\posscites}{%
%  \posscitealias%
%  \textcites


\newcommand\quelle[1]{{%
  \unskip\nobreak\hfil\penalty50
  \hskip2em\hbox{}\nobreak\hfil#1%
  \parfillskip=0pt \finalhyphendemerits=0 \par
}
}

\newcommand{\AREx}{%
  \a.[(\textbf{AR})]\label{ex:AREx} Natalie hammered the metal flat.
  \z.
}
\setcounter{secnumdepth}{3}
\graphicspath{ {./img/} }
\addbibresource{../Thesis.bib}
\degree{Doctor of Philosophy}
\department{Linguistics}
\gradyear{2018}
\author{Daniel A. Milway}
\title{Explaining the Resultative Parameter}
\begin{document}
\begin{preliminary}
  \maketitle
  \tableofcontents
  %\listoffigures
\end{preliminary}
\chapter{Introduction}
%\subfile{intro}

\fcolorbox{black}{lightgray}{
  \begin{minipage}[t]{0.9\textwidth}
	  \textbf{Note:} I will write a fuller introduction as the body of the thesis becomes more finalized. 
	  In place of that, here are a few bullet points on what my this thesis is about:
	  \begin{itemize}
		  \item Languages vary wrt whether they allow adjectival resultatives.
		  \item No one hase come up with a satisfactory account of resultatives and their variable nature
		  \item I am proposing an account of resultatives that includes:
			  \begin{itemize}
				  \item an analysis of their structure,
				  \item an analysis of the parametric variation, and
				  \item an argument that these two accounts are consistent with each other and that the parametric variation is acquirable.
			  \end{itemize}
		  \item Essentially I am trying to account for \Next and \NNext
	  \end{itemize}
	  \exg. Die Teekanne leer trinken.\\
the teapot empty drink.\\
``to drink the teapot dry'' \parencite[German;][]{kratzer2004building}

\exg.* Dario je ofarbao kucu crveno\\
Dario \textsc{cop} painted house red.\textsc{fem}\\
``Dario painted the house red.'' (Serbo-Croatian; Mia Sara Misic P.C.)

  \end{minipage}
}
\chapter{Previous Literature}\label{sec:litreview}
\subfile{LitReview}
\chapter{The Structure of Resultatives}
\subfile{MyAnalysis}
\chapter{Label Theory}
\subfile{Labels}
%\section{The architecture of a label-theoretic grammar}
%\subfile{agree}
\chapter{Deriving the Resultative Parameter}\label{sec:deriving}

\chapter{Some additional consequences of modified label theory}
\chapter{Conclusion}\label{sec:Conclusion}
\printbibliography
\end{document}


