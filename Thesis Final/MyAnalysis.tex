%        File: MyAnalysis.tex
%     Created: Mon Nov 06 10:00 AM 2017 E
% Last Change: Mon Nov 06 10:00 AM 2017 E
%
% arara: pdflatex: {options: "-draftmode"}
% arara: biber
% arara: pdflatex: {options: "-draftmode"}
% arara: pdflatex: {options: "-file-line-error-style"}
\documentclass[MilwayThesis]{subfiles}
\setcounter{chapter}{3}
\begin{document}
In the previous chapter, I discussed the failings of a number of previous analyses of adjectival resultatives.
In this chapter I will address the positive aspects of one of those analyses, and show how it can be modified to address the concerns raised in the previous chapter.
The analysis in question, that of \textcite{kratzer2004building}, has three features which I will retain in my final analysis: A small clause, theme raising, and a result head.
<+MoreHere+>
\section{Fixing the UTAH problem}
The one issue with Kratzer's analysis is that it seems to violate UTAH.
That is there is a single $\Theta$-relation between \textit{hammer} and \textit{the metal} in both sentneces in \Next, that is not represented by a single structural relation.
\ex.
\a. Joe hammered the metal flat.
\b. Joe hammered the metal.

According to Kratzer's analysis, \textit{the metal} is the specifier of \textit{hammer} in \Last[a], but a standard analysis of \Last[b] will place \textit{the metal} as the complement of \textit{hammer}
\ex.
\a. hammer the metal flat \parencite[following][]{kratzer2004building}\\
\begin{forest}
    nice empty nodes,sn edges,baseline,for tree={
    calign=fixed edge angles,
    calign primary angle=-30,calign secondary angle=70}
    [VP
	    [DP[the metal,roof,name=specV]]
	    [
		    [hammer]
		    [resP
			    [res]
			    [SC
				    [$\langle$DP$\rangle$,name=SCDP]
				    [flat]
			    ]
		    ]
	    ]
    ]
    \draw[->] (SCDP) to[out=south west, in=south] (specV);
\end{forest}
\b.hammer the metal\\
\begin{forest}
    nice empty nodes,sn edges,baseline,for tree={
    calign=fixed edge angles,
    calign primary angle=-30,calign secondary angle=70}
    [VP
	    [hammer]
	    [DP[the metal,roof]]
    ]
\end{forest}

If we were to modify Kratzer's analysis, such that \textit{the metal} is the complement of \textit{hammer} then we would need to attach the result phrase in a different position.
I propose that the result phrase is adjoined to the VP, which means the DP can be merged directly with the verb as shown in \Next.
\ex.\label{tree:hammer-flat}
\begin{forest}
    nice empty nodes,sn edges,baseline,for tree={
    calign=fixed edge angles,
    calign primary angle=-30,calign secondary angle=70}
    [VP
	    [VP
		    [hammer]
		    [DP[the metal,roof,name=compV]]
	    ]
	    [resP
		    [$\langle$DP$\rangle$,name=specRes]
		    [
			    [res]
			    [SC
				    [$\langle$DP$\rangle$,name=SCDP]
				    [flat]
			    ]
		    ]
	    ]
    ]
    \draw[->] (SCDP) to[out=south west, in=south east] (specRes);
    \draw[->] (specRes) to[out=south, in= south] (compV);
\end{forest}

The modified analysis no longer violates UTAH, but it introduces two issues new issues which I address below: Sideward movement and compositionality.
\subsection{Sideward movement}
In my proposed structure for resultatives, the object DP moves from [Spec res] to [Comp V].
The movement ``chain'' this operation forms is problematic due to the fact that the head of the chain does not c-command the tail.
Although this type of so-called sideward movement is generally barred, \textcite{nunes2001sideward} argues that our theory should allow a restricted version of sideward movement.
Nunes argues that head movement and parasitic gaps both require a sideward movement operation, as they both create non-c-command dependencies.
\ex.
\a. Head Hovement\\
\begin{forest}
    nice empty nodes,sn edges,baseline,for tree={
    calign=fixed edge angles,
    calign primary angle=-30,calign secondary angle=70}
    [TP
	    [DP]
	    [
		    [T
			    [T]
			    [V,name=head]
		    ]
		    [VP
			    [$\langle$V$\rangle$,name=tail]
			    [DP]
		    ]
	    ]
    ]
    \draw[->] (tail) to[out=south, in=south] (head);
\end{forest}
\b. Parasitic gaps\\
What did Mary hear without seeing.\\
\begin{forest}
    nice empty nodes,sn edges,baseline
    [CP
	    [DP[What,roof,name=specCP]]
	    [
		    [C+T\\did,align=center]
		    [TP
			    [DP[Mary,roof]]
			    [
				    [$\langle$T$\rangle$]
				    [VP
					    [VP
						    [V\\see,align=center]
						    [$\langle$DP$\rangle$,name=CompV1]
					    ]
					    [PP
						    [P\\without,align=center]
						    [VP
							    [V\\seeing,align=center]
							    [$\langle$DP$\rangle$,name=CompV2]
						    ]
					    ]
				    ]
			    ]
		    ]
	    ]
    ]
    \draw[->] (CompV2) to[out=south, in=south] (CompV1);
    \draw[->] (CompV1) to[out=south, in=south] (specCP);
\end{forest}

According to the standard definition of Merge, sideward movement should be impossible.
The facts of parasitic gaps and head movement, however, suggest that a possibly complex operation with the net effect of sideward movement must be active in the grammar.
Since \textcite{nunes1995diss,nunes2001sideward} has developed a theory of sideward movement, I will adopt that theory, which I explain below.

In order to explain sideward movement, Nunes hypothesizes that a movement operation is composed of a Copy operation followed by Merge.
The operation Copy adds an object X to the workspace of a derivation provided X is contained in an already constructed syntactic object.
\ex. For a workspace W and a syntactic object X, Copy(W, X) = $W\cup \{X\}$ iff there is a syntactic object Z $\in$ W and Z contains X.

Merge, then is a simpler operation which replaces two members of a workspace with the set containing them. 
To see how a Copy+Merge theory of movement works, consider the derivation of passivization in \Next.
\ex. 
\begin{tabular}[t]{lll}
	\textbf{Stage} & \textbf{Workspace} & \\
	\cline{1-2}
	1 & $\{$[T, [ \textsc{Voice}$_{pass}$ [see, [the, boy]]]]$\}$ & Copy([the, boy])\\
	2 & $
		\begin{Bmatrix*}[l]
			\text{[the, boy]},\\
			\text{[T, [ \textsc{Voice}$_{pass}$ [see, [the, boy]]]]}
		\end{Bmatrix*}
		$ & Merge([the, boy], [T \ldots])\\
	3 & $\{$[[the, boy], [T, [ \textsc{Voice}$_{pass}$ [see, [the, boy]]]]]$\}$ &\\
\end{tabular}

The Copy+Merge theory of movement allows us to derive sideward movement by holding the copied object in the workspace while another tree is built as in the derivation of \ref{tree:hammer-flat} in \Next.
\ex.
\begin{tabular}[t]{lll}
	\textbf{Stage} & \textbf{Workspace} & \\
	\cline{1-2}
	1 & $\left\{ \text{[[the, metal], [res, [\dots]]]} \right\}$ & Copy([the, metal])\\
	2 & $
	\begin{Bmatrix*}[l]
		\text{[the, metal]},\\
		\text{[[the, metal], [res, [\dots]]]}
	\end{Bmatrix*}
	$ & Select(hammer)\\
	3 & $
	\begin{Bmatrix*}[l]
		\text{hammer},\\
		\text{[the, metal]},\\
		\text{[[the, metal], [res, [\dots]]]}
	\end{Bmatrix*}
	$ & Merge(hammer, [the, metal])\\
	4 & $
	\begin{Bmatrix*}[l]
		\text{[hammer, [the, metal]]},\\
		\text{[[the, metal], [res, [\dots]]]}
	\end{Bmatrix*}
	$ & Merge$\begin{pmatrix*}[l]\text{[hammer, [the, metal]]},\\ \text{[[the, metal], [res [\dots]]]}\end{pmatrix*}$\\
	5 & \multicolumn{2}{l}{$\left\{\text{[[hammer, [the, metal]], [[the, metal], [res [\dots]]]]}\right\}$}\\
\end{tabular}

Note that in stage 5 of the derivation in \Last the syntactic object in the workspace is representable as \ref{tree:hammer-flat}.

In order to constrain sideward movement, Nunes notes that its immediate results such as the tree in \ref{tree:hammer-flat} are unlinearizable.
Assuming that decisions regarding linear order depend on c-command relations, 

\end{document}
