% arara: pdflatex: {options: "-draftmode"}
% arara: biber
% arara: pdflatex: {options: "-draftmode"}
% arara: pdflatex: {options: "-file-line-error-style"}
\documentclass[MilwayThesis]{subfiles}
\begin{document}

This dissertation rests on a number of non-standard theoretical assumptions and draws a few non-standard distinctions, which I will defend in this chapter.
My defense of the assumptions, however, will not be an argument that they are true, as the truth of any theoretical statement ultimately depends on the empirical facts.
Rather, my defense will actually be an offense; I will argue that the standard assumption is, in fact, ill-founded.
So, in a sense, I will be rejecting standard assumptions rather than making non-standard ones.
The distinctions I draw, in contrast, will not be defended, but rather explained and clarified.

\section{The $\Theta$-Criterion}
The $\theta$-criterion standardly assumed was first formulated by Chomsky in \textit{Lectures in Government and Binding} (LGB) as \Next.
\ex. Each argument bears one and only one $\theta$-role, and each $\theta$-role is assigned to one and only one argument. \parencite[36]{chomsky1981lectures}

In a footnote, Chomsky justifies this criterion, saying 
\begin{quote}
	The second clause of [the $\theta$-criterion] is well-motivated.
	To say that each $\theta$-role must be filled implies, for example, that a pure transitive verb such as \textit{hit} must have an object, that a verb such as \textit{put} or \textit{keep} (with the sense they have in \textit{put it in the corner}, \textit{keep it in the garage}) must have the associated PP slot filled, etc. 
	The additional requirement that each $\theta$-role must be filled by only one argument will, for example, exclude the possibility that a single trace is associated with several argument antecedents, a possibility ruled out in principle under the Move-$\alpha$ theory. 
	\parencite[139]{chomsky1981lectures}
\end{quote}
I would agree that the second clause of \Last, that each $\theta$-role is assigned to a single argument, is well-motivated by the empirical considerations Chomsky cites, and as such I will not reject that portion of the $\theta$-criterion.
The first clause, however, is motivated mainly by theoretical concerns of LGB, that is, its connection to empirical facts is indirect at best.

The nature of the LGB theory is such that its various hypotheses and principles are connected to each other in a web-like network.
As a result, the first clause of the $\theta$-criterion depends on various other theoretical statements and various other theoretical statements depend on it.

\end{document}
