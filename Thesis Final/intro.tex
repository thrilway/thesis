%        File: intro.tex
%     Created: Thu Jun 22 04:00 PM 2017 E
% Last Change: Thu Jun 22 04:00 PM 2017 E
%
% arara: pdflatex: {options: "-draftmode"}
% arara: biber
% arara: pdflatex: {options: "-draftmode"}
% arara: pdflatex: {options: "-file-line-error-style"}
\documentclass[MilwayThesis]{subfiles}

\begin{document}
This thesis is within the minimalist program, which does not mean that I necessarily assume a particular theory of grammar, but that I take the Strong Minimalist Thesis (SMT) to be the guiding methodological axiom of linguistic inquiry.
The SMT, which has been formulated in a number of ways, states that language is an optimal solution to the problem of generating structures which are legible to the Sensorimotor (SM) and Conceptual-Intentional (CI) interfaces.
In Chomsky's words, SMT ``becomes an empirical thesis insofar as we are able to determine interface conditions and to clarify notions of {`good design'}'' \parencite[1]{chomsky2001derivation}.
I take SMT to be a methodological axiom meaning it is assumed true and therefore not evaluable against data.
Furthermore, theoretical statements are to be admitted only if they are consistent with SMT.\footnote{
  This methodology is inspired by Imre Lakatos' (\citeyear{lakatos1978methodology}) philosophy os science.
  Lakatos argues that ``research programmes'' (\textit{sic}) are characterized by two sets of theoretical propositions: 
  the hard core, which is a set of propositions which are held constant, and the protective belt, which are auxiliary hypothesis proposed to ensure that data comports with the hard core.
  A programme is scientific only if its auxiliary hypotheses are progressive, meaning they generate novel testable empirical predictions.
}

As an example of this type of reasoning, consider Chomsky's (\citeyear[144]{chomsky2008on}) discussion of feature inheritance.
Minimalist reasoning led Chomsky to propose the No-Tampering Condition (NTC), a ban on the altering of previously constructed structures, as a condition that an optimal system should ensure.
Empirical facts, however, suggest that finite T inherits its $\varphi$-features from C, which constitutes a violation of NTC.
If inheritance is a language-particular operation (\textit{i.e.}, part of UG), then it violates SMT.
If, however, inheritance is required by the CI interface, then it is fully consistent with SMT.
Chomsky rejects the first possibility as unprincipled, and adopts the second.
If something other than SMT were the guiding axiom of Chomsky's theorizing, however, the first alternative might have been preferred.

As the quote above indicates, syntactic work guided by the SMT will largely consist of an investigation of the interfaces.
So, with the exception of those properties that fall out from the nature of the operation merge, syntactic phenomena (\textit{i.e.}, those phenomena discovered and discussed by syntacticians) are taken to be interface phenomena.
In particular, since the phenomenon under investigation in this thesis is a semantic one, I will be approaching it as an investigation of the CI interface.



\end{document}


