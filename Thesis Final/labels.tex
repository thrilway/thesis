%        File: labels.tex
%     Created: Tue Jun 06 02:00 PM 2017 E
% Last Change: Tue Jun 06 02:00 PM 2017 E
%
% arara: pdflatex: {options: "-draftmode"}
% arara: biber
% arara: pdflatex: {options: "-draftmode"}
% arara: pdflatex: {options: "-file-line-error-style"}
\documentclass[MilwayThesis]{subfiles}
\begin{document}
Chomsky begins his proposal of label theory with a discussion of the minimalist program in general.
In his estimation, the goal of the minimalist program has been to explain the universal properties of language as simply as possible.
The properties he identifies are (i) the structure-dependence of rules, (ii) displacement, (iii) linear order and (iv) projection/labelling.
He then argues that if we assume linear order is a reflex of transfer to the SM interface, properties (i) and (ii) can be explained by assuming that Narrow Syntax is only simplest Merge, as defined in \Next.
\ex. Merge($\alpha$, $\beta$) = $\left\{ \alpha, \beta \right\}$

Unlike previous versions of Merge, however, simplest Merge does not include labelling.
Chomsky argues that this is a welcome outcome, because labelling/projection is not as detectible in surface forms as the other properties of language, and has always been a theory internal notion.
What's more, Chomsky argues, previous theories that bundle labelling with structure building have always stipulated labelling rather than deriving it.
So, for instance, the phrase \textit{see the girl} is stipulated to be a VP rather than a DP.

Chomsky proposes that labels are assigned post-syntactically by a special instance of minimal search called the Labelling Algorithm (LA).
LA operates iteratively in a top-down manner, searching each syntactic object for a ``most prominent'' element which can serve as the label.
In the simplest case, an atomic element (a head X) merged with a complex object (a phrase YP), the atomic element is found the most prominent element and, as such, is the label of the object.
\ex. LA$(\left\{ X, YP \right\}) = X$

The head-phrase case in \Last is trivial due to the inherent asymmetry in the structure.
Labelling becomes more complicated when symmetric structures are considered, that is, when head-head and phrase-phrase structures are considered.
To discover how these structures could be labelled, Chomsky considers exmaples of head-head and phrase-phrase structures that are generated by grammars, and hypothesizes why they are generated and not other instances.
The only head-head structures that surface are those that result from the merger of acategorial roots and category-determining heads.
So, structures like \Next[a] are labellable, but those like \Next[b] and \Next[c] are not.
\ex.
\a. $\left\{ n, \sqrt{\textsc{water}}\right\}$
\b.* $\left\{ n, v\right\}$
\c.* $\left\{ \sqrt{\textsc{ice}}, \sqrt{\textsc{water}} \right\}$

Chomsky proposes that roots are completely featureless and, therefore, invisible to LA, meaning \Last[a] recieves the label $n$.

Phrase-phrase structures present more of a problem, because they surface in multiple forms.


\end{document}
