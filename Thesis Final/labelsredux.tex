%        File: labelsredux.tex
%     Created: Tue Jun 06 03:00 PM 2017 E
% Last Change: Tue Jun 06 03:00 PM 2017 E
%
% arara: pdflatex: {options: "-draftmode"}
% arara: biber
% arara: pdflatex: {options: "-draftmode"}
% arara: pdflatex: {options: "-file-line-error-style"}
\documentclass[MilwayThesis]{subfiles}

\begin{document}
In this section I will address two questions which \textcite{chomsky2013problems,chomsky2015problems} largely leaves open.
First there is the question of how to label Host-Adjunct structures.
These are cases of XP-YP structures, but generally involve neither movement of host or adjunct, nor agreement between the two.
Chomsky's LA, then, would crash when processing these structures.
I propose, following \textcite{hornstein2009theory} and \textcite{chametzky1996theory}, that Host-Adjunct structures are unlabelled.
How is it that a structure can be unlabelled?
To answer this question we must consider the nature of the labelling process. 

If we consider the labelling process to be a function Label from unlabelled structures to labelled structures, then an unlabelled structure is an impossible output of Label.
If Host-Adjunct structures are unlabelled, then they cannot be the output of Label.
This means one of two things, either Host-Adjunct structures are not interpreted at the CI interface, or they bypass Label.
The first possibility seems to be contradicted by the fact that Adjuncts are interpreted at CI, so I will follow the second possibility.
Consider, then, a structure [XP, ZP], where XP is a host and ZP is an adjunct.
The structure as a whole will bypass Label, but the host XP will be labelled.
It is reasonable to assume, then, that the adjunct ZP will bypass Label, meaning its internal structure will not be labelled in this cycle of Label.

<+MoreHere+>

The second question is why labels should be required by the CI interface at all. 

\end{document}


