%        File: learnability.tex
%     Created: Mon Feb 06 11:00 AM 2017 E
% Last Change: Mon Feb 06 11:00 AM 2017 E
%
% arara: pdflatex: {options: "-draftmode"}
% arara: biber
% arara: pdflatex: {options: "-draftmode"}
% arara: pdflatex: {options: "-file-line-error-style"}
\documentclass[Milway Thesis]{subfiles}

\begin{document}
The question of whether the resultative parameter is directly learnable from the PLD turns on the question of what we acquire when we acquire a language.
What do we need, for instance, to fully understand the sentence in \Next or produce it felicitously?
\ex. Some dogs bark loudly.

In addition to phonological knowledge, which I will set aside here, we need to syntactic and interpretive knowledge.
Assuming, following the Borer-Chomsky Conjecture, that what is acquired is the lexicon rather than the grammar, a person need only acquire the individual lexical items in \Last to understand the sentence.
This means we must learn the formal properties of \textit{e.g.} the lexical items making up \textit{dogs} as well as the concepts they express.

\end{document}


