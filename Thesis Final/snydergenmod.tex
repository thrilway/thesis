%        File: snydergenmod.tex
%     Created: Wed Apr 05 09:00 AM 2017 E
% Last Change: Wed Apr 05 09:00 AM 2017 E
%
% arara: pdflatex: {options: "-draftmode"}
% arara: biber
% arara: pdflatex: {options: "-draftmode"}
% arara: pdflatex: {options: "-file-line-error-style"}
\documentclass[MilwayThesis]{subfiles}

\begin{document}
To discuss Snyder's proposed compositional rule, let's consider the form of such rules more generally.
Consider, as a prototype, the standard statement of Predicate Modification (PM) in \ref{def:pm}.
\ex.\label{def:pm} \textbf{Predicate Modification} \parencite{heimkratzer1998semantics}\\
If $\alpha$ is a branching node, $\left\{ \beta,\gamma \right\}$ is the set of $\alpha$'s daughters, and $\llbracket\beta\rrbracket$ and $\llbracket\gamma\rrbracket$ are both in D$_{\langle e,t\rangle}$, then\\
$\llbracket\alpha\rrbracket = \lambda x \in D_e . \llbracket\beta\rrbracket(x) = \llbracket\gamma\rrbracket(x) = 1$.

As with other composition rules (\textit{e.g.} Function Application \parencite{heimkratzer1998semantics} and Event Identification \parencite{kratzer1996severing}), PM  has two parts: a domain statement ($\alpha$ is a branching node whose daughters are each property-denoting), and a range statement ($\llbracket\alpha\rrbracket = \lambda x \in D_e . \llbracket\beta\rrbracket(x) = \llbracket\gamma\rrbracket(x) = 1$).
The mapping of these parts can be evaluated both empirically and theoretically.
Empirically, we can ask whether such a mapping of domain to range holds in natural language.
In most cases, PM seems to be empirically accurate: the combination of two property-denoting expressions seems to be interpreted as the conjunction for those properties.

Theoretically, we can first ask whether a compositional rule is formulable in our system.
Answering such a question is rarely difficult, simply a matter of asking if it is well-formed, but often ignored.
A more difficult theoretical question is whether a compositional rule, as stated, improves the explanatory power of the theory it is to be added to.
This question is more difficult because, unlike well-formedness, explanatory power is a rather vague notion which resists definition.
Considering these questions with respect to PM, we can see that it is theoretically acceptable, given the framework it is couched in.
It is clearly well-formed, in that it uses only well-defined concepts and symbols (\textit{e.g.}, $\lambda$, branching nodes, D$_{\langle e,t\rangle}$), and uses them appropriately.
Also, it is arguably an improvement upon a system which only has one rule of composition (Function Application).
Although the addition of PM complicates the compositional system, it does so in order to keep the syntax simple.

Considering Snyder's (\citeyear{snyder2012parameter}) Generalized Modification (GM) rule, given below in \ref{def:snydergm}, in light of the above questions, I argue that it is not an acceptable rule of composition.
\ex. \textbf{Generalized Modification} \parencite{snyder2012parameter}\\
If $\alpha$ and $\beta$ are syntactic sisters under the node $\gamma$, where $\alpha$ is the \uline{head} of $\gamma$, and if $\alpha$ denotes a \uline{kind}, then\\
interpret $\gamma$ semantically as a \uline{subtype} of $\alpha$'s kind that stands in a pragmatically suitable \uline{relation} to the denotation of $\beta$. (underlining Snyder's)

\end{document}


