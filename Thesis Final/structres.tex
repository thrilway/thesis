%        File: structres.tex
%     Created: Thu Apr 20 10:00 AM 2017 E
% Last Change: Thu Apr 20 10:00 AM 2017 E
%
% arara: pdflatex: {options: "-draftmode"}
% arara: biber
% arara: pdflatex: {options: "-draftmode"}
% arara: pdflatex: {options: "-file-line-error-style"}
\documentclass[MilwayThesis]{subfiles}

\begin{document}
A structural analysis of adjectival resultatives such as (AR) should explain two general facts about them.
\AREx{}

The first fact is that the object of a resultative is interpreted as the theme of both the verb and the adjective; \textit{the metal} is hammered and \textit{the metal} is/becomes flat.
The second fact is that the verb and the adjective interpreted as describing a cause-effect relation; the \textit{hammer}ing event causes the \textit{flat}ness state.
In this section I will consider the current structural analyses of resultatives, and how they account or fail to account for the two general facts.

Before doing so, though, I must consider the theoretical notions that come into play when discussing the two facts of resultatives.
The first fact implicates the notion of $\Theta$-roles which has been an important notion in generative syntax since at least as far back as GB theory \parencite{chomsky1981lectures}.
Research on the syntax of $\Theta$-roles has to this day largely been guided by the Uniformity of Theta Hypothesis (UTAH), given, in its original form, below.
\ex. \textbf{The Uniformity of Theta Hypothesis (UTAH)}\\
Identical thematic relationships between items are represented by identical structural relationships between those items at the level of D-structure. \parencite[46]{baker1988incorporation}

Although UTAH, as formulated by Baker, requires clarification to be applied to specific data, it provides a good enough guideline for my discussion of the first fact about resultatives.
If we consider the thematc relations between verbs and the bolded DPs in \Next, and adjectives and the bolded DPs in \NNext we can see that they are identical to the corresponding thematic relations in (AR).
\ex. 
\a. Natalie hammered \textbf{the metal} for an hour.
\b. \textbf{The metal} was hammered by Natalie.

\ex. 
\a. \textbf{The metal} is flat.
\b. Anaximander considered \textbf{the earth} flat.

Following UTAH then, we expect that the relevant V-DP structural relationships in \LLast should be identical to that in (AR), as should the Adj-DP structural relationships in \Last.
In both cases, the standard assumption is that both structural relationships are local ones (\textit{i.e.} sisterhood relationships).

The second fact raises the notion of causativity, which is a topic often linked to transitivity alternations such as the causative-inchoative alternation, demonstrated in \Next.
\ex.
\a. \textbf{Inchoative}\\
The toast burned.
\b. \textbf{Causative}\\
Jeff burned the toast.

Specifically, \Last[b] is called a causative alternant because it means that an action of Jeff's caused the event described by \Last[b].
Within generative grammar, \Last[b] is commonly assumed to be in some way derived from \Last[a].\footnote{
  See \textcite{fodor1970three} for a dissenting view.
}
This derivation is generally achieved by augmenting inchoative \textit{burn} with a causative functional head to form causative \textit{burn}.
The content of that functional head is an open question, but the only relevant part is the causativity.

\end{document}


