%        File: ACCing.tex
%     Created: Tue Feb 21 02:00 PM 2017 E
% Last Change: Tue Feb 21 02:00 PM 2017 E
%
% arara: pdflatex: {options: "-draftmode"}
% arara: biber
% arara: pdflatex: {options: "-draftmode"}
% arara: pdflatex: {options: "-file-line-error-style"}
\documentclass[MilwayThesis]{subfiles}

\begin{document}
Another issue with my account has to do with the sideward movement operation.
Recall that in order to derive an adjectival resultative, a DP must move sideward from the resP adjunct to the VP as in \cref{fig:ResStruct}.
Note also that that sideward movement operation seems to be obligatory, that is, the DP that originates in the resP must also appear as the theme of the VP.
This obligatoriness can be seen in the fact that \cref{ex:double-theme-res} is ungrammatical.
\ex.*  Sam [$_{VP}$ hammered the nail] [$_{resP}$ the planks together].\label{ex:double-theme-res}\\
($\approx$ Sam hammered the nail and, as a result, the planks were fastened together)

An easy way of accounting for this would be to hypothesize that it is due to some property of the res head, and if this obligatory sideward movement were particular to resultatives, then, indeed, this would likely be the best way to proceed.
However, sideward wovement seems to be obligatory in other cases.
First, there is the case of depictives, which differ from resultatives only in the fact that they lack a res head.
Also, as I will argue in this chapter, certain so-called ACC-ing clauses in direct perception reports such as the embedded clause in \cref{ex:ACCing1} 
\ex. We heard [them shouting at the top of their lungs].\label{ex:ACCing1}

Furthermore, I will argue that this obligatory sideward movement is, in fact, a property of adjoined phrases.
Specifically, the generalization in \cref{ex:AdjunctGen} seems to hold.
\ex. <++>\label{ex:AdjunctGen}

Such a generalization, I argue cannot be accounted for in theory of grammar based on feature satisfaction.
Label theory, however, is able in principle to derive this generalization, but in order to do so, it must be modified and extended.
I will perfom such an extension and show that this modified label theory can derive \cref{ex:AdjunctGen}.

\section{On ACC-ing clauses}

\textcite{cinque1996pseudo} discusses ACC-ing clauses (ACs) under direct preception verbs as in \cref{ex:ACCingDPR} and argues that they are ambiguous, having the two structures in \cref{ex:ACCingStructs}.\footnote{
	The main object of Cinque's study, in fact, is pseudo relatives such as those in \cref{ex:PR}, which he argues are ambiguous between the three structures in \cref{ex:PRStructs}
	\ex.\label{ex:PR}
	\a. Ho visto Mario che correva a tutta velocit\'a. (Italian)
	\b. J'ai vu Mario qui courrait \'a tout vitesse. (French)

	\ex.\label{ex:PRStruct}
	\a. Ho [visto [$_\text{NP}$ Mario [$_\text{CP}$] che correva \ldots ]]
	\b. Ho [visto [$_\text{CP}$ Mario [$_{\text{C}^\prime}$ che [$_\text{IP}$ correva \ldots ]]]]
	\c. Ho [[visto Mario] [$_\text{CP}$ \textit{ec} che correva \ldots]]

	He mentions ACC-ing clauses briefly in order to point out that his remarks and claims about pseudo relatives largely apply to ACC-ing clauses.
	The main difference between the two constructions is that ACC-ing clauses are not analyzable as nominals, but a related form with nominal morphology serves this function.
	\ex. [Their shouting at the top of their lungs] didn't help matters.

}
\ex. I saw Mario running at full speed. \label{ex:ACCingDPR} 

\ex.\label{ex:ACCingStructs}
\a. I [saw [$_\text{ProgP}$ Mario [$_{\text{Prog}^\prime}$ -ing [$_\text{VP}$ run \ldots]]]].
\b. I [[saw Mario] [$_\text{ProgP}$ \textit{ec} running \ldots]].

These two structure, or, rather, the fact that a single grammar can generate both of these structures, present a serious problem for standard theories of grammar.
As such, I will discuss them in greater detail below.

In one structure, represented in \cref{fig:CompACCing}, the ACC-ing clause is merged as the complement of the perception verb.
The interpretation of this structure is one in which the running event was seen and by virtue of the meaning of \textit{run}, seeing a running event generally entail seeing the agent of that event.
\begin{figure}[h]
	\centering
\begin{forest}
    nice empty nodes,sn edges,baseline,for tree={
    calign=fixed edge angles,
    calign primary angle=-30,calign secondary angle=70}
    [VP
	    [V\\see,align=center]
	    [ProgP
		    [DP[Mario,roof]]
		    [Prog'[running at full speed,roof]]
	    ]
    ]
\end{forest}
	\caption{Complement ACC-ing structure}
	\label{fig:CompACCing}
\end{figure}
In the second structure, represented in \cref{fig:AdjunctACCing}, the ACC-ing subject is merged as the complement of the perception verb, while the ACC-ing clause (with a controlled subject) is adjoined to the VP.
The interpretation of this structure is one in which \textit{him} is seen and the event of \textit{him} being seen coincides with an event of \textit{him} running.
Again in this interpretation, due to the meaning for \textit{run}, seeing the agent of a running event generally entails seeing that event.
\begin{figure}[h]
	\centering
\begin{forest}
    nice empty nodes,sn edges,baseline,for tree={
    calign=fixed edge angles,
    calign primary angle=-30,calign secondary angle=70}
    [VP
	    [VP
		    [V\\see,align=center]
		    [DP$_i$[Mario,roof]]
	    ]		    
	    [ProgP
		    [$\langle$DP$_i\rangle$]
		    [Prog'[running at full speed,roof]]
	    ]
    ]
\end{forest}
	\caption{Adjunct ACC-ing structure}
	\label{fig:AdjunctACCing}
\end{figure}
By the assumptions of this thesis, the argument \textit{Mario} can only be shared by the verb \textit{see} and the verb \textit{run} if it is merged with both, meaning it must move from [Spec, Prog] to [Comp, V].
In the case of the complement ACC-ing clause in \cref{fig:CompACCing}, however, the argument \textit{Mario} seems to stay in situ in [Spec, Prog], suggesting that the movemnt operation represented in \cref{fig:AdjunctACCing} is, in fact, optional.
If the movement operation is optional, however, we would expect two additional structures for \cref{ex:ACCingDPR}: one, represented in \cref{fig:CompACCingMove}, in which the ProgP is the complement of \textit{saw} and \textit{Mario} has moved from [Spec, Prog], and another, represented in \cref{fig:AdjunctACCingStay}, in which the ProgP is an adjunct, but \textit{Mario} does not move from [Spec, Prog].
\begin{figure}[h]
	\centering
	\begin{forest}
	    nice empty nodes,sn edges,baseline,for tree={
	    calign=fixed edge angles,
	    calign primary angle=-30,calign secondary angle=70}
	    [AgrOP
		    [DP$_i$[Mario,roof]]
		    [
			    [AgrO]
			    [VP
				    [V\\see,align=center]
				    [ProgP
					    [$\langle\text{DP}_i\rangle$]
					    [Prog'[running at full speed,roof]]
				    ]
			    ]
		    ]
	    ]		
	\end{forest}
	\caption{Complement ACC-ing structure with object raising}
	\label{fig:CompACCingMove}
\end{figure}
\begin{figure}[h]
	\centering
	\begin{forest}
	    nice empty nodes,sn edges,baseline,for tree={
	    calign=fixed edge angles,
	    calign primary angle=-30,calign secondary angle=70}
	    [VP
		    [VP
			    [V\\see,align=center]
		    ]		    
		    [ProgP
			    [DP$_i$[Mario,roof]]
			    [Prog'[running at full speed,roof]]
		    ]
	    ]
	\end{forest}
	\caption{Adjunct ACC-ing structure without DP movement}
	\label{fig:AdjunctACCingStay}
\end{figure}
If we consider the consequences of this proposed optionality, we can see that it is not true.

First, consider the structure in \cref{fig:AdjunctACCingStay}, in particular the fact that \textit{see} does not have an internal argument.
This is not \textit{per se} problematic, as verbs may be optionally transitive, but we would expect that \textit{see} could have an internal argument other than \textit{Mario}.
That is, if \cref{fig:AdjunctACCingStay} is a possible structure for \cref{ex:ACCingDPR}, then we would expect that \cref{ex:ACCingDouble} to be a licit sentence.
\ex.* I [$_\text{VP}$ [$_\text{VP}$ saw Sue] [$_\text{ProgP}$ Mario running at full speed]]. \label{ex:ACCingDouble}

If \textit{Mario} can remain in [Spec, Prog], then we have no way to rule out \textit{Sue} merging with \textit{see} and behaving as a direct object.
Of course, \cref{ex:ACCingDouble} is ungrammatical, suggesting that \textit{Mario} cannot remain in [Spec, Prog] if ProgP is adjoined to VP.
Movement from [Spec, Prog], then, cannot be optional, strictly speaking.
If it is not optional, perhaps it is obligatory.

If movemnt from [Spec, Prog] is obligatory, then we must revise Cinque's analysis of complement ACC-ing clauses.
Suppose, then, that \textit{Mario}, in the complement ACC-ing analysis of \cref{ex:ACCingDPR}, must raise to object.
In other words, suppose \cref{fig:CompACCingMove} is a possible structure of \cref{ex:ACCingDPR} and \cref{fig:CompACCing} is not.
Note that in \cref{fig:CompACCingMove}, \textit{Mario} is the grammatical object but not the theme of \textit{see}.
If this is the case then we expect that \textit{Mario} can become the subject of a passive derived from \cref{fig:CompACCingMove}.

Indeed, subjects of ACCing clauses can become passive subjects as in \cref{ex:ACCingPassive}, but it is not immediately obvious whether \cref{ex:ACCingPassive} is derived from an Adjunct ACC-ing structure or a Complement ACC-ing structure.
\ex. Mario was seen running at full speed. \label{ex:ACCingPassive}

If \cref{ex:ACCingPassive} had been derived from a Complement ACC-ing structure, however, then \textit{Mario} would not have been 
\end{document}


