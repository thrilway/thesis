%        File: ACCing.tex
%     Created: Tue Feb 21 02:00 PM 2017 E
% Last Change: Tue Feb 21 02:00 PM 2017 E
%
% arara: pdflatex: {options: "-draftmode"}
% arara: biber
% arara: pdflatex: {options: "-draftmode"}
% arara: pdflatex: {options: "-file-line-error-style"}
\documentclass[MilwayThesis]{subfiles}

\begin{document}
Another issue with my account has to do with the sideward movement operation.
Recall that in order to derive an adjectival resultative, a DP must move sideward from the resP adjunct to the VP as in \cref{fig:ResStruct}.
Note also that that sideward movement operation seems to be obligatory, that is, the DP that originates in the resP must also appear as the theme of the VP.
This obligatoriness can be seen in the fact that \cref{ex:double-theme-res} is ungrammatical.
\ex.*  Sam [$_{VP}$ hammered the nail] [$_{resP}$ the planks together].\label{ex:double-theme-res}\\
($\approx$ Sam hammered the nail and, as a result, the planks were fastened together)

An easy way of accounting for this would be to hypothesize that it is due to some property of the res head, and if this obligatory sideward movement were particular to resultatives, then, indeed, this would likely be the best way to proceed.
However, sideward wovement seems to be obligatory in other cases.
First, there is the case of depictives, which differ from resultatives only in the fact that they lack a res head.
Also, as I will argue in this chapter, certain so-called ACC-ing clauses in direct perception reports such as the embedded clause in \cref{ex:ACCing1} 
\ex. We heard [them shouting at the top of their lungs].\label{ex:ACCing1}

Furthermore, I will argue that this obligatory sideward movement is, in fact, a property of adjoined phrases.
Specifically, the generalization in \cref{ex:AdjunctGen} seems to hold.
\ex. <++>\label{ex:AdjunctGen}

Such a generalization, I argue cannot be accounted for in theory of grammar based on feature satisfaction.
Label theory, however, is able in principle to derive this generalization, but in order to do so, it must be modified and extended.
I will perfom such an extension and show that this modified label theory can derive \cref{ex:AdjunctGen}.

\section{On ACC-ing clauses}

\textcite{cinque1996pseudo} discusses pseudo-relatives (PRs) and ACC-ing clauses (ACs) under direct preception verbs and argues that they are three-ways ambiguous.
\ex. 
\a. Ho visto Mario che correva a tutta velocit\'a. (Italian) 
\b. J'ai vu Mario qui courrait \'a tout vitesse. (French)
\c. I saw Mario running at full speed.

According to Cinque, \Last[a] would have three distinct structures 
\ex.
\a. Ho [visto [$_\text{NP}$ Mario [$_\text{CP}$] che correva \ldots ]]
\b. Ho [visto [$_\text{CP}$ Mario [$_{\text{C}^\prime}$ che [$_\text{IP}$ correva \ldots ]]]]
\c. Ho [[visto Mario] [$_\text{CP}$ \textit{ec} che correva \ldots]]

\end{document}


