
%
% arara: pdflatex: {options: "-draftmode"}
% arara: biber
% arara: pdflatex: {options: "-draftmode"}
% arara: pdflatex: {options: "-file-line-error-style"}
\documentclass[MilwayThesis]{subfiles}

\begin{document}
\begin{abstract}
	This thesis proposes an explanatory account for the fact that some, but not all languages generate adjectival resultatives.
It does so by synthesizing a number earlier proposals, namely \textcite{snyder1995language,snyder2016compound}, \textcite{kratzer2004building}, and \textcite{son2008microparameters}.
From Snyder, I adopt the proposal that since it is not directly acquirable, the setting of the resultative parameter must be acquired indirectly.
Specifically, I adopt the proposal that a positive setting of the resultative parameter is inferred from the presence of bare stem compounding in th primary linguistic data, and a negative setting is inferred from the absence of bare stem compounding.
From Kratzer, I adopt the proposal that Snyder's parameter is representable as the presence or absence of inflectional features on lexical categories, and a small clause analysis of resultatives.
Finally, from Son and Svenonius, I adopt the proposal that secondary predication structures like adjectival resultatives involve a single syntactic argument being shared by the primary and secondary predicates.

These previous proposals
\end{abstract}
\end{document}
