
%
% arara: pdflatex: {options: "-draftmode"}
% arara: biber
% arara: pdflatex: {options: "-draftmode"}
% arara: pdflatex: {options: "-file-line-error-style"}
\documentclass[MilwayThesis]{subfiles}

\begin{document}
\begin{abstract}
	This thesis proposes an explanatory account for the fact that some, but not all languages exhibit adjectival resultatives.
It does so by synthesizing a number of earlier proposals.
From \textcite{snyder1995language,snyder2016compound}, I adopt the proposal that the setting of the resultative parameter must be acquired indirectly, being inferred from the presence or absence of bare stem compounding in the primary linguistic data.
From \textcite{kratzer2004building}, I adopt a small clause analysis of resultatives and the proposal that Snyder's parameter is related to inflectional features on adjectives.
Finally, from \textcite{son2008microparameters}, I adopt the proposal that adjectival resultatives involve a single syntactic argument being shared by the primary and secondary predicates.

These proposals, combined with  minimalist assumptions, yield: 
\textit{(i)} a structural analysis of resultatives, in which they are represented by a resP adjoined to a VP, with a DP argument undergoing sideward movement between them,
and \textit{(ii)} an analysis of the parameter, according to which languages generate adjectival resultatives only if their lexicon contains uninflected categorizing heads.
I then show that, under Chomsky's (\citeyear{chomsky2013problems}, \citeyear{chomsky2015problems}) label theory, with some modifications, an adjectival resultative can be derived only if the result adjective is categorized by an uninflected $adj$ head (\textit{e.g.}, $\left\{ adj_{\emptyset}, \sqrt{flat} \right\}$).
This demonstration concludes the theoretical explanation  of the resultative parameter.

The remainder of the thesis addresses some consequences of the explanation.
First I address the appearance that my theory seems to wrongly rule out copular clauses and depictives in, for example, French.
I argue that this stems from our conception of Agree, and present a theory of postsyntactic Agree that gives the right empirical results.
Next, I address the previously unrecognized generalization that sideward movement from adjuncts is always necessary when possible.
Using Cinque's (\citeyear{cinque1996pseudo}) analysis for ACC-ing clauses, I show that this can be explained by a novel theory of adjunction synthesizing the pair-merge \parencite{chomsky2004beyond} and late adjunction \parencite{stepanov2001late} theories.
Finally, I discuss the semantics that follows from my syntactic analysis.
Although the semantic analysis seems implausible, I present corroborating evidence in its favour and argue that is more plausible than the alternatives.
\end{abstract}
\end{document}
