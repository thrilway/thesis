
%
% arara: pdflatex: {options: "-draftmode"}
% arara: biber
% arara: pdflatex: {options: "-draftmode"}
% arara: pdflatex: {options: "-file-line-error-style"}
\documentclass[MilwayThesis]{subfiles}

\begin{document}
\begin{abstract}
	This thesis proposes an explanatory account for the fact that some, but not all languages generate adjectival resultatives (\textit{e.g.}, \textit{Jackie hammered the metal flat.}).
It does so by synthesizing a number earlier proposals, namely \textcite{snyder1995language,snyder2016compound}, \textcite{kratzer2004building}, and \textcite{son2008microparameters}.
From Snyder, I adopt the proposal that since it is not directly acquirable, the setting of the resultative parameter must be acquired indirectly.
Specifically, I adopt the proposal that a positive setting of the resultative parameter is inferred from the presence of bare stem compounding in th primary linguistic data, and a negative setting is inferred from the absence of bare stem compounding.
From Kratzer, I adopt the proposal that Snyder's parameter is representable as the presence or absence of inflectional features on lexical categories, and a small clause analysis of resultatives.
Finally, from Son and Svenonius, I adopt the proposal that secondary predication structures like adjectival resultatives involve a single syntactic argument being shared by the primary and secondary predicates.

These previous proposals are filtered through a number of minimalist assumptions, resulting in a synthesis, consisting of a structural anlysis of resultatives, and an analysis of the parameter.
According to the structural analysis, resultatives are represented by an adjectival resP adjoined to a VP, with a DP argument shared between them by a sideward movement step.
According to the analysis of the parameter, languages generate adjecival resultatives only if their lexicon contains uninflected categorizing heads.
I then show that, under Chomsky's (\citeyear{chomsky2013problems}, \citeyear{chomsky2015problems}) label theory, an adjectival resultative can be derived if the result adjective is categorized by an uninflected $adj$ head (\textit{e.g.}, $\left\{ adj_{\emptyset}, \sqrt{flat} \right\}$), but not if the adjective is categorized by an inflected \textit{adj} head (\textit{e.g.}, $\left\{ adj_{\varphi}, \sqrt{plat} \right\}$).
This demonstration concludes the theoretical explation  of the resultative parameter, and part I of the thesis.

In part II, I address consequences, peculiarities and apparent flaws of the theoretical explanation presented in Part I.
First I address the an apparent flaw in my theory that seems to wronly rule out copular clauses and depictives in, for example, French.
I argue that this flaw stems from our conception of Agree, and present a theory of Agree as a postsyntactic operation that gives the right empirical results.
Next I address the previously unrecognized generalization that sideward movement from adjuncts is always necessary when possible.
Using Cinque's (\citeyear{cinque1996pseudo}) proposed ambiguity analysis for ACC-ing clauses under direct perception verbs (\textit{e.g.}, \textit{We saw Sadie running.}), I show that this can be explained by a novel theory of adjunction that synthesizes the competing pair-merge \parencite{chomsky2004beyond} and late adjunction \parencite{stepanov2001late} theories.
Finally, I address the semantic analysis that follows from my syntactic analysis.
Although these analyses may seem implausible at first, I argue that there is corroborating evidence in their favour and that they are much more plausible than the alternatives.
\end{abstract}
\end{document}
