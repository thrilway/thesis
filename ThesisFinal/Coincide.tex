%        File: Coincide.tex
%     Created: Sat Apr 14 12:00 PM 2018 E
% Last Change: Sat Apr 14 12:00 PM 2018 E
%
% arara: pdflatex: {options: "-draftmode"}
% arara: biber
% arara: pdflatex: {options: "-draftmode"}
% arara: pdflatex: {options: "-file-line-error-style"}
\documentclass[MilwayThesis]{subfiles}
\begin{document}
In this thesis I have made liberal use of a novel class of sideward movement structures which I schematize below in \cref{fig:SidewardSchema}.
\begin{figure}[h]
	\centering
\[\sbox0{$\begin{array}[]{ccc}
		\begin{forest}
	    nice empty nodes,
	    sn edges,baseline,
	    for tree={
	    calign=fixed edge angles,
	    calign primary angle=-30,calign secondary angle=70}
	    [YP
		    [DP$_i$]
		    [
			    [Y]
			    [ZP]
		    ]
	    ]
	\end{forest}			
	&
	\tikz[baseline=10ex,scale=1] \node[inner sep=0] at (0,-1) {\large,\,};
	&
	\begin{forest}
	    nice empty nodes,
	    sn edges,baseline,
		for tree={
	    calign=fixed edge angles,
	    calign primary angle=-30,calign secondary angle=70}
	    [VP
		    [V]
		    [DP$_i$]
	    ]
	    \end{forest}
		\end{array}$}
\mathopen{\resizebox{1.2\width}{\ht0}{$\Bigg\langle$}}
\usebox{0}
\mathclose{\resizebox{1.2\width}{\ht0}{$\Bigg\rangle$}}
\]
	\caption{A schema of the sideward movement structure}
	\label{fig:SidewardSchema}
\end{figure}
This structure was used in the analysis of adjectival resultatives and depictives, and for one available structure for direct perception reports with ACC-ing clauses.
The distinction between these constructions is due to the choice of head Y.
For resultatives, Y is instantiated by res, while for perception reports, it is instantiated by Prog.
In depictives, the adjoined phrase YP is a small clause, so Y is either absent or instantiated by a Pred head.
Note that, while the complement of Y, ZP, also varies from construction to construction, this variation can be derived from the selectional requirements of Y.

If we take VPs to be event descriptions, and we assume that host-adjunct structures compose by predicate conjunction, then a VP adjunct (SC for depictives, resP for resultatives, and ProgP for DPRs) must also be event description.
That is, in \cref{fig:SidewardSchema} the VP and YP are both interpreted as predicates of events, and, because they combine by predicate conjunction, they both describe the same event.
In this chapter, I discuss the interpretation of depictives and adjunct ACC-ing clauses.
As for the interpretation of resultatives, nothing needs to be added the discussion in \cref{sec:ResInterp}.

In the case of depictives, the interpretation is mostly straightforward, though not entirely without complications.
Consider \cref{ex:Depictive}, which describes an eventuality in which Natasha eats the fish while that fish is raw.
\ex. Natasha ate the fish raw.\label{ex:Depictive}

According to our analysis, this interpretation would be derived from the fact that the eating event and the rawness state are taken to to be identical.
That is the VP in \cref{ex:Depictive} has a logical form  as derived in \cref{ex:DepictiveLF}
\ex. \textsc{sem}(\textit{eat the fish raw}) =\\
$\lambda e [\textsc{sem}(\textit{eat the fish})(e) \, \&\, \textsc{sem}(\textit{the fish raw})(e)] =$\\
$\lambda e [\textbf{eating}(e)\, \&\, \textbf{raw}(e)\, \&\, \textsc{Theme}(\textbf{the\_fish})(e)]$\label{ex:DepictiveLF}

One might object that this LF is incoherent, as eating is an event, while rawness is a state, and an eventuality cannot be both an event and a state.
This objection, however, does not hold up under scrutiny.
Suppose we take the externalist perspective, according to which the entities that natural language expressions are predicated of are external to and independent of the mind, and the predicates and concepts of natural language correspond to natural kinds.
From this perspective, eventualities are regions of space-time, some of which are events, while others are states.
So, for instance to utter \cref{ex:EventDesc} truthfully is to refer to a particular region of space-time.
\ex.\label{ex:EventDesc} The officer ticketed the car.

Now, according to the objection at hand, the region of space-time referred to by \cref{ex:EventDesc} is an event, and, therefore, not a state.
However, it is entirely reasonable to assume we could truthfully utter \cref{ex:StateDesc}, a state description, referring to the same space-time region.
\ex.\label{ex:StateDesc} The car was parked illegally.

It seems, then, that, if there is an event/state contrast, it does not originate in the extra-mental world, or else \cref{ex:EventDesc} and \cref{ex:StateDesc} could not possibly refer to the same region of space-time.

Furthermore, many adverbs describe states, yet may modify event descriptions.
If adverbs are adjuncts, then they are interpreted as conjoined with their host, meaning that they will provide a partial description of an event, rather than a state.
Therefore, there doesn't seem to be any contradiction in my analysis of depictives.

The case of adjunct ACC-ing clauses is slightly more challenging, due to a subtlety in their meanings which I will discuss below.
Ultimately, however, their meaning can be explained from their structure, given in \cref{fig:ACCingPair}, and an assumption about the nature of eventualities.
\begin{figure}[h]
	\centering
\[\sbox0{$\begin{array}[]{ccc}
		\begin{forest}
	    nice empty nodes,
	    sn edges,baseline,
	    for tree={
	    calign=fixed edge angles,
	    calign primary angle=-30,calign secondary angle=70}
	    [ProgP
		    [DP$_i$]
		    [
			    [Prog]
			    [VoiceP[DP$_i$ run,roof]]
		    ]
	    ]
	\end{forest}			
	&
	\tikz[baseline=10ex,scale=1] \node[inner sep=0] at (0,-1) {\large,\,};
	&
	\begin{forest}
	    nice empty nodes,
	    sn edges,baseline,
		for tree={
	    calign=fixed edge angles,
	    calign primary angle=-30,calign secondary angle=70}
	    [VP
		    [saw]
		    [DP$_i$[the dog,roof]]
	    ]
	    \end{forest}
		\end{array}$}
\mathopen{\resizebox{1.2\width}{\ht0}{$\Bigg\langle$}}
\usebox{0}
\mathclose{\resizebox{1.2\width}{\ht0}{$\Bigg\rangle$}}
\]
	\caption{An adjunct ACC-ing structure}
	\label{fig:ACCingPair}
\end{figure}
As a first pass, we can say that the interpretation of the structure in \cref{fig:ACCingPair} is a description of an event of the dog being seen and an event of the dog running.
This alone is not sufficient, as an English speaker's intuition regarding the \cref{ex:ACC-ing} is that the $we$ referent saw both the dog and the event of the dog running.
\ex. We saw the dog running.\label{ex:ACC-ing}

This intuition seems to be inescapable; English speakers cannot seem to entertain an interpretation of \cref{ex:ACC-ing} in which \textit{we} saw the dog but not the running event, or the event but not the dog.
This suggests that both complement ACC-ing and adjunct ACC-ing versions of \cref{ex:ACC-ing} are interpreted as both the individual and the event being seen.
The strong version of UTAH that I assume in this thesis, however, predicts that the interpretation \textit{x was seen} can only be encoded if the expression denoting $x$ is merged as the complement of the verb \textit{see}.
In adjunct ACC-ing analysis of \cref{ex:ACC-ing}, as shown in \cref{fig:ACCingPair}, the event denoting expression, ProgP, is adjoined to VP, yet we interpret it as meaning that the event was seen.
Since this interpretation is not directly encoded, we must infer it from what is directly encoded.

To see how we would infer the perception of the event, consider what is directly encoded.
First, the VP is interpreted as a description of a seeing event which the dog is the theme of.
\ex.\label{ex:VPSEM} \textsc{sem}(VP) = $\lambda e [\textbf{see}(e)\,\&\,\textsc{theme}(\textbf{the\_dog})(e)]$

The interpretation of the ProgP, represented in \cref{ex:ProgPStruct}, however, is more complicated.
\ex.\label{ex:ProgPStruct} [$\langle$the dog$\rangle$, [Prog [$_\text{VoiceP} \langle\text{the dog}\rangle$ run]]]

I will make the simplifying assumption that the copy of \textit{the dog} in [Spec, Prog] is semantically vacuous\footnote{
	At this stage, this is purely stipulative.
	I suspect some version of this assumption is true, but a full investigation and justification of it is beyond the scope of this thesis.
} and discuss the Prog-VoiceP structure.
The VoiceP is unremarkable, so I assume its meaning is the complete but tenseless event description in \cref{ex:VoicePSEM}.
\ex.\label{ex:VoicePSEM} \textsc{sem}(VoiceP) = $\lambda e [\textbf{run}(e)\,\&\,\textsc{doer}(e)(\textbf{the\_dog})]$

Prog, then, takes this description as an argument, and ascribes progressive aspect to it.
The standard, though perhaps na\"ive analysis of progressive aspect (as proposed in \cite{klein1994time}) is that Prog takes a description of event $e$ as an argument, introduces a topic time $t$, and asserts that $t$ is included in the run-time of $e$.
This predicts that ProgP encodes the predicate of times in \cref{ex:ProgPSEM1}
\ex.\label{ex:ProgPSEM1} \textbf{First approximation:}\\\textsc{SEM}(ProgP) =  $\lambda t \exists e [t \subseteq \textsc{time}(e)\,\&\,\textbf{run}(e)\,\&\,\textsc{doer}(e)(\textbf{the\_dog})]$

However, since ProgP adjoins to VP and the resulting structure is interpreted as a conjunction, ProgP must be interpreted as a predicate of events.
Therefore, I will modify the semantic analysis of Prog such that it introduces an event $e^{\prime}$ and asserts that $e^{\prime}$ is included in $e$.
The final interpretation of the ProgP, is given in \cref{ex:ProgPSEM2} \parencite[cf.][]{bjorkman2018poster}.
\ex.\label{ex:ProgPSEM2} \textbf{Second approximation:}\\\textsc{sem}(ProgP) =  $\lambda e^{\prime} \exists e [e^{\prime} \subseteq e\,\&\,\textbf{run}(e)\,\&\,\textsc{doer}(e)(\textbf{the\_dog})]$

This interpretation will properly compose with \cref{ex:VPSEM} to yield the interpretation of the host-adjunct structure in \cref{ex:VPProgPSEM}.
\ex.\label{ex:VPProgPSEM} \textsc{sem}($\langle$ProgP, VP$\rangle$) = $\lambda e^{\prime} \exists e [ \textbf{see}(e^{\prime})\,\&\,\textsc{theme}(\textbf{the\_dog})(e^{\prime})\,\&\,e^{\prime} \subseteq e\,\&\,\textbf{run}(e)\,\&\,\textsc{doer}(e)(\textbf{the\_dog})]$

So, the event of seeing the dog is included in the event of the dog running, meaning that the seeing occurred in the same space-time region as the running and therefore we can infer that the running event was seen.
This denotation along with the very nature of seeing and running allows us to infer from \cref{ex:VPProgPSEM} that we saw the running event.

One could argue that this analysis is implausible as it requires that the seeing event is a part of the seemingly independent running event.
On its face, this seems to imply an interdependecy between the two events, and, while it seems reasonable to say that the perception event depends on the perceived event, it is far from obvious that the perceived event depends on the perception event.
This line of argumentation, I believe, is not enough to rule out the logical form in \cref{ex:VPProgPSEM}.


Saying that the perception of an event is a sub-part of that event, does imply that the event is dependent on it being perceived, but it does so in a very weak way.
It is perhaps a truism of set theory and mereology to say that two complex objects are identical only if they consist of the same parts.
So, if $x$ is a part of $e$ but not a part of $e^{\prime}$, the $e \neq e^{\prime}$.
Similarly, a particular running event which is seen by some individual $x$, cannot be identical to a running event which is not seen by $x$.
Note that this does not mean that the unseen running event is not a running event, only that it is a not a seen running event.

Furthermore, there is linguistic evidence that an event of \textit{x} being seen can be construed as a part of an event of \textit{x} running.
Consider, for instance, sentences of the form \textit{In the course of XP, S}, such as \cref{ex:ITCORunShops} and \cref{ex:ITCOInvestigationInterview}.
\ex.\label{ex:ITCORunShops} In the course of her morning run today, Sadie saw three new coffee shops.

\ex.\label{ex:ITCOInvestigationInterview} In the course of her investigation, Kima interviewed two hotel receptionists.

Sentences of this form seem to entail that the event described by \textit{S} is a proper subpart of the event described by \textit{XP}.
So seeing the new coffee shops was a part of the running event, as interviewing receptionists was part of the investigating event.
Compare these to the infelicitous uses of this construction in \cref{ex:ITCORunMeet} and \cref{ex:ITCOInvestigationMarried}.
\ex.\#? In the course of her morning run today, Sadie met her writing partner for coffee.\label{ex:ITCORunMeet}

\ex.\# In the course of her investigation, Kima got married.\label{ex:ITCOInvestigationMarried}

In both cases the would-be subevent is judged to be either incompatible with the would-be superevent, or not a natural subevent of the would-be superevent.
In the case of \cref{ex:ITCORunMeet}, a morning run is a more-or-less uninterrupted event, and meeting someone for coffee would constitute an interruption.
In the case of \cref{ex:ITCOInvestigationMarried}, even if the getting married event occurred while the investigation were ongoing, the sentence as given means that the marriage is somehow part of the investigation.
For instance \cref{ex:ITCOInvestigationMarried} would be true if Kima got married as part of an undercover operation.
Now consider \cref{ex:ITCORunSeen}, which entails that an event of someone seeing Sadie is included in an event of Sadie running.
\ex.\label{ex:ITCORunSeen} In the course of her morning run today, Sadie was seen by Declan.

Since this sentence is not judged to be odd, we can infer that there is no semantic or conceptual reason to rule out the interpretation in \cref{VPProgPSEM}.

That being said, I will now entertain two alternative analyses and discuss their flaws.

Suppose, for instance, that the semantics of Prog is about time rather than eventualities, as proposed by \textcite{klein1994time}.
This can be attained without the compositionality issues discussed above if we hypothesize the denotation in \cref{ex:ProgTauDenote}.
\ex.\label{ex:ProgTauDenote} \textsc{sem}(Prog) = $\lambda P_{\langle s,t\rangle} \lambda e^{\prime} \exists e [\tau(e^{\prime}) \subseteq \tau(e) \& P(e^{\prime})]$

This denotation would predict that the sentence in \cref{ex:ACC-ing}, under the adjunct ACC-ing interpretation, would mean that there was an event $e1$ of us seeing the dog, and an event $e2$ of the dog running, and that the time of $e1$ is included in the time of $e2$.
And while one could certainly infer that if one sees a dog at the same time as the dog is running, then one sees the running, such an inference does not hold in other cases.
Consider the proposed adjunct ACC-ing interpretation of \cref{ex:PlayingSoccer}, given in \cref{ex:PlayingSoccerTimes}.
\ex.\label{ex:PlayingSoccer} I [[$_{\text{VP}}$ saw Ronaldo$_{i}$][$_{\text{ProgP}}$ $t_{i}$ playing soccer on TV]].

\ex.\label{ex:PlayingSoccerTimes} 
$\begin{array}{rcl}
	\textsc{sem}(\cref{ex:PlayingSoccer}) & =  & \exists e, e^{\prime} [\textsc{Exper}(e)(\textbf{speaker}) \,\&\, \textbf{see}(e) \,\&\, \textsc{theme}(e)(\textbf{R})\\
	& & \&\, \textsc{Agent}(e^{\prime})(\textbf{R}) \,\&\, \textbf{playing\_soccer\_on\_TV}(e^{\prime})\\
	& & \&\, \tau(e) \subseteq \tau(e^{\prime})] 
\end{array}$

While this hypothesized interpretation is consistent with the actual interpretation of \cref{ex:PlayingSoccer}, it is also consistent with some non-existent interpretations.
For instance, \cref{ex:PlayingSoccerTimes} is consistent with a situation in which the speaker sees Ronaldo, while a replay of one of his matches plays on the TV in the other room.

To be more forceful, this hypothesized interpretation of Prog predicts that the illicit sentences in \cref{ex:SlanderedPassive} and \cref{ex:ParodiedPassive}, reproduced below, should be licit.
\ex.* The writer was heard being slandered.\label{ex:SlanderedPassive22}

\ex.* The singer was seen being parodied.\label{ex:ParodiedPassive2}

Since the coincidence between the perception event and the slandering/parodying event is strictly temporal, there is no reason to require that the two events occur in the same room, or even in the same hemisphere.
So, suppose I were sitting at a caf\'e with some singer, while at the same time, ``Weird Al'' Yankovic was recording a parody of that singer in a studio across town.
If Prog merely encodes temporal coincidence, then \cref{ex:ParodiedPassive2} should be licit and true of this situation.
These examples, however, are illicit, and therefore the temporal coincidence hypothesis does not fare as well as my hypothesis.

We could, of course, seek to stipulate that the ProgP event (\textit{e.g.}, the event of the dog running) is perceived, but I argue that such a move is untenable.
Suppose that the logical form in \cref{ex:AdHocLF} is final logical form of the VP in \cref{ex:ACC-ing}.
\ex.\label{ex:AdHocLF}
$
\begin{array}[t]{rcl}
	\textsc{sem}(\text{\cref{ex:ACC-ing}}) & = & \lambda e \exists e^{\prime} [ \textbf{see}(e) \,\&\, \textsc{theme1}(e)(\textbf{the\_dog}) \,\&\, \textsc{theme2}(e)(e^{\prime})\\
	& & \&\, \exists e^{\prime\prime}[ e^{\prime} \subseteq e^{\prime\prime} \,\&\, \textbf{run}(e^{\prime\prime}) \,\&\, \textsc{doer}(e^{\prime\prime})(\textbf{the\_dog})]]
\end{array}
$

There are, as far as I can tell, three possibilities for where the novel \textsc{theme2} predicate is encoded.
It is either (\textit{i}) encoded in Prog, (\textit{ii}) encoded in the perception verb, or (\textit{iii}) encoded on an independent functional head, which I will call $\Theta$.
I will discuss each of these below in turn, and show that they all fail empirically.

First, consider the option of encoding a theme predicate in the Prog head.
This obviously wouldn't do for the case in which the present participle is the main verb of the clause (\textit{e.g.}, \textit{The dog is running.}), so we need to hypothesize a new head which I will call Prog$_{\Theta}$, with the denotation in \cref{ex:ProgAdHoc}.
\ex.\label{ex:ProgAdHoc} $\textsc{sem}(\text{Prog}_{\Theta}) = \lambda P \lambda e \exists e^{\prime},e^{\prime\prime}[\textsc{theme}(e)(e^{\prime}) \,\&\, e^{\prime}\subseteq e^{\prime\prime} \,\&\, P(e^{\prime\prime})]$

Now consider the adjunct ACC-ing interpretation of \cref{ex:ACC-ing} under this hypothesis.
The Prog$_{\Theta}$P interpretation will be \cref{ex:ProgPAdHoc}, and the host VP interpretation will be \cref{ex:VPSimple}.
\ex.\label{ex:ProgPAdHoc} $
\begin{array}[t]{rcl}
	\textsc{sem}(\text{Prog}_{\Theta}\text{P}) &=& \lambda e \exists e^{\prime},e^{\prime\prime}[\textsc{theme}(e)(e^{\prime}) \,\&\, e^{\prime}\subseteq e^{\prime\prime}\\
	& & \&\, \textsc{doer}(e^{\prime\prime})(\textbf{the\_dog}) \& \textbf{run}(e^{\prime\prime})]
\end{array}
$

\ex.\label{ex:VPSimple} \textsc{sem}(VP) = $\lambda e [\textsc{theme}(e)(\textbf{the\_dog}) \,\&\, \textbf{see}(e)]$

If we these expressions were conjoined by adjunction, then we would get the expression in \cref{ex:AdHocLF1}, where the seeing event has two distinct themes.
\ex.\label{ex:AdHocLF1} $
\begin{array}[t]{rcl}
	\textsc{sem}(\langle \text{Prog}_{\Theta}\text{P}, \text{VP}\rangle) &=& \lambda e\exists e^{\prime},e^{\prime\prime}[\textbf{see}(e)\\
		& & \&\, \textsc{theme}(e)(\textbf{the\_dog}) \,\&\, \textsc{theme}(e)(e^{\prime})\\
	& & \&\, e^{\prime}\subseteq e^{\prime\prime} \,\&\, \textsc{doer}(e^{\prime\prime})(\textbf{the\_dog}) \& \textbf{run}(e^{\prime\prime})]
\end{array}
$

This is problematic since there seems to be a uniqueness restriction on thematic roles: in a given event description, each thematic role predicate must be true of at most one (possibly plural) individual.
We can see this in the case of the agent role in \cref{ex:DoubleAgent} and the beneficiary or recipient role in \cref{ex:DoubleBen}.
\ex.* \label{ex:DoubleAgent} Angela duped Phyllis by Pam.

\ex.? \label{ex:DoubleBen} Toby baked Kevin a cake for Oscar.\footnote{
	This sentence is acceptable if Kevin is the recipient and Oscar is the beneficiary or vice-versa, but not if Kevin and Oscar have the same role.
}

We could stipulate that Prog$_{\Theta}$ encodes a secondary theme predicate \textsc{theme2}, but this would be entirely ad hoc, as this role would only be used for these direct perception reports.
Note, of course, that this line of argumentation would apply to the other two possibilities investigated below.
I could, therefore, cut the discussion short and declare the analysis in \cref{ex:AdHocLF} irreparably flawed.
There are other arguments against the analysis though, so I will present those below.

Consider the option of encoding the secondary theme in the perception verb.
Our new version of \textit{see}, then, is given in \cref{ex:SeeAdHoc}.
\ex.\label{ex:SeeAdHoc}
$
\begin{array}[t]{rcl}
	\textsc{sem}(\text{see}_\Theta) & = & \lambda x_e \lambda P_{\langle s,t\rangle} \lambda e_{s} \exists e^{\prime} [\textbf{see}(e) \,\&\, \textsc{theme1}(e)(x) \\
		& & \&\, \textsc{theme2}(e)(e^{\prime}) \,\&\, P(e^{\prime})]
\end{array}
$

The issue with this hypothesis is that it requires the ACC-ing clause to be not an adjunct, but an argument.
Note that the second argument of the denotation of \textit{see}$_{\Theta}$ is a predicate of events, which is satisfied by the ACC-ing clause.
However, by hypothesis (see \cref{sec:modifications}), an expression can only be an argument of a verb if it is merged with, rather than adjoined to the VP.
Therefore, this proposal requires the ACC-ing clause to be merged with the VP, rather than adjoined to it, as in \cref{fig:AdHocTree2}.
\begin{figure}[h]
	\centering
	\begin{forest}
		nice empty nodes,sn edges, baseline
		[{$\gamma$}
			[{$\beta$}
				[{$\alpha$}
					[{$v_{\Theta}$}]
					[{$\sqrt{\textsc{see}}$}]
				]
				[DP$_{i}$[the dog,roof]]
			]
			[$\zeta$
				[{$\langle\text{DP}_{i}\rangle$}]
				[$\delta$
					[Prog]
					[VoiceP[{$\langle\text{DP}_{i}\rangle$ running},roof]]
				]
			]
		]
	\end{forest}
	\caption{The unlabelled structure of \cref{ex:ACC-ing} required by the \textit{see}$_{\Theta}$ proposal}
	\label{fig:AdHocTree2}
\end{figure}
This structure, however, is untenable without significant additional stipulation for two reasons.
First, it is unlabellable.
Consider how we would label $\gamma$, which is a phrase-phrase structure.
The labelling algorithm would search $\gamma$ and find the least-embedded, still-visible, functional LIs.
In this case, $v_{\Theta}$ and Prog are equally embedded, and thus are returned by LA.
This can result in an unambiguous label only if we make one of two additional stipulations.
Either (\textit{i}) $v_{\Theta}$ and Prog agree for some feature, or (\textit{ii}) the ProgP $\zeta$ undergoes movement.
Stipulation (\textit{i}) requires at least two additional stipulations: a stipulation as to the identity of the agreeing features, and a stipulation that agreement can occur at such a structural distance.
I strongly suspect that these sub-stipulation will require stipulations of their own, and so on, but I stop here.
Stipulation (\textit{ii}), requires us to stipulate a landing site for the movement operation; one that ensures the proper word order.
This new landing site implies a new functional head, whose identity will also likely be stipulated.
As with stipulation (\textit{i}), this stipulation likely begets more stipulations.\footnote{
	A skeptical reader might point out that this is the very nature of scientific inquiry: Explaining one phenomenon invariably brings to light several phenomena in need of explanation.
	This view might suggest that one person's stipulation is another's hypothesis.
	I would argue, however, that these hypotheses do not raise new areas of inquiry, but rather allow us to avoid raising new areas of inquiry.
	Recall that the impetus for these stipulations was to avoid a vague unease with the semantic analysis in \cref{ex:VPProgPSEM} that followed naturally from the syntactic analysis assumed in this chapter.
	Such an unease, I believe, does not justify this level of stipulation.
}
The level of stipulation required in order to label the structure in \cref{fig:AdHocTree2}, I believe, renders it untenable.

Even if we were to assume that the structure in \cref{fig:AdHocTree2}, or a structure derived from it, is labellable, there is another reason that this analysis cannot be maintained.
Recall that in \cref{sec:ACCing} I showed that, when an ACC-ing clause is merged as an argument, its subject must remain in situ.
This predicts, not only that ungrammatical sentences with two DP objects like \cref{ex:DoubleObj} should be grammatical, but also that \cref{ex:ACC-ing} should be ungrammatical.
\ex.* We saw the dog the cat running.\label{ex:DoubleObj}

In view of the problems with this proposed analysis, I will set it aside.

This brings us to our final possibility: encoding the \textsc{theme2} on a specialized functional head $\Theta$.
Under this analysis, \cref{ex:ACC-ing} would have the structure in \cref{fig:AdHocTree3}.
\begin{figure}[h]
	\centering
	\begin{forest}
		nice empty nodes,sn edges, baseline
		[$\alpha$
			[$\Theta$P
				[$\Theta$]
				[{$v$P}[see the dog,roof]]
			]
			[ProgP[$t$ running,roof]]
		]
	\end{forest}
	\caption{The structure of \cref{ex:ACC-ing} with an independent $\Theta$ head}
	\label{fig:AdHocTree3}
\end{figure}

Assuming $\Theta$ takes $v$P as an argument, and ProgP is adjoined to the resulting $\Theta$P, lets consider a possible meaning for $\Theta$ in \cref{ex:ThetaSEM1}
\ex. \textbf{First approximation:}\\ \textsc{sem}$(\Theta) = \lambda P_{\langle s,t\rangle} \lambda e \exists e^{\prime}\left[ P(e)\,\&\,\textsc{theme2}(e)(e^{\prime}) \right]$\label{ex:ThetaSEM1}

This \textsc{sem} says that there is some event which is the secondary theme of the event described by the $v$P.
Now, when this combines with the $v$P, it yields the predicate of events in \cref{ex:ThetaPSEM1}.
\ex.\label{ex:ThetaPSEM1} \textsc{sem}$(\Theta\text{P}) = \lambda e \exists e^{\prime}\left[ \textbf{see}(e)\,\&\,\textsc{theme}(e)(\textbf{the\_dog})\,\&\,\textsc{theme2}(e)(e^{\prime})\right]$

If this is then conjoined with the ProgP, we get the predicate of events in \cref{ex:ThetaPSEM12}.
\ex.\label{ex:ThetaPSEM12} 
$
\begin{array}[t]{rcl}
	\textsc{sem}(\alpha) & = & \lambda e \exists e^{\prime}[ \textbf{see}(e)\,\&\,\textsc{theme}(e)(\textbf{the\_dog})\,\&\,\textsc{theme2}(e)(e^{\prime})\\
		& & \&\,e\subseteq e^{\prime} \,\&\, \textsc{doer}(e^{\prime})(\textbf{the\_dog}) \& \textbf{run}(e^{\prime})]
\end{array}
$

This interpretation has two flaws.
First, it asserts that the entire running event $e^{\prime}$ is the secondary theme of seeing.
This is far too strong of an assertion; \cref{ex:ACC-ing} is compatible with only seeing a sliver of the running event.
Second, it asserts that the seeing event $e$ is a subpart of the running event $e^{\prime}$, which is precisely the assertion we are trying to avoid.
These issues can be avoided by swapping the variables $e$ and $e^{\prime}$ in our denotation of $\Theta$ as in \cref{ex:ThetaSEM2}.
\ex. \textbf{Second approximation:}\\ \textsc{sem}$(\Theta) = \lambda P_{\langle s,t\rangle} \lambda e^{\prime} \exists e\left[ P(e)\,\&\,\textsc{theme2}(e)(e^{\prime}) \right]$\label{ex:ThetaSEM2}

The resulting denotation of the structure in \cref{fig:AdHocTree3} is given in \cref{ex:ThetaPSEM2}
\ex.\label{ex:ThetaPSEM2} 
$
\begin{array}[t]{rcl}
	\textsc{sem}(\alpha) & = & \lambda e^{\prime} \exists e [ \textbf{see}(e)\,\&\,\textsc{theme}(e)(\textbf{the\_dog})\,\&\,\textsc{theme2}(e)(e^{\prime})\\
		& & \&\,e\subseteq e^{\prime} \,\&\, \textsc{doer}(e^{\prime})(\textbf{the\_dog}) \& \textbf{run}(e^{\prime})]
\end{array}
$

This no longer has the flaws of the denotation in \cref{ex:ThetaPSEM12}, but it has introduced a new one.
The denotation in \cref{ex:ThetaPSEM2} is a predicate of events, but it no longer describes a seeing event, but a running event.
So, if we continued the derivation of \cref{ex:ACC-ing}, and introduce the external argument \textit{we}, then that argument will be construed as the experiencer/agent of the running event, not the seeing event.

The independent $\Theta$ head analysis, it seems, cannot save us, and therefore, the secondary theme analysis cannot save us.
Since our attempts to avoid the analysis in \cref{ex:VPProgPSEM} have only made things worse, I see no alternative to adopting our first analysis, despite the sense of unease it may evoke.
Note that we are tied to the analysis in \cref{ex:VPProgPSEM} only insofar as we adopt the admittedly simplistic semantic analysis of progressive aspect used here.
I leave the task of exploring the consequences of more precise semantic analyses of progressive aspect to others.

All of this is just to say that \cref{ex:VPProgPSEM} is a \textit{plausible} semantic analysis for adjunct ACC-ing VPs.
Whether or not it is the correct analysis is, of course, an empirical question, one which may require quantitative experimental methods to answer.
Although such an investigation is beyond the scope of this thesis, I can make the question more precise.
The question comes down to a contrast between how argument ACC-ing structures and adjunct ACC-ing structures are interpreted, respectively, and therefore asking the question will require us to disambiguate the two structures.
We've seen that, while active perception clauses, such as \cref{ex:ACC-ing} (repeated below), are ambiguous between the two readings, passive perception clauses, such as \cref{ex:ACC-ingPass} seem to only admit the adjunct ACC-ing reading.
\ex.[\cref{ex:ACC-ing}] We saw the dog running.

\ex.\label{ex:ACC-ingPass} The dog was seen running.

By hypothesis, then, passives like \cref{ex:ACC-ingPass} should only admit readings under which the perceived event ``contains'' the perception event, While actives like \cref{ex:ACC-ing} should also admit readings under which the perceived event does not contain the perception event.
The best way to do this is to construct active-passive pairs like \cref{ex:ACC-ing} and \cref{ex:ACC-ingPass} in which the passive, but not the active is judged to be contradictory.
More precisely, we need to construct these active-passive pairs such that the source of the would be contradiction is the hypothesized containment relation.

To this end, I propose the use of direct perception reports in which the perceived event is potentially imperceptible.
For instance, mental activities are potentially imperceptible, so there is no contradiction in \cref{ex:SawThinkingBut}.
\ex.\label{ex:SawThinkingBut} Koorosh$_{i}$ saw Sahar thinking but he$_{i}$ didn't realize it.

Similarly, there are a host of physical events that are imperceptible to the naked eye---events such as paint drying, water heating, or skyscrapers swaying.
For this reason, neither the sentences in \cref{ex:ActiveImperceptibly} are contradictory, nor are the situations they describe impossible.
\ex.\label{ex:ActiveImperceptibly}
\a. We saw the paint drying imperceptibly.
\b. She saw the water heating up imperceptibly.
\c. You saw the skyscraper swaying imperceptibly.

If my hypothesis is correct, though, the passive versions of \cref{ex:SawThinkingBut} and \cref{ex:ActiveImperceptibly} should entail that an imperceptible event contains its own perception---a clear contradiction.
Therefore, the sentences in \cref{ex:PassiveImpercept} should be contradictory.
\ex.\label{ex:PassiveImpercept} 
\a. Sahar was seen thinking by Koorosh, but he didn't realize it.
\b. The paint was seen drying imperceptibly.
\c. The water was seen heating up imperceptibly.
\d. The skyscraper was seen swaying imperceptibly.

My own judgments about these sentences are unclear and, of course, likely tainted by self-interest, which is why I believe they should be tested experimentally.
As of the writing of this thesis, however, I lack the time, resources, and know-how to perform such an experiment.
I therefore leave it as a task for future research.

\end{document}
