%        File: Coincide.tex
%     Created: Sat Apr 14 12:00 PM 2018 E
% Last Change: Sat Apr 14 12:00 PM 2018 E
%
% arara: pdflatex: {options: "-draftmode"}
% arara: biber
% arara: pdflatex: {options: "-draftmode"}
% arara: pdflatex: {options: "-file-line-error-style"}
\documentclass[MilwayThesis]{subfiles}
\begin{document}
In this thesis I have made liberal use of a novel class of sideward movement structure which I schematize below in \cref{fig:SidewardSchema}.
\begin{figure}[h]
	\centering
\[\sbox0{$\begin{array}[]{ccc}
		\begin{forest}
	    nice empty nodes,
	    sn edges,baseline,
	    for tree={
	    calign=fixed edge angles,
	    calign primary angle=-30,calign secondary angle=70}
	    [YP
		    [DP$_i$]
		    [
			    [Y]
			    [ZP]
		    ]
	    ]
	\end{forest}			
	&
	\tikz[baseline=10ex,scale=1] \node[inner sep=0] at (0,-1) {\large,\,};
	&
	\begin{forest}
	    nice empty nodes,
	    sn edges,baseline,
		for tree={
	    calign=fixed edge angles,
	    calign primary angle=-30,calign secondary angle=70}
	    [VP
		    [V]
		    [DP$_i$]
	    ]
	    \end{forest}
		\end{array}$}
\mathopen{\resizebox{1.2\width}{\ht0}{$\Bigg\langle$}}
\usebox{0}
\mathclose{\resizebox{1.2\width}{\ht0}{$\Bigg\rangle$}}
\]
	\caption{A schema of the sideward movement structure}
	\label{fig:SidewardSchema}
\end{figure}
This structure was used as an analysis of adjectival resultatives, depictives, and one available structure for direct perception reports with ACC-ing clauses.
The distinction between these constructions is due to the choice of head Y.
For resultatives, Y is instantiated by res, while for precetion reports, it is instatntiated by Prog.
As for depictives, the adjoined phrase YP is a small clause, so Y is either absent or instantiated by a Pred head.
Note that, while the complement of Y, ZP, also varies between the constructions, this variation can be derived from the selectional requirements of Y.

If we take VPs to be event descriptions, and we assume that host-adjunct structures to compose by predicate conjunction, then a VP adjunct (SC for depictives, resP for resultatives, and ProgP for DPRs) must also be event description.
That is, in \cref{fig:SidewardSchema} the VP and YP are each interpreted as a predicate of events, and, because they combine by predicate conjunction, they both describe the same event.

In the case of depictives, the interpretation is mostly straightforward, though not without come complications.
Consider \cref{ex:Depictive}, which an eventuality in which Natasha eats the fish while that fish is raw.
\ex. Natasha ate the fish raw.\label{ex:Depictive}

According to our analysis, this interpretation would be derived from the fact that the eating event and the rawness state are taken to to be identical.
That is the VP in \cref{ex:Depictive} has a logical form  as derived in \cref{ex:DepictiveLF}
\ex. \textsc{sem}(\textit{eat the fish raw}) =\\
$\lambda e [\textsc{sem}(\textit{eat the fish})(e) \, \&\, \textsc{sem}(\textit{the fish raw})(e)] =$\\
$\lambda e [\textbf{eating}(e)\, \&\, \textbf{raw}(e)\, \&\, \textsc{Theme}(\textbf{the\_fish})(e)]$\label{ex:DepictiveLF}

One might object that this LF is incoherent, as eating is an event, while rawness is a state, and an eventuality cannot be both an event and a state.
This objection, however, does not hold up under scrutiny.
Suppose we take the externalist perspective, according to which, the entities that natural language expressions are predicated of are mind-external and -independant entities, and the predicates and concepts of natural language correspond to natural kinds.
From this perspective, eventualities are regions of space-time, some of which are events, while others are states.
So, for instance to utter \cref{ex:EventDesc} truthfully is to refer to a particular region of space-time.
\ex.\label{ex:EventDesc} The officer ticketed the car.

Now, according to the objection at hand, the region of space-time referred to by \cref{ex:EventDesc} is an event, and, therefore, not a state.
However, it is entirely reasonable to assume we could truthfully utter \cref{ex:StateDesc}, a state description, referring to the same space-time region.
\ex.\label{ex:StateDesc} The car was parked illegally.

It seems, then, that, if there is an event/state contrast, it does not originate in the extramental world, or else \cref{ex:EventDesc} and \cref{ex:StateDesc} could not possibly refer to the same region of space-time.

If the event/state contrast is not drawn from nature, then perhaps it comes from the mind.
<++>

The case of adjunct ACC-ing clauses is slightly more challenging.
\end{document}
