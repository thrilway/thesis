%        File: Coincide.tex
%     Created: Sat Apr 14 12:00 PM 2018 E
% Last Change: Sat Apr 14 12:00 PM 2018 E
%
% arara: pdflatex: {options: "-draftmode"}
% arara: biber
% arara: pdflatex: {options: "-draftmode"}
% arara: pdflatex: {options: "-file-line-error-style"}
\documentclass[MilwayThesis]{subfiles}
\begin{document}
In this thesis I have made liberal use of a novel class of sideward movement structure which I schematize below in \cref{fig:SidewardSchema}.
\begin{figure}[h]
	\centering
\[\sbox0{$\begin{array}[]{ccc}
		\begin{forest}
	    nice empty nodes,
	    sn edges,baseline,
	    for tree={
	    calign=fixed edge angles,
	    calign primary angle=-30,calign secondary angle=70}
	    [YP
		    [DP$_i$]
		    [
			    [Y]
			    [ZP]
		    ]
	    ]
	\end{forest}			
	&
	\tikz[baseline=10ex,scale=1] \node[inner sep=0] at (0,-1) {\large,\,};
	&
	\begin{forest}
	    nice empty nodes,
	    sn edges,baseline,
		for tree={
	    calign=fixed edge angles,
	    calign primary angle=-30,calign secondary angle=70}
	    [VP
		    [V]
		    [DP$_i$]
	    ]
	    \end{forest}
		\end{array}$}
\mathopen{\resizebox{1.2\width}{\ht0}{$\Bigg\langle$}}
\usebox{0}
\mathclose{\resizebox{1.2\width}{\ht0}{$\Bigg\rangle$}}
\]
	\caption{A schema of the sideward movement structure}
	\label{fig:SidewardSchema}
\end{figure}
This structure was used as an analysis of adjectival resultatives, depictives, and one available structure for direct perception reports with ACC-ing clauses.
The distinction between these constructions is due to the choice of head Y.
For resultatives, Y is instantiated by res, while for precetion reports, it is instatntiated by Prog.
As for depictives, the adjoined phrase YP is a small clause, so Y is either absent or instantiated by a Pred head.
Note that, while the complement of Y, ZP, also varies between the constructions, this variation can be derived from the selectional requirements of Y.
\end{document}


