%        File: Coincide.tex
%     Created: Sat Apr 14 12:00 PM 2018 E
% Last Change: Sat Apr 14 12:00 PM 2018 E
%
% arara: pdflatex: {options: "-draftmode"}
% arara: biber
% arara: pdflatex: {options: "-draftmode"}
% arara: pdflatex: {options: "-file-line-error-style"}
\documentclass[MilwayThesis]{subfiles}
\begin{document}
In this thesis I have made liberal use of a novel class of sideward movement structures which I schematize below in \cref{fig:SidewardSchema}.
\begin{figure}[h]
	\centering
\[\sbox0{$\begin{array}[]{ccc}
		\begin{forest}
	    nice empty nodes,
	    sn edges,baseline,
	    for tree={
	    calign=fixed edge angles,
	    calign primary angle=-30,calign secondary angle=70}
	    [YP
		    [DP$_i$]
		    [
			    [Y]
			    [ZP]
		    ]
	    ]
	\end{forest}			
	&
	\tikz[baseline=10ex,scale=1] \node[inner sep=0] at (0,-1) {\large,\,};
	&
	\begin{forest}
	    nice empty nodes,
	    sn edges,baseline,
		for tree={
	    calign=fixed edge angles,
	    calign primary angle=-30,calign secondary angle=70}
	    [VP
		    [V]
		    [DP$_i$]
	    ]
	    \end{forest}
		\end{array}$}
\mathopen{\resizebox{1.2\width}{\ht0}{$\Bigg\langle$}}
\usebox{0}
\mathclose{\resizebox{1.2\width}{\ht0}{$\Bigg\rangle$}}
\]
	\caption{A schema of the sideward movement structure}
	\label{fig:SidewardSchema}
\end{figure}
This structure was used as an analysis of adjectival resultatives and depictives, and for one available structure for direct perception reports with ACC-ing clauses.
The distinction between these constructions is due to the choice of head Y.
For resultatives, Y is instantiated by res, while for perception reports, it is instantiated by Prog.
As for depictives, the adjoined phrase YP is a small clause, so Y is either absent or instantiated by a Pred head.
Note that, while the complement of Y, ZP, also varies from construction to construction, this variation can be derived from the selectional requirements of Y.

If we take VPs to be event descriptions, and we assume that host-adjunct structures to compose by predicate conjunction, then a VP adjunct (SC for depictives, resP for resultatives, and ProgP for DPRs) must also be event description.
That is, in \cref{fig:SidewardSchema} the VP and YP are both interpreted as predicates of events, and, because they combine by predicate conjunction, they both describe the same event.
In this chapter, I discuss the interpretation of depictives and adjunct ACC-ing clauses.
As for the interpretation of resultatives, the discussion in \cref{sec:ResInterp} remains sufficient.

In the case of depictives, the interpretation is mostly straightforward, though not without some complications.
Consider \cref{ex:Depictive}, which describes an eventuality in which Natasha eats the fish while that fish is raw.
\ex. Natasha ate the fish raw.\label{ex:Depictive}

According to our analysis, this interpretation would be derived from the fact that the eating event and the rawness state are taken to to be identical.
That is the VP in \cref{ex:Depictive} has a logical form  as derived in \cref{ex:DepictiveLF}
\ex. \textsc{sem}(\textit{eat the fish raw}) =\\
$\lambda e [\textsc{sem}(\textit{eat the fish})(e) \, \&\, \textsc{sem}(\textit{the fish raw})(e)] =$\\
$\lambda e [\textbf{eating}(e)\, \&\, \textbf{raw}(e)\, \&\, \textsc{Theme}(\textbf{the\_fish})(e)]$\label{ex:DepictiveLF}

One might object that this LF is incoherent, as eating is an event, while rawness is a state, and an eventuality cannot be both an event and a state.
This objection, however, does not hold up under scrutiny.
Suppose we take the externalist perspective, according to which the entities that natural language expressions are predicated of are mind-external and -independent entities, and the predicates and concepts of natural language correspond to natural kinds.
From this perspective, eventualities are regions of space-time, some of which are events, while others are states.
So, for instance to utter \cref{ex:EventDesc} truthfully is to refer to a particular region of space-time.
\ex.\label{ex:EventDesc} The officer ticketed the car.

Now, according to the objection at hand, the region of space-time referred to by \cref{ex:EventDesc} is an event, and, therefore, not a state.
However, it is entirely reasonable to assume we could truthfully utter \cref{ex:StateDesc}, a state description, referring to the same space-time region.
\ex.\label{ex:StateDesc} The car was parked illegally.

It seems, then, that, if there is an event/state contrast, it does not originate in the extra-mental world, or else \cref{ex:EventDesc} and \cref{ex:StateDesc} could not possibly refer to the same region of space-time.

Furthermore, many adverbs describe states, yet may modify event descriptions.
If adverbs are adjuncts, then they are interpreted as conjoined with their host, meaning that they will provide a partial description of an event, rather than a state.
Therefore, there doesn't seem to be any contradiction in my analysis of depictives.

The case of adjunct ACC-ing clauses is slightly more challenging, due to a subtlety in their meanings which I will discuss below.
Ultimately, however, their meaning can be explained from their structure, given in \cref{fig:ACCingPair}, and an assumption about the nature of eventualities.
\begin{figure}[h]
	\centering
\[\sbox0{$\begin{array}[]{ccc}
		\begin{forest}
	    nice empty nodes,
	    sn edges,baseline,
	    for tree={
	    calign=fixed edge angles,
	    calign primary angle=-30,calign secondary angle=70}
	    [ProgP
		    [DP$_i$]
		    [
			    [Prog]
			    [VoiceP[DP$_i$ run,roof]]
		    ]
	    ]
	\end{forest}			
	&
	\tikz[baseline=10ex,scale=1] \node[inner sep=0] at (0,-1) {\large,\,};
	&
	\begin{forest}
	    nice empty nodes,
	    sn edges,baseline,
		for tree={
	    calign=fixed edge angles,
	    calign primary angle=-30,calign secondary angle=70}
	    [VP
		    [saw]
		    [DP$_i$[the dog,roof]]
	    ]
	    \end{forest}
		\end{array}$}
\mathopen{\resizebox{1.2\width}{\ht0}{$\Bigg\langle$}}
\usebox{0}
\mathclose{\resizebox{1.2\width}{\ht0}{$\Bigg\rangle$}}
\]
	\caption{An adjunct ACC-ing structure}
	\label{fig:ACCingPair}
\end{figure}
As a first pass, we can say that the interpretation of the structure in \cref{fig:ACCingPair} is a description of an event of the dog being seen and an event of the dog running.
This alone is not sufficient, as an English speaker's intuition regarding the \cref{ex:ACC-ing} is that the $we$ referent saw both the dog and the event of the dog running.
\ex. We saw the dog running.\label{ex:ACC-ing}

This intuition seems to be inescapable; English speakers cannot seem to entertain an interpretation of \cref{ex:ACC-ing} in which \textit{we} saw the dog but not the running event, or the event but not the dog.
This suggests that both complement ACC-ing and adjunct ACC-ing versions of \cref{ex:ACC-ing} are interpreted as both the individual and the event being seen.
The strong version of UTAH that I assume in this thesis, however, predicts that the interpretation \textit{x was seen} can only be encoded if the expression denoting $x$ is merged as the complement of the verb \textit{see}.
In adjunct ACC-ing analysis of \cref{ex:ACC-ing}, as shown in \cref{fig:ACCingPair}, the event denoting expression, ProgP, is adjoined to VP, yet we interpret it as meaning that the event was seen.
Since this interpretation is not directly encoded, we must infer it from what is directly encoded.

To see how we would infer the perception of the event, consider what is directly encoded.
First, the VP is interpreted as a description of a seeing event which the dog is the theme of.
\ex.\label{ex:VPSEM} \textsc{sem}(VP) = $\lambda e [\textbf{see}(e)\,\&\,\textsc{theme}(\textbf{the\_dog})(e)]$

The interpretation of the ProgP, represented in \cref{ex:ProgPStruct}, however, is more complicated.
\ex.\label{ex:ProgPStruct} [$\langle$the dog$\rangle$, [Prog [$_\text{VoiceP} \langle\text{the dog}\rangle$ run]]]

I will make the simplifying assumption that the copy of \textit{the dog} in [Spec, Prog] is semantically vacuous\footnote{
	At this stage, this is purely stipulative.
	I suspect some version of this assumption is true, but a full investigation and justification of it is beyond the scope of this thesis.
} and discuss the Prog-VoiceP structure.
The VoiceP is unremarkable, so I assume its meaning is the complete but tenseless event description in \cref{ex:VoicePSEM}.
\ex.\label{ex:VoicePSEM} \textsc{sem}(VoiceP) = $\lambda e [\textbf{run}(e)\,\&\,\textsc{doer}(e)(\textbf{the\_dog})]$

Prog, then, takes this description as an argument, and ascribes progressive aspect to it.
The standard, though perhaps na\"ive analysis of progressive aspect (as proposed in \cite{klein1994time}) is that Prog takes a description of event $e$ as an argument, introduces a topic time $t$, and asserts that $t$ is included in the run-time of $e$.
This predicts that ProgP encodes the predicate of times in \cref{ex:ProgPSEM1}
\ex.\label{ex:ProgPSEM1} \textsc{SEM}(ProgP) =  $\lambda t \exists e [t \subseteq \textsc{time}(e)\,\&\,\textbf{run}(e)\,\&\,\textsc{doer}(e)(\textbf{the\_dog})]$ (first pass)

However, since ProgP adjoins to VP and the resulting structure is interpreted as a conjunction, ProgP must be interpreted as a predicate of events.
Therefore, I will modify the semantic analysis of Prog such that it introduces an event $e^{\prime}$ and asserts that $e^{\prime}$ is included in $e$.
The final interpretation of the ProgP, is given in \cref{ex:ProgPSEM2} \parencite[cf.][]{bjorkman2018poster}.
\ex.\label{ex:ProgPSEM2} \textsc{sem}(ProgP) =  $\lambda e^{\prime} \exists e [e^{\prime} \subseteq e\,\&\,\textbf{run}(e)\,\&\,\textsc{doer}(e)(\textbf{the\_dog})]$ (second pass)

This interpretation will properly compose with \cref{ex:VPSEM} to yield the interpretation of the host-adjunct structure in \cref{ex:VPProgPSEM}
\ex.\label{ex:VPProgPSEM} \textsc{sem}($\langle$ProgP, VP$\rangle$) = $\lambda e^{\prime} \exists e [ \textbf{see}(e^{\prime})\,\&\,\textsc{theme}(\textbf{the\_dog})(e^{\prime})\,\&\,e^{\prime} \subseteq e\,\&\,\textbf{run}(e)\,\&\,\textsc{doer}(e)(\textbf{the\_dog})]$

So, the event of seeing the dog is included in the event of the dog running, meaning that the seeing occurred in the same space-time region as the running and therefore.
This denotation along with the very nature of seeing and running allows us to infer from \cref{ex:VPProgPSEM} that we saw the running event.

One could argue that this analysis is implausible as it requires that the seeing event is a part of the seemingly independent running event.
On its face, this seems to imply an interdependecy between the two events, and, while it seems reasonable to say that the perception event depends on the perceived event, it is far from obvious that the perceived event depends on the perception event.
This line of argumentation, I believe, confuses the issue at hand.

Saying that the perception of an event is a sub-part of that event, does imply that the event is dependent on it being perceived, but it does so in a very weak way.
It is perhaps a truism of set theory and mereology to say that two complex objects are identical only if they consist of the same parts.
So, if $x$ is a part of $e$ but not a part of $e^{\prime}$, the $e \neq e^{\prime}$.
Similarly, a particular running event which is seen by some individual $x$, cannot be identical to a running event which is not seen by $x$.
Note that this does not mean that the unseen running event is not a running event, only that it is a not a seen running event.

\end{document}
