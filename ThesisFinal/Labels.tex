% arara: pdflatex: {options: "-draftmode"}
% arara: biber
% arara: pdflatex: {options: "-draftmode"}
% arara: pdflatex: {options: "-file-line-error-style"}
\documentclass[MilwayThesis]{subfiles}
\begin{document}
In this chapter I will discuss the most recent iteration of Chomsky's syntactic theory, which he refers to as \textit{label theory}.
In the first section I will discuss the background and content of Chomsky's (\citeyear{chomsky2013problems,chomsky2015problems}) proposal.
In the second section I will draw out two questions that Chomsky's label theory leaves unanswered and hypothesize an answer for each.
This modified label theory will then be used to explain the resultative parameter in the following chapter.
\section{Label theory and its motivations}
Chomsky begins his proposal of label theory with a discussion of the minimalist program in general.
In his estimation, the goal of the minimalist program has been to explain the universal properties of language as simply as possible.
The properties he identifies are (i) the structure-dependence of rules, (ii) displacement, (iii) linear order and (iv) projection/labelling.
He then argues that if we assume linear order is a reflex of transfer to the SM interface, properties (i) and (ii) can be explained by assuming that Narrow Syntax is only simplest Merge, as defined in \Next.
\ex. Merge($\alpha$, $\beta$) = $\left\{ \alpha, \beta \right\}$

Unlike previous versions of Merge, however, simplest Merge does not include labelling.
Chomsky argues that this is a welcome outcome, because labelling/projection is not as detectible in surface forms as the other properties of language, and has always been a theory internal notion.
What's more, Chomsky argues, previous theories that bundle labelling with structure building have always stipulated labelling rather than deriving it.
So, for instance, the phrase \textit{see the girl} is stipulated to be a VP rather than a DP.

Chomsky proposes that labels are assigned post-syntactically by a special instance of minimal search called the Labelling Algorithm (LA).
LA operates iteratively in a top-down manner, searching each syntactic object for a ``most prominent'' element which can serve as the label.
In the simplest case, an atomic element (a head X) merged with a complex object (a phrase YP), the atomic element is found to be the most prominent element and, as such, is the label of the object.
\ex. LA$(\left\{ X, YP \right\}) = X$


The head-phrase case in \Last is trivial due to the inherent asymmetry in the structure.
Labelling becomes more complicated when symmetric structures are considered, that is, when head-head and phrase-phrase structures are considered.
To discover how these structures could be labelled, Chomsky considers examples of head-head and phrase-phrase structures that are generated by grammars, and hypothesizes why they are generated and not other instances.
The only head-head structures that surface are those that result from the merger of acategorial roots and category-determining heads.
So, structures like \Next[a] are labellable, but those like \Next[b] and \Next[c] are not.
\ex.
\a. $\left\{ n, \sqrt{\textsc{water}}\right\}$
\b.* $\left\{ n, v\right\}$
\c.* $\left\{ \sqrt{\textsc{ice}}, \sqrt{\textsc{water}} \right\}$

Chomsky proposes that roots are completely featureless and, therefore, invisible to LA, meaning \Last[a] recieves the label $n$.

As for phrase-phrase structures, Chomsky identifies two types that are able to surface.
The first type are what I will call phrase-trace structures.
These are phrase-phrase structure in which one of the constituent phrases is a lower copy.
Following \textcite{moro2000dynamic}, Chomsky proposes that lower copies are invisible to LA, meaning the labeling of a phrase-trace structure is as shown in \Next.
\ex. LA$(\left\{ XP, \langle YP\rangle \right\}) =$ LA$(XP)$\\
(Angle brackets here indicate that YP is a lower copy.)

The second type of phrase-phrase structure that can surface are what I will call agreement structures.
These are phrase-phrase structures in which the two constituent phrases agree with one another for some feature.
In these cases, the agreeing features serve as the label of the structure as in \Next.
\ex. LA$(\left\{ XP_F, YP_F \right\}) = \langle F,F\rangle$

If either of these two situations do not obtain for a given phrase-phrase structure, it will be unlabelable and result in a crash at the CI interface.
To see how this works, consider the raising construction in \Next and the ungrammatical version of it in \NNext.
\ex.\label{ex:raising}
\a. The dishes seem to be dirty.
\b. [$_\alpha$ The dishes [ seem [$_\beta \langle\text{the dishes}\rangle$, [ to be dirty]]]]

\ex.\label{ex:noraising}
\a.* It seems the dishes to be dirty
\b. [$_\gamma$ It [ seem [$_\delta$ the dishes, [ to be dirty]]]]

The sentence in \LLast has two relevant phrase-phrase structures which are labelable.
The first is a trace-phrase structure, given in \Next[a], which is labeled by the infinitive \textit{to}, as demonstrated in \Next[b].
\ex.
\a.  $\beta = \left\{ \langle\text{the dishes}\rangle, \left\{ \text{to}, \text{be dirty} \right\} \right\}$
\b. LA$(\beta) = \text{LA}(\left\{ \text{to}, \text{be dirty} \right\}) =$ to

The second is the agreement structure, given in \Next[b], which is labeled by the agreeing $\varphi$ features as in \Next[b].
\ex.
\a. $\alpha = \left\{ \text{the}_\varphi \text{ dishes} \left\{ T_\varphi \text{, seem} \left\{ \ldots \right\} \right\} \right\}$
\b. LA$(\alpha) = \langle\varphi, \varphi\rangle$

The derivation of \ref{ex:noraising}, however, crashes because the phrase-phrase structure, $\delta$ is unlabelable.
The DP \textit{the dishes} has not raised, so it is visible to LA in $\delta$, and there is no $\varphi$-agreement between \textit{the}$_\varphi$ and \textit{to}$_\emptyset$.
\ex.
\a. $\delta = \left\{ \text{the}_\varphi \text{ dishes} \left\{ \text{to}_\emptyset \left\{ \ldots \right\}\right\} \right\}$
\b. LA$(\delta) = $ Undefined


Also, Chomsky proposes that heads that bear only a partial set of features (\textit{e.g.} English finite T$_\varphi$) cannot label unless they agree for those features with some other head.
This is in contrast to heads that bear full feature sets (\textit{e.g.} Italian T${\langle\varphi,\varphi\rangle}$) or lack these features altogether (\textit{e.g.} English non-finite T$_\emptyset$) which are able to label without agreement.

At this point I should note an issue that arises when giving label-based explanations of syntactic derivations.
In previous theories, movement operations were described as being ``driven'' by some need.
For example, in Government and Binding theories DPs undergo movement in order to get abstract Case.
In minimalist theories, this has been generalized such that all movement is driven by the need to satisfy some feature.
This led debates around the mechanism for driving movement. 
Greed-based accounts, for instance, argue that an object moves if and only if such a move will satisfy one of its own features, while those that assume Enlightened Self Interest take the weaker stance that an object moves if and only if such a move will satisfy some feature on some argument.\footnote{
	See \textcite{lasnik1999last} for a discussion of these two types of accounts.
}
While these accounts were all based on an interface-based theory, they allow syntacticians to explain (un)grammaticality purely in terms of narrow syntax.

Label theory, however, assumes that all operations are free, that is, they don't require a trigger or a driver.
This means, however, that an explanation of why an operation occurs or does not occur in a given derivation is slightly more complicated in label theory.
The well-formedness of a structure is assessed at the interface, which means entire phases are assessed at once.
Consider, for instance, the successive \textit{wh}-movement in \Next, and how the two types of theories would account for it.
\ex. Who$_i$ does Mary say $t_i$ that Laura likes $t_i$?

The accounts of the final movement step ([Spec, C] to [Spec, C]) would be similar, as both theories assume that the highest C needs to agree with a \textit{wh}-word, either for labelling or feature-satisfaction.
The explanations of the first movement ([Comp, V] to [Spec, C]) however, are different.
If movement operations must be driven, then we would likely need to posit a feature on the lower C which must be satisfied by a \textit{wh}-word.
With free movement, however, the \textit{wh}-word must move to the lower [Spec, C] because, if it doesn't, it cannot move to the higher [Spec, C] without violating Subjacency.
Assuming a phase-based theory of subjacency, the first movement operation in \Last follows from the proposal that C is a phase head.

To summarize, a syntactic derivation in label theory proceeds as follows.
Structures are built by iteratively applying Merge (along with Select and Copy) to syntactic objects.
At certain points a portion of a structure (\textit{i.e.}, a phase) is transferred to the interfaces.
At the CI interface, the labelling algorithm labels the transferred structure and all of the structures contained within the transferred structure.
If the labelling algorithm fails to label any part of the transferred structure, the derivation crashes.
A summary of the labelling algorithm is given below in \Next.
\begin{table}
	\centering
	\begin{tabular}[t]{llp{5cm}}
		\textbf{SO} & \textbf{LA(SO)} & \\
		\cline{1-2}
		$X$ & $X$ & ($X$ is not a root, and does not have an incomplete feature set)\\
		$\left\{ X, R \right\}$ & LA$(X)$ & ($R$ is a root, $X$ is not a root)\\
		$\left\{ X, YP \right\}$ & LA$(X)$ & \\
		$\left\{ \langle XP\rangle, YP \right\}$ & LA$(YP)$ & \\
		$\left\{ XP_F, YP_F \right\}$ & $\langle F,F\rangle$ & ($XP$ and $YP$ agree for $F$)\\
		Otherwise & Undefined &
	\end{tabular}
	\caption{A summary of the labelling algorithm}
	\label{tab:LA-results}
\end{table}

\textcite{chomsky2015problems} demonstrates that label theory has some empirical advantage over previous theories of syntax, but leaves at least two questions unanswered.
The first question is why labels would be required by the CI interface at all, and the second question is how are host-adjunct structures labelled.
In the next section I will propose answers to those questions, before considering in the following section the ramifications label theory has for the architecture of the grammar.

\section{Summary}
In this chapter I reviewed Chomsky's (\citeyear{chomsky2013problems,chomsky2015problems}) label theory which says that labels are assigned algorithmically at the CI interface and are required for proper interpretation at that interface.
I identified two questions left open in Chomsky's proposal, which I then suggest answers to.
The first question, how would host-adjunct structures be labelled, was answered by the hypothesis that host-adjunct structures are ignored by the labelling algorithm and are null-labelled.
The second question, why labels are required at the CI interface, has answered with the hypothesis that the label of a structure determines how it composes.
In the next chapter, I will use this modified label theory to demonstrate that the (un)availability of resultative can be derived from the (un)availability of bare stem compounding.
\end{document}
