%        File: Milway Thesis.tex
%     Created: Mon Jan 02 03:00 PM 2017 E
% Last Change: Mon Jan 02 03:00 PM 2017 E
%
% arara: pdflatex
% arara: biber
% arara: pdflatex
% arara: pdflatex
\documentclass[
%	draft,
	12pt,
	twoside,
	narrowmargins
	]{ut-thesis}

%\usepackage[cmintegrals,cmbraces]{newtxmath}
%\usepackage{ebgaramond-maths}
\usepackage[T1]{fontenc}

%\usepackage[margin=1in]{geometry}
\usepackage[british]{babel}
\usepackage[]{csquotes}
\usepackage[
	backend=biber,
	style=authoryear,
	citestyle=authoryear-comp,
	language=british,
	dashed=true
]{biblatex}
\DeclareLanguageMapping{british}{british-apa}
\DeclareLabeldate{%
	\field{year}
  \field{date}
  \field{eventdate}
  \field{origdate}
  \field{urldate}
  \field{pubstate}
  \literal{nodate}
}
\renewbibmacro*{addendum+pubstate}{%
  \printfield{addendum}%
  \iffieldequalstr{labeldatesource}{pubstate}{}
  {\newunit\newblock\printfield{pubstate}}}
\usepackage[]{graphicx}

\usepackage[]{tipa}
\usepackage{stmaryrd}
\usepackage[]{amsmath}
\usepackage{amsfonts}
\usepackage{amssymb}
\usepackage{amsthm,thmtools}
\usepackage{mathtools}


\usepackage{nth}
\usepackage{combelow}

%\usepackage{marginnote}

\usepackage{enumitem}

\theoremstyle{definition}
\newtheorem{defn}{Definition}[chapter]

\usepackage{tikz}
\usepackage{forest}
\usepackage[]{longtable}
\usepackage{tabu}
\usepackage{subcaption}
\usepackage{caption}
\newcommand{\repeatcaption}[2]{%
	\renewcommand{\thefigure}{\ref{#1}}
	\captionsetup{list=no}
	\caption{#2 (repeated from page \pageref{#1})}
	\addtocounter{figure}{-1}
}
\usepackage[final,hidelinks]{hyperref}
\usepackage{linguex}
\makeatletter
\apptocmd{\gl@stop}{\nobreak}{}{}
\makeatother
\renewcommand{\Exarabic}{\thechapter.\arabic} 
% We want to reset the ExNo counter at each section
\usepackage{chngcntr}
\counterwithin{ExNo}{chapter}
\usepackage{cleveref}
\crefname{ExNo}{}{}
\creflabelformat{ExNo}{(#2#1#3)}
\crefname{SubExNo}{}{}

\crefname{figure}{figure}{figures}
\Crefname{figure}{Figure}{Figures}

\crefname{defn}{definition}{definitions}
\Crefname{defn}{Definition}{Definitions}

\usepackage{centernot}
%\usepackage{subfiles}
\usepackage{multirow}
\usepackage[normalem]{ulem}
\usepackage{xcolor}
%\listfiles

\usetikzlibrary{positioning,arrows}
\useforestlibrary{linguistics}

\usepackage{epigraph}

%\DeclareNameFormat{labelname:poss}{% Based on labelname from biblatex.def
%  \ifcase\value{uniquename}%
%  \usebibmacro{name:last}{#1}{#3}{#5}{#7}%
%  \or
%  \ifuseprefix
%  {\usebibmacro{name:first-last}{#1}{#4}{#5}{#8}}
%  {\usebibmacro{name:first-last}{#1}{#4}{#6}{#8}}%
%  \or
%  \usebibmacro{name:first-last}{#1}{#3}{#5}{#7}%
%  \fi
%  \usebibmacro{name:andothers}%
%  \ifnumequal{\value{listcount}}{\value{liststop}}{'s}{}
%}
%
%\DeclareFieldFormat{shorthand:poss}{%
%  \ifnameundef{labelname}{#1's}{#1}
%}
%
%\DeclareFieldFormat{citetitle:poss}{\mkbibemph{#1}'s}
%
%\DeclareFieldFormat{label:poss}{#1's}
%
%\newrobustcmd*{\posscitealias}{%
%  \AtNextCite{%
%    \DeclareNameAlias{labelname}{labelname:poss}%
%    \DeclareFieldAlias{shorthand}{shorthand:poss}%
%    \DeclareFieldAlias{citetitle}{citetitle:poss}%
%    \DeclareFieldAlias{label}{label:poss}
%  }
%}
%
%\newrobustcmd*{\posscite}{%
%  \posscitealias%
%  \textcite
%}
%
%\newrobustcmd*{\Posscite}{\bibsentence\posscite}
%
%\newrobustcmd*{\posscites}{%
%  \posscitealias%
%  \textcites

\usepackage[multiple]{footmisc}

\newcommand\quelle[1]{{%
  \unskip\nobreak\hfil\penalty50
  \hskip2em\hbox{}\nobreak\hfil#1%
  \parfillskip=0pt \finalhyphendemerits=0 \par
}
}

\newcommand{\AREx}{%
  \a.[(\textbf{AR})]\label{ex:AREx} Natalie hammered the metal flat.
  \z.
}
\newcommand{\rcommentg}[1]{\hfill\raisebox{1.2\baselineskip}[0pt][0pt]{#1}}
\makeatother
\renewcommand{\thmtformatoptarg}[1]{ -- #1}
\renewcommand{\listtheoremname}{List of definitions}
\setcounter{secnumdepth}{3}
\frenchspacing
\graphicspath{ {./img/} }
\addbibresource{../Thesis.bib}
\degree{Doctor of Philosophy}
\department{Linguistics}
\gradyear{2019}
\author{Daniel A. Milway}
\title{Explaining the Resultative Parameter}
\begin{document}
\begin{preliminary}
  \maketitle
  
%
% arara: pdflatex: {options: "-draftmode"}
% arara: biber
% arara: pdflatex: {options: "-draftmode"}
% arara: pdflatex: {options: "-file-line-error-style"}
\documentclass[MilwayThesis]{subfiles}

\begin{document}
\begin{abstract}
	This thesis proposes an explanatory account for the fact that some, but not all languages generate adjectival resultatives.
\end{abstract}
\end{document}

  \begin{acknowledgements}
	  This document was a long time in the making, and it certainly couldn't have happened without a supervisor with the generosity and intellectual dexterity of Elizabeth Cowper.
	  Elizabeth, who was already retired by the time I asked her to be my supervisor, never seemed to be fazed by supervising or advising on projects that were outside her wheelhouse, even though those projects were decidedly incompatible with each other.
	  But, my history with her goes farther back than the start of my thesis work, to LIN331 where she introduced me to the minimalist program, and later, in that time between undergrad and the MA when she met with me just to talk about syntax.
	  Between the course and those meetings, I'm quite certain I wouldn't be where I am without her.

	  I am also indebted to Diane Massam and Michela Ippolito, who, with Elizabeth, formed my thesis committee. 
	  My meetings with them helped me to ground my research and avoid tunnel-vision, and more generally helped keep me honest.
	  I also want to thank the rest of my defense panel: Susana B\'{e}jar, Nick LaCara, and Norbert Hornstein, for making my defense lively and actually fun.

	  Beyond my thesis committee I am indebted to the entire Linguistics Department at UofT.
	  Thank you to the faculty in general and the syntax and semantics faculty in particular for creating a model of a collegial atmosphere to nurture research and researchers.
	  Thank you to my fellow grad students, who were the best support network a young researcher could ask for.
	  And last but certainly not least, thank you to the administrative staffers who made the department run: Mary Hsu, Jill Given-King, Jennifer McCallum, and Deem Waham.

	  Research, of course, is only one aspect of graduate school; there's also teaching, TA-ing, and grading to fill our days.
	  There are too many to mention by name here, but I want to thank all the people I TA-ed for, and with, and those people who TA-ed for me.
	  I also must acknowledge my union, CUPE 3902, which tirelessly worked to protect and advocate for me and my fellow workers.
	  I came into grad school skeptical of unions in general but that skepticism evaporated in 2015 when the collective bargaining process and ensuing four week strike revealed the true colours of both the union and the university administration.
	  The union revealed itself to be supportive, open, and protective of all its members, while the administration revealed itself to be almost the opposite.
	  It definitely stung to see the administrators of the university that I had spent the better part of a decade at (undergrad and grad school) outright lie about me and my fellow workers, and try to pit us against each other, undergrads, and faculty, but that sting was lessened by the sense of solidarity I felt on the picket lines with my fellow members of that university.
	  The strike was a difficult time but it did remind me that, as is the case with any organization, the university is defined by its membership, not its administration.

	  When I was dithering over whether or not to pursue a PhD, wondering if it was worth it, Derek Denis, then just a grad student, pointed out one of the great side-benefits of the PhD: real lasting friendships.
	  Perhaps it's too early to say, but I think he was right, I've found a good number of people that I feel grateful to call friends.
	  Not least of all, I'm grateful to have had so many wonderful cohort-mates: Becky Tollan, Clarissa Forbes, Tomohiro Yokoyama, Jess Denniss, Emily Blamire, Michelle Yuan, and Jada Fung.
	  I'm grateful to have met them and to have shared the ups and downs of graduate school with them.
	  Thank you also to the many other wonderful people I got to know as fellow grad students from Joanna Chociej, Sarah Clarke, Richard Compton, Ail\'is Cournane, Radu Craioveanu, Derek Denis, Liisa Duncan, Ross Godfrey, Yu-Leng Lin, Mercedeh Mohaghegh, Alex Motut, Kenji Oda, Will Oxford, Christopher Spahr, Eugenia Suh, and others, who were the towering senior grad students who helped me to figure things out as I got started, to Julien Carrier, Emily Clare, Julie Doner, Erin Hall, Shayna Gardiner, Ruth Maddeaux, Emilia Melara, Patrick Murphy, and Na-Young Ryu, who were junior grad students figuring things out alongside me, to all the soon-to-be senior grad students, some of whom I was lucky enough to befriend, all of whom I'm proud to have shared a department with.
	  I hope that the line of graduate students continues on, being there for each other in rough times, celebrating achievements together, and just blowing off steam at pub night.

	  Thank you to Dawn Whitwell for teaching me how to write jokes and giving me a creative outlet other than linguistics. Thank you to my friends Shawna Edward, Lisa Feingold, and Dan Reilly (aka ``The Bit Acknowledgers'').
	  Thank you to the many people who put out the podcasts that informed me, made me laugh and generally helped me get through the PhD with my sanity:
	  Scott Aukerman, Adam Scott, Paul F. Tompkins, Dave Shumka, Graham Clark, Sean Clements, Hayes Davenport, John Hodgman, Jesse Thorn, Greg Kot, Jim DeRogatis, Howard Kremer, Kulap Vilaysack, Tim Batt, Guy Montgomery, Mike Mitchell, Nick Wiger, Jimmy Pardo, Matt Belknap, Garon Cockrell, Eliot Hochberg, Leo Laporte, Jeff Jarvis, Stacey Higginbotham, Gina Trapani, Julia Prescott, Allie Goertz, Lauren Lapkus, Russ Roberts, Jeremy Scahill, Jessica St Clair, Lennon Parham, and Andy Daly.


	  I'd like to thank my siblings and their families:
	  my brother Tom, his wife Karen, and their children Declan, Rose, Violet, and Sadie;
	  my brother Mike, his wife Brigid, and their children Imogen and Desmond;
	  my sister Joan and her partner Jared;
	  my brother Peter.


	  I'd like to thank Zoe, who came into my life only recently, but whose love and support has made this milestone all the sweeter.

	  Finally, Thank you to my parents Jim and Sheila.
	  I wouldn't be where I am now, I wouldn't have achieved anything worthwhile, I wouldn't be the person I am today were it not for their love and support. 
  \end{acknowledgements}
  \tableofcontents
  \listoftables
  \listoffigures
  \listoftheorems[ignoreall,show={defn}]
\end{preliminary}
\chapter{Introduction}
%        File: intro.tex
%     Created: Thu Jun 22 04:00 PM 2017 E
% Last Change: Thu Jun 22 04:00 PM 2017 E
%
% arara: pdflatex: {options: "-draftmode"}
% arara: biber
% arara: pdflatex: {options: "-draftmode"}
% arara: pdflatex: {options: "-file-line-error-style"}
\documentclass[MilwayThesis]{subfiles}

\usepackage{atveryend}

\BeforeClearDocument{
	\printbibliography
}
\addbibresource{../Thesis.bib}
\begin{document}
What follows would likely be placed by contemporary linguists under the broad umbrella of theoretical syntax for what seem to me to be purely negative reasons.
This thesis does not contain novel data gathered from fieldwork, laboratory techniques, or corpora, and, since it does not fit into any of those categories, it is considered ``theoretical.''
This negative definition of theoretical linguistics, however, does not properly characterize the variety of work that it subsumes.
\textcite{chametzky1996theory} argues that three distinct types of work are often subsumed under the name \textit{theoretical linguistics}: metatheoretical work, heoretical work, and analytical work, which he defines as follows.
\begin{quotation}
	\textit{Metatheoretical} work is theory of theory, and divides into two sorts: general and (domain) specific. 
	General metatheoretical work is concerned with developing and investigating adequacy conditions for any theory in any domain\dots
	Specific metatheoretical work is concerned with adequacy conditions for theory in a particular domain\dots

	\textit{Theoretical} work is concerned with developing and investigating primitives, derived concepts and architecture within a particular domain of inquiry.
	This work will also deploy and test concepts developed in metatheoretical work against the results of actual theory construction in a domain, allowing for both evaluating of the domain theory and sharpening of the metatheoretical concepts.
	Note this well: \textbf{deployment} of metatheoretical concepts is \textit{not} metatheoretical work; it is theoretical work.

	\textit{Analytic} work is concerned with investigating the (phenomena of the) domain in question.
	It deploys and tests concepts and architecture developed in theoretical work, allowing for both understanding of the domain and sharpening of the theoretical concepts.
	Note this well: \textbf{deployment} of theoretical concepts is \textit{not} theoretical work, it is analytic work.
	\quelle{(xvii--xviii)}
\end{quotation}
I adopt this trichotomy along with Chametzky's caveat that no actual work in linguistics fits neatly into any one of these categories.
Metatheory and theory bleed into each other as do theory and analysis.
I add a further caveat that, since Chametzky's definitions are all relative to a domain the the phenomena of that domain, the application of those definitions will depend on what we construe to be our empirical domain.
<+MoreHere+>

Generative theories of linguistics are and have always been computational theories, meaning the mental procedures they hypothesize are local and syntactic.
Since the terms \textit{computational}, \textit{local}, and \textit{syntactic}, are all quite ambiguous and none seem to have retained their original primary interpretations, it is worth defining them.
\ex. A procedure $P$ is local/syntactic/computational iff\\
for every object $X$ in the domain of $P$,\\
the value of $P(X)$ depends solely on properties of $X$.

This means that certain aspects of language will not be amenable to a generative theory.
For instance, many aspects of language use are very context-dependent and therefore are likely to be excluded from the domain of generative theories of linguistics.
\end{document}




\part{Explaining Resultatives}\label{sec:part1}
%\fcolorbox{black}{lightgray}{
%  \begin{minipage}[t]{0.9\textwidth}
%	  \textbf{Note:} I will write a fuller introduction as the body of the thesis becomes more finalized. 
%	  In place of that, here are a few bullet points on what my this thesis is about:
%	  \begin{itemize}
%		  \item Languages vary wrt whether they allow adjectival resultatives.
%		  \item No one has come up with a satisfactory account of resultatives and their variable nature
%		  \item I am proposing an account of resultatives that includes:
%			  \begin{itemize}
%				  \item an analysis of their structure,
%				  \item an analysis of the parametric variation, and
%				  \item an argument that these two accounts are consistent with each other and that the parametric variation is acquirable.
%			  \end{itemize}
%		  \item Essentially I am trying to account for \Next and \NNext
%	  \end{itemize}
%	  \exg. Die Teekanne leer trinken.\\
%the teapot empty drink.\\
%``to drink the teapot dry'' \parencite[German;][]{kratzer2004building}
%
%\exg.* Dario je ofarbao kucu crveno\\
%Dario \textsc{cop} painted house red.\textsc{fem}\\
%``Dario painted the house red.'' (Serbo-Croatian; Mia Sara Misic P.C.)
%
%  \end{minipage}
%}
\chapter{Non-standard Theoretical Assumptions}\label{sec:nonstandard}
% arara: pdflatex: {options: "-draftmode"}
% arara: biber
% arara: pdflatex: {options: "-draftmode"}
% arara: pdflatex: {options: "-file-line-error-style"}
\documentclass[MilwayThesis]{subfiles}
%\setcounter{chapter}{2}
\begin{document}

This dissertation rests on a number of non-standard theoretical assumptions and draws a few non-standard distinctions, which I will defend in this chapter.
My defense of the assumptions, however, will not be an argument that they are true, as the truth of any theoretical statement ultimately depends on the empirical facts.
Rather, my defense will actually be an offense; I will argue that the standard assumption is, in fact, ill-founded.
So, in a sense, I will be rejecting standard assumptions rather than making non-standard ones.
The distinctions I draw, in contrast, will not be defended, but rather explained and clarified.

\section{The $\Theta$-Criterion}
The $\theta$-criterion standardly assumed was first formulated by Chomsky in \textit{Lectures in Government and Binding} (LGB) as \Next.
\ex. Each argument bears one and only one $\theta$-role, and each $\theta$-role is assigned to one and only one argument. \parencite[36]{chomsky1981lectures}

In a footnote, Chomsky justifies this criterion, saying 
\begin{quote}
	The second clause of [the $\theta$-criterion] is well-motivated.
	To say that each $\theta$-role must be filled implies, for example, that a pure transitive verb such as \textit{hit} must have an object, that a verb such as \textit{put} or \textit{keep} (with the sense they have in \textit{put it in the corner}, \textit{keep it in the garage}) must have the associated PP slot filled, etc. 
	The additional requirement that each $\theta$-role must be filled by only one argument will, for example, exclude the possibility that a single trace is associated with several argument antecedents, a possibility ruled out in principle under the Move-$\alpha$ theory. 
	\parencite[139]{chomsky1981lectures}
\end{quote}
I would agree that the second clause of \Last, that each $\theta$-role is assigned to a single argument, is well-motivated by the empirical considerations Chomsky cites, and as such I will not reject that portion of the $\theta$-criterion.
The first clause, however, is motivated mainly by theoretical concerns of LGB, that is, its connection to empirical facts is indirect at best.

The nature of the LGB theory is such that its various hypotheses and principles are connected to each other in a web-like network.
As a result, the first clause of the $\theta$-criterion depends on various other theoretical statements and various other theoretical statements depend on it.
So, rather than attempting an exhaustive enumeration of the links between the $\theta$-criterion and the other theoretical statements of LGB, I will present what I consider to be the best argument in favour or the $\theta$-criterion, and argue that its premises have since been rejected within syntactic theory.

The first premise is the now familiar Y- or T-model of grammar shown in \autoref{fig:LGBYModel}, which LGB theory continues from earlier theories.
According to this model, a syntactic derivation has four levels of representation (D-structure, S-Structure, PF, and LF), and each step in the derivation is performed by the application of a subset of the transformational rules.
S-structures are derived by the applying of Move-$\alpha$ to D-structures, LFs are derived by applying QR (and maybe Move-$\alpha$) to S-structures, and PFs are derived by applying ``stylistic rules'' to S-structures \parencite[18]{chomsky1981lectures}.
\begin{figure}[h]
	\centering
\begin{tikzpicture}[baseline]
	\node[draw,rounded corners] (DS) at (0,0) {D-Structure};
	\node[draw,rounded corners] (SS) at (0,-2) {S-Structure};
	\node (PF) at (-2,-3) {PF};
	\node (LF) at (2,-3) {LF};
	\draw[->] (DS)--(SS) node[midway,right] {Move-$\alpha$};
	\draw[->] (SS)--(PF) node[midway,anchor=south east] {Stylistic rules};
	\draw[->] (SS)--(LF) node[midway,anchor=south west] {QR};
\end{tikzpicture}
	\caption{The architecture of the grammar in LGB}
	\label{fig:LGBYModel}
\end{figure}
The exact natures of the transformations are not important for this discussion.
What is important, is that all syntactic displacement is the result of one of these transformations.

The second premise is the projection principle, which states that lexical properties must be represented at all levels of syntax.
Since $\theta$-roles are lexical properties of (at least) verbs, they must be represented at all levels of syntax.
Consider the verb \textit{hit}, whose lexical entry specifies that it needs a patient argument.
The projection principle requires that at D-Structure and S-Structure \textit{hit} must have assigned a patient $\theta$-role to an argument and therefore, assuming the patient $\theta$-role is assigned to Comp,V, there must be a DP in the complement position of \textit{hit} at both D-Structure and S-Structure.

With these assumptions, it follows that no single argument can receive more than one $\theta$-role.
Suppose there is a derivation in which a single argument X receives two $\theta$-roles $\Theta1$ and $\Theta2$.
According to the projection principle, X must be marked with both $\Theta1$ and $\Theta2$ at D-Structure.
Since each $\theta$-role is associated with a unique structural position, it follows that X must be in two distinct positions at D-Structure.
The only way an argument can be in multiple positions is if it has undergone Move-$\alpha$.
Move-$\alpha$, however, maps D-Structures to S-Structures.
Therefore, an argument cannot be in two positions at D-Structure, and furthermore, cannot be multiply $\theta$-marked at D-Structure.
If an argument cannot be multiply $\theta$-marked at D-Structure, then it cannot be multiply $\theta$-marked at all.

Thus we are able to derive the first clause of the $\theta$-criterion from other principles.
These principles, however, have either been rejected or problematized since their statement in LGB.
Since \textit{The Minimalist Program} \parencite{chomsky1995minimalist}, generative theories have largely dispensed with D-Structure and S-Structure.
Without these levels of representation, the projection principle (as formulated in LGB) is effectively meaningless, and without the projection principle, there is no more basis for the first clause of the $\theta$-criterion.
Therefore, I will not be assuming the first clause of the $\theta$-criterion.

\section{Last Resort}
It almost goes without saying that the goal of the minimalist program is to clean up syntactic theory by explaining unnecessary principles in terms of necessary principles.
In nearly one fell swoop, \textcite{chomsky1995minimalist} eliminates a number of the complications built into LGB theory (S-Structure and D-Structure chief among them).
In \textit{The Minimalist Program} (MP), however, Chomsky was not able to explain two seeming imperfections: displacement and uninterpretable features.\footnote{
	The framework developed in MP also does not explain projection/labelling, but Chomsky does not recognize this as an imperfection until ``Problems of projection'' \parencite{chomsky2013problems}.
	More on this in Chapter \ref{sec:labels}
}
Chomsky formalized displacement as the operation Move, which is more complex than Merge, and proposes that, unlike Merge, which ``comes for free,'' every instance of Move must be triggered by the the need to satisfy an uninterpretable feature.
The simple operation Merge is preferred for economy reasons, while movement, Chomsky argues, is a ``last resort.''
In work subsequent to MP, however, Chomsky proposes that Move is actually a subtype of Merge, called Internal Merge.
Merge and Move are different names for the same operation, there is no reason to think that Move is more computationally complex, and therefore no reason to think that movement is a last resort.

In ``Beyond Explanatory Adequacy'' \parencite[][henceforth, \textit{BEA}]{chomsky2004beyond} Chomsky makes this rejection of Last Resort explicit:
\begin{quote}
	[Narrow Syntax] is based on the free operation Merge.
	[The Strong Minimalist Thesis] entails that Merge of $\alpha$, $\beta$ is unconstrained, therefore either external or internal.
	Under external Merge, $\alpha$ and $\beta$ are separate objects; 
	under internal Merge, one is part of the other, and Merge yields the property of ``displacement,'' which is ubiquitous in language and must be captured in some manner in any theory. 
	It is hard to think of a simpler approach than allowing	internal Merge (a grammatical transformation), an operation that is freely available.
	\parencite[110]{chomsky2004beyond}
\end{quote}
This line of reasoning bears discussion in light of the fact that Last Resort is still standardly assumed by self-described minimalist syntacticians.
In fact, several syntacticians expand Last Resort to External Merge arguing that neither type of Merge ``comes free'' \parencite{pesetsky2006probes,frampton2008crash,wurmbrand2014merge,yokoyama2015features}.
This proposal is understandable from a historical perspective, but ultimately misguided in my opinion.

Perhaps the most attractive aspect of a constrained Merge syntax is the purported gains in computational efficiency.
To illustrate this, \textcite{frampton2008crash} consider the incomplete product of a doomed derivation in \Next
\ex. it to be believed Max to be happy

A free Merge syntax, according to \textcite{frampton2008crash}, the derivation must continue for an indefinite time until a phase head is merged.
At this point the derivation will crash due to a Case Filter violation.
A constrained Merge syntax, however, could be able to halt as soon as the derivation becomes doomed, say, when \textit{it} is merged.
This would save us the indefinite number steps it takes to merge a phase head, and is therefore more efficient.
This I take to be a species of the argument that free Merge systems are inefficient because, in addition to the infinite array of convergent derivations they must generate, they also generate an infinite array of crashing derivations, whereas constrained Merge systems only generate to ``convergent'' derivations.
However, when we investigate the nature of constrained Merge, we can see that the purported gains in efficiency in one part of the system come at the expense of another part of that same system.
Constrained Merge theories, in effect, rob Peter to pay Paul.

At minimum, each version of the syntax will have a Merge operation and a Transfer operation.
So, the free Merge syntax will consist of a unconstrained Merge$_F$ and a constrained Transfer$_F$ as defined in \Next.
\ex.
\a. Merge$_F(\alpha,\beta) = \left\{ \alpha, \beta \right\}$ 
\b. Transfer$_F(\gamma) = \langle\textsc{sem}(\gamma), \textsc{phon}(\gamma)\rangle$ iff Filter($\gamma$) = F\\
(where $\alpha$, $\beta$, and $\gamma$ are syntactic objects.)

A constrained Merge syntax, then, would be consist of a constrained Merge$_C$ and an unconstrained Transfer$_C$.
\ex.
\a. Merge$_C(\alpha,\beta) =$ Merge$_F(\alpha,\beta)$ iff Satisfy($\alpha,\beta$) = T
\b. Transfer$_C(\gamma) = \langle\textsc{sem}(\gamma), \textsc{phon}(\gamma)\rangle$
(where $\alpha$, $\beta$, and $\gamma$ are syntactic objects.)

If we assume that both theories can be made descriptively adequate, then the Satisfy predicate will have the same net effect of the Filter predicate required for the free Merge system.
So, given any pair of syntactic objects $\alpha$ and $\beta$, Satisfy must be able to evaluate if Merge($\alpha,\beta$) is allowed.
And since there are an infinite amount of deriveable syntactic objects, even in a constrained Merge syntax, Satisfy must be able to evaluate an infinity of possible $\beta$'s against each possible $\alpha$.
For any given syntactic object, then, there is an indefinite amount of objects which will merge with that object and an indefinite amount that will not, and the only way to know if Satisfy is true of a pair of objects is to chack.
So, much like free Merge syntax suffers from an infinity of crashes, constrained Merge syntax suffers from an infinity of failed Merge operations.

A constrained Merge theorist, might still object by saying that Satisfy is a local operation, while Filter is a global operation, and local operations are to be preferred if we care about computational complexity.
Again, this is an intuitively attractive argument, but not obviously valid.
Consider the following thought experiment.
Suppose you are a TA charged with grading a quiz by your somewhat maniacal instructor.
Part of the instructor's mania is that they require all quizzes to consist of 40 equally weighted questions and be graded out of 10 points.
What is the most efficient procedure for assigning a grade to each quiz?
Two types of procedure suggest themselves.
The first, which I will call the Local Only procedure, is to assign each correct answer a value of 0.25 points and then add up all of the points.
The second, which I will call the Local/Global procedure, it to assign each correct answer a value of 1 point, add up all of the points, and divide by 4 (perhaps with a calculator).
Since humans are Very Good at counting by increments of 1, and calculators are Very Good at dividing by 4, while neither is Very Good at counting by increments of 0.25, the Local/Global procedure is likely to be more efficient than the Local Only procedure.
The moral of this story: One machine's global procedure is another machine's local procedure.

There is also a methodological rationale for preferring a free Merge framework which is slightly counterintuitive, so I would like to dwell on it for a moment.
My reason for assuming a free Merge framework is that it creates, or rather, lays bare, more problems for us to solve.
So, why is this preferrable?
Shouldn't we prefer the theory with fewer problems?
Inuitively, we should prefer the less problematic theory, but this all depends on how we count a theory's problems.
I would like to argue that, while free Merge theories pose a greater number of problems than constrained Merge theories, the sheer weight of the problems posed by each type of theory is equal to that of the other.
Furthermore, the problems of free Merge can be made into empirical questions more readily than those of constrained Merge.

In order to argue in favour of free Merge, I will present one argument against it and show how that argument actually strengthens my claims in the previous paragraph.
The argument comes from \textcite{frampton2008crash}, and states that a free Merge theory of grammar must posit ``filters'' to rule out the non-converging structures that its syntax generates, and the last thing we want is a flourishing of filters.
I could not agree more with their assessment, but where they see a bug, I see a feature.
So-called ``filters'' are not attempts at explanation, but descriptions of generalizations in need of explanation.

Take, for instance, the remaining clause of the $\theta$-criterion, given in \Next.
\ex. [E]ach $\theta$-role is assigned to one and only one argument. \parencite[36]{chomsky1981lectures}

To propose a $\theta$-filter, then would be to say that those derivations which violate \Last crash at an interface, presumably the CI interface.
For a theorist, this filter is actually a question or series of questions: Why is it that only those derived structures which satisfy \Last are valid CI objects?
That question may not be empirical, but it invites hypotheses which may lead to empirical questions which we don't currently know how to ask.
Very likely, due to the interface-nature of the questions, their answers will not be narrowly linguistic.

Now, consider the situation constrained Merge puts the theorist in.
For \textcite{frampton2008crash}, the $\theta$-criterion is expressed by ``selectional features'' on heads which must be satisfied immediately.
This leads to a number of questions: What is the nature of these selectional features?
How are they related to, say, $\varphi$-features?
Do they exist independently of the narrow syntax?
Why do they need to be satisfied first?
And so on.
I, for one, don't have the slightest clue how to proceed in answering or even sharpening these questions, and there don't seem to be any clues in the offing from constrained Merge theorists.
\section{Long Distance Agree}
Strongly associated with, but distinct from Last Resort is the notion of an Agree operation.
\textcite{chomsky2000minimalist} introduces Agree as <+Quote+>, and virtually every syntactician who identifies as a minimalist assumes that such an operation is part of the syntactic computation.
In fact there is an ongoing debate as to the exact nature of Agree <+CITE+>, but within there is a broad consensus with respect to the general properties of Agree.
<+MoreHere+>

Within the group of minimalist syntacticians, there is a somewhat disparate minority that rejects the orthodoxy of Long Distance Agree for a number of reasons.
In the remainder of this section I will discuss the main theoretical\footnote{
	\textcite{bobaljik2008wheres,arregi2013contextual} present empirical arguments <++> 
}<++> argument against a narrow syntactic notion of Agree, which is that this notion of Agree introduces a redundancy into the language faculty which violates the scientific principle of parsimony.
I will also introduce a novel argument against Agree as a syntactic operation, based on the computational complexity of feature valuation.


\section{Terminological notes}
The subject matter of this thesis is often called the ``syntax-semantics interface,'' a term which I have discovered is ambiguous, probably due to the fact that it is constructed from three ambiguous terms.

The term \textit{syntax} has (at least) three senses which seem to be used in generative grammar circles.
The first sense, which I will call the sociological sense, is that \textit{syntax} is what syntacticians do.
For instance, $\theta$-theory belongs to the domain of syntax under this sense, because syntacticians care about it, while semanticists tend not to.
However, $\theta$-theory deals at least partially with meaning, so it, at least, intersects with semantics.
This sense would be useful if this dissertation were an intellectual history of generative syntax, but since this is a work of syntactic theory, I will not use this sense.

The second sense, which I will call the broad sense, is that \textit{syntax} is the study (or description) of the form and arrangement of symbolic representations.
Under this sense, the study of syntax would be a part of the study of logic, programming languages, arithmetic, etc.
Furthermore,\textcite[174]{chomsky2000new}, discussing this sense, argues that most of what we call \textit{semantics} and \textit{phonology} would be classified as syntax under this sense.\footnote{Based on my discussion of this sense with phonologists and semanticists, this may be the most controversial claim Chomsky has ever made.}
This sense will prove useful in this dissertation, so I will retain it.

The third sense, which I will call the narrow sense, is that \textit{syntax} is a mental module characterized by a computational procedure that generates an unbounded array of structured form\footnote{I use the term \textit{form} here to refer to all possible expressive modalities of language}-meaning pairs.
This, I believe, is what generative syntacticians mean when they use the term \textit{syntax}.
The ``syntax module'' is one of the objects of study of this thesis, so I will retain this sense.

Since both the broad and narrow senses are useful to me, I will need to make a distinction for the sake of clarity.
I will use the term ``Narrow Syntax'' (or NS) to refer to the narrow sense, that is, the hypothesized mental module, and ``syntax'' (and derived terms) to indicate the broad sense.

Similar remarks apply to the term \textit{semantics}, which has at least three senses.
The first sense, as in the case of \textit{syntax}, is the sociological sense: semantics is what semanticists do.
I will not be using this sense for the same reasons as I cited above for the sociological sense of \textit{syntax}.

The broad sense of \textit{semantics} is that of the study of the relation of a symbolic system to some other system.
So, to various degrees, we can talk about the semantics of a logical system, a programming language, a natural language, etc.
I will use this sense only informally, when discussing notions of truth and reference associated with an instance or class of natural language expression.

The narrow sense of \textit{semantics} is that of the mental module (or system of modules) associated with computing the meaning of a linguistic expression.
Chomsky often refers to this mental entity as the Conceptual-Intentional (CI) system, and stresses that we know very little about it.
Insofar as this thesis makes claims or hypotheses about \textit{semantics}, it makes claims or hypotheses about the CI system.

The final ambiguous term I will discuss is \textit{interface}.
In recent years it has become common within generative linguistics to write papers, hold workshops, and compile books on \textit{the syntax-semantics interface}, but as I mentioned above, the term is ambiguous.
It largely seems to be ambiguous between a \textit{sociological} sense and a \textit{narrow} sense, with the sociological sense dominating discussion.

When used in the sociological sense, the syntax-semantics interface refers to a body of literature that mixes the formalisms and methods used by syntacticians with those used by semanticists.
That is, this type of work makes use of tree diagrams and expressions of typed lambda calculus.
In this sense, the interface is not an object of study per se, but a sub-discipline.

In the narrow sense, the syntax-semantics interface refers to the interface between the Narrow Syntax and the CI system.
This results in very different sorts of analyses compared to the standard analyses, analyses that posit computational procedures rather than merely representing expressions in two ways.
In many ways, however, the term \textit{interface} in its narrow sense is a misnomer, as there are likely no mental objects that we might call interfaces.
As best we can tell, the mind consists of a set of modules and a non-modular central system \parencite{fodor1983modularity,fodor2001mind}.
An interface, then, emerges wherever two modules interact with each other, or perhaps where a module interacts with the central system.
Restricting ourselves to the modules, we can see why positing interfaces as mental objects won't do.
Suppose we have two modules, M1 and M2, which seem to interact with each other.
Being modules, each will consist in a set of computational operations (P1 and P2) defined over a class of syntactically structured (in the broad sense) objects (L1 and L2).
Suppose we posit an interface I1, which consists in an operation P3 that converts objects of L1 into objects of L2.
What is I1, then, but a module that has interfaces with M1 and M2?
If I1 is a module, are its interfaces with M1 and M2 also modules?
If so, then we seem to be stuck with an infite regress.
If not, then interfaces are a special kind of module, but this would raise further questions with respect to their evolutionary origins.

If there are no mental objects that we might call interfaces, then how are we to study them?
The answer to this question is that, to study an interface, we must study the modules associated with that interface with the added assumption that such an interface exists.
So, studying the syntax-semantics interface involves studying the Narrow Syntax and the CI module with the assumption that there is an interface between them.
We will get a glimpse of how such a study would work in chapter \ref{sec:labels}.

\section{Summary}
In this chapter, I have made explicit two of my assumptions which would be considered non-standard among contemporary generative syntacticians.
In particular, I am not assuming the $\theta$-criterion as it is commonly stated, and I am assuming a free Merge syntax.
I have also clarified some terminology that many take for granted, specifically I clarified my use of the term \textit{syntax-semantics interface} and its constituent terms.
Now that the reader has a sense of my theoretical idiosyncrasies, we can move on to more specific concerns in the following chapters.
\end{document}

\chapter{Previous Literature}\label{sec:litreview}
\epigraph{
  There's always a siren\\
  Singing you to shipwreck.\\
  Steer away from these rocks.
}{``There, There''\\\textsc{Radiohead}}
%        File: LitReview.tex
%     Created: Thu Oct 05 09:00 AM 2017 E
% Last Change: Thu Oct 05 09:00 AM 2017 E
%
% arara: pdflatex: {options: "-draftmode"}
% arara: biber
% arara: pdflatex: {options: "-draftmode"}
% arara: pdflatex: {options: "-file-line-error-style"}
\documentclass[MilwayThesis]{subfiles}
\begin{document}
In this section I will review several previous analyses of adjectival resultatives and the parametric variation associated with them.
I will evaluate the anlyses against two desiderata.
First, I will evaluate whether the variation, as analyzed, is learnable.
Second, I will evaluate whether the analysis comports with the theoretical principles of the minimalist program.
Before reviewing the analyses, however, I will make these desiderata explicit and justify them.

\section{Desiderata for an analysis of resultatives}

\subsection{Desideratum 1: Learnability}
Most of the analyses of the structure of adjectival resultatives are packaged with an account of the associated parametric variation.
While few address directly address the acquisition of parametric variation, any analysis of variation makes implicit claims about acquistion.
Generally, when discussing parametric variation, the claims about acquisition can justifiably be left implict, as the acquisition task is trivial.
The nature of adjectival resultatives, however, is such that we must make those acquisition claims explicit.
To explain why, I will be comparing the resultative parameter to the V-to-T parameter.

Analyses of the V-to-T parameter do not need to address learnability because the ``parameter setting'' is directly learnable from the primary linguistic data.
It is directly learnable because the overt forms of (\textit{e.g.}) polar questions differs depending on the parameter setting.
In a language with V-to-T movement such as German, the language learner will observe that lexical verbs undergo inversion for polar questions, while in a language without V-to-T movement such as English, the learner will observe that lexical verbs do not invert for questions.
\ex. V-to-T movement (German)
\ag. Trinken sie Kaffee?\\
Drink.3plPres they Coffee\\
``Do they drink coffee?''
\b.* \textsc{do} sie Kaffee trinken?

\ex. *V-to-T Movement (English)
\a.* Drink they coffee?
\b. Do they drink coffee?

The form of polar questions, then, can be positive evidence for a parameter setting.
Since we can find direct positive evidence for a parameter setting, the task of the analyst/theoretician, then is merely to formalize the parameter is a way that is consistent with the broader theory.
\textcite{chomsky1995minimalist}, for instance, formalizes the V-to-T parameter in terms of feature strength, while \textcite{lasnik1999verbal} formalizes it in terms of the presence/absence of inflectional features on lexical verbs.
Neither, however, needs to explicitly describe how their parameter is set, but can assume that a certain setting is the default, and the other setting can be deduced from (\textit{e.g.}) the form of polar questions in the language.

Resultatives, on the other hand, are not directly learnable for two reasons
The first reason is that, on the surface, resultatives, which are parameterized, are indistinguishable from depictives, which appear to be universal.
The two construction types are indistinguishable in the sense that both correspond to the string template in \Next (modulo independent word order variation).
\ex. \textsc{Subj} V \textsc{Obj} Adj.

This indistinguishability is evident in the fact that one can construct examples which are truly ambiguous between resultative and depictive readings, as in \Next.
\ex. 
\a. He fried the fish dry.
\a. $\approx$ He fried the fish once it was dry. (\textbf{Depictive})
\b. $\approx$ He fried the fish until it was dry. (\textbf{Resultative})
\z.
\b. She painted the barn red.
\a. $\approx$ The barn is red in her painting. (\textbf{Depictive})
\b. $\approx$ She applied a coat of red paint to the door. (\textbf{Resultative})
\z.

Assuming a child acquiring either French or English encounters sentences with the form of \LLast in their PLD, there is no obvious way for the child to determine whether a given secondary predicate is to be interpreted depictively or resultatively.

An empiricist might object, arguing that the ambiguous examples above are highly constructed, and would easily by disambiguated in context.
They would insist that the learner would infer a positive setting of the resultative parameter from the use of a secondary predication construction in the presence of a resultative event.
So, an English learner, but not a French learner, might be exposed to the context-sentence pairing in \Next.
\ex. 
\a.[\textbf{Context:} ] A woman is methodically hammering a lump of metal.
A parent draws their child's attention to the hammering event and utters:
\b.[\textbf{A:}] She's hammering the metal flat.

Even this, however, is not fully unambiguous.
While it certainly couldn't be interpreted as a depictive, \textit{flat} could be interpreted as a manner adverb, modifying \textit{hammering}.
Such unavoidable ambiguity would make it difficult to employ any sort of semantic bootstrapping in the acquisition of the resultative parameter.

The second problem comes from the fact that both English-type and French-type languages can express resultative semantics periphrastically as in \Next.
\ex. periphrastic resultatives
\ag. Elle a aplati le m\'etal en martelant\\
She has flattened the metal in hammering\\
\b. She flattened the metal by hammering.


The resultative parameter, then, is a one-or-both parameter, unlike other parameters which are either/or choices.
The V-to-T parameter, in contrast, is an either/or choice which can be made on the basis of the PLD.

\end{document}



\chapter{The Structure of Resultatives}\label{sec:analysis}
\epigraph{Die Welt ist eine Glocke, die einen Ri\ss{} hat: sie klappert, aber klingt nicht.}{\textsc{Johann Wolfgang von Goethe}}
%        File: MyAnalysis.tex
%     Created: Mon Nov 06 10:00 AM 2017 E
% Last Change: Mon Nov 06 10:00 AM 2017 E
%
% arara: pdflatex: {options: "-draftmode"}
% arara: biber
% arara: pdflatex: {options: "-draftmode"}
% arara: pdflatex: {options: "-file-line-error-style"}
\documentclass[MilwayThesis]{subfiles}
\begin{document}
In the previous chapter, I discussed the failings of several previous analyses of adjectival resultatives.
In this chapter I will discuss two of those analyses---one structural and one parametric---and show how they can be modified to address the concerns raised in the previous chapter.
The structural analysis, that of \textcite{kratzer2004building}, was rejected as it stood because it did not comply with UTAH, but it has three features which I will retain: a small clause structure, theme raising, and a result head.
The parametric analysis, that of \textcite{snyder1995language,snyder2012parameter}, was rejected because it did not comply with the Lexical Parameterization Hypothesis, but it was based on a learnable pattern and so I will be adopting a modified version of it.
\section{Fixing the UTAH problem}
The one issue with Kratzer's analysis is that it seems to violate UTAH.
That is, there is a single $\theta$-relation between \textit{hammer} and \textit{the metal} in both sentences in \Next, that does not correspond to a single structural relation.
\ex.
\a. Joe hammered the metal flat.
\b. Joe hammered the metal.

According to Kratzer's analysis, \textit{the metal} is the specifier of \textit{hammer} in \Last[a], but a standard analysis of \Last[b] will place \textit{the metal} as the complement of \textit{hammer}
\begin{figure}[h]
	\centering
	\begin{subfigure}[b]{.55\textwidth}
	\begin{forest}
	    nice empty nodes,sn edges,baseline,for tree={
	    calign=fixed edge angles,
	    calign primary angle=-30,calign secondary angle=70}
	    [VP
		    [DP[the metal,roof,name=specV]]
		    [
			    [hammer]
			    [resP
				    [res]
				    [SC
					    [$\langle$DP$\rangle$,name=SCDP]
					    [flat]
				    ]
			    ]
		    ]
	    ]
	    \draw[->] (SCDP) to[out=south west, in=south] (specV);
	\end{forest}
	\caption{\textit{hammer the metal flat}}
	\end{subfigure}
	~
	\begin{subfigure}[b]{0.4\textwidth}
		\centering
	\begin{forest}
	    nice empty nodes,sn edges,baseline,for tree={
	    calign=fixed edge angles,
	    calign primary angle=-30,calign secondary angle=70}
	    [VP
		    [hammer]
		    [DP[the metal,roof]]
	    ]
	\end{forest}
	\vspace{6ex}
	\caption{\textit{hammer the metal}}
	\end{subfigure}
	\caption{Kratzer's (2004) analysis of resultatives}
	\label{fig:KratzerTree}
\end{figure}
If we were to modify Kratzer's analysis, so that \textit{the metal} is the complement of \textit{hammer}, then we would need to attach the result phrase in a different position.
I propose that the result phrase is adjoined to the VP, as shown in \cref{fig:hammer-flat}, allowing the DP to merge directly with the verb.
\begin{figure}[h]
	\centering
	{\small
	\begin{forest}
	    nice empty nodes,sn edges,baseline,
	    for tree={
	    calign=fixed edge angles,
	    calign primary angle=-35,calign secondary angle=60}
	    [VP
		    [VP
			    [hammer]
			    [DP[the metal,roof,name=compV]]
		    ]
		    [resP
			    [$\langle$DP$\rangle$,name=specRes]
			    [res$^{\prime}$
				    [res]
				    [SC
					    [$\langle$DP$\rangle$,name=SCDP]
					    [flat]
				    ]
			    ]
		    ]
	    ]
	    \draw[->] (SCDP) to[out=south west, in=south] (specRes);
	    \draw[->] (specRes) to[out=south, in=south] (compV);
	\end{forest}
	}
	\caption{The Structure of resultatives}
	\label{fig:hammer-flat}
\end{figure}
The modified analysis no longer violates UTAH, but it introduces two new issues.
First, the movement operation between [Spec, res] and [Comp, V] does not target a c-commanding position.
In other words, it is a sideward rather than an upward movement.
Second, resP and VP are adjoined, meaning that they compose by conjunction.
This is counterintuitive, however, since resultatives are inherently asymmetric, with the verb event causing the adjective state.
I will address each of these in turn below.
\subsection{Sideward movement}
In \cref{fig:hammer-flat}, the object DP moves from [Spec res] to [Comp V].
The movement ``chain'' this operation forms is problematic because the head of the chain does not c-command the tail.
Although this type of so-called sideward movement is generally barred, \textcite{nunes2001sideward} argues for a restricted version of sideward movement.
Nunes argues that head movement and parasitic gaps both require a sideward movement operation, as they both create non-c-command dependencies.
\ex.
\a. Head Movement\\
{\small
\begin{forest}
    nice empty nodes,sn edges,baseline,for tree={
    calign=fixed edge angles,
    calign primary angle=-30,calign secondary angle=70}
    [TP
	    [DP]
	    [
		    [T
			    [T]
			    [V,name=head]
		    ]
		    [VP
			    [$\langle$V$\rangle$,name=tail]
			    [DP]
		    ]
	    ]
    ]
    \draw[->] (tail) to[out=south, in=south] (head);
\end{forest}}
\b. Parasitic gaps\\
What did Mary hear without seeing.\\
{\small
\begin{forest}
    nice empty nodes,sn edges,baseline
    [CP
	    [DP[What,roof,name=specCP]]
	    [
		    [C+T\\did,align=center]
		    [TP
			    [DP[Mary,roof]]
			    [
				    [$\langle$T$\rangle$]
				    [VP
					    [VP
						    [V\\hear,align=center]
						    [$\langle$DP$\rangle$,name=CompV1]
					    ]
					    [PP
						    [P\\without,align=center]
						    [VP
							    [V\\seeing,align=center]
							    [$\langle$DP$\rangle$,name=CompV2]
						    ]
					    ]
				    ]
			    ]
		    ]
	    ]
    ]
    \draw[->] (CompV2) .. controls +(south:1cm) and +(south:5cm) .. (CompV1);
    \draw[->] (CompV1) .. controls +(south west:4.5cm) and +(south:2cm) ..  (specCP);
\end{forest}}

According to the standard definition of Merge, sideward movement should be impossible.
The facts of parasitic gaps and head movement, however, suggest that a possibly complex operation with the net effect of sideward movement must be active in the grammar.
I adopt the approach developed by \textcite{nunes1995diss,nunes2001sideward} as follows.

In order to explain sideward movement, Nunes hypothesizes that a movement operation is composed of a Copy operation followed by Merge.
The operation Copy adds an object X to the workspace of a derivation provided that X is contained in an already constructed syntactic object.
\ex. For a workspace W and a syntactic object X, Copy(W, X) = $W\cup \{X\}$ iff there is a syntactic object Z $\in$ W and Z contains X.

Merge, then is a simpler operation which replaces two members of a workspace with the set containing them. 
To see how a Copy+Merge theory of movement works, consider the derivation of passivization in \cref{tab:EngPass}.
\begin{longtabu}[t]{lll}
	\textbf{Stage} & \textbf{Workspace} & \\
	\cline{1-2}
	1 & $\{$[T, [ \textsc{Voice}$_{pass}$ [see, [the, boy]]]]$\}$ & Copy([the, boy])\\
	2 & $
		\begin{Bmatrix*}[l]
			\text{[the, boy]},\\
			\text{[T, [ \textsc{Voice}$_{pass}$ [see, [the, boy]]]]}
		\end{Bmatrix*}
		$ & Merge([the, boy], [T \ldots])\\
	3 & $\{$[[the, boy], [T, [ \textsc{Voice}$_{pass}$ [see, [the, boy]]]]]$\}$ &\\
	\caption{The derivation of an English Passive}
	\label{tab:EngPass}
\end{longtabu}

The Copy+Merge theory of movement allows us to derive sideward movement by holding the copied object in the workspace while another tree is built as in the derivation of \cref{fig:hammer-flat} in \cref{tab:hammer-flat}.
\begin{longtabu}[t]{lll}
	\textbf{Stage} & \textbf{Workspace} & \\
	\cline{1-2}
	1 & $\left\{ \text{[[the, metal], [res, [\dots]]]} \right\}$ & Copy([the, metal])\\
	2 & $
	\begin{Bmatrix*}[l]
		\text{[the, metal]},\\
		\text{[[the, metal], [res, [\dots]]]}
	\end{Bmatrix*}
	$ & Select(hammer)\\
	3 & $
	\begin{Bmatrix*}[l]
		\text{hammer},\\
		\text{[the, metal]},\\
		\text{[[the, metal], [res, [\dots]]]}
	\end{Bmatrix*}
	$ & Merge(hammer, [the, metal])\\
	4 & $
	\begin{Bmatrix*}[l]
		\text{[hammer, [the, metal]]},\\
		\text{[[the, metal], [res, [\dots]]]}
	\end{Bmatrix*}
	$ & Merge$\begin{pmatrix*}[l]\text{[hammer, [the, metal]]},\\ \text{[[the, metal], [res [\dots]]]}\end{pmatrix*}$\\
	5 & \multicolumn{2}{l}{$\left\{\text{[[hammer, [the, metal]], [[the, metal], [res [\dots]]]]}\right\}$}\\
	\caption{The derivation of an English resultative VP}
	\label{tab:hammer-flat}
\end{longtabu}

Note that at stage 5 of the derivation in \Last the syntactic object in the workspace is representable as \cref{fig:hammer-flat}.

In order to constrain sideward movement, Nunes notes that its immediate, results such as the tree in \cref{fig:hammer-flat}, are unpronounceable.
Assuming that decisions regarding linear order depend on c-command relations, and part of linearization is deciding which copy in a movement chain is to be pronounced, we would be unable to make a definitive linearization statement for the derived structure in \Last.
In order to linearize the movement chain of \textit{the hammer}, there must be a copy which c-commands all other copies, meaning there must be a subsequent move from theme position to grammatical object position, which I represent as [Spec, AgrO]\footnote{
	The choice to include AgrO in my structures does not indicate a particular commitment on my part to the existence of such a head, but rather to the fact that Object position seems to be distinct from internal argument position, and higher than VP.
	Further, I assume that either the movement to object position is covert, or there is head raising of the verb, such that it precedes the object.
}
in \cref{fig:HammerFlatAgrO}.
\begin{figure}[h]
	\centering
{\small
\begin{forest}
    nice empty nodes,sn edges,baseline,
%    for tree={
%    calign=fixed edge angles,
%    calign primary angle=-30,calign secondary angle=65}
    [AgrOP
	    [DP[the metal,roof,name=obj]]
	    [
		    [AgrO]
    [VP
	    [VP
		    [hammer]
		    [DP,name=compV]
	    ]
	    [resP
		    [$\langle$DP$\rangle$,name=specRes]
		    [
			    [res]
			    [SC
				    [$\langle$DP$\rangle$,name=SCDP]
				    [flat]
			    ]
		    ]
	    ]
    ]
    ]
    ]
    \draw[->] (SCDP) to[out=south west, in=south east] (specRes);
    \draw[->] (specRes) to[out=south, in= south] (compV);
    \draw[->] (compV) to[out=south west, in=south] (obj); 
\end{forest}
}
	\caption{An English resultative AgrOP}
	\label{fig:HammerFlatAgrO}
\end{figure}

Since the copy of \textit{the metal} in [Spec, AgrO] c-commands all of the other copies, it will be pronounced and the lower copies will be deleted at the SM interface.
So, assuming some mechanism for sideward movement, we are able to modify Kratzer's (\citeyear{kratzer2004building}) analysis of resultatives so that it is compatible with UTAH.

Since we have modified Kratzer's analysis, it is worth asking if our version will still compose semantically to give us the desired interpretation.
In the next section, I argue that not only can we retain the proper interpretation, but we can do so while assuming a simpler compositional system.
\section{Composing resultatives}\label{sec:ResInterp}
\textcite{kratzer2004building} adopts a neo-Davidsonian semantics for resultatives, meaning that they are analyzed as descriptions of eventualities rather than merely as relations between entities.
Her syntactic analysis is repeated in \cref{fig:KratzerTreeRedux} for reference.
\begin{figure}[h]
	\centering
\begin{forest}
    nice empty nodes,sn edges,baseline,for tree={
    calign=fixed edge angles,
    calign primary angle=-30,calign secondary angle=70}
    [VP
	    [DP[the metal,roof,name=specV]]
	    [
		    [hammer]
		    [resP
			    [res]
			    [SC
				    [$\langle$DP$\rangle$,name=SCDP]
				    [flat]
			    ]
		    ]
	    ]
    ]
    \draw[->] (SCDP) to[out=south west, in=south] (specV);
\end{forest}
	\caption{Kratzer's (2005) structural analysis of resultatives}
	\label{fig:KratzerTreeRedux}
\end{figure}
According to this analysis, the small clause \textit{the metal flat} is interpreted as the state description in \Next, where the domain $D_s$ is the domain of eventualities.
\ex. $\llbracket\text{SC}\rrbracket = \lambda s_s \left[ \textsc{state}(s) \& \textbf{flat}(\textbf{the\_metal})(s) \right]$

The verb \textit{hammer} is interpreted as a predicate of events.
\ex. $\llbracket\textit{hammer}\rrbracket = \lambda e_s \left[ \textsc{event}(e) \& \textbf{hammer}(e)\right]$

Note that Kratzer analyses resultative verbs as intransitives, meaning they do not take any entity arguments.
Finally, she analyses the result head as a higher order function, which expresses a causal relation between the event expressed by the verb and the state expressed by the small clause.
\ex. $\llbracket\textit{res}\rrbracket = \lambda P_{\langle s,t\rangle} \lambda e_s \exists s_s \left[\textsc{event}(e) \& \textsc{state}(s) \& P(s) \& \textsc{Cause}(s)(e)\right]$

So, for Kratzer, the typed LF of \textit{hammer the metal flat} is as in \cref{fig:TypedLFKratzer}, and, according to her, \textit{hammer} and the resP compose by an operation she calls Event Identification \parencite{kratzer1996severing} which, in this instance, is equivalent to Predicate Modification generalized to eventualities.
\begin{figure}[h]
	\centering
\begin{forest}
    nice empty nodes,sn edges,baseline,for tree={
    calign=fixed edge angles,
    calign primary angle=-30,calign secondary angle=70}
    [VP$_{\langle s,t\rangle}$
		    [hammer$_{\langle s,t\rangle}$]
		    [resP$_{\langle s,t\rangle}$
			    [res$_{\langle st, st\rangle}$]
			    [SC$_{\langle s,t\rangle}$
				    [DP$_e$]
				    [flat$_{\langle e, st\rangle}$]
			    ]
		    ]
	    ]
\end{forest}
	\caption{The LF of Kratzer's resultatives}
	\label{fig:TypedLFKratzer}
\end{figure}
\ex. \textbf{Predicate Modification (eventuality version)}\\
If $\alpha$ is a branching node with daughters $\beta$ and $\gamma$, both of which are of type $\langle s,t\rangle$, the $\llbracket\alpha\rrbracket = \lambda e_s [\llbracket\beta\rrbracket(e) \& \llbracket\gamma\rrbracket(e)]$

So, the interpretation of the VP in \LLast, can be derived as in \Next.
\ex.
\begin{enumerate}
	\item $\llbracket$VP$\rrbracket$ = \hfill (Predicate Modification)
	\item $\lambda e_s [\llbracket\text{hammer}\rrbracket(e) \& \llbracket\text{resP}\rrbracket(e)]$ = 
	\item $\lambda e_s [ \textbf{hammer}(e) \& \exists s_s[\textsc{Cause}(s)(e) \& \textbf{flat}(\textbf{the\_metal})(s)]]$
\end{enumerate}

So, the hammering event is identical to the event of causing the flatness state.

The sideward movement structure I propose, represented in \cref{fig:hammer-flat}, is only slightly different from Kratzer's.
The main difference is that, in my analysis, the DP \textit{the metal} is an argument of both the result adjective and the verb.
In Kratzer's analysis, The DP is solely an argument of the result adjective.
Despite this difference, the VP and resP in my structures are still predicted to be predicates of eventualities.
As such, the VP and resP can be combined by predicate modification.
As a demonstration of this, see the typed LF for the proposed structure in \cref{fig:TypedLFSideward}, and the partial derivation of its denotation in \cref{ex:sidewardSemDeriv}.
\begin{figure}[h]
	\centering
\begin{forest}
    nice empty nodes,sn edges,baseline,for tree={
    calign=fixed edge angles,
    calign primary angle=-30,calign secondary angle=70}
    [VP$^{\prime}_{\langle s,t\rangle}$
	    [VP$_{\langle s,t\rangle}$
		    [hammer$_{\langle e, st\rangle}$]
		    [DP$_e$]
	    ]
	    [resP$_{\langle s,t\rangle}$
		    [res$_{\langle st, st\rangle}$]
		    [SC$_{\langle s,t\rangle}$
			    [DP$_e$]
			    [flat$_{\langle e, st\rangle}$]
		    ]
	    ]
	    ]
\end{forest}
	\caption{The LF of a resultative with sideward movement}
	\label{fig:TypedLFSideward}
\end{figure}
\ex. \label{ex:sidewardSemDeriv}
\begin{enumerate} 
	\item $\llbracket \text{VP}^{\prime}\rrbracket =$ \hfill (Predicate Modification)
	\item $\lambda e_s [\llbracket \text{VP}\rrbracket(e) \& \llbracket \text{resP}\rrbracket(e)]$
	\item $\lambda e_s [ \lambda e^{\prime}_{s}[ \textbf{hammer}(e^{\prime})\, \&\, \textsc{theme}(\textbf{the\_metal})(e^{\prime})](e)\, \&\, \llbracket \text{resP}\rrbracket(e)]$
	\item $\lambda e_s [ \lambda e^{\prime}_{s}[ \textbf{hammer}(e^{\prime})\, \&\, \textsc{theme}(\textbf{the\_metal})(e^{\prime})](e)\, \&$\\
		$\lambda e^{\prime\prime}_{s}[\exists s_s[\textsc{Cause}(s)(e^{\prime\prime}) \& \textbf{flat}(\textbf{the\_metal})(s)]](e)]$
	\item $\lambda e_s [ \textbf{hammer}(e)\, \&\, \textsc{theme}(\textbf{the\_metal})(e)\, \&\, \exists s_s[\textsc{Cause}(s)(e) \& \textbf{flat}(\textbf{the\_metal})(s)]]$
\end{enumerate}

Thus, with these adaptations, Kratzer's analysis of resultatives can be made UTAH-compliant.
In the remainder of this chapter, I will discuss my parametric analysis of resultatives.

\section{Where does the resultative parameter come from?}
In the previous chapter, I discussed two desiderata for a parametric analysis.
First, the parameter must be learnable, meaning there must be some variable in the primary linguistic data which the learner can detect and deduce a particular parameter setting from.
Second, the variable must be represented in the lexicon.
For clarity, I will refer to the variable detectable in the PLD as the surface variable, and its lexical representation as the lexical variable.

To my knowledge, there is only one proposed candidate for the surface variable in the generative literature, that is, Snyder's (\citeyear{snyder1995language,snyder2012parameter}) compounding parameter.
According to the latest version of this parameter, a language allows resultatives iff it allows bare stem compounding.
As discussed in the previous chapter, Snyder rejects the Lexical Parameterization Hypothesis, meaning that he does not propose a lexical variable.
Instead, he situates the parameter in the operations of the CI interface.
However, I will propose a lexical variable from which both the (un)availability of bare stem compounding, and the (un)availability of adjectival resultatives can be derived.

To make such a proposal, we must make the intermediate hypothesis that a language allows bare stem compounding iff it allows bare stems, meaning there should be no languages that allow for bare stems but cannot compound them together.
Now, a bare stem is merely an independent word with no inflectional material.
Words are represented in most current theories of syntax as an acategorial root merged with a category-determining functional head (following Marantz \citeyear{marantz1997no}, but see also Borer \citeyear{borer2005name} for a similar proposal).
Since roots are, by definition, featureless, any inflectional features on stems must be due to their category-determining heads.
It follows from this that the (im)possibility of bare stems derives from the presence or absence of inflectional features on category-determining heads in the lexicon.

So, if we represent inflected category-determining heads as $v_\varphi, n_\varphi, adj_\varphi, etc.$ and their bare counterparts as $v_\emptyset, n_\emptyset, adj_\emptyset, etc.$, then the lexical version of Snyder's compounding parameter can be represented as in \Next.
\ex. \textsc{lex} $\left\{ \text{includes, does not include} \right\}$ $v_\emptyset, n_\emptyset, adj_\emptyset, etc.$

Note that this is a fairly weak claim.
A stronger claim would be that compounding languages have only uninflected category-determining heads.
The weak claim, however, is sufficient for present purposes, and is therefore adopted.

This version of the compounding parameter is lexical, and therefore complies with the Lexical Parameterization Hypothesis.
Furthermore, it is learnable from the primary linguistic data, since its external manifestation is the presence or absence of inflectional morphology.
Since the inflectional morphology is detectable on the surface, its absence must also be detectable or at least deducible.
This leaves us with questions regarding the initial state of the lexicon, and which parameter setting is the default, but those questions are beyond the scope of this thesis and will be set aside.

\end{document}

\chapter{Label Theory}\label{sec:labels}
\epigraph{``A rose by any other name would smell as sweet.''\\
``Not if you called 'em stench blossoms.''}{``The Principal and the Pauper''\\\textit{The Simpsons}}
% arara: pdflatex: {options: "-draftmode"}
% arara: biber
% arara: pdflatex: {options: "-draftmode"}
% arara: pdflatex: {options: "-file-line-error-style"}
\documentclass[MilwayThesis]{subfiles}
\begin{document}
In this chapter I discuss a recent development in Chomsky's syntactic theory, which he refers to as \textit{label theory}.
Specifically, I discuss the background and content of Chomsky's (\citeyear{chomsky2013problems,chomsky2015problems}) proposal.
Label theory will then be used in \cref{sec:deriving} to explain the resultative parameter.
This discussion is separate from the other theoretical background because as I write this thesis, label theory is in its nascent stage.
In fact, in \cref{sec:modifications} I will draw out two questions that Chomsky's label theory leaves unanswered and hypothesize an answer for each, thus modifying the theory.
\section{Label theory and its motivations}
Chomsky begins his proposal of label theory with a discussion of the minimalist program in general.
In his estimation, the goal of the minimalist program has been to explain the universal properties of language as simply as possible.
The properties he identifies are (i) the structure-dependence of rules, (ii) displacement, (iii) linear order and (iv) projection/labelling.
He then argues that if we assume that linear order is a reflex of transfer to the SM interface, properties (i) and (ii) can be explained by assuming that Narrow Syntax is reducible to simplest Merge, as defined in \Next.
\ex. Merge($\alpha$, $\beta$) = $\left\{ \alpha, \beta \right\}$

Unlike previous versions of Merge, simplest Merge does not include labelling.
Chomsky argues that this is a welcome outcome, because labelling/projection is not as detectible in surface forms as the other properties of language, and has always been a theory internal notion.
What's more, Chomsky argues, previous theories that bundle labelling with structure building have always stipulated labelling rather than deriving it.
So, for instance, the phrase \textit{see the girl} is stipulated to be a VP rather than a DP.

Chomsky proposes that labels are assigned post-syntactically by a special instance of minimal search called the Labelling Algorithm (LA).
LA operates iteratively in a top-down manner, searching each syntactic object for a ``most prominent'' element which can serve as the label.
In the simplest case, an atomic element (a head X) merged with a complex object (a phrase YP\footnote{
		Note that YP is actually a set $\left\{ \alpha,\beta \right\}$, where $\alpha$ and $\beta$ are arbitrary syntactic objects.
		The use of ``YP'' is not meant to indicate headedness, projection, maximality, or any notions other than the complexity of the object indicated by ``YP''. 
}), the atomic element is found to be the most prominent element and, as such, is the label of the object.
\ex. LA$(\left\{ X, YP \right\}) = X$


The head-phrase case in \Last is trivial due to the inherent asymmetry in the structure.
Labelling becomes more complicated when symmetric structures are considered, that is, when head-head and phrase-phrase structures are considered.
To discover how these structures could be labelled, Chomsky considers examples of head-head and phrase-phrase structures that \textit{are} generated by grammars, and hypothesizes why \textit{they} are generated, while other instances are not.
The only head-head structures that surface are those that result from the merger of acategorial roots and category-determining heads.
So, structures like \Next[a] are labellable, but those like \Next[b] and \Next[c] are not.
\ex.
\a. $\left\{ n, \sqrt{\textsc{water}}\right\}$
\b.* $\left\{ n, v\right\}$
\c.* $\left\{ \sqrt{\textsc{ice}}, \sqrt{\textsc{water}} \right\}$

Chomsky proposes that roots are completely featureless and, therefore, invisible to LA.
\Last[a] thus receives the label $n$, while neither \Last[b] nor \Last[c] can be labelled.

As for phrase-phrase structures, Chomsky identifies two types that can be generated.
The first type are what I will call phrase-trace structures.
These are phrase-phrase structures in which one of the constituent phrases is a lower copy of a moved element.
Following \textcite{moro2000dynamic}, Chomsky proposes that lower copies are invisible to LA.\footnote{
	While this assertion might be characterized as a stipulation, I believe it is best understood as a statement of fact in label-theoretic terms.
	By hypothesis, all convergent SOs are labelled.
	Since there is a class of $\left\{XP, \lfloor YP\rfloor \right\}$ objects that are convergent, those objects must be labelled.
	The labels of those trace-phrase objects depend only on the unmoved constituent.
	That is, an object $\left\{ \lfloor DP\rfloor, vP \right\}$ is labelled $v$ not D.
	It follows from this that the labelling algorithm does not take into account the properties of the moved constituent, or, in other words, the moved object is invisible to the algorithm.

	Note that, this being an empirical statement, it requires explanation.
	For instance, it remains to be explained how the LA can distinguish upper copies from lower copies.
	Such an explanation, however, is beyond the scope of this thesis.
}
The labelling of a phrase-trace structure is, therefore, as shown in \Next.
\ex. LA$(\left\{ XP, \lfloor YP\rfloor \right\}) =$ LA$(XP)$\\
(Floor brackets ($\lfloor\cdot\rfloor$) here indicate that YP is a lower copy.)

The second type of phrase-phrase structure that can surface is what I will call agreement structures.
These are phrase-phrase structures in which the two constituent phrases agree with one another for some feature.
In these cases, the agreeing features serve as the label of the structure as in \Next.
\ex. LA$(\left\{ XP_F, YP_F \right\}) = \langle F,F\rangle$

If neither of these two situations obtains for a given phrase-phrase structure, it will be unlabellable and result in a crash at the CI interface.
To see how this works, consider the raising construction in \Next and the ungrammatical version of it in \NNext.
\ex.\label{ex:raising}
\a. The dishes seem to be dirty.
\b. [$_\alpha$ The dishes [ seem [$_\beta \lfloor\text{the dishes}\rfloor$, [ to be dirty]]]]

\ex.\label{ex:noraising}
\a.* It seems the dishes to be dirty
\b. [$_\gamma$ It [ seem [$_\delta$ the dishes, [ to be dirty]]]]

The sentence in \LLast has two relevant phrase-phrase structures which are labellable.
The first is a trace-phrase structure, given in \Next[a], which is labelled by the infinitive \textit{to}, as demonstrated in \Next[b].
\ex.
\a.  $\beta = \left\{ \lfloor\text{the dishes}\rfloor, \left\{ \text{to}, \text{be dirty} \right\} \right\}$
\b. LA$(\beta) = \text{LA}(\left\{ \text{to}, \text{be dirty} \right\}) =$ to

The second is the agreement structure, given in \Next[b], which is labelled by the agreeing $\varphi$ features as in \Next[b].
\ex.
\a. $\alpha = \left\{ \text{the}_\varphi \text{ dishes} \left\{ T_\varphi \text{, seem} \left\{ \ldots \right\} \right\} \right\}$
\b. LA$(\alpha) = \langle\varphi, \varphi\rangle$

The derivation of \ref{ex:noraising}, however, crashes because the first phrase-phrase structure, shown as $\delta$ in \cref{ex:UnlabellableDelta}, is unlabellable.
The DP \textit{the dishes} has not raised, so it is visible to LA in $\delta$, and there is no $\varphi$-agreement between \textit{the}$_\varphi$ and \textit{to}$_\emptyset$.
\ex.\label{ex:UnlabellableDelta}
\a. $\delta = \left\{ \text{the}_\varphi \text{ dishes} \left\{ \text{to}_\emptyset \left\{ \ldots \right\}\right\} \right\}$
\b. LA$(\delta) = $ Undefined


Also, Chomsky proposes that heads that bear only a partial set of features (\textit{e.g.} English finite T$_\varphi$) cannot label unless they agree for those features with some other head.
This is in contrast to heads that bear full feature sets (\textit{e.g.} Italian T${\langle\varphi,\varphi\rangle}$) or lack these features altogether (\textit{e.g.} English non-finite T$_\emptyset$) which can label without agreement.

At this point I should note an issue that arises when giving label-based explanations of syntactic derivations.
In previous theories, movement operations were described as being ``driven'' by some need.
For example, in Government and Binding theories DPs undergo movement in order to get abstract Case.
In minimalist theories, this has been generalized such that all movement is driven by the need to satisfy some feature.
This led to debates about the exact mechanism that drives movement. 
Greed-based accounts, for instance, argue that an object moves if and only if such a move will satisfy one of its own features, while those that assume Enlightened Self Interest take the weaker stance that an object moves if and only if such a move will satisfy some feature on some argument.\footnote{
	See \textcite{lasnik1999last} for a discussion of these two types of accounts.
}
While these accounts assumed an interface-based theory, they allow syntacticians to explain (un)grammaticality purely in terms of narrow syntax.

Label theory, however, assumes that all operations are free, that is, they do not require a trigger or a driver.
This means, however, that an explanation of why an operation occurs or does not occur in a given derivation is slightly more complicated.
The well-formedness of a structure is assessed at the interface; this means that entire phases are assessed at once.
Consider, for instance, the successive \textit{wh}-movement in \Next, and how the two types of theories would account for it.
\ex. Who$_i$ does Mary say $t_i$ that Laura likes $t_i$?

The accounts of the final movement step ([Spec, C] to [Spec, C]) would be similar, as both theories assume that the highest C needs to agree with a \textit{wh}-word, either for labelling or feature-satisfaction.
The explanations of the first movement ([Comp, V] to [Spec, C]) however, are different.
If movement operations must be driven, then we would likely need to posit a feature on the lower C which must be satisfied by a \textit{wh}-word.
With free movement, however, the \textit{wh}-word must move to the lower [Spec, C] because, if it doesn't, it cannot move to the higher [Spec, C] without violating Subjacency.
Assuming a phase-based theory of subjacency, the first movement operation in \Last follows from the proposal that C is a phase head.

To be concrete: suppose that \textit{who} does not move from its base position to its intermediate position in [Spec, \textit{that}].
The phase head \textit{that} would ``trigger'' the transfer of its complement, which includes every instance of \textit{who}.
When C$_{Q}$ is finally merged, \textit{who} is unavailable, as shown in \Next. 
\ex. C$_{Q}$ Mary T say that \sout{Laura likes who}

Since C$_{Q}$ has only one Q feature, it cannot be the label of \Last and the derivation crashes.

To summarize, a syntactic derivation in label theory proceeds as follows.
Structures are built by iteratively applying Merge (along with Select and Copy) to syntactic objects.
At certain points a portion of a structure (\textit{i.e.}, a phase) is transferred to the interfaces.
At the CI interface, the labelling algorithm labels the transferred structure and all of the structures contained within the transferred structure.
If the labelling algorithm fails to label any part of the transferred structure, the derivation crashes.
A summary of the labelling algorithm is given in \cref{tab:LA-results}.
\begin{table}
	\centering
	\begin{tabular}[t]{llp{5cm}}
		\textbf{SO} & \textbf{LA(SO)} & \\
		\cline{1-2}
		$X$ & $X$ & ($X$ is not a root, and does not have an incomplete feature set)\\
		$\left\{ X, R \right\}$ & LA$(X)$ & ($R$ is a root, $X$ is not a root)\\
		$\left\{ X, YP \right\}$ & LA$(X)$ & \\
		$\left\{ \langle XP\rangle, YP \right\}$ & LA$(YP)$ & \\
		$\left\{ XP_F, YP_F \right\}$ & $\langle F,F\rangle$ & ($XP$ and $YP$ agree for $F$)\\
		Otherwise & Undefined &
	\end{tabular}
	\caption{A summary of the labelling algorithm}
	\label{tab:LA-results}
\end{table}

\textcite{chomsky2015problems} demonstrates that label theory has some empirical advantage over previous theories of syntax, but leaves at least two questions unanswered.
The first question is why labels would be required by the CI interface at all, and the second question is how are host-adjunct structures labelled.
In Part II I will propose answers to those questions, and consider the ramifications label theory has for the architecture of the grammar.

\section{Summary}
In this chapter I reviewed Chomsky's (\citeyear{chomsky2013problems,chomsky2015problems}) label theory, according to which labels are assigned algorithmically at the CI interface and are required for proper interpretation at that interface.
In the next chapter, I will use label theory to demonstrate that the (un)availability of resultative can be derived from the (un)availability of bare stem compounding.
\end{document}

%\section{The architecture of a label-theoretic grammar}
%\subfile{agree}
\chapter{Deriving the Resultative Parameter}\label{sec:deriving}
\epigraph{All happy families are alike; every unhappy family is unhappy in its own way.}{\textit{Anna Karenina}\\\textsc{Leo Tolstoy}}
% arara: pdflatex: {options: "-draftmode"}
% arara: biber
% arara: pdflatex: {options: "-draftmode"}
% arara: pdflatex: {options: "-file-line-error-style"}
\documentclass[MilwayThesis]{subfiles}
\setcounter{chapter}{5}

\begin{document}
\epigraph{All happy families are alike; every unhappy family is unhappy in its own way.}{\textit{Anna Karenina}\\\textsc{Leo Tolstoy}}
So far, I have proposed a theory of syntax, a structural analysis of resultatives, and an analysis of the resultative parameter.
Although they share a justification in minimalism, each of these is an independant proposal, and as such their combination must be justified.
In this chapter I show that the combination of my proposals work together.
I do so by first showing that the structure in \Next can be derived in a language with uninflected adjectives ($adj_\emptyset \in \textsc{lex}$), and then by showing how such a derivation fails in a language without uninflected adjectives ($adj_\emptyset \centernot\in \textsc{lex}$).
\ex.
{\small
\begin{forest}
    nice empty nodes,sn edges,baseline,
%    for tree={
%    calign=fixed edge angles,
%    calign primary angle=-30,calign secondary angle=60}
    [VP
	    [VP
		    [hammer]
		    [DP[the metal,roof,name=compV]]
	    ]
	    [resP
		    [$\langle$DP$\rangle$,name=specRes]
		    [
			    [res]
			    [SC
				    [$\langle$DP$\rangle$,name=SCDP]
				    [
					    [adj]
					    [\textsc{flat}]
				    ]
			    ]
		    ]
	    ]
    ]
    \draw[->] (SCDP) to[out=south west, in=south] (specRes);
    \draw[->] (specRes) to[out=south, in=south east] (compV);
\end{forest}
}

In the first section, I will describe a derivation of the English resultative VP \textit{hammer the metal flat} and show that it converges at the CI interface (taking SM convergence for granted).
In the second section, I will describe two derivations of the ungrammatical French resultative VP \textit{marteller le m\'etal plat}, and show that deriving that VP leads to a CI crash, while avoiding that crash blocks the derivation.

Before describing the derivations, I will reiterate and clarify my assumptions regarding the nature of the syntactic derivation.
I will adopt a slightly simplified version of the formal grammar developed by \textcite{collins2016formalization} which I will augment slightly based on new assumptions.
A \textit{derivation} is defined as a finite sequence of \textit{stages}, $\langle S_1, S_2 \ldots S_n\rangle$.
Each stage $S_i$ in a derivation is a pair $\langle LA_i, W_i\rangle$, where $LA_i$ is a set of lexical items called the \textit{lexical array} and $W_i$ is a set of syntactic objects called the \textit{workspace}.
The computational operations (Merge, Select, Copy, Transfer) play the role that rules on inference play in deductive systems, that is, they map derivational stages onto subsequent stages.
A given stage $S_i$ \textit{derives} a subsequent stage $S_{i+1}$ if and only if some operation, applied to $S_i$, yields $S_{i+1}$.


\section{A successful derivation in English}
In many ways, successful derivations, like happy families, are uninteresting, but they still must be demonstrated in order to show where the crashing derivations go wrong.

To begin with, we derive the result phrase.
The formal derivation and resulting unlabelled structure is given in \Next.
\ex.
\a.
\begin{tabular}[t]{llll}
	\textbf{Stage} & \textbf{LA} & \textbf{Workspace} & \\
	\cline{1-3}
	1 & $
	\begin{Bmatrix*}[l]
		\sqrt{\textsc{flat}},\\
		adj_\emptyset,\\
		res,\\
		\text{DP}
	\end{Bmatrix*}
	$ & $\emptyset$ & Select($\sqrt{\textsc{flat}}$)\\
	2 & $
	\begin{Bmatrix*}[l]
		adj_\emptyset,\\
		res,\\
		\text{DP}
	\end{Bmatrix*}
	$ & $\left\{\sqrt{\textsc{flat}}\right\}$ & Select($adj_\emptyset$)\\
	3 & $
	\begin{Bmatrix*}[l]
		res,\\
		\text{DP}
	\end{Bmatrix*}
	$ & $
	\begin{Bmatrix*}[l]
		adj_\emptyset,\\
		\sqrt{\textsc{flat}}
	\end{Bmatrix*}$
	& Merge($adj_\emptyset, \sqrt{\textsc{flat}}$)\\
	4 & $
	\begin{Bmatrix*}[l]
		res,\\
		\text{DP}
	\end{Bmatrix*}
	$ & $\left\{ \left\{_\alpha adj_\emptyset, \sqrt{\textsc{flat}} \right\} \right\}$ & Select(DP)\\
	5 & $\left\{ res \right\}$ & $
	\begin{Bmatrix*}[l]
		\text{DP},\\
		\left\{_\alpha adj_\emptyset, \sqrt{\textsc{flat}} \right\}
	\end{Bmatrix*}
	$ & Merge(DP, $\alpha$)\\
	6 & $\left\{ res \right\}$ & $ \left\{ \left\{_\beta \text{DP}, \left\{_\alpha adj_\emptyset, \sqrt{\textsc{flat}} \right\} \right\} \right\}$ &
	Select(\textit{res})\\
	7 & $\emptyset$ & $
	\begin{Bmatrix*}[l]
		res,\\
		\left\{_\beta \text{DP}, \left\{_\alpha adj_\emptyset, \sqrt{\textsc{flat}} \right\} \right\}
	\end{Bmatrix*}
	$ & Merge(\textit{res}, $\beta$)\\
	8 & $\emptyset$ & $\left\{ \left\{_\gamma res, \left\{_\beta \text{DP}, \left\{_\alpha adj_\emptyset, \sqrt{\textsc{flat}} \right\} \right\}\right\} \right\}$
	& Copy(DP)\\
	9 & $\emptyset$ & $
	\begin{Bmatrix*}[l]
		\text{DP},\\
		\left\{_\gamma res, \left\{_\beta \text{DP}, \left\{_\alpha adj_\emptyset, \sqrt{\textsc{flat}} \right\} \right\}\right\}
	\end{Bmatrix*}
	$
	& Merge(DP, $\gamma$)\\
	10 & $\emptyset$ & $
	\left\{\left\{_\delta \text{DP},\left\{_\gamma res, \left\{_\beta \text{DP}, \left\{_\alpha adj_\emptyset, \sqrt{\textsc{flat}} \right\} \right\}\right\}\right\}\right\}
	$
	& Transfer($\beta$)\\
\end{tabular}
\b.
{\small
  \begin{forest}
      nice empty nodes,sn edges,baseline,for tree={
    calign=fixed edge angles,
  calign primary angle=-30,calign secondary angle=70}
      [$\delta$
        [DP$_\varphi$[{\rm the metal},roof]]
        [$\gamma$
          [res]
          [$\beta$
        [DP$_\varphi$[{\rm the metal},name=SC DP,roof]]
        [$\alpha$
          [adj$_\emptyset$]
          [{\rm flat}]
        ]
          ]
        ]
      ]
      \draw[thick] ([xshift=-36pt, yshift=-24pt]SC DP) arc[start angle=170,end angle=130,radius=7.5cm];
  \end{forest}
}

Assuming res is a phase head, its complement $\beta$ is trasferred and must be labelled along with the SOs contained in $\beta$.
The small clause $\beta$ ($\left\{_\beta \langle \text{DP}\rangle, \left\{_\alpha adj_\emptyset, \sqrt{\textsc{flat}} \right\} \right\}$) is a Phrase-Phrase structure, but since one of its constituent parts, the DP, is a lower copy, that part is invisible to the labelling algorithm.
Therefore, only the adjective ($\left\{_\alpha adj_\emptyset, \sqrt{\textsc{flat}} \right\}$) is available to provide a label.
Assuming roots are inert for labelling, and uninflected categorizing heads are able to label, the label of $\alpha$, and therefore the label of $\beta$, is $adj_\emptyset$.
So, $\beta$ is succesfully labelled and therefore convergent at the CI interface.
\ex. LA($\left\{_\beta \langle\text{DP}\rangle, \left\{_\alpha adj_\emptyset, \sqrt{\textsc{flat}} \right\} \right\} = \left[_{adj} \langle\text{DP}\rangle, \left[_{adj} adj_\emptyset, \sqrt{\textsc{flat}} \right]  \right]$


\section{Two crashing derivations in French}

\subsection{Crashing Derivation \#1}

\subsection{Crashing Derivation \#2}

\end{document}


\part{Dealing with the consequences}\label{sec:part2}
\chapter{Small clauses in French-like languages}\label{sec:FreSC}
\epigraph{All the pieces matter}{David Simon}
%        File: agree.tex
%     Created: Fri Mar 24 09:00 AM 2017 E
% Last Change: Fri Mar 24 09:00 AM 2017 E
%
% arara: pdflatex: {options: "-draftmode"}
% arara: biber
% arara: pdflatex: {options: "-draftmode"}
% arara: pdflatex: {options: "-file-line-error-style"}
\documentclass[MilwayThesis]{subfiles}

\begin{document}
In the preceding chapters, I presented an explanation of why adjectival resultatives cannot be derived in French(-type) languages.
Without further comment, however, this explanation seems to wrongly predict that depictives and copular clauses are also ruled out in French.
In this chapter, I will refine my proposal so that this prediction is removed.
Rather than changing any of the claims and hypotheses I have made thus far, I will clarify another part of the grammar, namely, the Agree operation.
With this clarification, resultatives in French can be ruled out without barring depictives and copular clauses.
Furthermore, the clarified grammar will provide a straightforward explanation of \textit{wanna}-contraction in English.

\section{The faulty predictions: *depictives, *copular clauses}
In order to account for the lack  of resultatives in French(-type languages), I argued that DP movement from a small clause is barred in these languages.
The fact that French allows copular clauses and depictives, however, means that this restriction, as stated, does not hold across the board.
In other words, the grammar should generate \cref{ex:FreCop} and \cref{ex:FreDep} but not \cref{ex:FreRes}.
\exg. \label{ex:FreCop}Jeanne est grand -e.\\
Jeanne is tall -FSg\\
``Jeanne is tall.''

\exg. \label{ex:FreDep}Marie mange la viande crue\\
Marie eats the.FSg meat raw.Fsg\\
``Marie eats meat raw''

\exg.* \label{ex:FreRes}Sophie a martell\'e le m\'etal plat.\\
Sophie \textsc{aux} hammered the.MSg metal flat.MSg\\
``Sophie hammered the metal flat.''

Consider the case of the copular clause \cref{ex:FreCop}, whose (simplified) structure is given in \cref{fig:FreCop}
\begin{figure}[h]
	\centering
\begin{forest}
  nice empty nodes,sn edges,baseline,for tree={
    calign=fixed edge angles,
  calign primary angle=-30,calign secondary angle=70}
  [$\delta$
    [DP$_\varphi$[Jeanne,roof]]
    [$\gamma$
      [T$_\varphi$]
      [$\beta$
	[$\langle$DP$_\varphi\rangle$]
	[$\alpha$
	  [adj$_\varphi$]
	  [\textsc{grand}]
	]
      ]
    ]
  ]
\end{forest}
	\caption{Simplified structure of a French copular clause}
	\label{fig:FreCop}
\end{figure}
French \textit{adj} has a single $\varphi$-set, meaning it can only label if it is strengthened by agreement.
In the case of resultatives, DP movement out of a small clause bleeds agreement with \textit{adj}.
If this is also true of \cref{fig:FreCop}, then $\alpha$ and $\beta$ would be unlabellable.
The same reasoning apply to depictives as well.

This line of thinking is based on the implicit premise that lower copies are invisible to Agree as well as to Label.
We could permit \cref{ex:FreCop} by hypothesizing that lower copies are visible to Agree but invisible to Label.
However, such a move would predict that French generates resultatives, clearly an unwanted result.
I will argue that lower copies are invisible to Agree, but only under certain circumstances.
The next section will present a hypothesis regarding the nature of these ``certain circumstances'' by clarifying the Agree operation.

\section{The nature of the Agree operation}\label{sec:natureofagree}

As stated in \cref{sec:nonstandard}, I assume that syntactic agreement occurs outside of the Narrow Syntax.
Unsurprisingly, this assumption requires additional refinement.
If we take Agree to be an operation, we can ask where it fits in the grammatical architecture.
By hypothesis, it operates on the output of the Narrow Syntax, and it must operate before labelling.
Furthermore, its effects are phonetically overt.
These considerations suggest that Agree is part of Transfer, that is, it operates on derived syntactic objects before they are sent to the interfaces.
This is represented in \cref{fig:SepCycles}.
%Crucial to both label theory in general and its application in this thesis, is syntactic agreement.
%The highest XP in a given chain must agree with its sister YP in order to converge, and a subset of functional heads must agree in order to label (\textit{e.g.}, English T$_\varphi$).
%In this chapter I will discuss the theory of agreement, as it relates to labeling and show how the version of agree required for labeling avoids an undergeneration issue apparently predicted for predicative adjectives in French-type languages.
%
%Agreement is required in label theory to account for (\textit{e.g.},) Subject-TP structures as in \Next.
%\ex. [$_\alpha$ DP$_\varphi$ [$_\beta$ T$_\varphi$ ZP]]
%
%The labels of both $\alpha$ and $\beta$ depend on agreement between the subject and T.
%Since $\alpha$ is a Phrase-Phrase structure, its label will be $\langle\varphi,\varphi\rangle$ provided DP and T agree for $\varphi$.
%This agreement also renders $\beta$ labelable, since, prior to agreement, English T$_\varphi$, with an incomplete $\varphi$-set, is too weak to label \parencite{chomsky2013problems}.
%Agreement has the effect of strengthening T such that it can label.
%
%To understand how Agree and Label interact, we must first consider what sort of operations they each are abstracting away from their actual implementations.
%Both take syntactic objects as inputs and operate on them iteratively and locally.
%This means that labeling a structure like \Last requires labeling all of its substructures (Iterativity) and that labeling $\beta$ depends solely on the properties of $\beta$ (Locality).
%The same, then, is true for Agree, which iteratively considers each substructure and performs agreement is the conditions for agreement are met.
%
%Because labeling is sometimes contingent on agreement, the calculation of the latter must precede that of the former.
%Assuming both Agree and Label occur after narrow syntax and before transfer to CI, this leaves us with two possibilities for ordering the two operations.
%Either (i) individual iterations of Agree and Label are ordered with respect to each other forming a single Agree+Label cycle or (ii) Agree and Label each has its own cycle, and those cycles are ordered with respect to each other.
%To decide between these two alternatives we can consider how Agree and Label interact with other components of grammar, specifically the SM and CI interfaces.
%By hypothesis, Label feeds interpretation at CI but not at SM.
%Agree, on the other hand, feeds Label and interpretation at SM.
%This asymmetry points to the second alternative, where narrow syntax feeds an Agree cycle, which feeds SM interpretation and Label as in figure \ref{fig:SepCycles}.
%The first alternative, in which Agree and Label are bundled into a single cycle as shown in figure \ref{fig:OneCycle}, predicts that both Label and Agree feed both interfaces, which is not what we seem to see in the data. 
\begin{figure}[h]
  \centering
  \begin{tikzpicture}
    \node (syn) at (1,3) {Narrow Syntax};
    \node[draw,rounded corners] (agree) at (1,2) {Agree};
    \node[draw,rounded corners] (label) at (1,1) {Label};
    \node (SM) at (0,1.5) {SM};
    \node (CI) at (1,0) {CI};
    \path[->](syn)	edge			(agree)
	   (agree.south)	edge			(label)
			  edge [bend left]	(SM)
	  (label)		edge [loop right]	()
			  edge			(CI);
  \end{tikzpicture}
  \caption{The position of Agree in the grammar}
  \label{fig:SepCycles}
\end{figure}

This much is an almost unavoidable result of my assumptions and the general observations, but we require further hypotheses to arrive at the predictions we need.
The first hypothesis is that Agree feeds some operation that renders lower copies invisible to Label.
This could be deletion, impoverishment , or even some kind of cloaking -- the details of the operation are not important -- but crucially, it is this operation that renders a lower copy invisible to Label.
Presumably the effects of this operation will also be felt at the SM interface.
More on that in \cref{sec:wanna}.

If the Agree operation, in a sense, determines whether a given SO is visible to Label, what determines whether an SO is visible to Agree?
Supposing that the input to an Agree-cycle is a a phase P, I hypothesize that all and only those SO's that are \textit{contained} by P are visible to that cycle.
This may seem like a trivial hypothesis, but given the definition of ``SO'' and ``contain'' that I will adopt, it makes actual empirical predictions.

My definition will depend on distinguishing a syntactic object and an occurrence of a syntactic object.
This type of distinction has been made throughout the development of generative syntax, but perhaps the best known version is the notion of a chain.
In LGB, for instance, each nominal in an S-Structure was associated with a sequence of grammatical functions that represent its derivational history.
This sequence was called a function chain.
So, in the passive S-structure \cref{ex:passiveSStruct}, \textit{Jennifer} is associated with the chain \cref{ex:PassiveChain}.
\ex.\label{ex:passiveSStruct} Jennifer$_i$ was served $t_i$.

\ex.\label{ex:PassiveChain} $\langle\left[\text{NP,S}\right], \left[\text{NP,VP}\right]\rangle$

There is a sense in which the chain was the real grammatical object in LGB and later theories, as filters like the $\theta$-criterion and the Case filter were satisfied by chains rather than by their individual links.
So, \textit{Jennifer} in \cref{ex:passiveSStruct} has a $\theta$-role because a link in its chain ([NP, VP]) has a $\theta$-role.
Following \textcite{collins2016formalization}, I replace the terms ``chain'' and ``link'' with syntactic objects and occurrences defined below.
\begin{defn}[Syntactic Object]
  X is a \textit{syntactic object} (SO) iff\\
    X is a lexical item, or\\
    X is a set of syntactic objects. \parencite[Modified from][]{collins2016formalization}
  \label{def:so}
\end{defn}
\begin{defn}[Position]
  The \textit{position} of \I{SO}n in \I{SO}1 is a path, a sequence of syntactic objects $\langle\text{SO}_1,\text{SO}_2,\dots,\text{SO}_n\rangle$ where for all $0 < i < n$, $\text{SO}_{i + 1} \in \text{SO}_i$. \parencite{collins2016formalization}
  \label{def:position}
\end{defn}
\begin{defn}[Occurrence]
  B \textit{occurs} in A at position P iff P = $\langle\text{A},\dots,\text{B}\rangle$. We also say B has an occurrence in A at position P (written \I{B}P).
  \label{def:occurrence}
\end{defn}

Consider the abstract syntactic object and its tree representation below in \cref{ex:AbstractSO} and \cref{fig:AbstractTree}, respectively.
\ex.\label{ex:AbstractSO} $\left\{ \text{X}, \left\{ \text{Y} \left\{ \text{X}, \text{Z} \right\} \right\} \right\}$

\begin{figure}[h]
	\centering
	\begin{forest}
	  nice empty nodes,sn edges,baseline,for tree={
	    calign=fixed edge angles,
	    calign primary angle=-30,calign secondary angle=70
	  }
	  [$\alpha$
	    [X]
	    [$\beta$
	      [Y]
	      [$\gamma$
		[X]
		[Z]
	      ]
	    ]
	  ]
	\end{forest}
	\caption{A Tree representation of \cref{ex:AbstractSO}}
	\label{fig:AbstractTree}
\end{figure}

Based on the definitions above, we can say the following things about \cref{ex:AbstractSO}:
There are six SOs represented in \cref{ex:AbstractSO}: three lexical items (X, Y, Z) and three sets of SOs ($\alpha$, $\beta$, $\gamma$).
There is a single SO, X, with two occurrences in \cref{ex:AbstractSO}:
at $\langle \alpha, \text{X}\rangle$, and at $\langle \alpha, \beta, \gamma, \text{X}\rangle$

With this contrast between SOs and occurrences, we can limit the domain of Agree to complete chains without stipulating the existence of chains.
Consider the structure in \cref{ex:AbstractSO}, assuming that Y is a phase head, meaning its complement $\gamma$ has been transferred, rendering it inert. 
Assume that the computation must track two sets of SOs: the set of SOs in the derivation (\textsc{Terms}$_\text{SO}$), which includes the active and inert objects, and the set of active SOs (\textsc{Active}$_\text{SO}$), which excludes the inert objects.
For \cref{ex:AbstractSO}, the two sets are given in \cref{ex:AbstractTermsActive}.
\ex. \label{ex:AbstractTermsActive}
\a. \textsc{Terms}$_\alpha$ = $\left\{ \text{X}, \text{Y}, \text{Z}, \alpha, \beta, \gamma  \right\}$
\b. \textsc{Active}$_\alpha$ = $\left\{ \text{X}, \text{Y}, \alpha, \beta \right\}$

Since Agree operates on those objects which have been transferred, and, therefore, been rendered inert for the purposes of further computation, we can determine the input to Agree, then, by computing the set difference between the two sets in \cref{ex:AbstractTermsActive} as shown in \Next.
\ex.\label{ex:AbstractAgrInput} $\textsc{Terms}_\alpha \setminus \textsc{Active}_\alpha = \left\{ \text{Z}, \left[ \text{X}, \text{Z} \right] \right\}$ 

This derived set of SOs, I assume, is the input to Agree.
Note that X, which has moved to [Spec, Y], is not a member of the input to Agree, despite the fact that there is a member of the input which has X as a member.
As such, X is invisible to Agree and Label.
With that understood in the abstract, we can consider the concrete cases of copular clauses and resultatives.


\section{Correct predictions: French small clauses saved}\label{sec:FreSaved}
To begin, let's compare the two relevant structures: the copular clause in \cref{fig:cop-clause}, and the resultative adjunct in \cref{fig:result-adjunct}.
\begin{figure}[h]
	\centering
	\begin{forest}
	  nice empty nodes,sn edges,baseline,for tree={
	    calign=fixed edge angles,
	  calign primary angle=-30,calign secondary angle=70}
	  [$\zeta$
	    [C]
	    [$\delta$
	      [DP$_\varphi$[Jeanne,roof,name=subj]]
	      [$\gamma$
		[T$_\varphi$]
		[$\beta$
		  [$\langle$DP$_\varphi\rangle$]
		  [$\alpha$
		    [adj$_\varphi$]
		    [\textsc{grand}]
		  ]
		]
	      ]
	    ]
	  ]
	  \draw[thick] ([xshift=-12pt]subj.west) arc(180:130:5cm);
	\end{forest}	
	\caption{An unlabelled French copular clause}
	\label{fig:cop-clause}
\end{figure}
\begin{figure}[h]
	\centering
	\begin{forest}
	  nice empty nodes,sn edges,baseline,for tree={
	    calign=fixed edge angles,
	    calign primary angle=-30,calign secondary angle=70
	  }
	  [$\delta$
	    [DP$_\varphi$[le m\'etal,roof]]
	    [$\gamma$
	      [res]
	      [$\beta$
		[$\langle$DP$_\varphi\rangle$,name=insitu]
		[$\alpha$
		  [adj$_\varphi$]
		  [\textsc{plat}]
		]
	      ]
	    ]
	  ]
	  \draw[thick] (insitu.south west) arc(180:130:3cm);
	\end{forest}
	\caption{An unlabelled French resP (which will crash)}
	\label{fig:result-adjunct}
\end{figure}

The most salient difference between the DP ``chains'' in \cref{fig:cop-clause} and \cref{fig:result-adjunct} are that the ``chain'' in \cref{fig:result-adjunct} crosses a phase boundary, while the on in \cref{fig:cop-clause} does not.
This fact is relevant for Agree's visibility conditions.
In \cref{fig:cop-clause}, both occurrences of the DP are contained within the phase, meaning the entire syntactic object DP is contained within the phase and therefore is visible to Agree.
So, Agree takes $\delta$, which contains a full DP chain, and values $\varphi$ features on T and adj with $\varphi$ features of DP.
This has two relevant effects: first, it strengthens T and adj so that they can label, and second it renders the lower copy of DP inactive/invisible for Label.
Label then operates on the output of Agree and successfully sends a labelled phrase marker to CI.
\begin{figure}[h]
	\centering
	\begin{forest}
	  nice empty nodes,sn edges,baseline,for tree={
	    calign=fixed edge angles,
	    calign primary angle=-30,calign secondary angle=70
	  }
	  [$\delta$
	    [DP$_\varphi$[Jeanne,roof,name=subj]]
	    [$\gamma$
	      [T$_{\langle\varphi,\varphi\rangle}$]
	      [$\beta$
		[\sout{DP$_\varphi$}]
		[$\alpha$
		  [adj$_{\langle\varphi,\varphi\rangle}$]
		  [\textsc{grand}]
		]
	      ]
	    ]
	  ]
	\end{forest}
	\caption{The output of Agree for a copular clause}
	\label{fig:agree-cop-clause}
\end{figure}
\ex. The result of Label for \cref{fig:agree-cop-clause} 
\a. Label($\delta$) = $\langle\varphi,\varphi\rangle$
\b. Label($\gamma$) = T
\b. Label($\beta$) = Label($\alpha$) = adj

Thus, the derivation of a copular clause converges in French.

Next, consider the resultative adjunct in \cref{fig:result-adjunct}, which does not converge in French.
We start with a small clause which we merge with res, a phase head, forming $\gamma$.
We then merge the DP with $\gamma$, and commence our phase operations on $\beta$.
The phase complement, $\beta$, unlike that of the copular clause in \cref{fig:cop-clause}, contains only one occurrence of the DP.
Since Agree operates only on SOs, the DP, \textit{le m\'etal}, is invisible to it.
There is, therefore, no feature transfer between D and adj, and the DP, therefore cannot be ``deleted'' yet.\footnote{
	Recall from \cref{sec:natureofagree} that, along with feature transfer, one of the things that Agree does is render lower copies invisible to Label.
	It follows from this that only those objects which are visible to Agree can be rendered invisible to Label.
}
The output of Agree, then, is passed to Label which fails to produce a labelled structure for CI.
\begin{figure}[h]
	\centering
	\begin{forest}
	  nice empty nodes,sn edges,baseline,for tree={
	    calign=fixed edge angles,
	    calign primary angle=-30,calign secondary angle=70
	  }
	  [$\beta$
	    [$\langle$DP$_\varphi\rangle$,name=insitu]
	    [$\alpha$
	      [adj]
	      [\textsc{plat}]
	    ]
	  ]
	\end{forest}
	\caption{The output of Agree for a resP adjunct}
	\label{fig:agree-result-adjunct}
\end{figure}
Specifically, the adjective $\alpha$ cannot be labelled because its would-be labeller $adj_\varphi$ has not been strengthened to provide a label.
Furthermore, the small clause $\beta$ cannot be labelled, as it is a phrase-phrase structure which cannot be labelled under either of its available labelling strategies.
Since DP and $\alpha$ don't agree, $\beta$ cannot receive a feature-pair label, and even if the DP were rendered invisible by Agree, then the label would be the most prominent element in $\alpha$, but, as I just mentioned, that would be the too-weak $adj_\varphi$.
Therefore, the derivation will crash.
So, if we separate Agree from Label, we are able to fix the apparent under-generation, provided we assume Agree operates on chains, rather than occurrences.

Turning to depictives, we can see how they would be allowed in French, given our assumptions.
Taking resultatives and depictives to be minimally different, we can begin to investigate the source of the contrast in grammaticality between the two.
Both are secondary predication constructions, consisting of an eventive VP and a stative small clause.
The difference between the two is the semantic relation between the event and state.
Roughly speaking, resultatives describe an event causing a state, while depictives describe an event coinciding with a state.
I have chosen, following \textcite{kratzer2004building} and \textcite{pietroski2005events}, to assume a \textit{res} head that encodes causation, but I see no compelling reason to assume a \textit{dep} head to encode coincidence.
Most syntacticians assume a \textit{dep} head following \textcite{pylkkanen2008introducing}.
Pylkk\"anen, however, does not argue for the presence of a \textit{dep} head, but posits the head based on the assumption that it is necessary to encode depictive semantics.
I, on the other hand, will assume that no such \textit{dep} head is required.
So, a depictive VP has the structure represented in \cref{fig:FreDepVP}.
\begin{figure}[h]
	\centering
	\begin{forest}
		nice empty nodes,sn edges,baseline
		[AgrOP
			[DP [la viande,roof,name=specagro]]
			[
				[AgrO]
				[VP
					[VP
						[mange]
						[DP,name=theme]
					]
					[SC
						[DP,name=scsubj]
						[crue]
					]
				]
			]
		]
		\draw[->] (scsubj) to[out=south, in=south] (theme);
		\draw[->] (theme) to[out=south west, in=south] (specagro);
	\end{forest}
	\caption{A depictive VP}
	\label{fig:FreDepVP}
\end{figure}
As for the coincidence interpretation, I will postpone that discussion until a \cref{sec:coincidence}.
Note that, as in the case of copular clauses, and unlike the case of resultatives, the DP movement ``chain'' does not cross a phase boundary.
Therefore, the same reasoning that I applied to the copular clause, can be applied to the depictive case.
That is, depictives are allowed in French for the same reason that copular clauses are.

\section{On \textit{wanna}-contraction}\label{sec:wanna}
The proposal that Agree operates on chains, rather than syntactic objects, gains support when we consider a fact about A-bar traces.
As has been noted by several authors \parencite{lightfoot1976trace,jaeggli1980remarks,hornstein1999movement}, A-bar traces block \textit{wanna}-contraction.
\ex.\label{ex:wanna-contraction}
\a.\label{ex:wanna} Who$_i$ do you want to visit $t_i$? $\rightarrow$ Who do you wanna visit?
\b.\label{exwant-to} Who$_i$ do you want $t_i$ to visit Emma? $\rightarrow$ *Who do you wanna visit Emma?

The derivation of \Last[b] involves movement of \textit{who} across a phase boundary, creating a chain which is invisible to Agree.
Consider the structure of \Last[b] in \Next.
\ex. \label{fig:star-wanna-tree}
[$_\gamma$ Who$_i$ [$_\beta$ do$_C$ [$_\alpha$ you want $t_i$ to visit Emma]]]?

Upon $\gamma$ being formed, phase operations are performed on $\alpha$.
When Agree operates on $\alpha$, only the tail of the A-bar chain $\langle$Who$_i$, $t_i\rangle$ is available, meaning it is invisible to Agree.
Since Agree, in addition to valuing features, also deletes copies, t$_i$ will remain in $\alpha$ when it is spelled out, until the rest of $\gamma$ is spelled out.
Assuming morphophonological processes operate on the output of Agree, the input of the contraction process will be the string/structure in \Next.
\ex. you want who to visit Emma.

And assuming adjacency is a precondition for contraction, we wouldn't expect contraction to occur in \Last.

In \ref{ex:wanna}, however the input to contraction is the string/structure in \Next.
\ex. you want to visit who

In this case, \textit{want} and \textit{to} are adjacent (or at least, no phonologically overt material intervenes between them), meaning contraction can occur.

\section{Summary}
The analysis of the resultative parameter developed in \cref{sec:part1} seems to under-generate for languages that lack resultatives.
Specifically it predicts that languages like French should have no copular clauses and no depictives.
In this section, however, I have shown that the appearance of this under-generation was due to the fact that the grammatical architecture was not explicitly described.
In particular, once the nature of the Agree operation and its position in the language faculty was made explicit, I could show that a grammar that rules out resultatives need not rule out copular clauses and depictives.

In this clarified architecture, Agree is taken to be part of Transfer, operating after the Narrow Syntax and before Label.
The proposition that Agree is post-syntactic is assumed, but the hypothesis that it is part of Transfer follows from the fact that the effects of Agree are seen at both interfaces.
I further hypothesized that Agree operates only on complete syntactic objects, as opposed to occurrences of syntactic objects.
This means that if an object has moved across a phase boundary (as is the case for resultatives) then it will be invisible to Agree, and if agreement is required for labelling, then such a movement operation will bleed labelling.
Since the movements required for copular clauses and depictives do not cross phase boundaries, they do not bleed labelling, and therefore do not crash the derivations.
To further justify my hypotheses, I showed that this conception of Agree can be used to give a straightforward account of \textit{wanna}-contraction.

\end{document}

\chapter{Movement from Specifier of resP}\label{sec:ACCing}
%        File: ACCing.tex
%     Created: Tue Feb 21 02:00 PM 2017 E
% Last Change: Tue Feb 21 02:00 PM 2017 E
%
% arara: pdflatex: {options: "-draftmode"}
% arara: biber
% arara: pdflatex: {options: "-draftmode"}
% arara: pdflatex: {options: "-file-line-error-style"}
\documentclass[MilwayThesis]{subfiles}

\begin{document}
\textcite{cinque1996pseudo} discusses pseudo-relatives (PRs) and ACC-ing clauses (ACs) under direct preception verbs and argues that they are three-ways ambiguous.
\ex. 
\a. Ho visto Mario che correva a tutta velocit\'a. (Italian) 
\b. J'ai vu Mario qui courrait \'a tout vitesse. (French)
\c. I saw Mario running at full speed.

According to Cinque, \Last[a] would have three distinct structures 
\ex.
\a. Ho [visto [$_\text{NP}$ Mario [$_\text{CP}$] che correva \ldots ]]
\b. Ho [visto [$_\text{CP}$ Mario [$_{\text{C}^\prime}$ che [$_\text{IP}$ correva \ldots ]]]]
\c. Ho [[visto Mario] [$_\text{CP}$ \textit{ec} che correva \ldots]]

\end{document}



\chapter{Coincidence}\label{sec:coincidence}
%        File: Coincide.tex
%     Created: Sat Apr 14 12:00 PM 2018 E
% Last Change: Sat Apr 14 12:00 PM 2018 E
%
% arara: pdflatex: {options: "-draftmode"}
% arara: biber
% arara: pdflatex: {options: "-draftmode"}
% arara: pdflatex: {options: "-file-line-error-style"}
\documentclass[MilwayThesis]{subfiles}
\begin{document}
In this thesis I have made liberal use of a novel class of sideward movement structures which I schematize below in \cref{fig:SidewardSchema}.
\begin{figure}[h]
	\centering
\[\sbox0{$\begin{array}[]{ccc}
		\begin{forest}
	    nice empty nodes,
	    sn edges,baseline,
	    for tree={
	    calign=fixed edge angles,
	    calign primary angle=-30,calign secondary angle=70}
	    [YP
		    [DP$_i$]
		    [
			    [Y]
			    [ZP]
		    ]
	    ]
	\end{forest}			
	&
	\tikz[baseline=10ex,scale=1] \node[inner sep=0] at (0,-1) {\large,\,};
	&
	\begin{forest}
	    nice empty nodes,
	    sn edges,baseline,
		for tree={
	    calign=fixed edge angles,
	    calign primary angle=-30,calign secondary angle=70}
	    [VP
		    [V]
		    [DP$_i$]
	    ]
	    \end{forest}
		\end{array}$}
\mathopen{\resizebox{1.2\width}{\ht0}{$\Bigg\langle$}}
\usebox{0}
\mathclose{\resizebox{1.2\width}{\ht0}{$\Bigg\rangle$}}
\]
	\caption{A schema of the sideward movement structure}
	\label{fig:SidewardSchema}
\end{figure}
This structure was used as an analysis of adjectival resultatives and depictives, and for one available structure for direct perception reports with ACC-ing clauses.
The distinction between these constructions is due to the choice of head Y.
For resultatives, Y is instantiated by res, while for perception reports, it is instantiated by Prog.
As for depictives, the adjoined phrase YP is a small clause, so Y is either absent or instantiated by a Pred head.
Note that, while the complement of Y, ZP, also varies from construction to construction, this variation can be derived from the selectional requirements of Y.

If we take VPs to be event descriptions, and we assume that host-adjunct structures to compose by predicate conjunction, then a VP adjunct (SC for depictives, resP for resultatives, and ProgP for DPRs) must also be event description.
That is, in \cref{fig:SidewardSchema} the VP and YP are both interpreted as predicates of events, and, because they combine by predicate conjunction, they both describe the same event.
In this chapter, I discuss the interpretation of depictives and adjunct ACC-ing clauses.
As for the interpretation of resultatives, the discussion in \cref{sec:ResInterp} remains sufficient.

In the case of depictives, the interpretation is mostly straightforward, though not without some complications.
Consider \cref{ex:Depictive}, which describes an eventuality in which Natasha eats the fish while that fish is raw.
\ex. Natasha ate the fish raw.\label{ex:Depictive}

According to our analysis, this interpretation would be derived from the fact that the eating event and the rawness state are taken to to be identical.
That is the VP in \cref{ex:Depictive} has a logical form  as derived in \cref{ex:DepictiveLF}
\ex. \textsc{sem}(\textit{eat the fish raw}) =\\
$\lambda e [\textsc{sem}(\textit{eat the fish})(e) \, \&\, \textsc{sem}(\textit{the fish raw})(e)] =$\\
$\lambda e [\textbf{eating}(e)\, \&\, \textbf{raw}(e)\, \&\, \textsc{Theme}(\textbf{the\_fish})(e)]$\label{ex:DepictiveLF}

One might object that this LF is incoherent, as eating is an event, while rawness is a state, and an eventuality cannot be both an event and a state.
This objection, however, does not hold up under scrutiny.
Suppose we take the externalist perspective, according to which the entities that natural language expressions are predicated of are mind-external and -independent entities, and the predicates and concepts of natural language correspond to natural kinds.
From this perspective, eventualities are regions of space-time, some of which are events, while others are states.
So, for instance to utter \cref{ex:EventDesc} truthfully is to refer to a particular region of space-time.
\ex.\label{ex:EventDesc} The officer ticketed the car.

Now, according to the objection at hand, the region of space-time referred to by \cref{ex:EventDesc} is an event, and, therefore, not a state.
However, it is entirely reasonable to assume we could truthfully utter \cref{ex:StateDesc}, a state description, referring to the same space-time region.
\ex.\label{ex:StateDesc} The car was parked illegally.

It seems, then, that, if there is an event/state contrast, it does not originate in the extra-mental world, or else \cref{ex:EventDesc} and \cref{ex:StateDesc} could not possibly refer to the same region of space-time.

Furthermore, many adverbs describe states, yet may modify event descriptions.
If adverbs are adjuncts, then they are interpreted as conjoined with their host, meaning that they will provide a partial description of an event, rather than a state.
Therefore, there doesn't seem to be any contradiction in my analysis of depictives.

The case of adjunct ACC-ing clauses is slightly more challenging, due to a subtlety in their meanings which I will discuss below.
Ultimately, however, their meaning can be explained from their structure, given in \cref{fig:ACCingPair}, and an assumption about the nature of eventualities.
\begin{figure}[h]
	\centering
\[\sbox0{$\begin{array}[]{ccc}
		\begin{forest}
	    nice empty nodes,
	    sn edges,baseline,
	    for tree={
	    calign=fixed edge angles,
	    calign primary angle=-30,calign secondary angle=70}
	    [ProgP
		    [DP$_i$]
		    [
			    [Prog]
			    [VoiceP[DP$_i$ run,roof]]
		    ]
	    ]
	\end{forest}			
	&
	\tikz[baseline=10ex,scale=1] \node[inner sep=0] at (0,-1) {\large,\,};
	&
	\begin{forest}
	    nice empty nodes,
	    sn edges,baseline,
		for tree={
	    calign=fixed edge angles,
	    calign primary angle=-30,calign secondary angle=70}
	    [VP
		    [saw]
		    [DP$_i$[the dog,roof]]
	    ]
	    \end{forest}
		\end{array}$}
\mathopen{\resizebox{1.2\width}{\ht0}{$\Bigg\langle$}}
\usebox{0}
\mathclose{\resizebox{1.2\width}{\ht0}{$\Bigg\rangle$}}
\]
	\caption{An adjunct ACC-ing structure}
	\label{fig:ACCingPair}
\end{figure}
As a first pass, we can say that the interpretation of the structure in \cref{fig:ACCingPair} is a description of an event of the dog being seen and an event of the dog running.
This alone is not sufficient, as an English speaker's intuition regarding the \cref{ex:ACC-ing} is that the $we$ referent saw both the dog and the event of the dog running.
\ex. We saw the dog running.\label{ex:ACC-ing}

This intuition seems to be inescapable; English speakers cannot seem to entertain an interpretation of \cref{ex:ACC-ing} in which \textit{we} saw the dog but not the running event, or the event but not the dog.
This suggests that both complement ACC-ing and adjunct ACC-ing versions of \cref{ex:ACC-ing} are interpreted as both the individual and the event being seen.
The strong version of UTAH that I assume in this thesis, however, predicts that the interpretation \textit{x was seen} can only be encoded if the expression denoting $x$ is merged as the complement of the verb \textit{see}.
In adjunct ACC-ing analysis of \cref{ex:ACC-ing}, as shown in \cref{fig:ACCingPair}, the event denoting expression, ProgP, is adjoined to VP, yet we interpret it as meaning that the event was seen.
Since this interpretation is not directly encoded, we must infer it from what is directly encoded.

To see how we would infer the perception of the event, consider what is directly encoded.
First, the VP is interpreted as a description of a seeing event which the dog is the theme of.
\ex.\label{ex:VPSEM} \textsc{sem}(VP) = $\lambda e [\textbf{see}(e)\,\&\,\textsc{theme}(\textbf{the\_dog})(e)]$

The interpretation of the ProgP, represented in \cref{ex:ProgPStruct}, however, is more complicated.
\ex.\label{ex:ProgPStruct} [$\langle$the dog$\rangle$, [Prog [$_\text{VoiceP} \langle\text{the dog}\rangle$ run]]]

I will make the simplifying assumption that the copy of \textit{the dog} in [Spec, Prog] is semantically vacuous\footnote{
	At this stage, this is purely stipulative.
	I suspect some version of this assumption is true, but a full investigation and justification of it is beyond the scope of this thesis.
} and discuss the Prog-VoiceP structure.
The VoiceP is unremarkable, so I assume its meaning is the complete but tenseless event description in \cref{ex:VoicePSEM}.
\ex.\label{ex:VoicePSEM} \textsc{sem}(VoiceP) = $\lambda e [\textbf{run}(e)\,\&\,\textsc{doer}(e)(\textbf{the\_dog})]$

Prog, then, takes this description as an argument, and ascribes progressive aspect to it.
The standard, though perhaps na\"ive analysis of progressive aspect (as proposed in \cite{klein1994time}) is that Prog takes a description of event $e$ as an argument, introduces a topic time $t$, and asserts that $t$ is included in the run-time of $e$.
This predicts that ProgP encodes the predicate of times in \cref{ex:ProgPSEM1}
\ex.\label{ex:ProgPSEM1} \textsc{SEM}(ProgP) =  $\lambda t \exists e [t \subseteq \textsc{time}(e)\,\&\,\textbf{run}(e)\,\&\,\textsc{doer}(e)(\textbf{the\_dog})]$ (first pass)

However, since ProgP adjoins to VP and the resulting structure is interpreted as a conjunction, ProgP must be interpreted as a predicate of events.
Therefore, I will modify the semantic analysis of Prog such that it introduces an event $e^{\prime}$ and asserts that $e^{\prime}$ is included in $e$.
The final interpretation of the ProgP, is given in \cref{ex:ProgPSEM2} \parencite[cf.][]{bjorkman2018poster}.
\ex.\label{ex:ProgPSEM2} \textsc{sem}(ProgP) =  $\lambda e^{\prime} \exists e [e^{\prime} \subseteq e\,\&\,\textbf{run}(e)\,\&\,\textsc{doer}(e)(\textbf{the\_dog})]$ (second pass)

This interpretation will properly compose with \cref{ex:VPSEM} to yield the interpretation of the host-adjunct structure in \cref{ex:VPProgPSEM}
\ex.\label{ex:VPProgPSEM} \textsc{sem}($\langle$ProgP, VP$\rangle$) = $\lambda e^{\prime} \exists e [ \textbf{see}(e^{\prime})\,\&\,\textsc{theme}(\textbf{the\_dog})(e^{\prime})\,\&\,e^{\prime} \subseteq e\,\&\,\textbf{run}(e)\,\&\,\textsc{doer}(e)(\textbf{the\_dog})]$

So, the event of seeing the dog is included in the event of the dog running, meaning that the seeing occurred in the same space-time region as the running and therefore we can infer that the running event was seen.
This denotation along with the very nature of seeing and running allows us to infer from \cref{ex:VPProgPSEM} that we saw the running event.

One could argue that this analysis is implausible as it requires that the seeing event is a part of the seemingly independent running event.
On its face, this seems to imply an interdependecy between the two events, and, while it seems reasonable to say that the perception event depends on the perceived event, it is far from obvious that the perceived event depends on the perception event.
This line of argumentation, I believe, confuses the issue at hand.

Saying that the perception of an event is a sub-part of that event, does imply that the event is dependent on it being perceived, but it does so in a very weak way.
It is perhaps a truism of set theory and mereology to say that two complex objects are identical only if they consist of the same parts.
So, if $x$ is a part of $e$ but not a part of $e^{\prime}$, the $e \neq e^{\prime}$.
Similarly, a particular running event which is seen by some individual $x$, cannot be identical to a running event which is not seen by $x$.
Note that this does not mean that the unseen running event is not a running event, only that it is a not a seen running event.

That being said, I will now entertain two alternative analyses and discuss their flaws.

Suppose, for instance, that the semantics of Prog is about time rather than eventualities.
This can be attained without the compositionality issues discussed above if we hypothesise the denotation in \cref{ex:ProgTauDenote}.
\ex.\label{ex:ProgTauDenote} \textsc{sem}(Prog) = $\lambda P_{\langle s,t\rangle} \lambda e^{\prime} \exists e [\tau(e^{\prime}) \subseteq \tau(e) \& P(e^{\prime})]$

This denotation would predict that the sentence in \cref{ex:ACC-ing}, under the adjunct ACC-ing interpretation, would mean that there was an event $e1$ of us seeing the dog, and an event $e2$ of the dog running, and that the time of $e1$ is included in the time of $e2$.
And while one could certainly infer that if one sees a dog at the same time as the dog is running, then one sees the running, such an inference does not hold in other cases.
Consider the proposed adjunct ACC-ing interpretation of \cref{ex:PlayingSoccer}, given in \cref{ex:PlayingSoccerTimes}.
\ex.\label{ex:PlayingSoccer} I [[$_{\text{VP}}$ saw Ronaldo$_{i}$][$_{\text{ProgP}}$ $t_{i}$ playing soccer on TV]].

\ex.\label{ex:PlayingSoccerTimes} 
$\begin{array}{rcl}
	\textsc{sem}(\cref{ex:PlayingSoccer}) & =  & \exists e, e^{\prime} [\textsc{Exper}(e)(\textbf{speaker}) \,\&\, \textbf{see}(e) \,\&\, \textsc{theme}(e)(\textbf{R})\\
	& & \&\, \textsc{Agent}(e^{\prime})(\textbf{R}) \,\&\, \textbf{playing\_soccer\_on\_TV}(e^{\prime})\\
	& & \&\, \tau(e) \subseteq \tau(e^{\prime})] 
\end{array}$

While this hypothesized interpretation is consistent with the actual interpretation of \cref{ex:PlayingSoccer}, it is also consistent with some non-existent interpretations.
For instance, \cref{ex:PlayingSoccerTimes} is consistent with a situation in which the speaker sees Ronaldo, while a replay of one of his matches plays on the TV in the other room.

To be more forceful, this hypothesized interpretation of Prog predicts that the illicit sentences in \cref{ex:SlanderedPassive} and \cref{ex:ParodiedPassive}, reproduced below, should be licit.
\ex.* The writer was heard being slandered.\label{ex:SlanderedPassive22}

\ex.* The singer was seen being parodied.\label{ex:ParodiedPassive2}

Since the coincidence between the perception event and the slandering/parodying event is strictly temporal, there is no reason to require that the two events occur in the same room, or even in the same hemisphere.
So, suppose I were sitting at a caf\'e with some singer, while at the same time, ``Weird Al'' Yankovic was recording a parody of that singer in a studio across town.
If Prog merely encodes temporal concidence, then \cref{ex:ParodiedPassive2} should be licit and true of this situation.
These examples, however, are illicit meaning that the temporal coincidence hypothesis does not fare as well as my hypothesis.

Suppose, we instead modify the semantics of perception verbs, such that in the case of adjunct ACC-ing clauses, both the referant of ACC-ing subject and that of the ACC-ing Clause are perceived.
This new hypothesized denotation of \textit{see} is given in \cref{ex:SeeAdHoc}.
\ex.\label{ex:SeeAdHoc}
$
\begin{array}[t]{rcl}
	\textsc{sem}(\text{see}_2) & = & \lambda x_e \lambda P_{\langle s,t\rangle} \lambda e_{s} \exists e^{\prime} [\textbf{see}(e) \,\&\, \textsc{theme1}(e)(x) \\
		& & \&\, \textsc{theme2}(e)(e^{\prime}) \,\&\, P(e^{\prime})]
\end{array}
$

Immediately we can see a number of theoretical issues with this hypothesis.
<++>


\end{document}

\chapter{Conclusion}\label{sec:Conclusion}
%        File: conclusion.tex
%     Created: Wed Nov 28 10:00 AM 2018 E
% Last Change: Wed Nov 28 10:00 AM 2018 E
%
% arara: pdflatex: {options: "-draftmode"}
% arara: biber
% arara: pdflatex: {options: "-draftmode"}
% arara: pdflatex: {options: "-file-line-error-style"}
\documentclass[MilwayThesis]{subfiles}

\begin{document}
The parametric variation of resultative, what I call the resultative parameter, presents a puzzle for linguistic theory:
Children acquire their language's parameter-setting despite a seeming lack of direct evidence in the primary linguistic data.
Since there is no direct evidence, the parameter setting must follow from indirect evidence.
This line of reasoning leads to a two-part research question:
What aspect of a child's PLD provides indirect evidence for the setting of the resultative parameter, and how is the parameter setting deduced from that aspect?
Each part of that question, it turns out, calls for a dissertation-length answer.
The first part is largely answered by William Snyder's dissertation \parencite{snyder1995language} and refined in his later work \parencite{snyder2001nature,snyder2012parameter,snyder2016compound}.
Snyder's answer is that children use the availability of bare stem compounding in their PLD as indirect evidence for the availability of resultatives in their target grammar.
My dissertation begins with this result, and aims to provide an answer to the second part of the question: 
A language may generate both bare stem compounds and adjectival resultatives only if its lexicon has categorizing heads without $\varphi$-features.\footnote{
	This, of course, grossly oversimplifies both the range of possibilities available for lexical variation.
	Restricting ourselves to categorizing heads, we can express the logical range of possibilities as in (i)
	\ex.[(i)] For each category $cat$, a non-empty subset of $\left\{ cat_{\emptyset}, cat_{F} \right\}$ is included in the lexicon.

	Furthermore, the choice of lexicon will certainly affect the grammar in a variety of ways.
	This can be seen by comparing isolating languages such as Niuean, which seem to lack any morphological agreement, to languages such as Italian, which shows a great deal of agreement morphology.
}

Answering the question ``How are resultatives linked to bare stem compounding?'' is a theoretical task, and, as with any theoretical task, it begins with an explicit litany of theoretical assumptions.
I make many of the assumptions standardly made in early \nth{21} century generative syntax (Merge, the Y-model of grammar) and a number of non-standard assumptions.
First, I assume that the $\Theta$-criterion does not fully hold and that an argument may receive multiple $\theta$-roles.
Second, I assume that Merge operates freely, provided that there are two syntactic objects to be combined.
finally, I assume that there is no operation Agree active in the Narrow Syntax.
These assumptions, as I discuss in \cref{sec:nonstandard}, despite being non-standard, actually follow from the logic of the minimalist program.

In order to provide any answer to the question of how resultatives are related to bare stem compounds, we must have an idea of what adjectival resultatives are.
That is, we must give a syntactic analysis of resultatives.
Furthermore, we must provide what I call a parametric analysis---an analysis of how a parameter may be acquired and represented in the grammar.
To that end, I discuss previous analyses in \cref{sec:litreview} before offering my own in \cref{sec:analysis}.
The syntactic analysis I offer, reproduced in \cref{fig:hammer-flat-conc}, is one in which a result phrase is adjoined to the VP and a DP undergoes sideward movement between them.
\begin{figure}[h] 
	\centering
	{\small
	\begin{forest}
	    nice empty nodes,sn edges,baseline,
	    for tree={
	    calign=fixed edge angles,
	    calign primary angle=-35,calign secondary angle=60}
	    [VP
		    [VP
			    [hammer]
			    [DP[the metal,roof,name=compV]]
		    ]
		    [resP
			    [$\langle$DP$\rangle$,name=specRes]
			    [res$^{\prime}$
				    [res]
				    [SC
					    [$\langle$DP$\rangle$,name=SCDP]
					    [flat]
				    ]
			    ]
		    ]
	    ]
	    \draw[->] (SCDP) to[out=south west, in=south] (specRes);
	    \draw[->] (specRes) to[out=south, in=south] (compV);
	\end{forest}
	}
	\caption{The structure of resultatives}
	\label{fig:hammer-flat-conc}
\end{figure}
The parametric analysis, I offer is based on a similar one by \textcite{kratzer2004building}.
According to this analysis, the presence of bare stem compounding in a child's PLD signals that the child's lexicon should admit categorizing heads without $\varphi$-features.

In order to show that resultatives depend on $\varphi$-less heads, we must show that a structure such as \cref{fig:hammer-flat-conc} can be derived only if the lexicon contains $\varphi$-less categorizing heads.
I do so in \cref{sec:deriving}, but only after discussing the latest iteration (at least at the time of this thesis) of Chomsky's syntactic theory---label theory---in \cref{sec:labels}.
According to label theory, a syntactic derivation only converges if the structure it creates can be unambiguously labelled.
In \cref{sec:deriving}, I show that the structure in \cref{fig:hammer-flat-conc} can be derived and labelled if the result adjective \textit{flat} is categorized by a $\varphi$-less head $adj_{\emptyset}$.
I then show that if \textit{flat} is categorized by $adj_{\varphi}$, the derivation either fails or creates an unlabellable structure.
Thus I have answered the question at hand.

In part II, I bring to the forefront the apparently loose theoretical ends left by Part I.
Rather than tie these loose ends up with auxiliary or ad-hoc hypotheses, I investigate how they might inform our theory of the language faculty.
In \cref{sec:FreSC}, I argue that an apparent undergeneration problem of my proposed theory is actually due to the lack of a suitable theory of feature agreement.
Such a theory, I propose, is one in which agreement occurs postsyntactically.

In \cref{sec:ACCing}, I point out an odd fact---that movement from [Spec, res] to [Comp, V] in \cref{fig:hammer-flat-conc} seems to be obligatory---and show that it seems to generalize to other cases of sideward movement---objects in the specifier of adjoined phrases must move to the host phrase.
I argue that this fact is odd only if we make the standard (although often tacit) assumption that grammaticality is determined within the Narrow Syntax.
Under an interface-based theory, such as label theory, this fact can be accounted for.

Finally, in \cref{sec:coincidence} I discuss a semantic question raised by my proposal.
I assume that primary and secondary predicates compose via something like predicate modification. 
This mode of composition leads to what initially seem to be odd interpretations in which the events described by the two predicates are in fact the same event.
I argue that despite this apparent oddness, there is no other principled way of interpreting the structures in question (\textit{i.e.}, resultatives, depictives, and some direct perception reports), and furthermore, that the apparent oddness is only apparent.
A closer look at both structures and the ontology of eventualities significantly diminishes this oddness, but a closer investigation will be needed to corroborate the predicted interpretation.

There are, of course, loose ends left by this thesis, which will have to be tied up in later investigations.
My starting points for the thesis were works on adjectival resultatives by \textcite{snyder1995language,snyder2001nature,snyder2016compound} and \textcite{kratzer2004building} who drew a strong correlation between adjectival/nominal inflection and adjectival resultatives.
As with all empirical generalizations, there are exceptions to this correlation.
Exceptions, of course, are tricky things in any scientific inquiry.
They can either strengthen a theory or destroy it, and there is no way to tell which they will do without a full analysis

For instance, Italian, which is one of the prototypical *resultative languages, does seem to generate a form of adjectival resultative, but only under fairly restrictive conditions.
\textcite{napoli1992secondary}, for instance, gives the following examples of Italian adjectival resultatives.
\ex.
\a. Ha dipinto la macchina rossa.\\
``He painted the car red.''
\b. 
	\a. Ho stirato la camicia piatta piatta.\\
	``I ironed the shirt flat flat.''
	\b.* Ho stirato la camicia piatta.
	\z.
\z.

\textcite{folli2005prepositions} add to this list the following cases, in which the result AP is intensified with \textit{troppo}.
\ex.
\a. Gianni ha cucito la camicia *(troppo) stretta.\\
``John sewed the dress *(too) tight.''
\b. Gianni ha sciolto il cioccolato *(troppo) liquido.\\
``John melted the chocolate *(too) liquid.''

\textcite{napoli1992secondary} suggests a semantic/pragmatic analysis; namely, that Italian only allows resultatives when ``the verb can be interpreted as focusing on the endpoint of its activity'' (p75).
\textcite{folli2005prepositions}, on the other hand, suggest a syntactic analysis; Italian only allows resultatives when the result AP is complex.
Neither analysis is complete, though, and the case of Italian resultatives remains a puzzle.

If \citeauthor{napoli1992secondary} is correct, and the case of Italian resultatives is to be given a semantic/pragmatic analysis, then my syntactic explanation of the resultative parameter will face some difficulties.
If, on the other hand, \citeauthor{folli2005prepositions} are correct that this exception is to be given a syntactic analysis, then perhaps it will only strengthen my proposal.

\textcite{whelpton2007building}, addressing Kratzer's (\citeyear{kratzer2004building}) proposal, presents Icelandic as a possible counterexample. 
Recall that Kratzer's analysis was that resultatives could only be derived if the result adjective was uninflected, a proposal that is compatible with mine.
Whelpton shows that, while Icelandic allows resultatives, it also seems to require inflectional morphology on result adjectives as in the following examples.
\ex.
\ag. \'Eg k\'yldi l\"ogguna kalda.\\
I.Nom punched cop.the.FSgAcc cold.FSgAcc\\
``I punched the cop out cold.''
\bg.J\'{a}rnsmi\dh{}urinn hamra\dh{}i \'{a}lminn flatan.\\
blacksmith.the hammered metal.the.MSgAcc flat.MSgAcc\\
``the blacksmith hammered the metal flat.''
\bg. D\'{o}ra \ae{}pti sig h\'{a}sa.\\
D\'{o}ra screamed herself.FSgAcc hoarse.FSgAcc\\
``D\'{o}ra screamed herself hoarse.''

However, Whelpton also notes that Icelandic, unlike a prototypical *resultative	language, does have bare stem compounding.
Indeed, it has bare stem compounding that is interpreted as resultatives as in the following examples where bare result adjectives are compounded with deverbal adjectives.
\ex. 
\a. svart-lita\dh{}ur\\
black-coloured.mSgNom
\b. \th{}unnsneiddu sveppirnir\\
thin-cut.MPlNom mushrooms.the
\b. f\'{i}nmuldu piparkornin\\
fine-ground.NPlNom peppercorns.the
\b. hreinskr\'{u}bbu\dh{}u p\"{o}nnurnar\\
clean-scrubbed.FPlNom pans.the
\b. mj\'{u}kbr\ae{}dda s\'{u}kkula\dh{}i\\
soft-meltedNSgNom chocolate

Whelpton presents this and other data as a rebuttal to Kratzer's (\citeyear{kratzer2004building}) analysis but offers no deep analysis or counter-proposal.
Without a deeper analysis, it is difficult to estimate the importance of his data as counterevidence to my proposal.
Therefore I leave it to further investigation.

In addition to possible counterexamples, there are a number of phenomena related to adjectival resultatives which may be amenable to an analysis/explanation along the lines of what I propose here.
First off, there is the case of directionalized locatives such as one of the (a) reading of \Next.
\ex. Kate kicked the ball between the posts.
\a. $\approx$ Kate kicked the ball such that it passed/landed between the posts. (directionalized)
\b. $\approx$ Kate stood between the posts and kicked the ball. (plain locative)

While these PPs are standardly assumed to be PathPs like PPs headed by, say, \textit{through} or \textit{around}, I argue elsewhere \parencite{milway20xxmodifying} that such an assumption is unfounded.
Rather, directionalized locatives are perhaps analyzable as PP resultatives based on their semantics.
Furthermore, they show a parametric variation similar to that of adjectival resultatives.
So, Germanic languages seem to have directionalized locatives, but Romance languages do not.
There are, however, reports that certain varieties of Acadian French allow directionalized locatives.
For instance, according to Ruth King and Yves Roberge \parencite[p.c. cited in][253--254]{rooryck1996prepositions} report that sentences like \Next, while they only receive a plain locative reading in Metropolitan and Laurentian French, receive a directionalized locative reading in PEI French.
\ex.La bouteille flottait [sous le pont].\\
The bottle floated under the bridge. \parencite{rooryck1996prepositions}

In previous work \parencite{milway2015generals}, I hypothesized that this could be linked to the fact that, unlike Metropolitan and Laurentian French, Acadian French tends to allow P-stranding.
So, for instance, the sentences in \Next are acceptable in PEI French but ungrammatical in most other varieties of French.
\ex.
\ag. Le ciment a \'{e}t\'{e} march\'{e} dedans.\\
the cement has been walked in\\
``The cement was walked in''
\bg. O\'{u} il vient de?\\
where he comes from\\
``Where does he come from?'' \parencite{roberge2013preposition}

A full analysis and explanation would require an in-depth empirical study, perhaps of the sort \textcite{snyder1995language} performed on adjectival resultatives.
I leave such a study for future research.

Secondly, there are serial verb constructions (SVCs) of the type studied by \textcite{stewart2013serial,bakerstewart1999double}.
Consider, for example, the Edo SVC in \Next.
\exg. \`{O}z\'{o}  gh\'{a}d\`{i}y\'{a}n    r\`{e}.\\
Ozo  FUT   buy   yam    eat\\
``Ozo will buy yams and eat them.'' \parencite{bakerstewart1999double}

SVCs and resultatives are similar in that both involve a single argument shared between two predicates which are related to each other by more than mere coincidence.
So, in \Last, the buying event is a prerequisite of the eating event, and the latter is, in some sense the goal of the former, and yams are the theme of both events.
Also like resultatives, SVCs are parameterized, though they are rarer typologically than resultatives.
Indeed, \textcite{stewart2013serial} proposes that the two constructions are linked and that the SVC parameter may be a subparameter of the resultative parameter.
Further research would be required to integrate my results with those of \textcite{stewart2013serial,bakerstewart1999double},

Finally, there is the case of Romanian bare noun resultatives\footnote{\textcite{irimia2012secondary} calls these ``bare noun \textit{pseudo}results.''} as discussed by \textcite[220--224]{irimia2012secondary} and \textcite{farkas2011predicative}.
Romanian, like other Romance languages, disallows adjectival resultatives as shown in \Next.
\exg. *Femeia a cur\u{a}\cb{t}at casa str\u{a}lucitoare.\footnotemark\\
Woman.the has cleaned.PstPrt house.the spotless.FSg\\
``The woman cleaned the house spotless.''\parencite{irimia2012secondary}

Unlike the other Romance languages, however, Romanian has a bare nominal resultative as shown in \Next.
\footnotetext{\textcite{irimia2012secondary} reports that this sentence is grammatical in Romanian, but only receives a depictive reading.}
\exg. Studentul s -a sup\u{a}rat foc.\\
student-the CL.3ReflAcc has get angry.Perf fire\\
``The student has got so angry that he became as red as fire.''\parencite{farkas2011predicative}

As the name suggests, the result nominal in a bare nominal resultative, despite the fact that Romanian allows nominal inflection.
\ex. a se sup\u{a}ra foc/*focul/*un foc/*focuri/*focurile\\
``to get angry fire/fire-the/a fire/fires/fires-the'' \parencite{farkas2011predicative}

Although this clashes with the generalization that Romance languages disallow resultatives, it is entirely consistent with my proposal.
If we propose that the Romanian lexicon has the $n_{\emptyset}$ head but not the $adj_{\emptyset}$ head, then the bare noun resultative can be integrated with my analysis.
This does, however, raise the question of why the bare noun resultative does not show up in other languages.
I leave this question to future research.


The proposals made here are, of course, provisional as is the case for any scientific proposal.
That is, they are subject to revisions, clarifications, and perhaps outright refutation.
That said, I believe that with this thesis I have made two broad contributions to the ongoing study of the human language faculty.
First, I have presented a template for the explanation of parametric variation, especially parametric semantic variation.
Such variation can be explained by first finding a surface correlate of that parameter, and then showing how that correlate can be connected to the parameter.
Second, I have incrementally developed the theory of the language faculty by identifying and fixing flaws in our understanding of such things as the syntax-semantics interface and adjunction.
The flaws were found by applying the logic of the minimalist program to these domains, as were the proposed solutions to those flaws.
I believe my solutions to be intriguing and suggestive, but they may, of course, be dead ends.
The flaws, themselves, however, are more important; they represent domains that we previously thought we understood.
Finding gaps in our understanding, such as these, is what makes scientific inquiry worth it.
A failure of understanding is merely an opportunity to understand.
\end{document}



\printbibliography[heading=bibintoc]
\end{document}


