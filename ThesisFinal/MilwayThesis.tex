%        File: Milway Thesis.tex
%     Created: Mon Jan 02 03:00 PM 2017 E
% Last Change: Mon Jan 02 03:00 PM 2017 E
%
% arara: pdflatex
% arara: biber
% arara: pdflatex
% arara: pdflatex
\documentclass[
%	draft,
	12pt,
	twoside,
	narrowmargins
	]{ut-thesis}

%\usepackage[cmintegrals,cmbraces]{newtxmath}
%\usepackage{ebgaramond-maths}
\usepackage[T1]{fontenc}

%\usepackage[margin=1in]{geometry}
\usepackage[british]{babel}
\usepackage[]{csquotes}
\usepackage[
	backend=biber,
	style=authoryear,
	citestyle=authoryear-comp,
	language=british,
	dashed=true
]{biblatex}
\DeclareLanguageMapping{british}{british-apa}
\DeclareLabeldate{%
	\field{year}
  \field{date}
  \field{eventdate}
  \field{origdate}
  \field{urldate}
  \field{pubstate}
  \literal{nodate}
}
\renewbibmacro*{addendum+pubstate}{%
  \printfield{addendum}%
  \iffieldequalstr{labeldatesource}{pubstate}{}
  {\newunit\newblock\printfield{pubstate}}}
\usepackage[]{graphicx}

\usepackage[]{tipa}
\usepackage{stmaryrd}
\usepackage[]{amsmath}
\usepackage{amsfonts}
\usepackage{amssymb}
\usepackage{amsthm,thmtools}
\usepackage{mathtools}


\usepackage{nth}
\usepackage{combelow}

%\usepackage{marginnote}

\usepackage{enumitem}

\theoremstyle{definition}
\newtheorem{defn}{Definition}[chapter]

\usepackage{tikz}
\usepackage{forest}
\usepackage[]{longtable}
\usepackage{tabu}
\usepackage{subcaption}
\usepackage{caption}
\newcommand{\repeatcaption}[2]{%
	\renewcommand{\thefigure}{\ref{#1}}
	\captionsetup{list=no}
	\caption{#2 (repeated from page \pageref{#1})}
	\addtocounter{figure}{-1}
}
\usepackage[final,hidelinks]{hyperref}
\usepackage{linguex}
\makeatletter
\apptocmd{\gl@stop}{\nobreak}{}{}
\makeatother
\renewcommand{\Exarabic}{\thechapter.\arabic} 
% We want to reset the ExNo counter at each section
\usepackage{chngcntr}
\counterwithin{ExNo}{chapter}
\usepackage{cleveref}
\crefname{ExNo}{}{}
\creflabelformat{ExNo}{(#2#1#3)}
\crefname{SubExNo}{}{}

\crefname{figure}{figure}{figures}
\Crefname{figure}{Figure}{Figures}

\crefname{defn}{definition}{definitions}
\Crefname{defn}{Definition}{Definitions}

\usepackage{centernot}
%\usepackage{subfiles}
\usepackage{multirow}
\usepackage[normalem]{ulem}
\usepackage{xcolor}
%\listfiles

\usetikzlibrary{positioning,arrows}
\useforestlibrary{linguistics}

\usepackage{epigraph}

%\DeclareNameFormat{labelname:poss}{% Based on labelname from biblatex.def
%  \ifcase\value{uniquename}%
%  \usebibmacro{name:last}{#1}{#3}{#5}{#7}%
%  \or
%  \ifuseprefix
%  {\usebibmacro{name:first-last}{#1}{#4}{#5}{#8}}
%  {\usebibmacro{name:first-last}{#1}{#4}{#6}{#8}}%
%  \or
%  \usebibmacro{name:first-last}{#1}{#3}{#5}{#7}%
%  \fi
%  \usebibmacro{name:andothers}%
%  \ifnumequal{\value{listcount}}{\value{liststop}}{'s}{}
%}
%
%\DeclareFieldFormat{shorthand:poss}{%
%  \ifnameundef{labelname}{#1's}{#1}
%}
%
%\DeclareFieldFormat{citetitle:poss}{\mkbibemph{#1}'s}
%
%\DeclareFieldFormat{label:poss}{#1's}
%
%\newrobustcmd*{\posscitealias}{%
%  \AtNextCite{%
%    \DeclareNameAlias{labelname}{labelname:poss}%
%    \DeclareFieldAlias{shorthand}{shorthand:poss}%
%    \DeclareFieldAlias{citetitle}{citetitle:poss}%
%    \DeclareFieldAlias{label}{label:poss}
%  }
%}
%
%\newrobustcmd*{\posscite}{%
%  \posscitealias%
%  \textcite
%}
%
%\newrobustcmd*{\Posscite}{\bibsentence\posscite}
%
%\newrobustcmd*{\posscites}{%
%  \posscitealias%
%  \textcites

\usepackage[multiple]{footmisc}

\newcommand\quelle[1]{{%
  \unskip\nobreak\hfil\penalty50
  \hskip2em\hbox{}\nobreak\hfil#1%
  \parfillskip=0pt \finalhyphendemerits=0 \par
}
}

\newcommand{\AREx}{%
  \a.[(\textbf{AR})]\label{ex:AREx} Natalie hammered the metal flat.
  \z.
}
\newcommand{\rcommentg}[1]{\hfill\raisebox{1.2\baselineskip}[0pt][0pt]{#1}}
\makeatother
\renewcommand{\thmtformatoptarg}[1]{ -- #1}
\renewcommand{\listtheoremname}{List of definitions}
\setcounter{secnumdepth}{3}
\frenchspacing
\graphicspath{ {./img/} }
\addbibresource{../Thesis.bib}
\degree{Doctor of Philosophy}
\department{Linguistics}
\gradyear{2019}
\author{Daniel A. Milway}
\title{Explaining the Resultative Parameter}
\begin{document}
\begin{preliminary}
  \maketitle
  
%
% arara: pdflatex: {options: "-draftmode"}
% arara: biber
% arara: pdflatex: {options: "-draftmode"}
% arara: pdflatex: {options: "-file-line-error-style"}
\documentclass[MilwayThesis]{subfiles}

\begin{document}
\begin{abstract}
	This thesis proposes an explanatory account for the fact that some, but not all languages generate adjectival resultatives.
\end{abstract}
\end{document}

  \begin{acknowledgements}
	  This document was a long time in the making, and it certainly couldn't have happened without a supervisor with the generosity and intellectual dexterity of Elizabeth Cowper.
	  Elizabeth, who was already retired by the time I asked her to be my supervisor, never seemed to be fazed by supervising or advising on projects that were outside her wheelhouse, even though those projects were decidedly incompatible with each other.
	  But, my history with her goes farther back than the start of my thesis work, to LIN331 where she introduced me to the minimalist program, and later, in that time between undergrad and the MA when she met with me just to talk about syntax.
	  Between the course and those meetings, I'm quite certain I wouldn't be where I am without her.

	  I am also indebted to Diane Massam and Michela Ippolito, who, with Elizabeth, formed my thesis committee. 
	  My meetings with them helped me to ground my research and avoid tunnel-vision, and more generally helped keep me honest.
	  I also want to thank the rest of my defense panel: Susana B\'{e}jar, Nick LaCara, and Norbert Hornstein, for making my defense lively and actually fun.

	  Beyond my thesis committee I am indebted to the entire Linguistics Department at UofT.
	  Thank you to the faculty in general and the syntax and semantics faculty in particular for creating a model of a collegial atmosphere to nurture research and researchers.
	  Thank you to my fellow grad students, who were the best support network a young researcher could ask for.
	  And last but certainly not least, thank you to the administrative staffers who made the department run: Mary Hsu, Jill Given-King, Jennifer McCallum, and Deem Waham.

	  Research, of course, is only one aspect of graduate school; there's also teaching, TA-ing, and grading to fill our days.
	  There are too many to mention by name here, but I want to thank all the people I TA-ed for, and with, and those people who TA-ed for me.
	  I also must acknowledge my union, CUPE 3902, which tirelessly worked to protect and advocate for me and my fellow workers.
	  I came into grad school skeptical of unions in general but that skepticism evaporated in 2015 when the collective bargaining process and ensuing four week strike revealed the true colours of both the union and the university administration.
	  The union revealed itself to be supportive, open, and protective of all its members, while the administration revealed itself to be almost the opposite.
	  It definitely stung to see the administrators of the university that I had spent the better part of a decade at (undergrad and grad school) outright lie about me and my fellow workers, and try to pit us against each other, undergrads, and faculty, but that sting was lessened by the sense of solidarity I felt on the picket lines with my fellow members of that university.
	  The strike was a difficult time but it did remind me that, as is the case with any organization, the university is defined by its membership, not its administration.

	  When I was dithering over whether or not to pursue a PhD, wondering if it was worth it, Derek Denis, then just a grad student, pointed out one of the great side-benefits of the PhD: real lasting friendships.
	  Perhaps it's too early to say, but I think he was right, I've found a good number of people that I feel grateful to call friends.
	  Not least of all, I'm grateful to have had so many wonderful cohort-mates: Becky Tollan, Clarissa Forbes, Tomohiro Yokoyama, Jess Denniss, Emily Blamire, Michelle Yuan, and Jada Fung.
	  I'm grateful to have met them and to have shared the ups and downs of graduate school with them.
	  Thank you also to the many other wonderful people I got to know as fellow grad students from Joanna Chociej, Sarah Clarke, Richard Compton, Ail\'is Cournane, Radu Craioveanu, Derek Denis, Liisa Duncan, Ross Godfrey, Yu-Leng Lin, Mercedeh Mohaghegh, Alex Motut, Kenji Oda, Will Oxford, Christopher Spahr, Eugenia Suh, and others, who were the towering senior grad students who helped me to figure things out as I got started, to Julien Carrier, Emily Clare, Julie Doner, Erin Hall, Shayna Gardiner, Ruth Maddeaux, Emilia Melara, Patrick Murphy, and Na-Young Ryu, who were junior grad students figuring things out alongside me, to all the soon-to-be senior grad students, some of whom I was lucky enough to befriend, all of whom I'm proud to have shared a department with.
	  I hope that the line of graduate students continues on, being there for each other in rough times, celebrating achievements together, and just blowing off steam at pub night.

	  Thank you to Dawn Whitwell for teaching me how to write jokes and giving me a creative outlet other than linguistics. Thank you to my friends Shawna Edward, Lisa Feingold, and Dan Reilly (aka ``The Bit Acknowledgers'').
	  Thank you to the many people who put out the podcasts that informed me, made me laugh and generally helped me get through the PhD with my sanity:
	  Scott Aukerman, Adam Scott, Paul F. Tompkins, Dave Shumka, Graham Clark, Sean Clements, Hayes Davenport, John Hodgman, Jesse Thorn, Greg Kot, Jim DeRogatis, Howard Kremer, Kulap Vilaysack, Tim Batt, Guy Montgomery, Mike Mitchell, Nick Wiger, Jimmy Pardo, Matt Belknap, Garon Cockrell, Eliot Hochberg, Leo Laporte, Jeff Jarvis, Stacey Higginbotham, Gina Trapani, Julia Prescott, Allie Goertz, Lauren Lapkus, Russ Roberts, Jeremy Scahill, Jessica St Clair, Lennon Parham, and Andy Daly.


	  I'd like to thank my siblings and their families:
	  my brother Tom, his wife Karen, and their children Declan, Rose, Violet, and Sadie;
	  my brother Mike, his wife Brigid, and their children Imogen and Desmond;
	  my sister Joan and her partner Jared;
	  my brother Peter.


	  I'd like to thank Zoe, who came into my life only recently, but whose love and support has made this milestone all the sweeter.

	  Finally, Thank you to my parents Jim and Sheila.
	  I wouldn't be where I am now, I wouldn't have achieved anything worthwhile, I wouldn't be the person I am today were it not for their love and support. 
  \end{acknowledgements}
  \tableofcontents
  \listoftables
  \listoffigures
  \listoftheorems[ignoreall,show={defn}]
\end{preliminary}
\chapter{Introduction}
%        File: intro.tex
%     Created: Thu Jun 22 04:00 PM 2017 E
% Last Change: Thu Jun 22 04:00 PM 2017 E
%
% arara: pdflatex: {options: "-draftmode"}
% arara: biber
% arara: pdflatex: {options: "-draftmode"}
% arara: pdflatex: {options: "-file-line-error-style"}
\documentclass[MilwayThesis]{subfiles}

\begin{document}
This thesis asks a seemingly simple question: Why do some but not all languages allow their users to generate adjectival resultatives?
I write ``\textit{seemingly} simple'' for reasons that are likely obvious to anyone reading this, but will clarify the reasons as a way of introducing the actual content of the thesis that follows.
There are, as far as I can tell, two complications inherent in questions of the form \textit{Why P?}: one linguistic, one metaphysical.
The linguistic complication is that \textit{Why P?} presupposes that \textit{P}.
So, in order to answer, or even be justified in asking, the question at hand, we must first demonstrate that there is a class of expressions that can be called adjectival resultatives, and that they are not found in every language.
The metaphysical complication is due to the fact that a given \textit{Why} question has an indefinite number of true responses, yet the appropriate response depends on the level of explanation that is sought.
So, In order to answer the stated question, we must clarify the level of explanation we are seeking.

Beginning with the presupposition: is it the case that some but not all natural language grammars generate adjectival resultatives?
The first thing we need to answer that question is a working definition of adjectival resultatives, which I give in \cref{def:AdjRes} and which, in turn, depends on the definition of \textit{secondary predicate} in \cref{def:SecPred}.
\begin{defn}[Adjectival Resultative]\label{def:AdjRes}
	An \textit{adjectival resultative} is a secondary predication structure, whose secondary predicate is an adjective (phrase) and is interpreted as describing the a state directly caused by the event described by the primary predicate.
\end{defn}
\begin{defn}[Secondary Predication]\label{def:SecPred}
	A \textit{secondary predication} structure is a monoclausal structure containing constituent consisting of a verb (phrase) (V) an argument (DP) and another element (SP) such that\\
	SP is interpreted as a predicate,\\
	and DP is an argument of both V and SP.
\end{defn}

A canonical example of an adjectival resultative is give in (AR).
\AREx

This is a secondary predication structure in the sense that it contains a constituent \textit{hammer the metal flat} which contains a verbal and an adjectival predicate (\textit{hammer} and \textit{flat}, respectively) and a DP \textit{the metal} which is an argument of both predicates.
Furthermore, it is an adjectival resultative because its secondary predicate is the adjective \textit{flat}, which describes a state caused by the hammering event.
Resultatives contrast minimally with depictives, secondary predication structures whose secondary predicate discribes a state not caused by the primary predicate.
The sentence in \cref{ex:DepictiveCanon}, is a canonical example of a depictive.
\ex.\label{ex:DepictiveCanon} Heather ate the fish raw.

This is an example of secondary predication, with \textit{ate} being the verb, \textit{raw} being the secondary predicate, and the argument \textit{the fish} being shared between the two.
It is not a resultative because, in the situation it describes, the rawness state is in no way caused by the eating event.
So, part of the presupposition is true: Resultatives exist in a least one language.
\textcite{snyder1995language,snyder2001nature}, however, demonstrates that adjectival resultatives exist in a number of other languages, including ASL, Dutch, German, Khmer, Japanese, Korean, Hungarian, Mandarin, and Thai.
\ex. Examples of grammatical resultatives\footnote{Unless otherwise noted, these examples are drawn from \textcite{snyder2001nature}}\\
\a. J-O-H-N PAINT CHAIR RED\hfill\parencite[ASL,][]{kentner2018wh}\\
``John painted the chair red.''
\bg. Hans h\"ammert das Metall flach.\\
Hans hammered the metal flat\\
``Hans hammered the metal flat.'' \hfill \raisebox{1.4\baselineskip}[0pt][0pt]{(German)}
\bg. A munk\'as lapos-ra kalap\'acsolta a f\'emet.\\
the worker flat-\textsc{trans} hammer-\textsc{pst} the metal\\
``The worker hammered the metal flat.'' \rcommentg{(Hungarian)}
\bg. John-ga teeburu-o kiree-ni hui-ta.\\
John-\textsc{nom} table-\textsc{acc} clean wipe-\textsc{pst}\\
``John wiped the table clean.''\rcommentg{(Japanese)}
\bg. Kira wai daik kpaet.\\
Kira hit metal flat\\
``Kira beat the metal flat.''\rcommentg{(Khmer)}
\bg. John-i teibl-ul kekuti tak-at-ta\\
John-\textsc{nom} table-\textsc{acc} clean polish-\textsc{pst}-\textsc{complementizer}\\
``John wiped/polished the table clean.''\rcommentg{(Korean)}
\bg. Ta ba tie guan da ping.\\
(s)he \textsc{ba} iron pipe hit flat\\
``(S)he beat the iron pipe flat.''\rcommentg{(Mandarin [tones omitted])}
\bg. Ja: t'up lo:ha\textglotstop{} haj 

Furthermore, Snyder demonstrates that a number of languages seem to be incapable of generating adjectival resultatives, expressing resultatives periphrastically instead.
\ex. <StarResultatives+>

Our presupposition, then, seems to hold; some, but not all languages exhibit adjectival resultatives.

Our second issue---that of deciding what we mean by \textit{why}---I believe is a far more interesting one, as answering it requires us to be explicit about the broader goals of our inquiry.
If our interest is historical linguistics or language variation and change, then we might be interested in the migration patterns and language contact situations, and how they do or do not correlate with a language's ability to generate resultatives, or with the social factors linked to resultatives.
This thesis, however, is a work of largely theoretical generative syntax, so our \textit{why} question is actually two questions: What essential property (or properties) do grammars that generate resultatives have that grammars that do not generate resultatives lack? And how is that property (or set of properties) acquirable by children from the primary linguistic data?
Note that I have framed the acquisition question as dependent on the grammatical question---likely a reflection of my training as a syntactician---but I don't believe that one question is \textit{logically} prior to the other.
\textcite{snyder1995language,snyder2001nature}, for instance, takes the grammatical question to be dependent on the acquisition question.
I believe the questions are interdependent, meaning that the correct answer to one should at least be consistent with the correct answer to the other.
The simplest situation, however, would be that the correct answer to each question entails the correct answer to the other; in other words, that a single statement would provide an answer to both questions.

For reasons that have little to do with the content of linguistic theory or its empirical base, and a great deal to do with the social, cultural, and political atmosphere of modern scientific research, the two questions that I pose above are not commonly treated as interdependent.
Syntacticians focus on the grammatical question, and consider the acquisition question to be secondary at best, while acquisitionists consider the reverse to be the case.
This leads to syntactic proposals where the acquisition question is ignored or treated as an afterthought, and acquisition studies which do not fully address how their results could be integrated into linguistic theory.
With this thesis, I hope to avoid this pitfall.
That is, I aim to develop a theoretical explanation of the resultative parameter that takes the acquisition question to be a crucial criterion for the success of my proposal; in other words, I assume that a grammatical theory of resultatives is adequate only if it answers the acquisition.
This statement is likely uncontroversial among generative syntacticans, indeed it is perhaps an unstated criterion of all generative syntax.
I make it a stated criterion here as a way of ensuring that readers can hold me to it.

The answer I argue for is that a grammar generates resultatives only if it also generates adjectives without $\varphi$-features.
I argue that this parameter is both acquirable and consistent with Snyder's results.
Since $\varphi$-features on adjectives manifest themselves as agreement morphology, their presence/absence is directly detectable by a language acquisition device, and therefore acquirable.
The demonstration that Snyder's results can be derived from the presence/absence of $\varphi$-fetures on adjectives, however, requires a great deal of theoretical discussion, which is beyond the scope of this introduction.

\section{Theoretical Context}
The general theory that I assume here is a variety of what is called \textit{minimalist syntax} after Chomsky's (\citeyear{chomsky1995minimalist}) \textit{The Minimalist Program}.
Using the term \textit{minimalism} to refer to a theory of grammar, however, is perhaps incorrect, as minimalism is a metatheoretical position.
The contrast between theory and metatheory that I assume here is due to \textcite{chametzky1996theory}\footnote{
	Chametzky makes a three way distinction between metatheoretical, theoretical, and analytic work:
	\begin{quote}
		\textit{Metatheoretical} work is theory of theory, and divides into two sorts: general and (domain) specific.
		\textit{General} metatheoretical work is concerned with developing and investigating adequacy conditions for any theory in any domain.
		So, for example, it is generally agreed that theories should be (1) consisted and coherent, both internally and with other well-established theories; (2) explicit; and (3) simple. This sort of work is philosophical in nature \dots.
		\textit{Specific} metatheoretical work is concerned with adequacy conditions for theory in a particular domain.
		So, for example, in linguistics	we have Chomsky's (1964; 1965) familiar distinctions among observational, descriptive, and explanatory adequacy.
		Whether such work is ``philosophy'' or, in this case, ``linguistics'' seems to me a pointless question.
		
		\textit{Theoretical} work is concerned with developing and investigating primitives, derived concepts, and architecture within a particular domain of inquiry. 
		This work will also deploy and test concepts developed in metatheoretical work against the results of actual theory construction in a domain, allowing for both evaluation of the domain theory and sharpening of the metatheoretical concepts. 
		Note this well: deployment of metatheoretical concepts is \textit{not} metatheoretical work; it is theoretical work.
		
		\textit{Analytic} work is concerned with investigating the (phenomena of the) domain in question.
		It deploys and tests concepts and architecture developed in theoretical work, allowing for both understanding of the domain and sharpening of the theoretical concepts. 
		Note this well: deployment of theoretical concepts is \textit{not} theoretical work, it is analytic work. 
		Analytic work is what overwhelmingly most linguists do overwhelmingly most of the time.
		This is as it should, and indeed must, be: an empirical discipline only exists insofar as there is a community of scientists investigating the domain.
		For linguistics to be the science of language, this must be where linguists do their work.
		\parencite[xvii\textit{ff}]{chametzky1996theory}
	\end{quote}
	}
This distinction is evident when one considers the stark contrasts between the theories of grammar that are referred to as \textit{minimalist}.
For instance, \textcite{chomsky2000minimalist,hornstein2009theory,frampton2008crash,epstein2006derivations,borer2005name,borer2005normal,borer2013taking} all develop distinct minimialist theories of syntax.
They all, however, share a set of assumptions, likely due to their shared chomskyan heritage.
Since this thesis shares that heritage, it also shares those assumptions which I will list and explain below.

Most fundamentally, minimalist theories of grammar share a model of the language faculty called the \textit{Y Model} or the \textit{T Model}.
In this model, a narrowly syntactic ``module'' operates on items drawn from a lexicon to generate structures that are evaluated by a pair of modules:
the Sensorimotor (SM) module, commonly called the morphonology, or PF, which is responsible for external expression, and the Conceptual-Intentional (CI) module, commonly called the semantics, or LF, which is responsible for interpreting structures for use in internal system of thought.
\begin{figure}[h]
	\centering
	\caption{The Y Model of the language faculty}
	\label{fig:YModel}
\end{figure}
Beyond this, there is significant disagreement among minimalist theorists.

Another assumption common to minimalist syntacticians is the syntactic operation Merge.
The primary (and in some cases only) syntactic operation, Merge combines pairs of syntactic objects (\textit{i.e.}, lexical items or syntactic structures) to form larger syntactic objects.
The standard formulation of Merge is given in \cref{ex:MergeStd}\footnote{
	\textcite{hornstein2009theory} differs from the standard formulation, defining Merge as concatenation rather than set formation.
}.
\ex.\label{ex:MergeStd} Merge$(\alpha, \beta) = \left\{ \alpha, \beta \right\}$\\
iff both $\alpha$ and $\beta$ are syntactic objects.

Merge is responsible not only for creating new structures, but also for syntactic displacement.
To demonstrate this ability, \textcite{chomsky2004beyond} distinguishes between two cases of Merge: external and internal.
An instance of Merge$(\alpha, \beta)$ is external if neither $\alpha$, nor $\beta$ contains the other, and internal if $\alpha$ contains $\beta$ or vice versa.
Again, beyond the basics discussed above, there is little consensus among minimalist syntacticians.

While the Y Model and Merge seem to be the only instances of true consensus among minimalist syntacticians, there is growing accord about the underlying representation of certin types of word.
Specifically, many minimalist syntacticians now assume that a lexical word, like the noun \textit{chair}, minimally consists of an acategorial root and a categorizing head \parencite{borer2005name,marantz1997no}.
This is commonly expressed in the formalism of a vocabulary insertion rule from the theory of Distributed Morphology as in \cref{ex:VIRule}.
\ex.\label{ex:VIRule} \textit{chair} $\leftrightarrow \left\{ n, \sqrt{\textsc{chair}} \right\}$

There are competing views of the syntactic nature of morphological words, but as of this dissertation's writing this is the standard view, a view to which I subscribe, and therefore, I will not explicitly argue for it.

The theory I assume is not 100\% standard, though.
There are a number of assumptions that I make, which will certainly raise the eybrows of many if not most contemporary syntacticians.
I discuss these assumptions in \cref{sec:nonstandard}.

\subsection{Minimalism and the SMT}
The minimalist program can be viewed as an effort to simplify GB theory without losing its empirical coverage.
That is, a minimalist analysis is one that compares two hypotheses that have roughly equivalent empirical power, and chooses the simpler one.
However, as \textcite{chomsky1965aspects} discusses, there is no such thing as an absolute measure of simplicity.
Consider, for instance, the following equivalent expressions of arithmetic using the more standard infix notation in \cref{ex:StdAddition} and lambda calculus in \cref{ex:ChurchAddition}.
\ex.\label{ex:StdAddition} $3 + 2$

\ex.\label{ex:ChurchAddition} $\lambda f \lambda x . ((\lambda f \lambda x . f(f(f x))) f(\lambda f \lambda x . (f(f x))(f x)))$

While it may seem obvious that \cref{ex:StdAddition} which consists of a mere three symbols is simpler than \cref{ex:ChurchAddition} with its 41 characters, it becomes less obvious when we compare as wholes, the systems that they are drawn from.
Performing arithmetic with infix notation requires rote memorization of the results of single digit addition, multiplication, subraction, and division and rather complex algorithms for larger numbers (\textit{e.g.}, long division).
The lambda calculus, on the other hand, uses two very simple operations, requiring no rote memorization for their application.
From this standpoint the lambda calculus is vastly simpler.
The point here is that a judgment of simplicity depends on the choice of simplicity metric.

This is not to say that the choice of simplicity metric is arbitrary.
On the contrary, since any choice of simplicity metric will be a major factor in the deciding between theories, it must be justified.
The main simplicity metric of the minimalist program is the Strong Minimalist Thesis (SMT) which states that the language faculty is an optimal solution to interface conditions \parencite{chomsky2001derivation}.
One of the justifications for SMT, which is often repeated by Chomsky, comes from evolutionary biology.
It begins with two observations, the results of several decades of linguistics research.
The first observation is that the human language faculty is unique in the biological world, that nothing does language like we do, to use Norbert Hornstein's formulation.
The second observation is that the language faculty is uniform across our species, that a child born in a remote African village, but raised in Dublin would acquire an Irish variety of English with the same ease as a child born and raised in Dublin.
These observations suggest that the language faculty emerged quite suddenly, likely due to a single genetic mutation in a single individual.
It follows from this that whatever portion of our cognitive system that is specific to language must be incredibly simple.

The SMT provides the following principles for minimalist syntactic analysis and theorizing:
(1) Assume the simplest possible recursive syntax (\textit{i.e.}, one that consists only of simplest Merge).
(2) Assume that no other module of the language faculty is capable of recursion.
(3) When you encounter a proposed property of the language faculty or principle of linguistic theory either 
	(a) show that it can be reduced to Merge,
	(b) show that it can be reduced to interface conditions,
	(c) show that it can be reduced to independent principles, or
	(d) show that it can be reduced to a combination of Merge, interface conditions, and independent principles.
It is only when we can show that any such reduction is impossible that we may modify our assumptions.
One might object that this places an unreasonable burden on linguists.
These principles, however, follow from the basic principles of science, and they are inapplicable to the study of human language only insofar as human language is immune to scientific inquiry.
So, to abandon these principles at the first sight of difficulty is to abandon the scientific approach to understanding the language faculty.


\end{document}                                                                                            


\part{Explaining Resultatives}\label{sec:part1}
%\fcolorbox{black}{lightgray}{
%  \begin{minipage}[t]{0.9\textwidth}
%	  \textbf{Note:} I will write a fuller introduction as the body of the thesis becomes more finalized. 
%	  In place of that, here are a few bullet points on what my this thesis is about:
%	  \begin{itemize}
%		  \item Languages vary wrt whether they allow adjectival resultatives.
%		  \item No one has come up with a satisfactory account of resultatives and their variable nature
%		  \item I am proposing an account of resultatives that includes:
%			  \begin{itemize}
%				  \item an analysis of their structure,
%				  \item an analysis of the parametric variation, and
%				  \item an argument that these two accounts are consistent with each other and that the parametric variation is acquirable.
%			  \end{itemize}
%		  \item Essentially I am trying to account for \Next and \NNext
%	  \end{itemize}
%	  \exg. Die Teekanne leer trinken.\\
%the teapot empty drink.\\
%``to drink the teapot dry'' \parencite[German;][]{kratzer2004building}
%
%\exg.* Dario je ofarbao kucu crveno\\
%Dario \textsc{cop} painted house red.\textsc{fem}\\
%``Dario painted the house red.'' (Serbo-Croatian; Mia Sara Misic P.C.)
%
%  \end{minipage}
%}
\chapter{Non-standard Theoretical Assumptions}\label{sec:nonstandard}
% arara: pdflatex: {options: "-draftmode"}
% arara: biber
% arara: pdflatex: {options: "-draftmode"}
% arara: pdflatex: {options: "-file-line-error-style"}
\documentclass[MilwayThesis]{subfiles}
%\setcounter{chapter}{2}
\begin{document}

This dissertation rests on a number of non-standard theoretical assumptions and draws a few non-standard distinctions, which I will defend in this chapter.
My defense of the assumptions, however, will not be an argument that they are true, as the truth of any theoretical statement ultimately depends on the empirical facts.
Rather, my defense will actually be an offense; I will argue that the standard assumption is, in fact, ill-founded.
So, in a sense, I will be rejecting standard assumptions rather than making non-standard ones.
The distinctions I draw, in contrast, will not be defended, but rather explained and clarified.

\section{The $\Theta$-Criterion}
The $\theta$-criterion standardly assumed was first formulated by Chomsky in \textit{Lectures in Government and Binding} (LGB) as \Next.
\ex. Each argument bears one and only one $\theta$-role, and each $\theta$-role is assigned to one and only one argument. \parencite[36]{chomsky1981lectures}

In a footnote, Chomsky justifies this criterion, saying 
\begin{quote}
	The second clause of [the $\theta$-criterion] is well-motivated.
	To say that each $\theta$-role must be filled implies, for example, that a pure transitive verb such as \textit{hit} must have an object, that a verb such as \textit{put} or \textit{keep} (with the sense they have in \textit{put it in the corner}, \textit{keep it in the garage}) must have the associated PP slot filled, etc. 
	The additional requirement that each $\theta$-role must be filled by only one argument will, for example, exclude the possibility that a single trace is associated with several argument antecedents, a possibility ruled out in principle under the Move-$\alpha$ theory. 
	\parencite[139]{chomsky1981lectures}
\end{quote}
I would agree that the second clause of \Last, that each $\theta$-role is assigned to a single argument, is well-motivated by the empirical considerations Chomsky cites, and as such I will not reject that portion of the $\theta$-criterion.
The first clause, however, is motivated mainly by theoretical concerns of LGB, that is, its connection to empirical facts is indirect at best.

The nature of the LGB theory is such that its various hypotheses and principles are connected to each other in a web-like network.
As a result, the first clause of the $\theta$-criterion depends on various other theoretical statements and various other theoretical statements depend on it.
So, rather than attempting an exhaustive enumeration of the links between the $\theta$-criterion and the other theoretical statements of LGB, I will present what I consider to be the best argument in favour or the $\theta$-criterion, and argue that its premises have since been rejected within syntactic theory.

The first premise is the now familiar Y- or T-model of grammar shown in \Next, which LGB theory continues from earlier theories.
According to this model, a syntactic derivation has four levels of representation (D-structure, S-Structure, PF, and LF), and each step in the derivation is performed by the application of a subset of the transformational rules.
S-structures are derived by the applying of Move-$\alpha$ to D-structures, LFs are derived by applying QR (and maybe Move-$\alpha$) to S-structures, and PFs are derived by applying ``stylistic rules'' to S-structures \parencite[18]{chomsky1981lectures}.
\ex. 
\begin{tikzpicture}[baseline]
	\node[draw,rounded corners] (DS) at (0,0) {D-Structure};
	\node[draw,rounded corners] (SS) at (0,-2) {S-Structure};
	\node (PF) at (-2,-3) {PF};
	\node (LF) at (2,-3) {LF};
	\draw[->] (DS)--(SS) node[midway,right] {Move-$\alpha$};
	\draw[->] (SS)--(PF) node[midway,anchor=south east] {Stylistic rules};
	\draw[->] (SS)--(LF) node[midway,anchor=south west] {QR};
\end{tikzpicture}

The exact natures of the transformations are not important for this discussion.
What is important, is that all syntactic displacement is the result of one of these transformations.

The second premise is the projection principle, which states that lexical properties must be represented at all levels of syntax.
Since $\theta$-roles are lexical properties of (at least) verbs, they must be represented at all levels of syntax.
Consider the verb \textit{hit}, whose lexical entry specifies that it needs a patient argument.
The projection principle requires that at D-Structure and S-Structure \textit{hit} must have assigned a patient $\theta$-role to an argument and therefore, assuming the patient $\theta$-role is assigned to Comp,V, there must be a DP in the complement position of \textit{hit} at both D-Structure and S-Structure.

With these assumptions, it follows that no single argument can receive more than one $\theta$-role.
Suppose there is a derivation in which a single argument X receives two $\theta$-roles $\Theta1$ and $\Theta2$.
According to the projection principle, X must be marked with both $\Theta1$ and $\Theta2$ at D-Structure.
Since each $\theta$-role is associated with a unique structural position, it follows that X must be in two distinct positions at D-Structure.
The only way an argument can be in multiple positions is if it has undergone Move-$\alpha$.
Move-$\alpha$, however, maps D-Structures to S-Structures.
Therefore, an argument cannot be in two positions at D-Structure, and furthermore, cannot be multiply $\theta$-marked at D-Structure.
If an argument cannot be multiply $\theta$-marked at D-Structure, then it cannot be multiply $\theta$-marked at all.

Thus we are able to derive the first clause of the $\theta$-criterion from other principles.
These principles, however, have either been rejected or problematized since their statement in LGB.
Since \textit{The Minimalist Program} \parencite{chomsky1995minimalist}, generative theories have largely dispensed with D-Structure and S-Structure.
Without these levels of representation, the projection principle (as formulated in LGB) is effectively meaningless, and without the projection principle, there is no more basis for the first clause of the $\theta$-criterion.
Therefore, I will not be assuming the first clause of the $\theta$-criterion.

\section{Last Resort}
It almost goes without saying that the goal of the minimalist program is to clean up syntactic theory by explaining unnecessary principles in terms of necessary principles.
In nearly one fell swoop, \textcite{chomsky1995minimalist} eliminates a number of the complications built into LGB theory (S-Structure and D-Structure chief among them).
In \textit{The Minimalist Program} (MP), however, Chomsky was not able to explain two seeming imperfections: displacement and uninterpretable features.\footnote{
	The framework developed in MP also does not explain projection/labelling, but Chomsky does not recognize this as an imperfection until ``Problems of projection'' \parencite{chomsky2013problems}.
	More on this in Chapter \ref{sec:labels}
}
Chomsky formalized displacement as the operation Move, which is more complex than Merge, and proposes that, unlike Merge, which ``comes for free,'' every instance of Move must be triggered by the the need to satisfy an uninterpretable feature.
The simple operation Merge is preferred for economy reasons, while movement, Chomsky argues, is a ``last resort.''
In work subsequent to MP, however, Chomsky proposes that Move is actually a subtype of Merge, called Internal Merge.
Merge and Move are different names for the same operation, there is no reason to think that Move is more computationally complex, and therefore no reason to think that movement is a last resort.

In ``Beyond Explanatory Adequacy'' \parencite[][henceforth, \textit{BEA}]{chomsky2004beyond} Chomsky makes this rejection of Last Resort explicit:
\begin{quote}
	[Narrow Syntax] is based on the free operation Merge.
	[The Strong Minimalist Thesis] entails that Merge of $\alpha$, $\beta$ is unconstrained, therefore either external or internal.
	Under external Merge, $\alpha$ and $\beta$ are separate objects; 
	under internal Merge, one is part of the other, and Merge yields the property of ``displacement,'' which is ubiquitous in language and must be captured in some manner in any theory. 
	It is hard to think of a simpler approach than allowing	internal Merge (a grammatical transformation), an operation that is freely available.
	\parencite[110]{chomsky2004beyond}
\end{quote}
This line of reasoning bears discussion in light of the fact that Last Resort is still standardly assumed by self-described minimalist syntacticians.
In fact, several syntacticians expand Last Resort to External Merge arguing that neither type of Merge ``comes free'' \parencite{pesetsky2006probes,frampton2008crash,wurmbrand2014merge,yokoyama2015features}.
This proposal is understandable from a historical perspective, but ultimately misguided in my opinion.

Perhaps the most attractive aspect of a constrained Merge syntax is the purported gains in computational efficiency.
To illustrate this, \textcite{frampton2008crash} consider the incomplete product of a doomed derivation in \Next
\ex. it to be believed Max to be happy

A free Merge syntax, according to \textcite{frampton2008crash}, the derivation must continue for an indefinite time until a phase head is merged.
At this point the derivation will crash due to a Case Filter violation.
A constrained Merge syntax, however, could be able to halt as soon as the derivation becomes doomed, say, when \textit{it} is merged.
This would save us the indefinite number steps it takes to merge a phase head, and is therefore more efficient.
This I take to be a species of the argument that free Merge systems are inefficient because, in addition to the infinite array of convergent derivations they must generate, they also generate an infinite array of crashing derivations, whereas constrained Merge systems only generate to ``convergent'' derivations.
However, when we investigate the nature of constrained Merge, we can see that the purported gains in efficiency in one part of the system come at the expense of another part of that same system.
Constrained Merge theories, in effect, rob Peter to pay Paul.

At minimum, each version of the syntax will have a Merge operation and a Transfer operation.
So, the free Merge syntax will consist of a unconstrained Merge$_F$ and a constrained Transfer$_F$ as defined in \Next.
\ex.
\a. Merge$_F(\alpha,\beta) = \left\{ \alpha, \beta \right\}$ 
\b. Transfer$_F(\gamma) = \langle\textsc{sem}(\gamma), \textsc{phon}(\gamma)\rangle$ iff Filter($\gamma$) = F\\
(where $\alpha$, $\beta$, and $\gamma$ are syntactic objects.)

A constrained Merge syntax, then, would be consist of a constrained Merge$_C$ and an unconstrained Transfer$_C$.
\ex.
\a. Merge$_C(\alpha,\beta) =$ Merge$_F(\alpha,\beta)$ iff Satisfy($\alpha,\beta$) = T
\b. Transfer$_C(\gamma) = \langle\textsc{sem}(\gamma), \textsc{phon}(\gamma)\rangle$
(where $\alpha$, $\beta$, and $\gamma$ are syntactic objects.)

If we assume that both theories can be made descriptively adequate, then the Satisfy predicate will have the same net effect of the Filter predicate required for the free Merge system.
So, given any pair of syntactic objects $\alpha$ and $\beta$, Satisfy must be able to evaluate if Merge($\alpha,\beta$) is allowed.
And since there are an infinite amount of deriveable syntactic objects, even in a constrained Merge syntax, Satisfy must be able to evaluate an infinity of possible $\beta$'s against each possible $\alpha$.
For any given syntactic object, then, there is an indefinite amount of objects which will merge with that object and an indefinite amount that will not, and the only way to know if Satisfy is true of a pair of objects is to chack.
So, much like free Merge syntax suffers from an infinity of crashes, constrained Merge syntax suffers from an infinity of failed Merge operations.

A constrained Merge theorist, might still object by saying that Satisfy is a local operation, while Filter is a global operation, and local operations are to be preferred if we care about computational complexity.
Again, this is an intuitively attractive argument, but not obviously valid.
Consider the following thought experiment.
Suppose you are a TA charged with grading a quiz by your somewhat maniacal instructor.
Part of the instructor's mania is that they require all quizzes to consist of 40 equally weighted questions and be graded out of 10 points.
What is the most efficient procedure for assigning a grade to each quiz?
Two types of procedure suggest themselves.
The first, which I will call the Local Only procedure, is to assign each correct answer a value of 0.25 points and then add up all of the points.
The second, which I will call the Local/Global procedure, it to assign each correct answer a value of 1 point, add up all of the points, and divide by 4 (perhaps with a calculator).
Since humans are Very Good at counting by increments of 1, and calculators are Very Good at dividing by 4, while neither is Very Good at counting by increments of 0.25, the Local/Global procedure is likely to be more efficient than the Local Only procedure.
The moral of this story: One machine's global procedure is another machine's local procedure.

There is also a methodological rationale for preferring a free Merge framework which is slightly counterintuitive, so I would like to dwell on it for a moment.
My reason for assuming a free Merge framework is that it creates, or rather, lays bare, more problems for us to solve.
So, why is this preferrable?
Shouldn't we prefer the theory with fewer problems?
Inuitively, we should prefer the less problematic theory, but this all depends on how we count a theory's problems.
I would like to argue that, while free Merge theories pose a greater number of problems than constrained Merge theories, the sheer weight of the problems posed by each type of theory is equal to that of the other.
Furthermore, the problems of free Merge can be made into empirical questions more readily than those of constrained Merge.

In order to argue in favour of free Merge, I will present one argument against it and show how that argument actually strengthens my claims in the previous paragraph.
The argument comes from \textcite{frampton2008crash}, and states that a free Merge theory of grammar must posit ``filters'' to rule out the non-converging structures that its syntax generates, and the last thing we want is a flourishing of filters.
I could not agree more with their assessment, but where they see a bug, I see a feature.
So-called ``filters'' are not attempts at explanation, but descriptions of generalizations in need of explanation.

Take, for instance, the remaining clause of the $\theta$-criterion, given in \Next.
\ex. [E]ach $\theta$-role is assigned to one and only one argument. \parencite[36]{chomsky1981lectures}

To propose a $\theta$-filter, then would be to say that those derivations which violate \Last crash at an interface, presumably the CI interface.
For a theorist, this filter is actually a question or series of questions: Why is it that only those derived structures which satisfy \Last are valid CI objects?
That question may not be empirical, but it invites hypotheses which may lead to empirical questions which we don't currently know how to ask.
Very likely, due to the interface-nature of the questions, their answers will not be narrowly linguistic.

Now, consider the situation constrained Merge puts the theorist in.
For \textcite{frampton2008crash}, the $\theta$-criterion is expressed by ``selectional features'' on heads which must be satisfied immediately.
This leads to a number of questions: What is the nature of these selectional features?
How are they related to, say, $\varphi$-features?
Do they exist independently of the narrow syntax?
Why do they need to be satisfied first?
And so on.
I, for one, don't have the slightest clue how to proceed in answering or even sharpening these questions, and there don't seem to be any clues in the offing from constrained Merge theorists.

\section{Terminological notes}
The subject matter of this thesis is often called the ``syntax-semantics interface,'' a term which I have discovered is ambiguous, probably due to the fact that it is constructed from three ambiguous terms.

The term \textit{syntax} has (at least) three senses which seem to be used in generative grammar circles.
The first sense, which I will call the sociological sense, is that \textit{syntax} is what syntacticians do.
For instance, $\theta$-theory belongs to the domain of syntax under this sense, because syntacticians care about it, while semanticists tend not to.
However, $\theta$-theory deals at least partially with meaning, so it, at least, intersects with semantics.
This sense would be useful if this dissertation were an intellectual history of generative syntax, but since this is a work of syntactic theory, I will not use this sense.

The second sense, which I will call the broad sense, is that \textit{syntax} is the study (or description) of the form and arrangement of symbolic representations.
Under this sense, the study of syntax would be a part of the study of logic, programming languages, arithmetic, etc.
Furthermore,\textcite[174]{chomsky2000new}, discussing this sense, argues that most of what we call \textit{semantics} and \textit{phonology} would be classified as syntax under this sense.\footnote{Based on my discussion of this sense with phonologists and semanticists, this may be the most controversial claim Chomsky has ever made.}
This sense will prove useful in this dissertation, so I will retain it.

The third sense, which I will call the narrow sense, is that \textit{syntax} is a mental module characterized by a computational procedure that generates an unbounded array of structured form\footnote{I use the term \textit{form} here to refer to all possible expressive modalities of language}-meaning pairs.
This, I believe, is what generative syntacticians mean when they use the term \textit{syntax}.
The ``syntax module'' is one of the objects of study of this thesis, so I will retain this sense.

Since both the broad and narrow senses are useful to me, I will need to make a distinction for the sake of clarity.
I will use the term ``Narrow Syntax'' (or NS) to refer to the narrow sense, that is, the hypothesized mental module, and ``syntax'' (and derived terms) to indicate the broad sense.

Similar remarks apply to the term \textit{semantics}, which has at least three senses.
The first sense, as in the case of \textit{syntax}, is the sociological sense: semantics is what semanticists do.
I will not be using this sense for the same reasons as I cited above for the sociological sense of \textit{syntax}.

The broad sense of \textit{semantics} is that of the study of the relation of a symbolic system to some other system.
So, to various degrees, we can talk about the semantics of a logical system, a programming language, a natural language, etc.
I will use this sense only informally, when discussing notions of truth and reference associated with an instance or class of natural language expression.

The narrow sense of \textit{semantics} is that of the mental module (or system of modules) associated with computing the meaning of a linguistic expression.
Chomsky often refers to this mental entity as the Conceptual-Intentional (CI) system, and stresses that we know very little about it.
Insofar as this thesis makes claims or hypotheses about \textit{semantics}, it makes claims or hypotheses about the CI system.

The final ambiguous term I will discuss is \textit{interface}.
In recent years it has become common within generative linguistics to write papers, hold workshops, and compile books on \textit{the syntax-semantics interface}, but as I mentioned above, the term is ambiguous.
It largely seems to be ambiguous between a \textit{sociological} sense and a \textit{narrow} sense, with the sociological sense dominating discussion.

When used in the sociological sense, the syntax-semantics interface refers to a body of literature that mixes the formalisms and methods used by syntacticians with those used by semanticists.
That is, this type of work makes use of tree diagrams and expressions of typed lambda calculus.
In this sense, the interface is not an object of study per se, but a sub-discipline.

In the narrow sense, the syntax-semantics interface refers to the interface between the Narrow Syntax and the CI system.
This results in very different sorts of analyses compared to the standard analyses, analyses that posit computational procedures rather than merely representing expressions in two ways.
In many ways, however, the term \textit{interface} in its narrow sense is a misnomer, as there are likely no mental objects that we might call interfaces.
As best we can tell, the mind consists of a set of modules and a non-modular central system \parencite{fodor1983modularity,fodor2001mind}.
An interface, then, emerges wherever two modules interact with each other, or perhaps where a module interacts with the central system.
Restricting ourselves to the modules, we can see why positing interfaces as mental objects won't do.
Suppose we have two modules, M1 and M2, which seem to interact with each other.
Being modules, each will consist in a set of computational operations (P1 and P2) defined over a class of syntactically structured (in the broad sense) objects (L1 and L2).
Suppose we posit an interface I1, which consists in an operation P3 that converts objects of L1 into objects of L2.
What is I1, then, but a module that has interfaces with M1 and M2?
If I1 is a module, are its interfaces with M1 and M2 also modules?
If so, then we seem to be stuck with an infite regress.
If not, then interfaces are a special kind of module, but this would raise further questions with respect to their evolutionary origins.

If there are no mental objects that we might call interfaces, then how are we to study them?
The answer to this question is that, to study an interface, we must study the modules associated with that interface with the added assumption that such an interface exists.
So, studying the syntax-semantics interface involves studying the Narrow Syntax and the CI module with the assumption that there is an interface between them.
We will get a glimpse of how such a study would work in chapter \ref{sec:labels}.

\section{Summary}
In this chapter, I have made explicit two of my assumptions which would be considered non-standard among contemporary generative syntacticians.
In particular, I am not assuming the $\theta$-criterion as it is commonly stated, and I am assuming a free Merge syntax.
I have also clarified some terminology that many take for granted, specifically I clarified my use of the term \textit{syntax-semantics interface} and its constituent terms.
Now that the reader has a sense of my theoretical idiosyncrasies, we can move on to more specific concerns in the following chapters.
\end{document}

\chapter{Previous Literature}\label{sec:litreview}
\epigraph{
  There's always a siren\\
  Singing you to shipwreck.\\
  Steer away from these rocks.
}{``There, There''\\\textsc{Radiohead}}
%        File: LitReview.tex
%     Created: Thu Oct 05 09:00 AM 2017 E
% Last Change: Thu Oct 05 09:00 AM 2017 E
%
% arara: pdflatex: {options: "-draftmode"}
% arara: biber
% arara: pdflatex: {options: "-draftmode"}
% arara: pdflatex: {options: "-file-line-error-style"}
\documentclass[MilwayThesis]{subfiles}
\begin{document}
In this chapter I will review several previous analyses of adjectival resultatives and the parametric variation associated with them.
I will evaluate the anlyses against two desiderata.
First, I will evaluate whether the variation, as analyzed, is learnable.
Second, I will evaluate whether the analysis comports with the theoretical principles of the minimalist program.
Before reviewing the analyses, however, I will make these desiderata explicit and justify them.

\section{Desiderata for an analysis of resultatives}

\subsection{Desideratum 1: Learnability}
Most of the analyses of the structure of adjectival resultatives are packaged with an account of the associated parametric variation.
While few address directly address the acquisition of parametric variation, any analysis of variation makes implicit claims about acquistion.
Generally, when discussing parametric variation, the claims about acquisition can justifiably be left implict, as the acquisition task is trivial.
The nature of adjectival resultatives, however, is such that we must make those acquisition claims explicit.
To explain why, I will be comparing the resultative parameter to the V-to-T parameter.

Analyses of the V-to-T parameter do not need to address learnability because the ``parameter setting'' is directly learnable from the primary linguistic data.
It is directly learnable because the overt forms of (\textit{e.g.}) polar questions differs depending on the parameter setting.
In a language with V-to-T movement such as German, the language learner will observe that lexical verbs undergo inversion for polar questions, while in a language without V-to-T movement such as English, the learner will observe that lexical verbs do not invert for questions.
\ex. V-to-T movement (German)
\ag. Trinken sie Kaffee?\\
Drink.3plPres they Coffee\\
``Do they drink coffee?''
\b.* \textsc{do} sie Kaffee trinken?

\ex. *V-to-T Movement (English)
\a.* Drink they coffee?
\b. Do they drink coffee?

The form of polar questions, then, can be positive evidence for a parameter setting.
Since we can find direct positive evidence for a parameter setting, the task of the analyst/theoretician, then is merely to formalize the parameter is a way that is consistent with the broader theory.
\textcite{chomsky1995minimalist}, for instance, formalizes the V-to-T parameter in terms of feature strength, while \textcite{lasnik1999verbal} formalizes it in terms of the presence/absence of inflectional features on lexical verbs.
Neither, however, needs to explicitly describe how their parameter is set, but can assume that a certain setting is the default, and the other setting can be deduced from (\textit{e.g.}) the form of polar questions in the language.

Resultatives, on the other hand, are not directly learnable for two reasons
The first reason is that, on the surface, resultatives, which are parameterized, are indistinguishable from depictives, which appear to be universal.
The two construction types are indistinguishable in the sense that both correspond to the string template in \Next (modulo independent word order variation).
\ex. \textsc{Subj} V \textsc{Obj} Adj.

This indistinguishability is evident in the fact that one can construct examples which are truly ambiguous between resultative and depictive readings, as in \Next.
\ex. 
\a. He fried the fish dry.
\a. $\approx$ He fried the fish once it was dry. (\textbf{Depictive})
\b. $\approx$ He fried the fish until it was dry. (\textbf{Resultative})
\z.
\b. She painted the barn red.
\a. $\approx$ The barn is red in her painting. (\textbf{Depictive})
\b. $\approx$ She applied a coat of red paint to the door. (\textbf{Resultative})
\z.

Assuming a child acquiring either French or English encounters sentences with the form of \LLast in their PLD, there is no obvious way for the child to determine whether a given secondary predicate is to be interpreted depictively or resultatively.

An empiricist might object, arguing that the ambiguous examples above are highly constructed, and would easily by disambiguated in context.
They would insist that the learner would infer a positive setting of the resultative parameter from the use of a secondary predication construction in the presence of a resultative event.
So, an English learner, but not a French learner, might be exposed to the context-sentence pairing in \Next.
\ex. 
\a.[\textbf{Context:} ] A woman is methodically hammering a lump of metal.
A parent draws their child's attention to the hammering event and utters:
\b.[\textbf{A:}] She's hammering the metal flat.

Even this, however, is not fully unambiguous.
While it certainly couldn't be interpreted as a depictive, \textit{flat} could be interpreted as a manner adverb, modifying \textit{hammering}.
Such unavoidable ambiguity would make it difficult to employ any sort of semantic bootstrapping in the acquisition of the resultative parameter.

The second problem comes from the fact that both English-type and French-type languages can express resultative semantics periphrastically as in \Next.
\ex. periphrastic resultatives
\ag. Elle a aplati le m\'etal en le martelant\\
She has flattened the metal in the hammering\\
\b. She flattened the metal by hammering.


The resultative parameter, then, is a one-or-both parameter, unlike other parameters which are either/or choices.
The V-to-T parameter, in contrast, is an either/or choice which can be made on the basis of the PLD.

Since there is no direct way to determine from the primary linguistic data whether a language allows or disallows resultatives, we can reject any analysis of resultatives that requires the parameter to be directly set.
Rather the setting of the resultative parameter must be determined indirectly.

\subsection{Desideratum 2: Theoretical consistency}
The second quality an analysis of resultatives should have is consistency with the broader theory of grammar.
While this is a desideratum of all analyses of grammatical phenomena, I will outline the subset of minimalist hypotheses that I will use to evaluate proposed analyses of adjectival resultatives.
First, I will review what is called the Uniformity of Theta Hypothesis (UTAH) which gives us a baseline for our theory of $\Theta$-role assignments.
Second, I will look at the Borer-Chomsky Conjecture, which is the minimalist hypothesis which will guide our theory of variation.
Each of these will be discussed with reference to a particular property of adjectival resultatives.

The canonical formulation of UTAH is that of \textcite{baker1988incorporation} working in pre-minimalist Principles and Parameters framework, given below in \Next.
\ex. \textbf{The Uniformity of Theta Hypothesis (UTAH)}\\
Identical thematic relationships between items are represented by identical structural relationships between those items at the level of D-structure. \parencite[46]{baker1988incorporation}

While I will be assuming something of this flavour in my discussion, the reference to D-structure, which was eliminated in the minimalist program, renders Baker's hypothesis unusable as it stands.
<+MoreDiscussion+>

I propose the following more precise formulation of UTAH:
\ex. \textbf{Minimal UTAH}\\
If a thematic relation $f$ holds between items X and Y in distinct expressions S$_1$ and S$_2$, then there is a structural relation $g$ that also holds between X and Y in S$_1$ and S$_2$.

To understand what this means, consider the following sentence pairs:
\ex. \label{ex:broke-bottle}
\a. Sara broke the bottle.
\b. The bottle broke.

\ex. \label{ex:katie-ran}
\a. Katie ran.
\b. Katie ran a kilometre.

\ex. \label{ex:emily-dropped}
\a. Emily dropped her book.
\b. Emily dropped the phone.

In the pair \ref{ex:}<++>
\end{document}



\chapter{The Structure of Resultatives}\label{sec:analysis}
\epigraph{Die Welt ist eine Glocke, die einen Ri\ss{} hat: sie klappert, aber klingt nicht.}{\textsc{Johann Wolfgang von Goethe}}
%        File: MyAnalysis.tex
%     Created: Mon Nov 06 10:00 AM 2017 E
% Last Change: Mon Nov 06 10:00 AM 2017 E
%
% arara: pdflatex: {options: "-draftmode"}
% arara: biber
% arara: pdflatex: {options: "-draftmode"}
% arara: pdflatex: {options: "-file-line-error-style"}
\documentclass[MilwayThesis]{subfiles}
\setcounter{chapter}{3}
%\usepackage{atveryend}
%
%\BeforeClearDocument{
%	\printbibliography
%}
\begin{document}
In the previous chapter, I discussed the failings of several previous analyses of adjectival resultatives.
In this chapter I will address two of those analyses -- one structural, and one parametric -- and show how they can be modified to address the concerns raised in the previous chapter.
The structural analysis, that of \textcite{kratzer2004building}, was rejected because it did not comply with UTAH, but has three features which I will retain: a small clause structure, theme raising, and a result head.
The parametric analysis, that of \textcite{snyder1995language,snyder2012parameter}, was rejected because it did not comply with the Lexical Parameterization Hypothesis, but was based on a learnable pattern and so I will be adopting a modified version of it.
\section{Fixing the UTAH problem}
The one issue with Kratzer's analysis is that it seems to violate UTAH.
That is, there is a single $\theta$-relation between \textit{hammer} and \textit{the metal} in both sentences in \Next, that is not represented by a single structural relation.
\ex.
\a. Joe hammered the metal flat.
\b. Joe hammered the metal.

According to Kratzer's analysis, \textit{the metal} is the specifier of \textit{hammer} in \Last[a], but a standard analysis of \Last[b] will place \textit{the metal} as the complement of \textit{hammer}
\ex.
\a. hammer the metal flat \parencite[following][]{kratzer2004building}\\
\begin{forest}
    nice empty nodes,sn edges,baseline,for tree={
    calign=fixed edge angles,
    calign primary angle=-30,calign secondary angle=70}
    [VP
	    [DP[the metal,roof,name=specV]]
	    [
		    [hammer]
		    [resP
			    [res]
			    [SC
				    [$\langle$DP$\rangle$,name=SCDP]
				    [flat]
			    ]
		    ]
	    ]
    ]
    \draw[->] (SCDP) to[out=south west, in=south] (specV);
\end{forest}
\b.hammer the metal\\
\begin{forest}
    nice empty nodes,sn edges,baseline,for tree={
    calign=fixed edge angles,
    calign primary angle=-30,calign secondary angle=70}
    [VP
	    [hammer]
	    [DP[the metal,roof]]
    ]
\end{forest}

If we were to modify Kratzer's analysis, so that \textit{the metal} is the complement of \textit{hammer}, then we would need to attach the result phrase in a different position.
I propose that the result phrase is adjoined to the VP, allowing the DP to merge directly with the verb as shown in \Next.
\ex.\label{tree:hammer-flat}
{\small
\begin{forest}
    nice empty nodes,sn edges,baseline,
%    for tree={
%    calign=fixed edge angles,
%    calign primary angle=-30,calign secondary angle=60}
    [VP
	    [VP
		    [hammer]
		    [DP[the metal,roof,name=compV]]
	    ]
	    [resP
		    [$\langle$DP$\rangle$,name=specRes]
		    [
			    [res]
			    [SC
				    [$\langle$DP$\rangle$,name=SCDP]
				    [flat]
			    ]
		    ]
	    ]
    ]
    \draw[->] (SCDP) to[out=south west, in=south] (specRes);
    \draw[->] (specRes) to[out=south, in=south east] (compV);
\end{forest}
}

The modified analysis no longer violates UTAH, but it introduces two issues new issues which I address in the proceeding sections.
The first issue is that the movement operation between [Spec, res] and [Comp, V] does not target a c-commanding positions.
In other words it is a sideward rather than upward movement.
The second issue is that resP and VP are now adjoined, meaning they compose by conjunction.
This is counterintuitive, however, since resultatives are inherently asymmetric, with the verb event causing the adjective state.
I will address each of these in turn below.
\subsection{Sideward movement}
In \ref{tree:hammer-flat}, the object DP moves from [Spec res] to [Comp V].
The movement ``chain'' this operation forms is problematic due to the fact that the head of the chain does not c-command the tail.
Although this type of so-called sideward movement is generally barred, \textcite{nunes2001sideward} argues for a restricted version of sideward movement.
Nunes argues that head movement and parasitic gaps both require a sideward movement operation, as they both create non-c-command dependencies.
\ex.
\a. Head Hovement\\
\begin{forest}
    nice empty nodes,sn edges,baseline,for tree={
    calign=fixed edge angles,
    calign primary angle=-30,calign secondary angle=70}
    [TP
	    [DP]
	    [
		    [T
			    [T]
			    [V,name=head]
		    ]
		    [VP
			    [$\langle$V$\rangle$,name=tail]
			    [DP]
		    ]
	    ]
    ]
    \draw[->] (tail) to[out=south, in=south] (head);
\end{forest}
\b. Parasitic gaps\\
What did Mary hear without seeing.\\
\begin{forest}
    nice empty nodes,sn edges,baseline
    [CP
	    [DP[What,roof,name=specCP]]
	    [
		    [C+T\\did,align=center]
		    [TP
			    [DP[Mary,roof]]
			    [
				    [$\langle$T$\rangle$]
				    [VP
					    [VP
						    [V\\see,align=center]
						    [$\langle$DP$\rangle$,name=CompV1]
					    ]
					    [PP
						    [P\\without,align=center]
						    [VP
							    [V\\seeing,align=center]
							    [$\langle$DP$\rangle$,name=CompV2]
						    ]
					    ]
				    ]
			    ]
		    ]
	    ]
    ]
    \draw[->] (CompV2) to[out=south west, in=south] (CompV1);
    \draw[->] (CompV1) to[out=south, in=south] (specCP);
\end{forest}

According to the standard definition of Merge, sideward movement should be impossible.
The facts of parasitic gaps and head movement, however, suggest that a possibly complex operation with the net effect of sideward movement must be active in the grammar.
I adopt the approach developed by \textcite{nunes1995diss,nunes2001sideward} as follows.

In order to explain sideward movement, Nunes hypothesizes that a movement operation is composed of a Copy operation followed by Merge.
The operation Copy adds an object X to the workspace of a derivation provided that X is contained in an already constructed syntactic object.
\ex. For a workspace W and a syntactic object X, Copy(W, X) = $W\cup \{X\}$ iff there is a syntactic object Z $\in$ W and Z contains X.

Merge, then is a simpler operation which replaces two members of a workspace with the set containing them. 
To see how a Copy+Merge theory of movement works, consider the derivation of passivization in \Next.
\ex. 
\begin{tabular}[t]{lll}
	\textbf{Stage} & \textbf{Workspace} & \\
	\cline{1-2}
	1 & $\{$[T, [ \textsc{Voice}$_{pass}$ [see, [the, boy]]]]$\}$ & Copy([the, boy])\\
	2 & $
		\begin{Bmatrix*}[l]
			\text{[the, boy]},\\
			\text{[T, [ \textsc{Voice}$_{pass}$ [see, [the, boy]]]]}
		\end{Bmatrix*}
		$ & Merge([the, boy], [T \ldots])\\
	3 & $\{$[[the, boy], [T, [ \textsc{Voice}$_{pass}$ [see, [the, boy]]]]]$\}$ &\\
\end{tabular}

The Copy+Merge theory of movement allows us to derive sideward movement by holding the copied object in the workspace while another tree is built as in the derivation of \ref{tree:hammer-flat} in \Next.
\ex.
\begin{tabular}[t]{lll}
	\textbf{Stage} & \textbf{Workspace} & \\
	\cline{1-2}
	1 & $\left\{ \text{[[the, metal], [res, [\dots]]]} \right\}$ & Copy([the, metal])\\
	2 & $
	\begin{Bmatrix*}[l]
		\text{[the, metal]},\\
		\text{[[the, metal], [res, [\dots]]]}
	\end{Bmatrix*}
	$ & Select(hammer)\\
	3 & $
	\begin{Bmatrix*}[l]
		\text{hammer},\\
		\text{[the, metal]},\\
		\text{[[the, metal], [res, [\dots]]]}
	\end{Bmatrix*}
	$ & Merge(hammer, [the, metal])\\
	4 & $
	\begin{Bmatrix*}[l]
		\text{[hammer, [the, metal]]},\\
		\text{[[the, metal], [res, [\dots]]]}
	\end{Bmatrix*}
	$ & Merge$\begin{pmatrix*}[l]\text{[hammer, [the, metal]]},\\ \text{[[the, metal], [res [\dots]]]}\end{pmatrix*}$\\
	5 & \multicolumn{2}{l}{$\left\{\text{[[hammer, [the, metal]], [[the, metal], [res [\dots]]]]}\right\}$}\\
\end{tabular}

Note that at stage 5 of the derivation in \Last the syntactic object in the workspace is representable as \ref{tree:hammer-flat}.

In order to constrain sideward movement, Nunes notes that its immediate, results such as the tree in \ref{tree:hammer-flat}, are unpronounceable.
Assuming that decisions regarding linear order depend on c-command relations, and part of linearization is deciding which copy in a movement chain is to be pronounced, we would be unable to make a definitive linearization statement for the derived structure in \Last.
In order to linearize the movement chain of \textit{the hammer}, there must be a copy which c-commands all other copies, meaning there must be a subsequent move from theme position to grammatical object position, which I represent as [Spec, AgrO] in \Next.
\ex.
{\small
\begin{forest}
    nice empty nodes,sn edges,baseline,
%    for tree={
%    calign=fixed edge angles,
%    calign primary angle=-30,calign secondary angle=65}
    [AgrOP
	    [DP[the metal,roof,name=obj]]
	    [
		    [AgrO]
    [VP
	    [VP
		    [hammer]
		    [DP,name=compV]
	    ]
	    [resP
		    [$\langle$DP$\rangle$,name=specRes]
		    [
			    [res]
			    [SC
				    [$\langle$DP$\rangle$,name=SCDP]
				    [flat]
			    ]
		    ]
	    ]
    ]
    ]
    ]
    \draw[->] (SCDP) to[out=south west, in=south east] (specRes);
    \draw[->] (specRes) to[out=south, in= south] (compV);
    \draw[->] (compV) to[out=south west, in=south] (obj); 
\end{forest}
}

Since the copy of \textit{the metal} in [Spec, AgrO] c-commands all of the other copies, it will be pronounced and the lower copies will be deleted at the SM interface.
So, assuming some mechanism for sideward movement, we are able to modify Kratzer's (\citeyear{kratzer2004building}) analysis of resultatives to be compatible with UTAH.

Before continuing, I would like to briefly defend my use of AgrO in \Last.
First proposed by \textcite{chomsky1995minimalist}, the AgrO phrase was proposed as a way to unify nominative and accusative Case assignment.
Both Cases were proposed to be checked in the specifier of an AgrP.
<++>

Since we have modified Kratzer's analysis, it is worth asking if our version will still compose semantically to give us the desired interpretation.
In the next section, I argue that not only are we able to retain the proper interpretation, but we are able to do so while assuming a simpler compositional system.
\section{Composing resultatives}
\textcite{kratzer2004building} adopts a neo-Davidsonian semantics for resultatives, meaning they are analyzed as descriptions of eventualities rather that merely as relations between entities.
Her syntactic analysis is given in \Next for reference.
\ex. \textbf{Kratzer's (2005) structural analysis of resultatives}\\
\begin{forest}
    nice empty nodes,sn edges,baseline,for tree={
    calign=fixed edge angles,
    calign primary angle=-30,calign secondary angle=70}
    [VP
	    [DP[the metal,roof,name=specV]]
	    [
		    [hammer]
		    [resP
			    [res]
			    [SC
				    [$\langle$DP$\rangle$,name=SCDP]
				    [flat]
			    ]
		    ]
	    ]
    ]
    \draw[->] (SCDP) to[out=south west, in=south] (specV);
\end{forest}

According to this analysis, the small clause \textit{the metal flat} is interpreted as the state description in \Next, where the domain $D_s$ is the domain of eventualities.
\ex. $\llbracket\text{SC}\rrbracket = \lambda s_s \left[ \textsc{state}(s) \& \textbf{flat}(\textbf{the\_metal})(s) \right]$

The verb \textit{hammer} is interpreted as a predicate of events.
\ex. $\llbracket\textit{hammer}\rrbracket = \lambda e_s \left[ \textsc{event}(e) \& \textbf{hammer}(e)\right]$

Note that Kratzer analyses resultative verbs as intransitives, meaning they do not take any entity arguments.
Finally, Kratzer analyses the result head as a higher order function, which expresses a causal relation between the event expressed by the verb and the state expressed by the small clause.
\ex. $\llbracket\textit{res}\rrbracket = \lambda P_{\langle s,t\rangle} \lambda e_s \exists s_s \left[\textsc{event}(e) \& \textsc{state}(s) \& P(s) \& \textsc{Cause}(s)(e)\right]$

So, for Kratzer, the typed LF of \textit{hammer the metal flat} is as in \Next.
\ex.
\begin{forest}
    nice empty nodes,sn edges,baseline,for tree={
    calign=fixed edge angles,
    calign primary angle=-30,calign secondary angle=70}
    [VP$_{\langle s,t\rangle}$
		    [hammer$_{\langle s,t\rangle}$]
		    [resP$_{\langle s,t\rangle}$
			    [res$_{\langle st, st\rangle}$]
			    [SC$_{\langle s,t\rangle}$
				    [DP$_e$]
				    [flat$_{\langle e, st\rangle}$]
			    ]
		    ]
	    ]
\end{forest}

Kratzer proposes that \textit{hammer} and the resP compose by an operation she calls Event Identification \parencite{kratzer1996severing} which, in this instance, is equivalent to Predicate Modification generalized to eventualities.
\ex. \textbf{Predicate Modification (eventuality version)}\\
If $\alpha$ is a branching node with daughters $\beta$ and $\gamma$, both of which are of type $\langle s,t\rangle$, the $\llbracket\alpha\rrbracket = \lambda e_s [\llbracket\beta\rrbracket(e) \& \llbracket\gamma\rrbracket(e)]$

So, the interpretation of the VP in \LLast, can be derived as in \Next.
\ex.
\begin{enumerate}
	\item $\llbracket$VP$\rrbracket$ = \hfill (Predicate Modification)
	\item $\lambda e_s [\llbracket\text{hammer}\rrbracket(e) \& \llbracket\text{resP}\rrbracket(e)]$ = 
	\item $\lambda e_s [ \textbf{hammer}(e) \& \exists s_s[\textsc{Cause}(s)(e) \& \textbf{flat}(\textbf{the\_metal})(s)]]$
\end{enumerate}

So, the hammering event is identical to the event of causing the flatness state.
The same compositional process can be adapted to the sideward movement structure I propose, as the VP and resP in my structures are still predicted to be predicates of eventualities.
\ex.
\begin{forest}
    nice empty nodes,sn edges,baseline,for tree={
    calign=fixed edge angles,
    calign primary angle=-30,calign secondary angle=70}
    [VP$_{\langle s,t\rangle}$
	    [VP$_{\langle s,t\rangle}$
		    [hammer$_{\langle e, st\rangle}$]
		    [DP$_e$]
	    ]
	    [resP$_{\langle s,t\rangle}$
		    [res$_{\langle st, st\rangle}$]
		    [SC$_{\langle s,t\rangle}$
			    [DP$_e$]
			    [flat$_{\langle e, st\rangle}$]
		    ]
	    ]
	    ]
\end{forest}

Thus, with these adaptations, Kratzer's analysis of resultatives can be made UTAH-compliant.
In the remainder of this chapter, I will discuss the parametric analysis of resultatives that I will be assuming.

\section{Where does the resultative parameter come from?}
In the previous chapter, I discussed two desiderata for a parametric analysis.
First, the parameter must be learnable, meaning there must be some variable in the primary linguistic data which the learner can detect and deduce a particular parameter setting from.
Second, the variable must be represented in the lexicon.
For the sake of expediency, I will refer to the variable detectable in the PLD as the surface variable, and its lexical representation as the lexical variable.

To my knowledge, there is only one proposed candidate for the surface variable in the generative literature, that is, Snyder's (\citeyear{snyder1995language,snyder2012parameter}) compounding parameter.
According to the latest version of this parameter, a language allows resultatives iff it allows bare stem compounding.
As I discussed in the previous chapter, Snyder rejects the Lexical Parameterization Hypothesis, meaning he does not propose a lexical variable, instead situating the parameter in the operations of the CI interface.
However, I will propose a lexical variable from which both the (un)availability of bare stem compounding, and the (un)availability of adjectival resultatives can be derived.

To make such a proposal, we must make the intermediate hypothesis that a language allows bare stem compounding iff it allows bare stems, meaning there should be no languages that allow for bare stems but cannot compound them together.
Now, a bare stem is merely an independent word with no inflectional material.
Words are represented in most current theories of syntax as an acategorial root merged with a category-determining functional head (following Marantz \citeyear{marantz1997no}, but see also Borer \citeyear{borer2005name} for a similar proposal).
Since roots are, by definition, featureless, any inflectional features on stems must be due to their category-determining heads.
It follows from this that the (im)possibility of bare stems derives from the presence or absence of inflectional features on category-determining heads in the lexicon.

So, if we represent inflected category-determining heads as $v_\varphi, n_\varphi, adj_\varphi, etc.$ and their bare counterparts as $v_\emptyset, n_\emptyset, adj_\emptyset, etc.$, then the lexical version of Snyder's compounding parameter can be represented as in \Next.
\ex. \textsc{lex} $\left\{ \text{includes, does not include} \right\}$ $v_\emptyset, n_\emptyset, adj_\emptyset, etc.$

Note that this is a fairly weak claim.
A stronger claim would be that compounding languages have only uninflected category-determining heads.
The weak claim, however, is sufficient for present purposes, and is therefore adopted.

This version of the compounding parameter is lexical, and therefore complies with the Lexical Parameterization Hypothesis.
Furthermore, it is learnable from the primary linguistic data, since its external manifestation is the presence or absence of inflectional morphology.
Since the inflectional morphology is detectable on the surface, its absence must also be detectable or at least deducible.
This leaves us with questions regarding the initial state of the lexicon, and which parameter setting is the default, but those questions are beyond the scope of this thesis and will be set aside.


\end{document}

\chapter{Label Theory}\label{sec:labels}
\epigraph{``A rose by any other name would smell as sweet.''\\
``Not if you called 'em stench blossoms.''}{``The Principal and the Pauper''\\\textit{The Simpsons}}
% arara: pdflatex: {options: "-draftmode"}
% arara: biber
% arara: pdflatex: {options: "-draftmode"}
% arara: pdflatex: {options: "-file-line-error-style"}
\documentclass[Proposal]{subfiles}

\begin{document}
\begin{frame}
  {Label Theory}
  {\textcite{chomsky2013problems,chomsky2015problems}}
  \begin{itemize}
    \item UG is reducible to simplest merge
  \end{itemize}
  \ex. Merge($\alpha,\beta$) = $\left\{ \alpha,\beta \right\}$

  \begin{itemize}
    \item Accounts for the fundamental properties of language (\textit{e.g.}, structure-dependence of rules, displacement)
      \begin{itemize}
	\item Except for Projection/Labeling
      \end{itemize}
    \item Chomsky's proposal: Labels are assigned at the CI interface by a Labeling Algorithm (LA)
  \end{itemize}
\end{frame}
\begin{frame}
  {The Labeling Algorithm}
  \begin{itemize}
    \item LA is a special instance of Minimal Search.
      \begin{itemize}
	\item Picks out the most prominent item as a syntactic object's label
      \end{itemize}
      \pause
    \item There are three relevant classes of syntactic objects for LA:
  \end{itemize}
  \begin{overprint}
    \onslide<3>
    \begin{block}
      {(i) Head-Phrase Structures}
      \begin{itemize}
	\item Label($\left\{ \text{X, YP} \right\}$) = X
      \end{itemize}
    \end{block}
    \onslide<4-5>
    \begin{block}
      {(ii) Head-Head Structures}
      \begin{itemize}
	\item<4-5> Label($\left\{ \text{X, }\textsc{root} \right\}$) = X
	  \begin{itemize}
	    \item<4-5> Roots cannot label
	  \end{itemize}
	\item<5> Undefined otherwise
      \end{itemize}
    \end{block}
    \onslide<6-8>
    \begin{block}
      {(iii) Phrase-Phrase Structures}
      \begin{itemize}
	\item<6-8> Label($\left\{ \text{XP}, \langle\text{YP}\rangle \right\}$) = Label(XP)
	  \begin{itemize}
	    \item<6-8> Lower copies are invisible to LA.
	  \end{itemize}
	\item<7-8> Label($\left\{ \text{XP}_F, \text{YP}_F \right\}$)= $\langle\text{F,F}\rangle$
	  \begin{itemize}
	    \item<7-8> Iff XP and YP agree for some feature F
	  \end{itemize}
	\item <8> Undefined otherwise
      \end{itemize}
    \end{block}
  \end{overprint}
\end{frame}
\begin{frame}
  {My extensions to Label Theory}
  \begin{block}
    {Labels determine composition}
    \begin{itemize}
      \item<2-> If Label(SO) $\in$ SO, then SO composes by function application
	\begin{itemize}
	  \item<3-> $\llbracket\left\{ \text{X, YP} \right\}\rrbracket$ = X(YP)
	\end{itemize}
      \item<4-> If Label(SO) = $\langle\text{F,F}\rangle$, then SO is interpreted as an Operator-variable structure
	\begin{itemize}
	  \item<5-> $\llbracket\left\{ \text{DP}_Q, \text{CP}_Q \right\}\rrbracket$ = $(\text{Wh}x)(\dots x \dots)$
	\end{itemize}
    \end{itemize}
  \end{block}
\end{frame}
\begin{frame}
  {My extensions to Label Theory}
  \begin{block}
    {Adjunction structures are unlabeled}
    \begin{itemize}
      \item Adjuncts are ignored by LA.
	\begin{itemize}
	  \item Label($\left\{ \text{XP, ZP} \right\}$) = $\emptyset$ if ZP is an adjunct
	  \item Since ZP is ignored by LA, it is internally unlabelled. (Label(ZP)=$\emptyset$)
	\end{itemize}
	\pause
      \item Unlabeled SOs compose by conjuction.
	\begin{itemize}
	  \item $\llbracket\left\{ \text{XP, ZP} \right\}\rrbracket$ = XP \& ZP (if $\left\{ \text{XP, ZP} \right\}$ is unlabelled)
	\end{itemize}
    \end{itemize}
  \end{block}
\end{frame}
\begin{frame}
  {What can this version of Label Theory get us?}
  \begin{block}
    {Subjects of ACC-ing clauses and pseudo-relatives}
    \begin{itemize}
      \item<2-> \textcite{cinque1996pseudo} shows that the position of an ACC-ing/pseudo-relative clause affects the behaviour of its subject.
    \end{itemize}
    \only<2->{
    \ex. \textbf{Italian pseudo-relative (PR)}\\
    {\rm Mario che correva a tutti velocit\`a}

    \ex. \textbf{English ACC-ing clause (AC)}\\
    {\rm Mario running at full speed}

  }
    \begin{itemize}
      \item<3-> If the AC/PR is a complement of V, the subject cannot move.
      \item<3-> If the AC/PR is a VP adjunct, the subject must move.
    \end{itemize}
  \end{block}
\end{frame}
\begin{frame}
  {What can this version of Label Theory get us?}

    \begin{columns}
    \begin{column}[T]{0.6\textwidth}
      {\rm *Mario$_i$ was [$_{VP}$ [seen[$t_i$ running]]]}
      \begin{block}
	{Complement ACs}
	\begin{itemize}
	  \item Subject is frozen in the AC
	  \item Prog$^\circ$ is too weak to label $\zeta$
	    \begin{itemize}
	      \item Compare Chomsky's (2015) discussion of EPP
	    \end{itemize}
	  \item If {\rm Mario} were \textit{in situ}, Label($\zeta$) = $\langle\text{F,F}\rangle$
	  \item Lower copies are invisible, so $\zeta$ is unabelable.
	  \item The derivation crashes at CI
	\end{itemize}
      \end{block}
    \end{column}
    \begin{column}[T]{0.4\textwidth}
	{\small
	  \begin{forest}
	    nice empty nodes,sn edges,baseline
	    [$\alpha$
	      [{\rm Mario}]
	      [$\beta$
		[T]
		[$\gamma$
		  [Voice$_{pass}$]
		  [$\delta$
		    [{\rm see}]
		    [$\zeta$
		      [$\langle${\rm Mario}$\rangle$]
		      [Prog]
		    ]
		  ]
		]
	      ]
	    ]
	  \end{forest}
	}
    \end{column}
  \end{columns}
\end{frame}
\begin{frame}
  {What can this version of Label Theory get us?}
  \begin{columns}
    \begin{column}[T]{0.55\textwidth}
      {\rm *I [ [$_{VP}$ saw Bill] [Mario running]]}
      \begin{block}
	{Adjunct ACs}
	\begin{itemize}
	  \item Subjects must move to theme position
	  \item $\eta$ is adjoined, therefore ignored by LA
	  \item $\eta$ is interpreted as the conjunction of {\rm Mario} and {\rm running}
	    \begin{itemize}
	      \item This is an ill-formed interpretation
	    \end{itemize}
	\end{itemize}
      \end{block}
    \end{column}
    \begin{column}[T]{0.45\textwidth}
      {\small
	  \begin{forest}
	    nice empty nodes,sn edges,baseline
	    [$\alpha$
	      [{\rm I}]
	      [$\beta$
		[Voice]
		[$\gamma$
		  [$\zeta$
		    [{\rm see}]
		    [{\rm Bill}]
		  ]
		  [$\eta$
		    [{\rm Mario}]
		    [Prog]
		  ]
		]
	      ]
	    ]
	  \end{forest}
	}
    \end{column}
  \end{columns}
\end{frame}
\begin{frame}
  {What can this version of Label Theory get us?}
    
\end{frame}
\end{document}

%\section{The architecture of a label-theoretic grammar}
%\subfile{agree}
\chapter{Deriving the Resultative Parameter}\label{sec:deriving}
\epigraph{All happy families are alike; every unhappy family is unhappy in its own way.}{\textit{Anna Karenina}\\\textsc{Leo Tolstoy}}
So far, I have clarified and developed my theoretical assumptions and proposed a structural analysis of resultatives, shown in \cref{fig:ResStruct}, and an analysis of the resultative parameter, given in \cref{ex:ResParam}.
\ex.\label{ex:ResParam} A language L allows resultatives only if the lexicon of L includes \textit{adj}$_{\emptyset}$.

Although they share a justification in minimalism, the structural analysis and the parametric analysis are independent proposals, and as such their combination must be justified.
That is, I have argued that each component is plausible on its own, but it is altogether possible that they are not consistent with each other, or they are not able to give the correct empirical results.
In this chapter, I argue that they are consistent with each other, and that, together, they can provide an explanation of the resultative parameter.
I do so by first showing that the structure in \autoref{fig:ResStruct} can be derived in a language with uninflected adjectives ($adj_\emptyset \in \textsc{lex}$), and then showing how such a derivation fails in a language without uninflected adjectives ($adj_\emptyset \centernot\in \textsc{lex}$).
\begin{figure}[h]
	\centering
{\small
\begin{forest}
    nice empty nodes,sn edges,baseline,
%    for tree={
%    calign=fixed edge angles,
%    calign primary angle=-30,calign secondary angle=60}
    [VP
	    [VP
		    [hammer]
		    [DP[the metal,roof,name=compV]]
	    ]
	    [resP
		    [$\lfloor$DP$\rfloor$,name=specRes]
		    [
			    [res]
			    [SC
				    [$\lfloor$DP$\rfloor$,name=SCDP]
				    [
					    [adj]
					    [\textsc{flat}]
				    ]
			    ]
		    ]
	    ]
    ]
    \draw[->] (SCDP) to[out=south west, in=south] (specRes);
    \draw[->] (specRes) to[out=south, in=south east] (compV);
\end{forest}
}
	\caption{The structure of a resultative}
	\label{fig:ResStruct}
\end{figure}

In the first section, I will give a derivation of the English resultative VP \textit{hammer the metal flat} and show that it converges at the CI interface (taking SM convergence for granted).
In the second section, I will give two possible derivations of the ungrammatical French resultative VP \textit{marteller le m\'etal plat}, and show that deriving that VP leads to a CI crash while avoiding that crash blocks the derivation.

Before describing the derivations, I will reiterate and clarify my assumptions regarding the syntactic derivation.
I adopt a slightly simplified version of the formal grammar developed by \textcite{collins2016formalization} which I will augment slightly based on new assumptions.
A \textit{derivation} is defined as a finite sequence of \textit{stages}, $\langle S_1, S_2 \ldots S_n\rangle$.
Each stage $S_i$ in a derivation is a pair $\langle LA_i, W_i\rangle$, where $LA_i$ is a set of lexical items called the \textit{lexical array} and $W_i$ is a set of syntactic objects called the \textit{workspace}.
The computational operations (Merge, Select, Copy, Transfer) play the role that rules of inference play in deductive systems, that is, they map derivational stages onto subsequent stages.
A given stage $S_i$ \textit{derives} a subsequent stage $S_{i+1}$ if and only if some operation, applied to $S_i$, yields $S_{i+1}$.

\section{A successful derivation in English}\label{sec:EngDeriv}
In many ways, successful derivations, like happy families, are uninteresting, but they still must be demonstrated in order to show where the crashing derivations go wrong.

To begin with, we derive the result phrase.
The formal derivation of the resP is given in \autoref{tab:EngResP} and the resulting unlabelled structure is given in \autoref{fig:EngResP}.
	\begin{longtabu}{llll}
	\textbf{Stage} & \textbf{LA} & \textbf{Workspace} & \\
	\cline{1-3}
	1 & $
	\begin{Bmatrix*}[l]
		\sqrt{\textsc{flat}},\\
		adj_\emptyset,\\
		res,\\
		\text{DP}
	\end{Bmatrix*}
	$ & $\emptyset$ & Select($\sqrt{\textsc{flat}}$)\\
	2 & $
	\begin{Bmatrix*}[l]
		adj_\emptyset,\\
		res,\\
		\text{DP}
	\end{Bmatrix*}
	$ & $\left\{\sqrt{\textsc{flat}}\right\}$ & Select($adj_\emptyset$)\\
	3 & $
	\begin{Bmatrix*}[l]
		res,\\
		\text{DP}
	\end{Bmatrix*}
	$ & $
	\begin{Bmatrix*}[l]
		adj_\emptyset,\\
		\sqrt{\textsc{flat}}
	\end{Bmatrix*}$
	& Merge($adj_\emptyset, \sqrt{\textsc{flat}}$)\\
	4 & $
	\begin{Bmatrix*}[l]
		res,\\
		\text{DP}
	\end{Bmatrix*}
	$ & $\left\{ \left\{_\alpha adj_\emptyset, \sqrt{\textsc{flat}} \right\} \right\}$ & Select(DP)\\
	5 & $\left\{ res \right\}$ & $
	\begin{Bmatrix*}[l]
		\text{DP},\\
		\left\{_\alpha adj_\emptyset, \sqrt{\textsc{flat}} \right\}
	\end{Bmatrix*}
	$ & Merge(DP, $\alpha$)\\
	6 & $\left\{ res \right\}$ & $ \left\{ \left\{_\beta \text{DP}, \left\{_\alpha adj_\emptyset, \sqrt{\textsc{flat}} \right\} \right\} \right\}$ &
	Select(\textit{res})\\
	7 & $\emptyset$ & $
	\begin{Bmatrix*}[l]
		res,\\
		\left\{_\beta \text{DP}, \left\{_\alpha adj_\emptyset, \sqrt{\textsc{flat}} \right\} \right\}
	\end{Bmatrix*}
	$ & Merge(\textit{res}, $\beta$)\\
	8 & $\emptyset$ & $\left\{ \left\{_\gamma res, \left\{_\beta \text{DP}, \left\{_\alpha adj_\emptyset, \sqrt{\textsc{flat}} \right\} \right\}\right\} \right\}$
	& Copy(DP)\\
	9 & $\emptyset$ & $
	\begin{Bmatrix*}[l]
		\text{DP},\\
		\left\{_\gamma res, \left\{_\beta \text{DP}, \left\{_\alpha adj_\emptyset, \sqrt{\textsc{flat}} \right\} \right\}\right\}
	\end{Bmatrix*}
	$
	& Merge(DP, $\gamma$)\\
	10 & $\emptyset$ & $
	\left\{\left\{_\delta \text{DP},\left\{_\gamma res, \left\{_\beta \text{DP}, \left\{_\alpha adj_\emptyset, \sqrt{\textsc{flat}} \right\} \right\}\right\}\right\}\right\}
	$
	& Transfer($\beta$)\\
	\caption{The derivation of an English resP}
	\label{tab:EngResP}
\end{longtabu}

\begin{figure}[h]
	\centering
{\small
  \begin{forest}
      nice empty nodes,sn edges,baseline,for tree={
    calign=fixed edge angles,
  calign primary angle=-30,calign secondary angle=70}
      [$\delta$
        [DP$_\varphi$[{\rm the metal},roof]]
        [$\gamma$
          [res]
          [$\beta$
        [DP$_\varphi$[{\rm the metal},name=SC DP,roof]]
        [$\alpha$
          [adj$_\emptyset$]
          [{\rm flat}]
        ]
          ]
        ]
      ]
      \draw[thick] ([xshift=-36pt, yshift=-24pt]SC DP) arc[start angle=170,end angle=130,radius=7.5cm];
  \end{forest}
}
	\caption{An unlabelled resP}
\label{fig:EngResP}
\end{figure}

Assuming res is a phase head, its complement $\beta$ is transferred and must be labelled along with the SOs it contains.
The small clause $\beta$ ($\left\{_\beta \lfloor \text{DP}\rfloor, \left\{_\alpha adj_\emptyset, \sqrt{\textsc{flat}} \right\} \right\}$) is a Phrase-Phrase structure, but since one of its constituent parts, the DP, is a lower copy, that part is invisible to the labelling algorithm.
Therefore, only the adjective ($\left\{_\alpha adj_\emptyset, \sqrt{\textsc{flat}} \right\}$) is available to provide a label.
Assuming that roots are inert for labelling, and that uninflected categorizing heads can label, $adj_\emptyset$ is selected to label $\beta$.
Since $\alpha$ is a head-root structure, it is labelled by the categorizing head $adj_\emptyset$.
So, $\beta$ is successfully labelled and therefore convergent at the CI interface.
\ex. LA$(\left\{_\beta \lfloor\text{DP}\rfloor, \left\{_\alpha adj_\emptyset, \sqrt{\textsc{flat}} \right\} \right\}) = \left[_{adj}\, \lfloor\text{DP}\rfloor, \left[_{adj}\, adj_\emptyset, \sqrt{\textsc{flat}} \right]  \right]$

Since $\gamma$ and $\delta$ are not transferred along with $\beta$, we do not need to discuss their labels yet.

We then derive the next phase as in \autoref{tab:EngVP}.
Note that resP and the DP are in the initial lexical array for this derivation.
While this stipulation is necessary to derive the next phase, a number of aspects of it are poorly understood.
I believe that this lack of understanding is directly related to the nature of the Transfer operation, which seems to be a stand-in for the interfaces.
Since a full understanding of the interfaces requires an entire research program, it is decidedly beyond the scope of this thesis, and I will make do with stipulation here.

The unlabelled structure is given in \autoref{fig:EngVP}.
{\small
\begin{longtabu}{llll}
\textbf{Stage} & \textbf{LA} & \textbf{Workspace} &\\
\cline{1-3}
1 & $
\begin{Bmatrix*}[l]
	\text{resP},\\
	\text{DP},\\
	v,\\
	\sqrt{\textsc{hammer}},\\
	\text{AgrO}
\end{Bmatrix*}
$ & $\left\{  \right\}$ & Select$(\sqrt{\textsc{hammer}})$\\
2 & $
\begin{Bmatrix*}[l]
	\text{resP},\\
	\text{DP},\\
	v,\\
	\text{AgrO},
\end{Bmatrix*}
$ & $\left\{ \sqrt{\textsc{hammer}} \right\}$ & Select$(v)$\\
3 & $
\begin{Bmatrix*}[l]
	\text{resP},\\
	\text{DP},\\
	\text{AgrO}
\end{Bmatrix*}
$ & $ 
\begin{Bmatrix*}[l]
	\sqrt{\textsc{hammer}},\\
	v
\end{Bmatrix*}
$ & Merge$(v, \sqrt{\textsc{hammer}})$\\
4 & $
\begin{Bmatrix*}[l]
	\text{resP},\\
	\text{DP},\\
	\text{AgrO}
\end{Bmatrix*}
$ & $ 
\left\{\left\{_\alpha v, \sqrt{\textsc{hammer}}\right\}\right\}
$ & Select(DP)\\
5 & $
\begin{Bmatrix*}[l]
	\text{resP},\\
	\text{AgrO}
\end{Bmatrix*}
$ & $ 
\begin{Bmatrix*}[l]
	\text{DP}\\
	\left\{_\alpha v, \sqrt{\textsc{hammer}}\right\}
\end{Bmatrix*}
$ & Merge(DP, $\alpha$)\\
6 & $
\begin{Bmatrix*}[l]
	\text{resP},\\
	\text{AgrO}
\end{Bmatrix*}
$ & $ \left\{\left\{_\beta\text{DP}, \left\{_\alpha v, \sqrt{\textsc{hammer}}\right\}\right\}\right\}$ &
Select(resP)\\
7 & $\left\{ \text{AgrO} \right\}$ & $
\begin{Bmatrix*}[l]
	\text{resP},\\
	\left\{_\beta\text{DP}, \left\{_\alpha v, \sqrt{\textsc{hammer}}\right\}\right\}
\end{Bmatrix*}
$ & Merge\footnotemark($\beta$, resP)\\
8 & $\left\{ \text{AgrO} \right\}$ & $
\left\{\left\{_\zeta\left\{_\beta\text{DP}, \left\{_\alpha v, \sqrt{\textsc{hammer}}\right\}\right\}, \text{resP}\right\}\right\}
$ & Select(AgrO)\\
9 & $\emptyset$ & $
\begin{Bmatrix*}[l]
	\text{AgrO},\\
	\left\{_\zeta\left\{_\beta\text{DP}, \left\{_\alpha v, \sqrt{\textsc{hammer}}\right\}\right\}, \text{resP}\right\}
\end{Bmatrix*}
$ & Merge(AgrO, $\zeta$)\\
10 & $\emptyset$ & $
\left\{\left\{_\eta \text{AgrO}, \left\{_\zeta \left\{_\beta\text{DP}, \left\{_\alpha v, \sqrt{\textsc{hammer}}\right\}\right\}, \text{resP}\right\}\right\}\right\}$ & Copy(DP) \\
11 & $\emptyset$ & $
\begin{Bmatrix*}[l]
	\text{DP},\\
	\left\{_\eta \text{AgrO}, \left\{_\zeta\left\{_\beta\text{DP}, \left\{_\alpha v, \sqrt{\textsc{hammer}}\right\}\right\}, \text{resP}\right\}\right\}
\end{Bmatrix*}
$ & Merge(DP, $\eta$) \\
12 & $\emptyset$ & $
\left\{\left\{_\kappa
	\text{DP},\left\{_\eta \text{AgrO}, \left\{_\zeta\left\{_\beta\text{DP}, \left\{_\alpha v, \sqrt{\textsc{hammer}}\right\}\right\},\text{resP}\right\}\right\}
\right\}\right\}$ & \dots\\
\caption{The derivation of an English resultative VP}
\label{tab:EngVP}
\end{longtabu}}
\footnotetext{This instance of ``Merge'' is, in fact, an instance of adjunction.
I represent it as Merge in order to maintain the simplicity of the formal grammar.}

\begin{figure}[h]
\centering
{\small
	\begin{forest}
		nice empty nodes,sn edges,baseline
		[$\kappa$
			[DP$_\varphi$[the metal,roof]]
			[$\eta$
				[AgrO$_\varphi$]
				[$\zeta$
					[$\beta$
						[$\alpha$
							[$v$]
							[$\sqrt{\textsc{hammer}}$]
						]
						[$\lfloor\text{DP}\rfloor$]
					]
					[$\delta$
						[$\lfloor\text{DP}\rfloor$]
						[$\gamma$
							[res]
							[flat]
						]
					]
				]
			]
		]
	\end{forest}
}
\caption{An unlabelled English resultative}
\label{fig:EngVP}
\end{figure}
When this is transferred, triggered, presumably, by the merging of the phase head Voice, it is labelled just as any transitive VP would be.
The largest object $\kappa$ is a phrase-phrase structure with agreeing features, so it will receive a $\langle\varphi,\varphi\rangle$ label.
The remaining objects will receive head-labels, with the exception of the host-adjunct structure $\zeta$.

As in previous theories of grammar, host-adjunct structures are problematic in label theory.
Structures like $\zeta$ are phrase-phrase structures, meaning they can only be labelled if the constituent parts agree for some feature, or one of the constituent parts is somehow invisible.
Since, almost by definition, adjuncts are not selected by their hosts\footnote{
	Cartographic approaches to syntax \parencite[][and references therein]{cinque2009cartography}, however, assume that adjectives and adverbs are selected by functional heads.
	This assumption does not, to my knowledge, extend to phrase- or clause-sized modifiers though.
} it is unlikely that there is agreement between adjuncts and their hosts.
If there is no agreement, then the only way for $\zeta$ to be labellable is if one of its parts is inert.
Since host-adjunct structures, again almost by definition, have the properties of the host and not those of the adjunct, it is reasonable to think that the host is active and the adjunct is inert.
I will therefore provisionally assume that the host $\beta$ provides the label for $\zeta$, and $\delta$ is inert.
This matter will be addressed in \cref{sec:modifications}.
\begin{figure}[h]
	\centering
{\small
	\begin{forest}
		nice empty nodes,sn edges,baseline
		[{$\langle\varphi,\varphi\rangle$}
			[DP$_\varphi$
				[the metal,roof]
			]
			[AgrO
				[AgrO$_{\varphi}$]
				[$v$
					[$v$
						[$v$
							[$v$]
							[$\sqrt{\textsc{hammer}}$]
						]
						[$\lfloor\text{DP}\rfloor$]
					]
					[res
						[$\lfloor\text{DP}\rfloor$]
						[$res$
							[res]
							[flat]
						]
					]
				]
			]
		]
	\end{forest}
}
	\caption{A labelled English resultative}
	\label{fig:EngVPLabelled}
\end{figure}

In this section, we have seen how a convergent resultative is derived in English.
The next section, however, is truly where the rubber meets the road.
There I show that the same grammar that generates resultatives in English will fail to generate them in French.
As we will see, the crucial operation, the one which will be blocked in French, is the movement of DP from the adjectival Small Clause.

\section{Two crashing derivations in French}\label{sec:Fre-deriv}
By hypothesis, the only relevant difference between English and French is that the lexicon of French contains only inflected category heads (specifically $adj_\emptyset \centernot\in \textsc{lex}$ and $adj_\varphi \in \textsc{lex}$).
In this section, I will attempt to derive a resultative with an $adj_\varphi$ and show that such a derivation inevitably either crashes due to failure to label or simply does not derive a resultative.
The first attempt will reproduce an English derivation and crash, while the second will avoid that crash but fail to move the object DP into the VP, and thus will be unable to derive the proper structure.

\subsection{Crashing Derivation}\label{sec:crash}
Consider the derivation described in \autoref{sec:EngDeriv} with $adj_\varphi$ replacing $adj_\emptyset$.
The resP will be derived in the same fashion, as shown in \autoref{tab:FreResP1}.
\begin{longtabu}{llll}
	\textbf{Stage} & \textbf{LA} & \textbf{Workspace} & \\
	\cline{1-3}
	1 & $
	\begin{Bmatrix*}[l]
		\sqrt{\textsc{plat-}},\\
		adj_\varphi,\\
		res,\\
		\text{DP}
	\end{Bmatrix*}
	$ & $\emptyset$ & Select($\sqrt{\textsc{plat-}}$)\\
	2 & $
	\begin{Bmatrix*}[l]
		adj_\varphi,\\
		res,\\
		\text{DP}
	\end{Bmatrix*}
	$ & $\left\{\sqrt{\textsc{plat-}}\right\}$ & Select($adj_\varphi$)\\
	3 & $
	\begin{Bmatrix*}[l]
		res,\\
		\text{DP}
	\end{Bmatrix*}
	$ & $
	\begin{Bmatrix*}[l]
		adj_\varphi,\\
		\sqrt{\textsc{plat-}}
	\end{Bmatrix*}$
	& Merge($adj_\varphi, \sqrt{\textsc{plat-}}$)\\
	4 & $
	\begin{Bmatrix*}[l]
		res,\\
		\text{DP}
	\end{Bmatrix*}
	$ & $\left\{ \left\{_\alpha adj_\varphi, \sqrt{\textsc{plat-}} \right\} \right\}$ & Select(DP)\\
	5 & $\left\{ res \right\}$ & $
	\begin{Bmatrix*}[l]
		\text{DP},\\
		\left\{_\alpha adj_\varphi, \sqrt{\textsc{plat-}} \right\}
	\end{Bmatrix*}
	$ & Merge(DP, $\alpha$)\\
	6 & $\left\{ res \right\}$ & $ \left\{ \left\{_\beta \text{DP}, \left\{_\alpha adj_\varphi, \sqrt{\textsc{plat-}} \right\} \right\} \right\}$ &
	Select(\textit{res})\\
	7 & $\emptyset$ & $
	\begin{Bmatrix*}[l]
		res,\\
		\left\{_\beta \text{DP}, \left\{_\alpha adj_\varphi, \sqrt{\textsc{plat-}} \right\} \right\}
	\end{Bmatrix*}
	$ & Merge(\textit{res}, $\beta$)\\
	8 & $\emptyset$ & $\left\{ \left\{_\gamma res, \left\{_\beta \text{DP}, \left\{_\alpha adj_\varphi, \sqrt{\textsc{plat-}} \right\} \right\}\right\} \right\}$
	& Copy(DP)\\
	9 & $\emptyset$ & $
	\begin{Bmatrix*}[l]
		\text{DP},\\
		\left\{_\gamma res, \left\{_\beta \text{DP}, \left\{_\alpha adj_\varphi, \sqrt{\textsc{plat-}} \right\} \right\}\right\}
	\end{Bmatrix*}
	$
	& Merge(DP, $\gamma$)\\
	10 & $\emptyset$ & $
	\left\{\left\{_\delta \text{DP},\left\{_\gamma res, \left\{_\beta \text{DP}, \left\{_\alpha adj_\varphi, \sqrt{\textsc{plat-}} \right\} \right\}\right\}\right\}\right\}
	$
	& Transfer($\beta$)\\
	\caption{The derivation of a French resP}
	\label{tab:FreResP1}
\end{longtabu}
\begin{figure}[h]
	\centering
{\small
  \begin{forest}
      	nice empty nodes,sn edges,baseline,for tree={
    	calign=fixed edge angles,
	calign primary angle=-30,calign secondary angle=70}
      [$\delta$
        [DP$_\varphi$[{\rm le m\'etal},roof]]
        [$\gamma$
          [res]
          [$\beta$
        [DP$_\varphi$[{\rm le m\'etal},name=SC DP,roof]]
        [$\alpha$
          [adj$_\varphi$]
          [{\rm plat-}]
        ]
          ]
        ]
      ]
      \draw[thick] ([xshift=-36pt, yshift=-24pt]SC DP) arc[start angle=170,end angle=130,radius=7.5cm];
  \end{forest}
}
	\caption{An unlabelled resP}
\label{fig:FreResP}
\end{figure}
Upon Transfer, $\beta$ must be labelled and since the DP has been moved, it is invisible to the labelling algorithm.
The label of $\beta$, then will be the label of $\alpha$ ($\left\{ adj_\varphi, \sqrt{\textsc{plat-}} \right\}$).
In the English case, $adj_\emptyset$ was able to provide a label, but following \textcite{chomsky2015problems}, the French $adj_\varphi$ is too weak to label without being strengthened by $\varphi$-agreement.
The DP which is merged with $\alpha$ would agree with $adj_\varphi$, but since it is a lower copy, it is inert, and therefore cannot take part in agreement.
Since $adj_\varphi$ has not been agreed with, it remains too weak to label $\alpha$ and, by extension, too weak to label $\beta$.
The derivation, then, crashes due to a failure to label.

So, attempting to derive a resultative in French as we did in English yields a crash at the interfaces.
Perhaps, though, there is another way to derive resultatives without causing a crash.
In the next section I attempt such an alternative derivation, but ultimately this attempt, while it doesn't crash, will not derive a resultative.
\subsection{Failed Derivation}\label{sec:fail}
The fatal flaw in the previous derivation was moving the DP from the small clause before it could agree with $adj_\varphi$.
Consider the following derivation of \autoref{fig:FreResP2} given in \autoref{tab:FreResP2}.
\begin{longtabu}{llll}
	\textbf{Stage} & \textbf{LA} & \textbf{Workspace} & \\
	\cline{1-3}
	1 & $
	\begin{Bmatrix*}[l]
		\sqrt{\textsc{plat-}},\\
		adj_\varphi,\\
		res,\\
		\text{DP}
	\end{Bmatrix*}
	$ & $\emptyset$ & Select($\sqrt{\textsc{plat-}}$)\\
	2 & $
	\begin{Bmatrix*}[l]
		adj_\varphi,\\
		res,\\
		\text{DP}
	\end{Bmatrix*}
	$ & $\left\{\sqrt{\textsc{plat-}}\right\}$ & Select($adj_\varphi$)\\
	3 & $
	\begin{Bmatrix*}[l]
		res,\\
		\text{DP}
	\end{Bmatrix*}
	$ & $
	\begin{Bmatrix*}[l]
		adj_\varphi,\\
		\sqrt{\textsc{plat-}}
	\end{Bmatrix*}$
	& Merge($adj_\varphi, \sqrt{\textsc{plat-}}$)\\
	4 & $
	\begin{Bmatrix*}[l]
		res,\\
		\text{DP}
	\end{Bmatrix*}
	$ & $\left\{ \left\{_\alpha adj_\varphi, \sqrt{\textsc{plat-}} \right\} \right\}$ & Select(DP)\\
	5 & $\left\{ res \right\}$ & $
	\begin{Bmatrix*}[l]
		\text{DP},\\
		\left\{_\alpha adj_\varphi, \sqrt{\textsc{plat-}} \right\}
	\end{Bmatrix*}
	$ & Merge(DP, $\alpha$)\\
	6 & $\left\{ res \right\}$ & $ \left\{ \left\{_\beta \text{DP}, \left\{_\alpha adj_\varphi, \sqrt{\textsc{plat-}} \right\} \right\} \right\}$ &
	Select(\textit{res})\\
	7 & $\emptyset$ & $
	\begin{Bmatrix*}[l]
		res,\\
		\left\{_\beta \text{DP}, \left\{_\alpha adj_\varphi, \sqrt{\textsc{plat-}} \right\} \right\}
	\end{Bmatrix*}
	$ & Merge(\textit{res}, $\beta$)\\
	8 & $\emptyset$ & $\left\{ \left\{_\gamma res, \left\{_\beta \text{DP}, \left\{_\alpha adj_\varphi, \sqrt{\textsc{plat-}} \right\} \right\}\right\} \right\}$
	& Transfer($\beta$)\\
	\caption{The derivation of a French resP with an \textit{in situ} DP}
	\label{tab:FreResP2}
\end{longtabu}
\begin{figure}[h]
	\centering
{\small
  \begin{forest}
      	nice empty nodes,sn edges,baseline,for tree={
    	calign=fixed edge angles,
	calign primary angle=-30,calign secondary angle=70}
        [$\gamma$
          [res]
          [$\beta$
        [DP$_\varphi$[{\rm le m\'etal},name=SC DP,roof]]
        [$\alpha$
          [adj$_\varphi$]
          [{\rm plat-}]
        ]
          ]
        ]
      \draw[thick] ([xshift=-36pt, yshift=-24pt]SC DP) arc[start angle=170,end angle=130,radius=7.5cm];
  \end{forest}
}
\caption{An unlabelled French resP with an \textit{in situ} DP}
\label{fig:FreResP2}
\end{figure}
Unlike the case in \autoref{sec:crash}, the transferred object $\beta$ will be labellable.
The \textit{in situ} DP will $\varphi$-agree with $adj_\varphi$, and $\beta$ will be labelled with the pair $\langle\varphi,\varphi\rangle$.
Furthermore, since $adj_\varphi$ has been strengthened by agreement, it will be able to label $\alpha$.
Thus, the transferred object is labelled as in \ref{ex:FreresPlabel}.
\ex. LA$(\beta) = \left[_{\langle\varphi,\varphi\rangle} \text{DP} \left[_{adj} adj, \sqrt{\textsc{plat}}  \right]  \right]$\label{ex:FreresPlabel}

The resP can be derived without a crash, but this will turn out to be something of a Pyrrhic victory.
The French small clause is labellable because the DP remains \textit{in situ}, but this same fact means that the DP is now inaccessible to further operations such as Copy and Merge.
If we cannot copy and remerge the DP, we will unable to continue the derivation.
We can't merge a DP with a verb if that DP is inaccessible to Merge.
Thus, our attempt to avert a crash has, in fact, doomed the derivation.
\section{On Bare Stem Compounding}
So far, I have demonstrated that, given my theoretical assumptions, resultatives can be generated only if the result adjective is categorized by a featureless $adj_{\emptyset}$ head.
I also proposed that a child acquires featureless categorizing heads if they encounter productive bare-stem compounding (BSC) in their PLD.
In this section, I will propose an analysis of bare-stem compounding that is consistent with these proposals and my theoretical assumptions.

If we restrict ourselves to endocentric bare-stem compounding like \textit{bourbon bar}, for instance, then our analysis must be consistent with the possibility of endocentricity.
That is, a proper analysis of \textit{bourbon bar} must naturally explain why it describes a type of bar rather than a type of bourbon.
Furthermore, an analysis of BSC must allow for the fact that a language's ability to generate these compounds depends on the properties of the categorizing heads in that language.

I will discuss three possibilities below:
	one in which a compound is formed by directly merging roots together,
	a second in which two categorized roots ($\left\{ cat, \sqrt{\textsc{Root}} \right\}$, or \textit{stems}) are directly merged together,
	and a third in which a stem merges with a root.
As we shall see, only the third analysis can account for endocentricity and parametric variation without major stipulation.
\subsection{Root-root compounding}
The first analysis that I will consider is one in which roots merge directly with each other as in \cref{fig:RootRoot}.
\begin{figure}[h]
	\centering
	\begin{forest}
    nice empty nodes,sn edges,baseline,
		[$\beta$
			[$n$]
			[$\alpha$
				[$\sqrt{\textsc{Bourbon}}$]
				[$\sqrt{\textsc{Bar}}$]
			]
		]
	]
	\end{forest}
	\caption{A root-root analysis of compounding}
	\label{fig:RootRoot}
\end{figure}
Immediately, we can see that the symmetrical nature of merge (\textit{i.e.} the fact that merge creates an unordered set) renders endocentricity impossible; \textit{bourbon bar} would be indistinguishable from \textit{bar bourbon}.

Setting this problem aside for the moment, could this analysis account for the parametric variation?
That is, can we show that \cref{ex:FreRootRoot} crashes, while \cref{ex:EngRootRoot} converges?
\ex.* $[_{\beta}\, n_{F}, [_{\alpha} \sqrt{\textsc{Bourbon}}, \sqrt{\textsc{Bar}}  ]  ]$ \label{ex:FreRootRoot}

\ex. $[_{\beta}\, n_{\emptyset}, [_{\alpha} \sqrt{\textsc{Bourbon}}, \sqrt{\textsc{Bar}}  ]  ]$ \label{ex:EngRootRoot}

Since $n$ is the least embedded atomic element in both \cref{ex:FreRootRoot,ex:EngRootRoot}, it would label both phrases if it were strong enough.
The French $n_{F}$ in \cref{ex:FreRootRoot} will, of course, need to be strengthened by Agree, but compare the proposed compound structure with that of a simple noun in \cref{ex:FreNoun}.
\ex. $[_{\alpha} n_{F}, \sqrt{\textsc{bar}}]$\label{ex:FreNoun}

Note that in both the simple noun \cref{ex:FreNoun} and the compound noun \cref{ex:FreRootRoot}, the categorizing head $n_{F}$ is an immediate constituent of the phrase in question.
Furthermore, in both cases, $n_{F}$ is the only possible labeller, as all of the other constituents are roots.
It follows, then, that $\beta$ in \cref{ex:FreRootRoot} and $\alpha$ in \cref{ex:FreNoun} are indistinguishable with respect to labelling.
Therefore, a root-root analysis of BSC gives us no principled way of ruling out compound nouns in French-like languages, without also ruling out simple nouns.

Since this analysis lacks both necessary properties for compounds, I will set it aside.
\subsection{Stem-stem compounding}
In the second analysis, bare-stem compounds are created by merging two stems (\textit{i.e.}, \{$n$, $\sqrt{\textsc{root}}$\}).
Like the root-root possibility, the stem-stem possibility is symmetrical, as we can see in \cref{fig:StemStem}
\begin{figure}[h]
	\centering
	\begin{forest}
    sn edges,baseline,
    [$\gamma$,tier=zero
	    [$\alpha$,tier=one
		    [$n$,tier=two]
				[$\sqrt{\textsc{bourbon}}$,tier=two]
			]
			[$\beta$,tier=one
				[$n$,tier=two]
				[$\sqrt{\textsc{bar}}$,tier=two]
			]
		]
	\end{forest}
	\caption{A stem-stem analysis of compounding}
	\label{fig:StemStem}
\end{figure}
As with the root-root option, the symmetry of this structure precludes endocentricity.
Again setting this problem aside, let's consider how it fares with respect to parametric variation.

If we consider the compound $\gamma$ as a whole, we can see that it is a phrase-phrase structure, and therefore labellable in only two situations:
	either the constituent phrases has moved, or there is agreement between the ``heads'' of the two constituent phrases (the two $n$ heads in this case).
The first is inapplicable since both members of the compound remain \textit{in situ}.
As for the second, in the case of an English-type language, there is certainly no agreement between the two $n$ heads, as they have, by hypothesis, no features, which are required for Agree.
A stem-stem analysis of compounds, therefore, seems to wrongly predict that English-type grammars do not generate bare-stem compounds; another strike against them.
French-like languages, on the other hand, have feature-bearing $n$ heads which are in the ideal structural configuration to agree with each other, but it is unlikely that they would undergo agreement with each other due to the types of features that they have.
The standard cases of agreement involve a featural asymmetry between the agreeing heads---an interpretable feature checks an uninterpretable feature; a valued feature values an unvalued feature---but in the structure in \cref{fig:StemStem}, the would-be agreeing heads are of the same type, and therefore have identical featural endowments.
Since there is no featural asymmetry, it is likely that there can be no agreement between the two $n_{F}$ heads in French-type languages.
If they cannot agree with each other, then, just as in the case of English-type languages, we would expect the structure in \cref{fig:StemStem} to be unlabellable.
If they can agree with each other, then $\gamma$ in \cref{fig:StemStem} should be labelled $\langle F,F\rangle$, and therefore, the structure in \cref{fig:StemStem} should be a licit compound.
So, the stem-stem analysis of compounds may wrongly predict that French-type grammars could generate bare-stem compounds.

Since a stem-stem analysis accounts for neither endocentricity nor the parametric variation with respect to bare-stem compounding, we will set it aside and move on to the third possible analysis.
\subsection{Root-stem compounding}
The final possibility is that compounds are formed by merging a root with a stem, as in \cref{fig:RootStem}.
\begin{figure}[h]
	\centering
	\begin{forest}
    sn edges,baseline,
    [$\beta$,tier=zero
	    [$\sqrt{\textsc{bourbon}}$,tier=one]
	    [$\alpha$,tier=one
		    [$n$,tier=two]
		    [$\sqrt{\textsc{bar}}$,tier=two]
			]
		]
	\end{forest}
	\caption{A root-stem analysis}
	\label{fig:RootStem}
\end{figure}
We can immediately see that this is an asymmetric structure, and, as such, one that is in principle able to capture endocentricity.
Under this view, \textit{bourbon bar} names a type of bar rather than a type of bourbon by virtue of the fact that the \textit{bar} root merges directly with the category-determining head, while \textit{bourbon} does so indirectly.
Compare this with the case of a Saxon genitive, such as \textit{the president's men}, which names some men rather than the president.
This endocentricity can be captured by the structure in \cref{fig:SaxonGenitive}, by virtue of the fact that \textit{men} is the complement, merging directly with the determiner \textit{-'s}, while \textit{the president} is its specifier.
\begin{figure}[h]
	\centering
	\begin{forest}
    nice empty nodes,sn edges,baseline,
		[DP
			[DP[the president,roof]]
			[
				['s]
				[$n$P[men,roof]]
			]
		]
	\end{forest}
	\caption{The Saxon genitive}
	\label{fig:SaxonGenitive}
\end{figure}

Now, what about the parametric variation?
Let's consider how the structure in \cref{fig:RootStem} would be labelled in the case of an English-type language.
The least embedded atom in $\beta$ is the root \textsc{bourbon}, which is invisible for labelling.
Therefore we must look for the next-least-embedded atom, which, assuming the root \textsc{bar} is invisible, would be $n_{\emptyset}$.
So, the label of $\beta$ would be $n_{\emptyset}$, as it would be for $\alpha$ as well.

Now consider the French-type language, where the featureless $n_{\emptyset}$ is replaced by $n_{F}$ with an incomplete feature set as in \cref{fig:RootStemFrench}.
Recall that $n_{F}$ cannot label a phrase unless it is strengthened to $n_{\langle F,F\rangle}$ by Agree.
So, the structure in \cref{fig:RootStemFrench}, is unlabellable on its own,
\begin{figure}[h]
	\centering
	\begin{forest}
    nice empty nodes,sn edges,baseline,
		[$\beta$
			[$\sqrt{\textsc{bourbon}}$]
			[$\alpha$
				[$n_{F}$]
				[$\sqrt{\textsc{bar}}$]
			]
		]
	\end{forest}
	\caption{A French-like Root-Stem structure}
	\label{fig:RootStemFrench}
\end{figure}
Furthermore, the $n_{F}$ head is embedded beneath a root, making it a more remote target for agreement with some head to be merged later.
If $n_{F}$ is too remote to be agreed with, then it will be too weak to label $\alpha$ or $\beta$, and therefore the structure in \cref{fig:RootStemFrench} would crash.
This difficulty would not, however, emerge in the case of a simple noun (\textit{e.g.}, $\left\{ n_{F}, \sqrt{\textsc{bar}} \right\}$), because the categorizing head $n_{F}$ is not embedded, and therefore will be available for agreement with a higher head.

A Root-Stem analysis of BSC, unlike the other two alternatives described above, can capture the fact that languages without featureless $cat_{\emptyset}$ heads cannot produce bare stem compounds.
Since a Root-Stem analysis of BSC seems to be the only one that correctly accounts for both endocentricity and parametric variation, it is likely the correct analysis.\footnote{
	It is not immediately obvious how this analysis would work for compounds formed by more complex stems such as \textit{attachment disorder}, or \textit{deionized water bottle}. 
	In order to extend my analysis to these cases, though, we would need an explicit theory of morphological derivation.
	Since such theorizing is outside the scope of this thesis, I will leave it for later research.
}

In her analysis of BSC, which she refers to as \textit{primary compounds}, \textcite{harley2009compounding} adopts what is essentially a Root-Stem structure, although her version, demonstrated in \cref{fig:HarleyCpd} differs from mine in a few interesting ways.

\begin{figure}[h]
	\centering
	\begin{forest}
		nice empty nodes,sn edges,baseline,
		[$n$P,
			calign=fixed edge angles,
			calign primary angle=-60,
			calign secondary angle=80
			[{$n^\circ$},name=n2
				[$\sqrt{}$
					[{$n^\circ$}
						[$\sqrt{\textsc{bourbon}}$]
						[{$n^\circ$}]
					]
					[$\sqrt{\textsc{bar}}$]
				]
				[{$n^\circ$}]
			]
			[$\sqrt{}$P,for tree={
	    		calign=fixed edge angles,
	    calign primary angle=-30,calign secondary angle=70}
    [$\sqrt{\textsc{bar}}$,name=bar]
				[$n$P
					[$n^\circ$,name=n1]
					[$\sqrt{\textsc{bourbon}}$,name=bourbon]
				]
			]
		]
		\draw[->] (bourbon)	to[out=south,in=south]		(n1);
		\draw[->] (n1) 		to[out=south west,in=south]	(bar);
		\draw[->] (bar) 	to[out=west, in=east]	(n2);
	\end{forest}
	\caption{Harley's (2009) analysis for \textit{bourbon bar}}
	\label{fig:HarleyCpd}
\end{figure}

Perhaps the biggest difference between Harley's analysis and mine is that the ``modifier'' (\textit{bourbon}) is merged below the ``head'' (\textit{bar}).
Related to that difference, is that Harley assumes that the surface appearance of \textit{bourbon bar} is due to a series of head movement operations, which Harley refers to as \textit{incorporation}.
Despite the similarities and differences between this analysis and mine, it would be difficult to compare the respective merits of the two, as they follow from two distinct sets of theoretical assumptions.
Harley, for instance, assumes that roots can both project phrases and select complements, assumptions that are explicitly ruled out under Chomsky's (2013) label theory, which this thesis adopts.
Comparing the two analyses, then, would require a deep comparison of the assumptions underlying them.
Such a comparison is beyond the scope of this thesis so I will set Harley's analysis aside.
In this chapter, I have shown that resultatives are, in fact, derivable in a label-theoretic grammar, provided the resultative adjective is categorized by an uninflected head $adj_\emptyset$.
I then demonstrated that the same derivation runs into difficulties if the resultative adjective is categorized by an inflected head $adj_\varphi$.
I do not claim to have demonstrated that deriving resultatives with $adj_\varphi$ is completely impossible (such a demonstration may be impossible).
Rather, I have merely pointed out two ways not to derive resultatives.
If we make the hypothesis that the parameter in \Next determines whether a language allows resultatives, then the demonstrations in this chapter represent an explanation of the resultative parameter, that is, an answer to the question of how resultatives are acquired, parameterized, and generated.
\ex. \textsc{lex} $\left\{ \text{includes, does not include} \right\}$ $v_\emptyset, n_\emptyset, adj_\emptyset, etc.$

Such an explanation was the stated goal of this thesis, but the hypotheses made in its service open up a number of questions, which I will address in Part II of the thesis.


\part{Dealing with the consequences}\label{sec:part2}
\chapter{Small clauses in French-like languages}\label{sec:FreSC}
\epigraph{All the pieces matter}{David Simon}
%        File: agree.tex
%     Created: Fri Mar 24 09:00 AM 2017 E
% Last Change: Fri Mar 24 09:00 AM 2017 E
%
% arara: pdflatex: {options: "-draftmode"}
% arara: biber
% arara: pdflatex: {options: "-draftmode"}
% arara: pdflatex: {options: "-file-line-error-style"}
\documentclass[MilwayThesis]{subfiles}

\begin{document}
Crucial to both label theory in general and its application in this thesis, is syntactic agreement.
The highest XP in a given chain must agree with its sister YP in order to converge, and a subset of functional heads must agree in order to label (\textit{e.g.}, English T$_\varphi$).
In this chapter I will discuss the theory of agreement, as it relates to labeling and show how the version of agree required for labeling avoids an undergeneration issue apparently predicted for predicative adjectives in French-type languages.

Agreement is required in label theory to account for (\textit{e.g.},) Subject-TP structures as in \Next.
\ex. [$_\alpha$ DP$_\varphi$ [$_\beta$ T$_\varphi$ ZP]]

The labels of both $\alpha$ and $\beta$ depend on agreement between the subject and T.
Since $\alpha$ is a Phrase-Phrase structure, its label will be $\langle\varphi,\varphi\rangle$ provided DP and T agree for $\varphi$.
This agreement also renders $\beta$ labelable, since, prior to agreement, English T$_\varphi$, with an incomplete $\varphi$-set, is too weak to label \parencite{chomsky2013problems}.
Agreement has the effect of strengthening T such that it can label.

To understand how Agree and Label interact, we must first consider what sort of operations they each are abstracting away from their actual implementations.
Both are take syntactic objects as inputs and operate on them iteratively and locally.
This means that labeling a structure like \Last requires labeling all of its substructures (Iterativity) and that labeling $\beta$ depends solely on the properties of $\beta$ (Locality).
The same, then, is true for Agree, which iteratively considers each substructure and performs agreement is the conditions for agreement are met.

Because labeling is sometimes contingent on agreement, the calculation of the latter must precede that of the former.
Assuming both Agree and Label occur after narrow syntax and before transfer to CI, this leaves us with two possibilities for ordering the two operations.
Either (i) individual iterations of Agree and Label are ordered with respect to each other forming a single Agree+Label cycle or (ii) Agree and Label each has its own cycle, and those cycles are ordered with respect to each other.
To decide between these two alternatives we can consider how Agree and Label interact with other components of grammar, specifically the SM and CI interfaces.
By hypothesis, Label feeds interpretation at CI but not at SM.
Agree, on the other hand, feeds Label and interpretation at SM.
This asymmetry points to the second alternative, where narrow syntax feeds an Agree cycle, which feeds SM interpretation and Label as in \ref{fig:SepCycles}.
The first alternative, in which Agree and Label are bundled into a single cycle as shown in \ref{fig:OneCycle}, predicts that both Label and Agree feed both interfaces, which is not what we seem to see in the data. 
\begin{figure}[h]
  \centering
  \begin{tikzpicture}
    \node (syn) at (1,3) {Narrow Syntax};
    \node[draw,rounded corners] (agree) at (1,2) {Agree};
    \node[draw,rounded corners] (label) at (1,1) {Label};
    \node (SM) at (0,1.5) {SM};
    \node (CI) at (1,0) {CI};
    \path[->](syn)	edge			(agree)
    (agree)		edge [loop right]	()
	   (agree.south)	edge			(label)
			  edge [bend left]	(SM)
	  (label)		edge [loop right]	()
			  edge			(CI);
  \end{tikzpicture}
  \caption{Agree and Label as separate cycles}
  \label{fig:SepCycles}
\end{figure}
\begin{figure}[h]
  \centering
  \begin{tikzpicture}
    \node (syn) at (1,3) {Narrow Syntax};
    \node[draw,rounded corners] (agree) at (1,2) {Agree};
    \node[draw,rounded corners] (label) at (1,1) {Label};
    \node (SM) at (0,0.25) {SM};
    \node (CI) at (1,0) {CI};
    \path[->](syn)edge 		(agree)
    (label.south)	edge		(CI)
    (label.south)	edge[bend left]	(SM);
    \draw[->] (label.east) arc(270:450:0.5cm);
    \draw[->] (agree.west) arc(90:270:0.5cm);
  \end{tikzpicture}
  \caption{Agree and Label bundled in a single cycle}
  \label{fig:OneCycle}
\end{figure}

<+FinishThisThought+>

Separating Agree and Label allows us to fix an apparent undergeneration problem in the account of *resultatives in chapter \ref{sec:deriving}, above.
Specifically, the account offered seems to predict that French disallows adjectives in predicate position as in \Next below.
\exg. Jeanne est grand -e\\
Joan is tall -\textsc{FSg}\\
``Joan is tall''

Indeed, without any further clarifications, the system proposed would bar \Last and similar structures.
Assuming the simplified sructure in \Next, below, for \Last, we expect the small clause $\beta$ and the adjP $\alpha$ to be unlabelable.
\ex. 
\begin{forest}
  nice empty nodes,sn edges,baseline,for tree={
    calign=fixed edge angles,
  calign primary angle=-30,calign secondary angle=70}
  [$\delta$
    [DP$_\varphi$[Jeanne,roof]]
    [$\gamma$
      [T$_\varphi$]
      [$\beta$
	[$\langle$DP$_\varphi\rangle$]
	[$\alpha$
	  [adj$_\varphi$]
	  [\textsc{grand}]
	]
      ]
    ]
  ]
\end{forest}

French \textit{adj} has a single $\varphi$-set, meaning it can only label if it is strengthened by an Agree operation.
If we were to assume that Label and Agree were bundled, we might assume that, just as lower copies are invisible to Label, they are invisible to Agree.
This would mean that movement of the DP \textit{Jeanne} would bleed agreement with \textit{adj}, thus rendering $\alpha$ and $\beta$ unlabelable.
We could save \LLast, by hypothesizing that lower copies are visible to Agree but invisible to label, but this would predict that French generates resultatives, clearly an unwanted result.
This means that the visibility conditions for Agree must be distinct from the visibility conditions for Label.
In the remainder of this section I will discuss the visibility conditions for Agree and show how \LLast can be derived.

To begin, let's compare the two relevant structures: the copular clause in \Next, and the resultative adjunct in \NNext.

\ex. Copular clause\label{fig:cop-clause}\\
\begin{forest}
  nice empty nodes,sn edges,baseline,for tree={
    calign=fixed edge angles,
  calign primary angle=-30,calign secondary angle=70}
  [$\zeta$
    [C]
    [$\delta$
      [DP$_\varphi$[Jeanne,roof,name=subj]]
      [$\gamma$
	[T$_\varphi$]
	[$\beta$
	  [$\langle$DP$_\varphi\rangle$]
	  [$\alpha$
	    [adj$_\varphi$]
	    [\textsc{grand}]
	  ]
	]
      ]
    ]
  ]
  \draw[thick] ([xshift=-12pt]subj.west) arc(180:130:5cm);
\end{forest}

\ex. Resultative adjunct \label{fig:result-adjunct}\\
\begin{forest}
  nice empty nodes,sn edges,baseline,for tree={
    calign=fixed edge angles,
    calign primary angle=-30,calign secondary angle=70
  }
  [$\delta$
    [DP$_\varphi$[le m\'etal,roof]]
    [$\gamma$
      [res]
      [$\beta$
	[$\langle$DP$_\varphi\rangle$,name=insitu]
	[$\alpha$
	  [adj]
	  [\textsc{plat}]
	]
      ]
    ]
  ]
  \draw[thick] (insitu.south west) arc(180:130:3cm);
\end{forest}

The most salient difference between the DP chains in \LLast and \Last are that the latter crosses a phase boundary, while the former does not.
This fact, I propose, is relevant for Agree's visibility conditions.
Whether or not a movement chain crosses a phase boundary can be made relevant to Agree if Agree operates not on syntactic objects but on chains.
To see how this would work, I will first clarify the notion of chains and then consider the derivations of \LLast and \Last in turn below.

Strictly speaking, I do not take chains to be theoretical primitives, rather they are expository conveniences. 
<+BarriersDefn+>
Following \textcite{collins2016formalization}, I replace the chains and links with syntactic objects and occurrences defined below.
\begin{defn}
  X is a \textit{syntactic object} (SO) iff\\
    X is a lexical item, or\\
    X is a set of syntactic objects. \parencite[Modified from][]{collins2016formalization}
  \label{def:so}
\end{defn}
\begin{defn}
  The \textit{position} of \I{SO}n in \I{SO}1 is a path, a sequence of syntactic objects $\langle\text{SO}_1,\text{SO}_2,\dots,\text{SO}_n\rangle$ where for all $0 < i < n$, $\text{SO}_{i + 1} \in \text{SO}_i$. \parencite{collins2016formalization}
  \label{def:position}
\end{defn}
\begin{defn}
  B \textit{occurs} in A at position P iff P = $\langle\text{A},\dots,\text{B}\rangle$. We also say B has an occurrence in A at position P (written \I{B}P).
  \label{def:occurrence}
\end{defn}

Consider the abstract syntactic object and its tree representation below in \Next and \NNext, respectively.
\ex. $\left\{ \text{X}, \left\{ \text{Y} \left\{ \text{X}, \text{Z} \right\} \right\} \right\}$

\ex.
\begin{forest}
  nice empty nodes,sn edges,baseline,for tree={
    calign=fixed edge angles,
    calign primary angle=-30,calign secondary angle=70
  }
  [$\alpha$
    [X]
    [$\beta$
      [Y]
      [$\gamma$
	[X]
	[Z]
      ]
    ]
  ]
\end{forest}

Based on the definitions above, we can say the following things about \LLast:
There are six SOs represented in \LLast: three lexical items (X, Y, Z) and three sets of SOs ($\alpha$, $\beta$, $\gamma$).
There is a single SO, X, with two occurrences in \LLast:
At $\langle \alpha, \text{X}\rangle$, and at $\langle \alpha, \beta, \gamma, \text{X}\rangle$

With this contrast between SOs and occurrences, we can limit the domain of Agree to complete chains without stipulating the existence of chains.
Consider the structure in \LLast, assuming that Y is a phase head. 
At a given stage of a derivation, it is reasonable to assume that the computation must track two sets of SOs: The set of SOs in the derivation (\textsc{Terms}$_\text{SO}$), and the set of active SOs (\textsc{Active}$_\text{SO}$).
For \LLast, re-represented in \Next, the two sets are given in \NNext.
\ex.
\begin{forest}
  nice empty nodes,sn edges,baseline,for tree={
    calign=fixed edge angles,
    calign primary angle=-30,calign secondary angle=70
  }
  [$\alpha$
    [X]
    [$\beta$
      [Y]
      [$\gamma$
	[X, name=x]
	[Z]
      ]
    ]
  ]
  \draw[thick] ([xshift=-1cm]x.west) arc(170:110:3cm);
\end{forest}

\ex.
\a. \textsc{Terms}$_\alpha$ = $\left\{ \text{X}, \text{Y}, \text{Z}, \alpha \beta, \gamma  \right\}$
\b. \textsc{Active}$_\alpha$ = $\left\{ \text{X}, \text{Y}, \alpha, \beta \right\}$

We can determine the input to Agree, then, by computing the set difference between the two sets as shown in \Next.
\ex. $\textsc{Terms}_\alpha \setminus \textsc{Active}_\alpha = \left\{ \text{Z}, \left[ \text{X}, \text{Z} \right] \right\}$ 

Note that X, which has moved to [Spec Y] is not a member of the input to Agree, despite the fact that there is a member of the input which has X as a member.
As such, X is invisible to Agree and Label.
With that understood in the abstract, we can consider the concrete cases of copular clauses and resultatives.

The copular clause, \ref{fig:cop-clause}, is derived from a small clause ($\beta = \left[ \text{DP,} \left[ \text{adj, }\textsc{grand} \right] \right]$) which merges with a finite T$_\varphi$ to form $\gamma$.
The DP, then, merges with $\gamma$ and C is merged triggering phase operations (Agree, Label, Transfer) on it's complement $\delta$.
Agree takes $\delta$, which contains a full DP chain, and values $\varphi$ features on T and adj with $\varphi$ features of DP.
This has two relevant effects: first, it strengthens T and adj such that they can label, and second it renders the lower copy of DP inactive/invisible for Label.
Label then operates on the output of Agree and successfully sends a labelled phrase marker to CI.
\ex. Agree(\ref{fig:cop-clause})\label{fig:agree-cop-clause}\\
\begin{forest}
  nice empty nodes,sn edges,baseline,for tree={
    calign=fixed edge angles,
    calign primary angle=-30,calign secondary angle=70
  }
  [$\delta$
    [DP$_\varphi$[Jeanne,roof,name=subj]]
    [$\gamma$
      [T$_{\langle\varphi,\varphi\rangle}$]
      [$\beta$
	[\sout{DP$_\varphi$}]
	[$\alpha$
	  [adj$_{\langle\varphi,\varphi\rangle}$]
	  [\textsc{grand}]
	]
      ]
    ]
  ]
\end{forest}

\ex. Label(\ref{fig:agree-cop-clause})
\a. Label($\delta$) = $\langle\varphi,\varphi\rangle$
\b. Label($\gamma$) = T
\b. Label($\beta$) = Label($\alpha$) = adj

Thus, the derivation of a copular clause converges in French.

Next, consider the resultative adjunct in \ref{fig:result-adjunct} which does not converge in French.
We start with a small clause which we merge with res, a phase head, forming $\gamma$.
We then merge the DP with $\gamma$, and commence our phase operations on $\beta$.
The phase complement, $\beta$, unlike that of the copular clause in \ref{fig:cop-clause}, contains only the tail of the DP chain.
Since Agree operates only on complete chains, the DP, \textit{le m\'etal}, is invisible to it, so there is no feature transfer between D and adj, nor is there any deletion of the DP.
The output of Agree, then, is passed to Label which fails to produce a labeled structure for CI.
\ex. Agree(\ref{fig:result-adjunct})\label{fig:agree-result-adjunct}\\
\begin{forest}
  nice empty nodes,sn edges,baseline,for tree={
    calign=fixed edge angles,
    calign primary angle=-30,calign secondary angle=70
  }
  [$\beta$
    [$\langle$DP$_\varphi\rangle$,name=insitu]
    [$\alpha$
      [adj]
      [\textsc{plat}]
    ]
  ]
\end{forest}

\ex. Label(\ref{fig:agree-result-adjunct})
\a. Label($\beta$) = Undefined \hfill (<+Reason+>)\\
\fcolorbox{black}{lightgray}{
  \begin{minipage}[t]{0.8\textwidth}
  \textbf{Note: }There are two possible explanations. Either DP is visible, and Label($\beta$) fails due to its symmetry, or DP is invisible and and adj is too weak to label.
  Either option will require some discussion.    
  \end{minipage}
}
\b. Label($\alpha$) = Undefined \hfill(adj is too weak to label)

So, if we separate Agree from Label, we are able to fix the apparent undergeneration, provided we assume Agree operates on chains, rather than occurrences.

The proposal that Agree operates on chains, rather than syntactic objects, gains support when we consider a fact about A-bar traces.
As has been noted by several authors (????), A-bar traces block \textit{wanna}-contraction.
\ex.\label{ex:wanna-contraction}
\a.\label{ex:wanna} Who$_i$ do you want to visit $t_i$? $\rightarrow$ Who do you wanna visit?
\b.\label{exwant-to} Who$_i$ do you want $t_i$ to visit Emma? $\rightarrow$ *Who do you wanna visit Emma?

The derivation of \Last[b] involves movement of \textit{who} across a phase boundary, creating a chain which is invisible to Agree.
Consider the structure of \Last[b] in \Next.
\ex. \label{fig:star-wanna-tree}
[$_\gamma$ Who$_i$ [$_\beta$ do$_C$ [$_\alpha$ you want $t_i$ to visit Emma]]]?

Upon $\gamma$ being formed, phase operations are preformed on $\alpha$.
When Agree operates on $\alpha$, only the tail of the A-bar chain $\langle$Who$_i$, $t_i\rangle$ is available, meaning it is invisible to Agree.
Since Agree, in addition to valuing features, also deletes copies, t$_i$ will remain in $\alpha$ when it is spelled out, until the rest of $\gamma$ is spelled out.
Assuming morphophonological processes operate on the output of Agree, the input of the contraction process will be the string/structure in \Next.
\ex. you want who to visit Emma.

And assuming adjacency is a precondition for contraction, we wouldn't expect contraction to occur in \Last.

In \ref{ex:wanna}, however the input to contraction is the string/structure in \Next.
\ex. you want to visit who

In this case, \textit{want} and \textit{to} are adjacent (or at least, no phonologically overt material intervenes between them), meaning contraction can occur.


\end{document}

\chapter{Movement from Specifier of resP}\label{sec:ACCing}
%        File: ACCing.tex
%     Created: Tue Feb 21 02:00 PM 2017 E
% Last Change: Tue Feb 21 02:00 PM 2017 E
%
% arara: pdflatex: {options: "-draftmode"}
% arara: biber
% arara: pdflatex: {options: "-draftmode"}
% arara: pdflatex: {options: "-file-line-error-style"}
\documentclass[MilwayThesis]{subfiles}

\begin{document}
Another issue with my account has to do with the sideward movement operation.
Recall that in order to derive an adjectival resultative, a DP must move sideward from the resP adjunct to the VP as in \cref{fig:ResStruct}.
\begin{figure}[h]
	\centering
	{\small
\begin{forest}
    nice empty nodes,sn edges,baseline,
%    for tree={
%    calign=fixed edge angles,
%    calign primary angle=-30,calign secondary angle=60}
    [VP
	    [VP
		    [hammer]
		    [DP[the metal,roof,name=compV]]
	    ]
	    [resP
		    [$\langle$DP$\rangle$,name=specRes]
		    [
			    [res]
			    [SC
				    [$\langle$DP$\rangle$,name=SCDP]
				    [
					    [adj]
					    [\textsc{flat}]
				    ]
			    ]
		    ]
	    ]
    ]
    \draw[->] (SCDP) to[out=south west, in=south] (specRes);
    \draw[->] (specRes) to[out=south, in=south east] (compV);
\end{forest}
}
\repeatcaption{fig:ResStruct}{The structure of a resultative}
\end{figure}
Note also that that sideward movement operation seems to be obligatory; that is, the DP that originates in the resP must also appear as the theme of the VP.
This obligatoriness can be seen in the fact that \cref{ex:double-theme-res} is ungrammatical.
\ex.*  Sam [$_{VP}$ hammered the nail] [$_{resP}$ the planks together].\label{ex:double-theme-res}\\
($\approx$ Sam hammered the nail and, as a result, the planks were fastened together)

An easy way of accounting for this would be to hypothesize that it is due to some property of the res head, and if this obligatory sideward movement were particular to resultatives, then, indeed, this would likely be the best way to proceed.
However, sideward movement seems to be obligatory in other cases.
First, there is the case of depictives, which differ from resultatives only in the fact that they lack a res head.
Also, as I will argue in this chapter, there is  obligatory sideward movement in certain so-called ACC-ing clauses in direct perception reports such as the embedded clause in \cref{ex:ACCing1}.
\ex. We heard [them shouting at the top of their lungs].\label{ex:ACCing1}

Furthermore, I will argue that this obligatory sideward movement is, in fact, a property of adjoined phrases.
Specifically, the generalization in \cref{ex:AdjunctGen} seems to hold.
\ex.\label{ex:AdjunctGen} Internally merged specifiers of adjoined phrases must move to the host phrase.

Such a generalization, I argue in \cref{sec:paradox}, cannot be accounted for in a theory of grammar based on feature satisfaction.
Label theory, however, is able in principle to derive this generalization, but in order to do so, it must be modified and extended.
I will perform such an extension and show that the resulting modified label theory can derive \cref{ex:AdjunctGen}.

\section{On ACC-ing clauses}

\textcite{cinque1996pseudo} discusses ACC-ing clauses (ACs) under direct perception verbs as in \cref{ex:ACCingDPR} and argues that they are ambiguous, having the two structures in \cref{ex:ACCingStructs}.\footnote{
	The main object of Cinque's study, in fact, is pseudo-relatives such as those in (i), which he argues are ambiguous between the three structures in (ii).
	\ex.[(i)]\label{ex:PR}
	\a. Ho visto Mario che correva a tutta velocit\`a. (Italian)
	\b. J'ai vu Mario qui courrait \'a tout vitesse. (French)

	\ex.[(ii)]\label{ex:PRStruct}
	\a. Ho [visto [$_\text{NP}$ Mario [$_\text{CP}$ che correva \ldots ]]]
	\b. Ho [visto [$_\text{CP}$ Mario [$_{\text{C}^\prime}$ che [$_\text{IP}$ correva \ldots ]]]]
	\c. Ho [[visto Mario] [$_\text{CP}$ \textit{ec} che correva \ldots]]

	He mentions ACs briefly in order to point out that his remarks and claims about pseudo-relatives largely apply to ACs.
	The main difference between the two constructions is that ACs are not analyzable as nominals, but a related form with nominal morphology serves this function.
	\ex.[(iii)] [Their singing of the national anthem] caused an international incident.

}
\ex. I saw Mario running at full speed. \label{ex:ACCingDPR} 

\ex.\label{ex:ACCingStructs}
\a. I [saw [$_\text{ProgP}$ Mario [$_{\text{Prog}^\prime}$ -ing [$_\text{VP}$ run \ldots]]]].
\b. I [[saw Mario] [$_\text{ProgP}$ \textit{ec} running \ldots]].

I argue in \cref{sec:paradox}, the fact that a single grammar can generate both of these structures presents a serious problem for standard theories of grammar.
I will therefore discuss them in greater detail in the remainder of this section.

In one structure, represented in \cref{fig:CompACCing}, the AC is merged as the complement of the perception verb.
The interpretation of this structure is one in which the running event was seen and by virtue of the meaning of \textit{run}, seeing a running event generally entails seeing the agent of that event.
\begin{figure}[h]
	\centering
\begin{forest}
    nice empty nodes,sn edges,baseline,for tree={
    calign=fixed edge angles,
    calign primary angle=-30,calign secondary angle=70}
    [VP
	    [V\\see,align=center]
	    [ProgP
		    [DP[Mario,roof]]
		    [Prog'[running at full speed,roof]]
	    ]
    ]
\end{forest}
	\caption{Complement ACC-ing structure}
	\label{fig:CompACCing}
\end{figure}
In the second structure, represented in \cref{fig:AdjunctACCing}, the ACC-ing subject is merged as the complement of the perception verb, while the AC (with a controlled subject) is adjoined to the VP.
The interpretation of this structure is one in which \textit{Mario} is seen and the event of \textit{Mario} being seen coincides with an event of \textit{Mario} running.
Again in this interpretation, due to the meaning of \textit{run}, seeing the agent of a running event generally entails seeing the event itself.
\begin{figure}[h]
	\centering
\begin{forest}
    nice empty nodes,sn edges,baseline,for tree={
    calign=fixed edge angles,
    calign primary angle=-30,calign secondary angle=70}
    [VP
	    [VP
		    [V\\see,align=center]
		    [DP$_i$[Mario,roof]]
	    ]		    
	    [ProgP
		    [$\langle$DP$_i\rangle$]
		    [Prog'[running at full speed,roof]]
	    ]
    ]
\end{forest}
	\caption{Adjunct ACC-ing structure}
	\label{fig:AdjunctACCing}
\end{figure}
By the assumptions made here, the argument \textit{Mario} can only be shared by the verb \textit{see} and the verb \textit{run} if it is merged with both, meaning it must move from [Spec, Prog] to [Comp, V].\footnote{
	Cinque assumes that [Spec, Prog] is occupied by a controlled PRO.
}
In the case of the complement AC in \cref{fig:CompACCing}, however, the argument \textit{Mario} seems to stay in situ in [Spec, Prog], suggesting that the movement operation represented in \cref{fig:AdjunctACCing} is, in fact, optional.
If the movement operation is optional, however, we would expect two additional structures for \cref{ex:ACCingDPR}: one, represented in \cref{fig:CompACCingMove}, in which the ProgP is the complement of \textit{saw} and \textit{Mario} has moved from [Spec, Prog], and another, represented in \cref{fig:AdjunctACCingStay}, in which the ProgP is an adjunct, but \textit{Mario} does not move from [Spec, Prog].
\begin{figure}[h]
	\centering
	\begin{forest}
	    nice empty nodes,sn edges,baseline,for tree={
	    calign=fixed edge angles,
	    calign primary angle=-30,calign secondary angle=70}
	    [AgrOP
		    [DP$_i$[Mario,roof]]
		    [
			    [AgrO]
			    [VP
				    [V\\see,align=center]
				    [ProgP
					    [$\langle\text{DP}_i\rangle$]
					    [Prog'[running at full speed,roof]]
				    ]
			    ]
		    ]
	    ]		
	\end{forest}
	\caption{Complement ACC-ing structure with object raising}
	\label{fig:CompACCingMove}
\end{figure}
\begin{figure}[h]
	\centering
	\begin{forest}
	    nice empty nodes,sn edges,baseline,for tree={
	    calign=fixed edge angles,
	    calign primary angle=-30,calign secondary angle=70}
	    [VP
		    [VP
			    [V\\see,align=center]
		    ]		    
		    [ProgP
			    [DP$_i$[Mario,roof]]
			    [Prog'[running at full speed,roof]]
		    ]
	    ]
	\end{forest}
	\caption{Adjunct ACC-ing structure without DP movement}
	\label{fig:AdjunctACCingStay}
\end{figure}
If we consider the consequences of this proposed optionality hypothesis, we can see that it cannot be true.

First, consider the structure in \cref{fig:AdjunctACCingStay}, in particular the fact that \textit{see} does not have an internal argument.
This is not \textit{per se} problematic, as verbs may be optionally transitive, but we would expect that \textit{see} in this structure could have an internal argument other than \textit{Mario}.
That is, if \cref{fig:AdjunctACCingStay} is a possible structure for \cref{ex:ACCingDPR}, then we would expect \cref{ex:ACCingDouble} to also be a licit sentence.
\ex.* I [$_\text{VP}$ [$_\text{VP}$ saw Sue] [$_\text{ProgP}$ Mario running at full speed]]. \label{ex:ACCingDouble}

If \textit{Mario} can remain in [Spec, Prog], then we have no way to rule out \textit{Sue} merging with \textit{see} and behaving as a direct object.
Of course, \cref{ex:ACCingDouble} is ungrammatical, suggesting that \textit{Mario} cannot remain in [Spec, Prog] if ProgP is adjoined to VP.
Movement from [Spec, Prog], then, cannot be optional, strictly speaking.
If it is not optional, perhaps it is obligatory.

If movement from [Spec, Prog] is obligatory, then we must revise Cinque's analysis of complement ACs.
Suppose, then, that \textit{Mario}, in the complement ACC-ing analysis of \cref{ex:ACCingDPR}, must raise to object.
In other words, suppose \cref{fig:CompACCingMove} is a possible structure of \cref{ex:ACCingDPR} and \cref{fig:CompACCing} is not.
Note that in \cref{fig:CompACCingMove}, \textit{Mario} is the grammatical object but not the theme of \textit{see}.
If this is the case then we expect that \textit{Mario} can become the subject of a passive derived from \cref{fig:CompACCingMove}.

Indeed, subjects of ACs can become passive subjects as in \cref{ex:ACCingPassive}, but it is not immediately obvious whether \cref{ex:ACCingPassive} is derived from an Adjunct AC structure or a Complement AC structure.
\ex. Mario was seen running at full speed. \label{ex:ACCingPassive}

If \cref{ex:ACCingPassive} had been derived from a Complement ACC-ing structure, however, then \textit{Mario} would not have been $\theta$-marked by \textit{see}.
So, in order to test whether passives like \cref{ex:ACCingPassive} can be generated in which the subject is not interpreted as the theme of the verb, that is, we need a clause of the form in \cref{ex:ACCingPassiveTempl} where the event of V-ing was perceived without the individual X being perceived.
\ex. X was $\left\{ \text{seen/heard/felt} \right\}$ V-ing \ldots\label{ex:ACCingPassiveTempl}

There are certain classes of predicates which we can use as diagnostics due to a non-canonical event/argument structure.
Consider the ACC-ing versions of weather reports and clausal idioms, for instance, as given in \cref{ex:ACCingWeather} and \cref{ex:ACCingIdiom}
\ex. Bill saw it snowing. $\centernot\implies$ Bill saw it. \label{ex:ACCingWeather}

\ex. Bill heard all hell breaking loose. $\centernot\implies$ Bill heard all hell. \label{ex:ACCingIdiom}

Consider, also, the predicates \textit{be slandered} and \textit{be parodied}.
Events of parodying or slandering an individual $x$ generally do not include $x$ as a participant the way, for instance, events of hitting $x$ or speaking to \textit{x} do.
This is demonstrated in \cref{ex:ACCingSlander} and \cref{ex:ACCingParody}
\ex. We heard the writer being slandered. $\centernot\implies$ We heard the writer. \label{ex:ACCingSlander}

\ex. They saw the singer being parodied. $\centernot\implies$ They saw the singer. \label{ex:ACCingParody}

Unlike most direct perception reports with ACs, then, \crefrange{ex:ACCingWeather}{ex:ACCingParody} are not ambiguous.
Rather, they have only the complement AC structures.
Therefore, if the perception reports in \crefrange{ex:ACCingWeather}{ex:ACCingParody} can be passivized, this will be evidence for the proposal that DPs are able to move out of complement ACs.
In fact, it seems that they cannot be passivized.
\ex.* It was seen raining.

\ex.* All hell was heard breaking loose.

\ex.* The writer was heard being slandered.\label{ex:SlanderedPassive}

\ex.* The singer was seen being parodied.\label{ex:ParodiedPassive}

I can think of no principled explanation of these facts except to propose that DP movement out of a complement AC is barred.
Absent any evidence or argument to the contrary, then, I will assume that Cinque's initial analysis of complement ACs was correct.

Thus we are led to the following generalization with respect to ACs:
If an AC is adjoined to a VP, then its subject must move, but if an AC is merged as the complement of a verb, then its subject cannot move.
This is an unexpected, perhaps unprecedented syntactic generalization.
In fact, I will argue in the following section that such a pattern is predicted to be impossible by any theory that defines grammaticality solely in terms of feature satisfaction, as standard minimalist theories do.
\section{Feature satisfaction cannot account for ACC-ing clauses}\label{sec:paradox}
As I discussed in \cref{sec:nonstandard}, standard minimalist theories tend to assume that a derived syntactic object converges at the interfaces iff it contains no unsatisfied features.
There is, of course, debate as to what it means for a feature to be unsatisfied, and what sort of operations are able to satisfy these features.
Therefore, I will attempt to abstract away from the details of particular theories and discuss what I take to be their shared assumptions.

The first common assumption (or set of assumptions) is about features.
Lexical items bear or consist of features, each of which is inherently either satisfied or unsatisfied,\footnote{
	The two most common versions of unsatisfied features and satisfaction operations are \textit{uninterpretable} features which must be checked, and \textit{unvalued} features which must be valued.
}
 and each of which is also specified for the information it encodes (person, gender, tense, etc.).
For instance, a determiner may bear a satisfied definiteness feature and an unsatisfied Case feature.

The second common assumption is that some computational operation converts a token of some unsatisfied feature into a token of the corresponting satisfied feature under the influence of some other feature token.
In order for feature token F on lexical item token X to satisfy feature token G on lexical item token Y, X and Y must stand in some structural relation to each other, and F and G must be of the same type.
Furthermore, feature satisfaction is automatic; if the conditions are met for F to satisfy G, then F satisfies G.

The third common assumption, is that there is no operation which undoes feature satisfaction.
This is never made explicit, but it is nonetheless assumed to be true.
Such an assumption, in effect, ensures a certain monotonicity in syntactic derivations, in that if at some derivational stage S$_n$ feature token F is satisfied, then there is no later stage S$_{n+i} (i > 0)$ at which F is unsatisfied.

The fourth common assumption, which I have already alluded to, is that a syntactic object is well-formed iff it contains no lexical items with any unsatisfied features.
This will be used as a diagnostic for satisfied features.
If a sentence, phrase, or word is well-formed, then it must contain no unsatisfied features, and if an expression is ill-formed, then it must contain unsatisfied features.

With these assumptions in place, consider the complement AC case in \cref{ex:CompACCing}, which is well-formed; this means that all of its sub-parts are well-formed.
\ex. They [$_\text{VP}$ saw [$_\text{ProgP}$ the raccoon [walking across the street]]].\label{ex:CompACCing}

Since the ProgP is well-formed, it must contain no unsatisfied features, and, therefore, the DP \textit{the raccoon} must contain no unsatisfied features.
Assuming that \textit{the raccoon} is in [Spec, Prog], rather than its base position in [Spec, $v$/Voice], then at least one of its unsatisfied features can only be satisfied in [Spec, Prog]. 
\ex.
\a.* They [$_\text{VP}$ saw [$_\text{ProgP}$ walking [$_\text{DP}$ the raccoon] across the street]].
\b.* They [$_\text{VP}$ saw [$_\text{ProgP}$ [-ing [[$_\text{DP}$ the raccoon] walk across the street]]]].

In other words, \textit{the raccoon} is licensed in [Spec, Prog].

However, when we consider the adjunct AC in \cref{ex:AdjuACCing}, we come very quickly to a contradiction.
\ex. They [$_\text{VP}$ [$_\text{VP}$ saw the raccoon][$_\text{ProgP}$ $t$ walking across the street]].\label{ex:AdjuACCing}

The fact that a lower copy/trace occupies [Spec, Prog] indicates that \textit{the raccoon} is not licensed there.
In fact, the data adduced above in \cref{ex:ACCingDouble} and the discussion thereof indicates this fact even more forcefully.
If \textit{the raccoon} is not licensed in [Spec, Prog], this means it bears some feature which cannot be satisfied there.
This is a direct contradiction of the conclusion I came to above, and yet this contradiction arises from an analysis of facts based on an axiom set.
The conclusion I draw from this contradiction is that the axiom set (a.k.a. the feature-satisfaction theory of syntax) is fundamentally flawed.

In order to pinpoint this flaw, we should consider how the facts could be generalized.
I believe a proper expression of the generalization is given in \cref{ex:AdjunctACCingGen} and \cref{ex:CompACCingGen}.
\ex. A syntactic object X (= $\left\{ t, \left\{ \text{Prog, YP}  \right\} \right\}$) is well-formed only if X is an adjunct.\label{ex:AdjunctACCingGen}

\ex. A syntactic object X (= $\left\{ \text{DP}, \left\{ \text{Prog, YP}  \right\} \right\}$) is well-formed only if X is an argument.\label{ex:CompACCingGen}

This is a problem for the feature-satisfaction theory because implicit in the theory is the claim that the well-formedness of an object depends solely on its internal structure.
The generalizations here, however, make reference, not just to the internal structure of an object, but also to the larger structure that the object is a part of.
This focus on internal structure is to be expected if the grammaticality of an expression is solely determined by whether the narrow syntax operates properly.
That is, if a given operation in the NS must be justified locally, it follows that it cannot be justified on the basis of an operation which has yet to occur.

However, A theory which bases grammaticality, at least partially, on interface conditions, as label theory does, will be able to account for the generalizations in \cref{ex:AdjunctACCingGen} and \cref{ex:CompACCingGen}.
Specifically, label theory must treat adjuncts and arguments differently and therefore, provides a good candidate for an explanatory theory of the generalization in question.
Since host-adjunct structures are always a species of phrase-phrase structures,\footnote{
	The phrasal nature of clausal and PP adjuncts for instance is uncontroversial, but adjectives and adverbs are also phrasal given the theory of categories assumed here.
	An adverb, for instance, consists minimally of a root and an $adv$ head.
}
we would expect a labelling paradox as with other phrase-phrase structures.
Unlike other phrase-phrase structures, however, there does not seem to be an agreement-based ``repair strategy'' for host-adjunct structures.
Almost by definition, adjuncts neither move, nor agree. 
Adjuncts, it seems, simply do not enter into the labelling calculation.
However, label theory as formulated in \cite{chomsky2013problems} and \cite{chomsky2015problems} does not address host-adjunct structures at all, and therefore is not quite suitable for my purposes.
Thus I will modify it in the following section.

\section{Modifications to label theory}\label{sec:modifications}

In this section I will address two questions that \textcite{chomsky2013problems,chomsky2015problems} largely leaves open.
First, there is the question of how to label Host-Adjunct structures, which I address in \cref{sec:adjuncts}.
Second, There is the question of why labels are required at all.
I address this question in \cref{sec:label-sem}.
%\subfile{Modifications}
\subsection{Labelling Host-Adjunct Structures}\label{sec:adjuncts}
To understand how adjuncts behave with respect to labelling, let's consider their basic properties: optionality, iterativity, and freedom of order.
These can be demonstrated in the series of sentences in \Next.
\ex. 
\a. The protester was brought to the police station.
\b. The protester was brought to the police station, against her will.
\b. The protester was brought to the police station, against her will, after the demonstration.
\b. The protester was brought to the police station after the demonstration, against her will.
\z.

If we assume that the adjuncts in \Last are adjoined to TP, then the TPs in each of the sentences in \Last, are, in some sense, grammatically indistinguishable.
If we take this much for granted, then we can view the task of developing a theory of adjuncts to be the task of making explicit the sense in which the sentences in \Last are indistinguishable.
Assuming label theory, we can make a fairly trivial explication of the indistinguishability of these sentences: the TPs in \Last are indistinguishable in the sense that they are labelled identically.
If the sentences in \Last are constructed purely by Merge (and Select, and Copy), then they have the structures in \Next.
\ex.
\a. [$_\alpha$ The protester was brought to the police station].
\b. [$_{\beta} [_{\alpha}$The protester was brought to the police station], [against her will]].
\b. $[_{\gamma}[_{\beta}[_{\alpha}$The protester was brought to the police station], [against her will]], [following the demonstration]].
\b. $[_{\eta} [_{\delta} [_{\alpha}$The protester was brought to the police station] [following the demonstration]], [against her will]].
\z.

If we take $\alpha$ to be the TP without adjuncts, then its label will be the basis for the label of the modified TPs $\beta, \gamma, \delta$ and $\eta$.
Since $\alpha$ is a finite TP with a subject, its label will be $\langle\varphi,\varphi\rangle$, and by assumption, the label of each of the modified TPs will be $\langle\varphi,\varphi\rangle$.
If this is the case, then the adjoined phrases contribute nothing to the labelling algorithm; in other words, they are invisible to LA.
The invisibility of adjuncts cannot, however, be the same phenomenon as the invisibility of lower copies, as the latter arises from a movement operation, and there is no reason to think that adjunct phrases as a class undergo movement.
Furthermore, even if adjuncts did move, this would only explain why lower copies are invisible; we would still need to explain why higher copies are invisible.

If the invisibility of adjuncts cannot be derived syntactically, perhaps it is inherent.
That is, perhaps the set of adjuncts is a natural class of objects which are invisible to LA.
This suggestion, however, runs into problems almost immediately due to the fact that there are phrases which can be arguments, predicates, or adjuncts as in \crefrange{ex:Adjective}{ex:PP}.
\ex. \label{ex:Adjective}
\a. The green room. (\textit{green} as an adjunct)
\b. The room is green. (\textit{green} as a predicate)

\ex. \label{ex:PP} 
\a. Meryl swam [in the pool]. (PP adjunct)
\b. Cameron fell [in the pool]. (PP complement)

It seems that, without a significant amount of stipulation, this is not a promising approach, so I will not pursue it further.

Adjunction, then, cannot be reduced to simplest Merge.
This leaves two broad options for assimilating it into our theory.
The first is to propose a new operation in Narrow Syntax (NS) that generates adjunction structures.
\textcite{chomsky2004beyond} proposes an operation of pair-Merge, which, given a host object $\beta$ and an adjunct $\alpha$, creates the object $\langle\alpha,\beta\rangle$ ($\alpha$ adjoined to $\beta$).
The new object $\langle\alpha,\beta\rangle$, however, has all of the syntactic properties (c-command relations, $\theta$-roles, selectional properties, etc) of the previously generated object $\beta$.
So, as far as NS is concerned, $\langle\alpha,\beta\rangle$ is equivalent to $\beta$.
For preciseness, I will use the term $\sigma$\textit{-equivalent}: an object $\langle\alpha,\beta\rangle$ (created by pair-Merge) is $\sigma$-equivalent to $\beta$.
There are two problems with this proposal, which I address in turn in the following paragraphs.

The first problem with adding an operation of pair-Merge to the system arises from the question of whether that change violates SMT.
Recall that SMT states that the language faculty is an optimal solution to interface problems, meaning that a minimalist theory of grammar should only admit complications if they are required due to interface conditions.
So, is pair-Merge required by one of the interfaces?
The fact that the information expressed by pair-Merge can be expressed periphrastically, as shown in \cref{ex:Periphrasis} suggests that pair-Merge is extraneous.
\ex.\label{ex:Periphrasis}
\a. I'd like a large burger with ketchup.\label{ex:succinct}
\b. I'd like a burger. I'd like it to be large. I'd like it to have ketchup.\label{ex:verbose}

The series of sentences in \cref{ex:verbose} express the same proposition as the single sentence in \cref{ex:succinct} and they would do so without any instances of pair-Merge.
The same cannot be said about structures formed by set-Merge; an expression constructed by set-Merge cannot be paraphrased without set-Merge.
Since it is not required by the interfaces, the addition of pair-Merge to NS would constitute a violation of SMT.

The second problem with pair-Merge arises from concerns about economy of derivation.
There are two facts about pair-Merge that are relevant to this issue.
First, pair-Merge is a more complex operation than set-Merge, as the former induces order, while the latter does not.
And second, when we adjoin an object to a host by pair-Merge, the resulting object is $\sigma$-equivalent to the host without the adjoined object, in the sense that a noun phrase with an adjective adjoined to it has all of the same syntactic properties as that same noun phrase without any adjunct.
Consider the two sub-derivations in \cref{ex:pmerge-deriv} and \cref{ex:smerge-deriv}.
The results of the two derivations are $\sigma$-equivalent to each other, but the first dervation is more complex than the second one.
\ex.\label{ex:pmerge-deriv} pMerge(X, Y) = $\langle\text{X, Y}\rangle$\\
Merge(Z$, \langle\text{X, Y}\rangle$) = $\left\{ \text{Z}, \langle\text{X, Y}\rangle \right\}$

\ex.\label{ex:smerge-deriv} Merge(Z, Y) = $\left\{ \text{Z, Y} \right\}$

From the view of NS, then, pair-Merge does a lot of work to no effect; the object derived in \cref{ex:pmerge-deriv} with an adjunct is syntactically indistinguishable from the object derived in \cref{ex:smerge-deriv} without an adjunct.
Therefore, for every derivation D that uses pair-Merge there is a simpler derivation D$^\prime$ such that the result of D is $\sigma$-equivalent to that of D$^\prime$.
This is exactly the type of situation that derivational economy rules out.

So, if, as I argue above, adjunction does not occur in NS, then it must occur after NS---the secon broad option for incorporating adjunction into our theory.
However, if we make the standard minimalist assumption that NS is the only module capable of recursively combining expressions to form larger expressions, then there can be no recursive combinatory operation outside of NS.
Therefore, adjunction -- being outside of NS -- cannot be a recursive combinatory operation. 
That is, adjunction does not create new syntactic objects.
This means that our way of representing adjunction in tree structures is misleading.

Consider, for instance, the modified VoiceP in \cref{ex:WithGusto} as represented in \cref{fig:WithGusto}.
\ex. Mary sang the song with gusto.\label{ex:WithGusto}

\begin{figure}[h]
	\centering{\small
	\begin{forest}
	    nice empty nodes,sn edges,baseline,
%	    for tree={
%	    calign=fixed edge angles,
%	    calign primary angle=-30,calign secondary angle=70}
	    [$\beta$
		    [$\alpha$
			    [DP[Mary,roof]]
			    [
				    [Voice]
				    [VP[sing the song,roof]]
			    ]
		    ]
		    [PP[with gusto,roof]]
	    ]
	\end{forest}}
	\caption{A standard representation of a modified VoiceP}
	\label{fig:WithGusto}
\end{figure}
The object $\beta$ is usually taken to be created by adjoining the PP \textit{with gusto} to $\alpha$, but, as I argued above, adjunction cannot create new objects.
Therefore, there is no object $\beta$.
This is, no doubt, a surprising conclusion, yet it follows from the basic facts of adjunction and SMT, so it behooves us to entertain it as a possibility.

This conclusion, in fact, does resolve the immediate question of how Host-Adjunct structures are labelled, not by answering it but by dissolving it.
If LA is a function from unlabelled SOs to labelled SOs and Host-Adjunct structures are not SOs, then they are outside the domain of LA.
However, there are a few caveats that bear mentioning.
The first is that, while Host-Adjunct structures are not properly SOs, the same cannot be said for the adjuncts \textit{per se}.
So, to consider a concrete case, $\beta$ in \cref{fig:WithGusto} is not an SO, but the PP \textit{with gusto} is an SO, meaning it was derived in NS and labelled by LA.
The second caveat is that not everything that looks like a Host-Adjunct structure is one.
For instance the topicalized PP in \cref{ex:WithSorrow} is likely an argument of, say, a functional projection Topic.
\ex.\label{ex:WithSorrow} With sorrow in her heart, Mary sang the song.

The third caveat, which perhaps is more of a promissory note, is that asserting that Host-Adjunct structures are not SOs leaves us with the question of what they are.
This is far from an easy question to answer, and I will not attempt a complete answer here.
Instead, I will stipulate that, at the CI interface, a host-adjunct structure is a complex object which is asymmetric and unlabelled.
It is asymmetric in the sense that the host is more prominent than its adjuncts.
A proper theory of adjunction, if one exists, will derive this asymmetry from intrinsic properties of the host and adjuncts, but I will stipulate it here.
It is unlabelled because only SOs are labelled, and host-adjunct structures are not SOs.

The notion that there can be complex linguistic objects which are not generated by Merge may seem to contradict the evolutionary version of SMT, which states that the evolution of the language faculty consists in the sudden appearence of Merge.
If adjunction is a non-Merge method for constructing complex linguistic expressions, then we would expect there to be a language faculty even without Merge.
While this expectation is not, strictly speaking, borne out, there does seem to be an extra-linguistic cognitive system that makes use of complex language-like representations.
Consider the system of propositional attitudes that \textcite{fodor1975language} discusses, and the structures employed in the study of discourse pragmatics.

For Fodor, all cognition involves several sets of propositions that a given organism has certain attitudes toward.
For instance, every animal has a set of beliefs and a set of desires, which are populated by propositions.
The fact that humans have language means that these propositions can be of arbitrary complexity, but they are still beliefs and desires.
Also, perhaps the most central notion of discourse pragmatics is the common ground, which is a set of propositions believed to be shared between discourse participants.
While the common ground interacts with linguistic expressions, it does not seem to be one itself.
These examples are complex cognitive objects that are non-linguistic, and, like host-adjunct structures, they ``compose'' by conjunction.

While these proposition sets are not usually considered to be compositional, it seems rather obvious that, holding an attitude towards a set of propositions is logically equivalent to holding that same attitude towards the conjunction of the proposition in the set.
\ex. \textsc{bel}(p) \& \textsc{bel}(q) $\leftrightarrow$ \textsc{bel}(p \& q)

Note, of course, that this is a logical equivalence but not a representational equivalence,\footnote{
	\citeauthor{fodor1975language}'s (\citeyear[50--100]{fodor2010lot}) discussion of referential opacity, provides, I believe, an excellent argument for distinguishing logical equivalence from representational equivalence.
} and since, according to the computational theory of mind, representations matter, I would not like to claim that holding an attitude towards a set of propositions P necessarily requires holding that attitude towards the conjunction of every subset of P.
Rather, I claim that if two propositions, p and q, are members of the same set in the mind, then the operation of adding the conjunction of those propositions, p\&q is available, but an operation of adding other possible compositions of p and q (p$\vee$q, p$\rightarrow$q, etc.) is not available.
So, perhaps the process of interpreting a host-adjunct structure involves adding the conjunction it expresses to some proposition set.

To summarize, the above discussion was a long-winded way of hypothesizing that host-adjunct structures not only are not labelled by LA, but are not processed by LA.

\subsection{Why are labels needed at all?}\label{sec:label-sem}
The second question is why labels should be required by the CI interface at all.
My proposed answer is that the label of a complex object determines how that object composes semantically.
While this may seem \textit{ad hoc}, it is actually a fairly reasonable hypothesis.
Consider Chomsky's labelling hypothesis as phrased in \Next, and the more standard theory of the CI interface in \NNext.
\ex. A syntactic object is a valid CI object iff it is labellable.

\ex. A syntactic object is a valid CI object iff it composes semantically.

At first glance, these hypotheses are incompatible, giving us three options for resolving the conflict.
The first option would be to reject one of the conflicting hypotheses.
There is no strong evidence, however, for rejecting either \LLast or \Last, so I will not choose this option.
The second option is to conjoin the iff clauses as in \Next.
\ex. A syntactic object is a valid CI object iff it is labellable and it composes semantically.

This option is unattractive for reasons of theoretical parsimony, so I will not choose it.
The third option is to hypothesize that labelling and composition are two sides of the same coin, and therefore the conflicting hypotheses are equivalent.
We can, then, replace our two conflicting statements with the two compatible statements in \Next and \NNext below.
\ex. A syntactic object composes iff it is labellable.

\ex. A syntactic object is a valid CI object iff it composes.

This move is theoretically attractive partially due to the fact that it mirrors the logic of antisymmetry on the SM interface \parencite{kayne1994antisymmetry}.
In the case of antisymmetry, Kayne identifies asymmetric c-command with linear order, and there is no compelling reason to think that the CI interface should be more complex than the SM interface.

So, what would it mean for labelling and composition to be two sides of the same coin?
Again, it is helpful to consider the SM interface, where asymmetric c-command and linear order are associated because they are isomorphic.
We should expect a similar isomorphism to hold between composition and labels, and, in fact, there seems to be good reason to think that there is such an isomorphism.
Consider the main modes of composition generally assumed by semanticists \parencite[\textit{e.g.}, by][]{heimkratzer1998semantics}, given schematically in \Next.
\ex. 
\a. \textbf{Lexical insertion}\\
\textsc{sem}($\alpha$) = $\alpha^\prime$
\b. \textbf{Function application}\\
\textsc{sem}($\left[ \alpha, \beta \right]$) = \textsc{sem}($\alpha$)(\textsc{sem}($\beta$))
\b. \textbf{Predicate modification}\\
\textsc{sem}($\left[ \alpha, \beta \right]$) = \textsc{sem}$(\alpha)(x) \&$ \textsc{sem}$(\beta)(x)$
\b. \textbf{Predicate abstraction}\\
\textsc{sem}($\left[ \alpha, \beta \right]$) = (Op$x$)(\textsc{sem}($\beta$)($x$))

Each of these modes of composition has a corresponding structure type as identified by the version of label theory developed here.
Abstracting away from phrasal idioms, lexical insertion operates on a single syntactic atom, \textit{i.e.}, a head, which label theory necessarily distinguishes from other syntactic objects.
Predicate modification is the next most complex operation: it conjoins two (possibly complex) objects without requiring or inducing any ordering of the two, exactly isomorphic with the output of merge: unlabelled and unordered syntactic objects.
Function application, likewise, requires two objects, but these objects are ordered.
Unlike conjunction structures created by predicate modification, which are commutative ($X \& Y = Y \& X$), the function-argument structures created by function application are inherently asymmetric ($X(Y) \neq Y(X)$).
This matches with head-labelled structures, which encode a pair of objects (the contents of the structure) and an ordering statement (the label).
Finally, predicate abstraction, which creates structures similar to quantifier structures, requires the content of the two expressions, an ordering between the two, and a variable.
Pair-labelled structures provide this information.

First consider those structures labelled by heads.
The classes of structures which get head labels are given in \Next.
\ex. \textbf{Head-labelled structures}
\a. $\left\{ \text{X, }\textsc{Root} \right\} \xrightarrow{Label} \left[_\text{X} \text{X, }\textsc{root}  \right]$
\b. $\left\{ \text{X, YP} \right\} \xrightarrow{Label} \left[_\text{X} \text{X, YP} \right]$
\c. $\left\{ t_\text{ZP}, \left\{ \text{X, YP} \right\} \right\}\xrightarrow{Label}\left[_\text{X} t_\text{ZP}, \text{XP} \right]$

I propose that in these cases, the objects compose by function application, with the label being the function and the non-labelling constituent being the argument.
So, for instance, a DP is interpreted as the function D, with NP as an argument.
\ex. \textsc{sem}($\left[_\textit{the} \textit{the, ball} \right]$) = \textsc{sem}(\textit{the})(\textsc{sem}(\textit{ball}))

Next, consider the structures labelled by feature-pairs.
These structures tend to be the result of internal Merge, which is generally associated with operator-variable structures.\footnote{
	I use the hedges \textit{tend} and \textit{generally} here to indicate that I was not able to perform an exhaustive enumeration of all pair-labelled structures.
	I perhaps could have formulated this generalization without hedges, in which case this footnote would be explaining that perhaps I should have hedged the generalization slightly.
}
I hypothesize, then, that feature-pair labels signal that a complex object is to be interpreted as an operator-variable structure.
For instance the Wh-question structure in \Next[a] is interpreted as in \Next[b].
\ex. \textsc{sem}($\left[_{\langle Q,Q \rangle} \textit{Who}_Q, \left[ \text{C}_Q+\textit{did}, \left[ \textit{Mary see } t_{Who} \right] \right]  \right]$) = (Wh\textit{x})(\textsc{sem}(\textit{Mary saw x}))

Finally, we come to the case of unlabelled structures, which is identical to the case of Host-Adjunct structures.
As I discussed in \cref{sec:adjuncts}, however, Host-Adjunct structures are unlabelled because they are not SOs and the domain of the Labelling Algorithm is restricted to SOs.
These structures are given the default interpretation of conjunction, as discussed in \cref{sec:adjuncts}.

I have identified Kayne's (\citeyear{kayne1994antisymmetry}) theory of the SM interface as an inspiration for my proposal, and I would like to say a little more about the similarities between his and my hypotheses.
Rather than positing an active process, be it simple or intricate, for linearizing a hierarchical structure, Kayne suggests that linear order is the product of a passive isomorphism.
Since asymmetric c-command \textit{is} a linear, or total, order, a hierarchical structure can be mapped to a linear string based purely on properties of that structure.
The idea that an interface between mental modules should be passive is in keeping with the very idea of modularity.
If the SM module and the Narrow Syntax module are truly independent, then we would not expect there to be any specialization of one in order to interact with the other.
My proposal for the CI interface is, I believe, a step towards a passive interface.
Although evidence for the nature of the CI module is not as readily available as evidence for the nature of the SM module, the working, albeit tacit, assumption seems to be that the CI module deals in representations that are formally very similar to formulas of predicate logics.
If we assume a predicate logic with operators\footnote{Since the term \textit{operator} already has a particular meaning in generative syntax, I will refer to the operators of predicate logic as \textit{l-operators}.} ($\forall$, $\exists$, $Wh$, M,\ldots), functions/predicates ($P, Q, f, g,$\ldots), and variables ($x, y, z,$\ldots), then we can see how there could be an isomorphism between labelled syntactic objects and formulas of this logic.

While it seems that SM objects are linear structures (strings), CI objects seem to be more complex, including notions of scope and variable binding.
I therefore cannot able to give a simple order-theoretic explanation of the CI interface as \textcite{kayne1994antisymmetry} gives for the SM interface.
What I can do, however, is give a coarse-grained mapping between labelled SOs and expressions in predicate logic.
The class of labelled SOs and that of complex expressions of predicate logic can each be divided into two sub-classes.
Labelled SOs can be head-labelled, or pair-labelled, while expressions of predicate logic can be function-argument expressions, or l-operator expressions.
Furthermore, these sub-classes seem to map to each other as shown in \Next.
\ex.
\begin{tabular}[t]{ll}
	Labelled SO & Predicate logic\\
	\hline
	\hline
	Head labelled & Function-argument\\
	$\left\{ \text{in}, \left\{ \text{in, the snow} \right\} \right\}$ & $\textbf{in}(\textbf{the\_snow})$\\
	\hline
	Pair-labelled & L-operator expressions\\
	$\left\{ \langle Q,Q\rangle, \left\{\text{Who}_Q, \left\{\text{C}_Q, \text{fell} \right\}  \right\} \right\}$ & $\text{Wh}x(\textbf{fell}(x))$\\
	\hline
\end{tabular}

This mapping is, of course, a first approximation of a theory of the CI interface.
Being a first approximation, it will face empirical and theoretical challenges.
For instance, there are likely to be cases where an expression's label does not seem to map to its interpretation.
These cases, however, might be cases in which our syntactic or semantic analysis is incorrect.

Furthermore, even if the hypothesized mapping in \Last were shown to be empirically adequate, it would still require theoretical explanation.
That is, we would need to explain why that particular mapping holds.
This would require us to show that there is a mathematically sound isomorphism between head-labelled SOs and function-argument expressions, and between pair-labelled SOs and l-operator expressions.
Such a demonstration, however, is beyond the scope of this thesis.
\section{Explaining ACC-ing clause subjects}\label{sec:ExplainingACCingSubjs}
With these modifications of label theory in place, we can consider how to account for the distribution of ACC-ing clause (AC) subjects.
First, I will explain the fact that, in the case of complement ACs, the subject DP cannot move.
This explanation will be essentially the same as the explanation of ECP effects given in \cite{chomsky2015problems}.
I will then explain the adjunct AC case, in which the subject DP must move.
This explanation will require the modified version of label theory and will support the general restriction in \cref{ex:AdjunctGen}.

As in Chomsky's (\citeyear{chomsky2015problems}) analysis of \textit{that}-trace effects and my explanation of the non-generation of resultatives in French-like languages in \cref{sec:Fre-deriv}, a DP can be blocked from moving out of a phrase $\left\{ \text{DP, XP} \right\}$ if the following conditions hold.
First, the head X of XP bears features which must be valued by a DP in order to label XP.
That is, X must bear an incomplete set of, say, $\varphi$-features.
Second, if the DP moved, then that movement would occur before DP and XP agree for the features in question.
In other words, if a movement operation bleeds agreement and labelling requires that agreement relation, then that movement operation will be ruled out.
Applying this logic to the complement AC, I hypothesize that the English Prog head has an incomplete $\varphi$-set which must be strengthened by Agree in order to provide a label.
If this is the case, then we can explain the impossibility of movement from complement ACs if such a movement would bleed agreement.

The case of adjunct ACs, however, requires a more complicated explanation.
The explanation will be split into two subsections: \Cref{sec:DPCanMove} will show that a DP can move out of an adjunct AC without causing a problem, and \cref{sec:DPMustMove} will argue that an in-situ DP in an adjunct AC runs into problems.

\subsection{DPs can move from adjunct ACC-ing clauses}\label{sec:DPCanMove}
In order to show that movement from an adjunct AC is permitted, I will demonstrate, in some detail, how an adjunct AC structure is derived and interpreted.
First, let's consider how the AC in \cref{ex:JoannaTeaching} is derived.
\ex.\label{ex:JoannaTeaching} Joanna teaching her students.

Since the ``contentful'' portion of the AC---the portion that encodes lexical information and thematic structure---is almost entirely independent of the larger AC structure, I will assume its derivation is uncomplicated.
That is, I will take for granted that, if the complement of Prog is a VoiceP, and that that VoiceP is derived and labelled as it would be in a grammatical finite clause.
Further support for this assumption comes from \textcite{harwood2015being} who argues that Prog is a phase head, and therefore its introduction triggers the transfer of its complement, although not before the ACC-ing subject raises and merges with ProgP.
So, the AC which is to be adjoined to the host VP has the structure in \cref{fig:ProgPStruct}
\begin{figure}[h]
	\centering
	\begin{forest}
		nice empty nodes,sn edges,baseline
		[$\beta$
			[DP$_{i}$[Joanna,roof]]
			[$\alpha$
				[Prog]
				[VoiceP
					[$t_{i}$ teach her students,roof]
				]
			]
		]
	\end{forest}
	\caption{The structure of a ProgP}
	\label{fig:ProgPStruct}
\end{figure}
The next stages of the derivation requires the DP \textit{Joanna} to move out of the AC, leaving us with an unlabellable structure as per the discussion immediately preceding this subsection.\footnote{
	Prog bears a single $\varphi$-feature set, and cannot label unless that feature set is enriched by agreeing with a DP in its specifier.
	Since the DP moves before agreement can happen, Prog's feature set is not strengthened and cannot label.
}
However, a derivation doesn't necessarily crash if it's derived structure is unlabellable.
Rather it crashes when LA tries to label an unlabellable structure.
Since the AC in \cref{fig:ProgPStruct} is adjoined to VP in this case, and, by hypothesis, the labelling algorithm operates only on the host of a host-adjunct structure, the AC is not processed by the labelling algorithm.
If the AC is not processed by the labelling algorithm, then it doesn't matter whether the AC is labellable or unlabellable.
The content of an adjunct does not matter in determining the labelling of the host.

\subsection{DPs must move from adjunct ACC-ing clauses}\label{sec:DPMustMove}
Thus far, I have argued that the labellability of an adjunct is immaterial, which suggests that DP movement from an adjunct AC would be optional.
The facts of ACs, however, suggest that DPs must move from adjunct ACs.
In order to explain why the movement operation in question is obligatory, we must consider how adjunct ACs are interpreted.
Consider the illicit adjunct AC structure in \cref{fig:IllicitAdjunctAC} corresponding to the ungrammatical string in \cref{ex:IllicitAdjunctAC}.
\ex.* We can Mario the woman teaching her students.\label{ex:IllicitAdjunctAC}

\begin{figure}[h]
	\centering
\[\sbox0{$\begin{array}[]{ccc}
		\begin{forest}
	    nice empty nodes,
	    sn edges,baseline,
	    for tree={
	    calign=fixed edge angles,
	    calign primary angle=-30,calign secondary angle=70}
		    [$\delta$
			    [DP$_i$[the woman,roof]]
			    [$\gamma$
				    [Prog]
				    [VoiceP[$t_{i}$ teach her students,roof]]
			    ]
		    ] 
	\end{forest}			
	&
	\tikz[baseline=10ex,scale=1] \node[inner sep=0] at (0,-1) {\large,\,};
	&
	\begin{forest}
	    nice empty nodes,
	    sn edges,baseline,
		for tree={
	    calign=fixed edge angles,
	    calign primary angle=-30,calign secondary angle=70}
		    [$\beta$
			    [$\alpha$
				    [$v$]
				    [$\sqrt{see}$]
			    ]
			    [DP[Mario,roof]]
		    ]
	    \end{forest}
		\end{array}$}
\mathopen{\resizebox{1.2\width}{\ht0}{$\Bigg\langle$}}
\usebox{0}
\mathclose{\resizebox{1.2\width}{\ht0}{$\Bigg\rangle$}}
\]
\caption{An illicit adjunct AC structure ($\delta$ adjoined to $\beta$)}
	\label{fig:IllicitAdjunctAC}
\end{figure}
When the host-adjunct structure is finally transferred as part of a larger phase, only the host $\beta$ will be labelled.
Neither the pair $\langle\delta,\beta\rangle$, representing $\delta$ adjoined to $\beta$, nor the adjunct $\delta$ itself will undergo labelling at this stage.
As such, both of these objects will be null-labelled at CI, and therefore, interpreted conjunctively.
This is expected for host-adjunct structures, but when we consider the interpretation of $\delta$, we can see an issue with the structure in \cref{fig:IllicitAdjunctAC}.

The adjunct $\delta$ will be null-labelled, as represented in \cref{ex:AdjunctCI}, and therefore interpreted conjunctively as shown in \cref{ex:AdjunctSEM}.
\ex.\label{ex:AdjunctCI} [$_{\emptyset}$ [the woman]$_i$, [$_{\emptyset}$ Prog, [$_\text{VoiceP}$ the woman$_i$ teach her students]]]

\ex.\label{ex:AdjunctSEM} $\textsc{sem}(\delta) = \textsc{sem}(\text{the woman})(e) \& \textsc{sem}(\text{Prog})(e) \& \textsc{sem}(\text{VoiceP})(e)$

So, DP \textit{the woman} and the VoiceP \textit{the woman teach her students} are predicated of the same extra-mental entity $e$.
In other words, there is some entity $e$ which is both \textit{the woman} and an event of the woman teaching her students.
Notice, however, that the two instances of \textit{the woman} in $\delta$ are not distinct SOs but occurrences of a single SO.
Despite the fact that these two expressions are identical, they are interpreted as being predicated of two distinct entities.
The upper copy is predicated of some event $e$, while the lower copy is predicated of some entity $x$ which participates in the event of teaching students.
If these copies are supposed to be identical, it seems like a contradiction to say that they are predicated of distinct entities.
This contradiction, I propose, is the reason that DPs must move from specifiers of adjunct ACs.

If, on the other hand the DP in [Spec, Prog] is a lower copy, then it can either be ignored by the CI system or treated as a variable.
In either case, it will not be treated as a predicate, and therefore cannot be predicated of two distinct sorts of entities.
\subsection{Generalizing the ACC-ing results}
In the previous section, I offered an explanation for the fact that, if an AC is adjoined to a VP, then its subject must move out of the AC.
No part of the explanation, however, depended on any inherent property of the AC, but rather on the fact that the AC is adjoined, and the fact that the subject DP was internally merged in subject position. 
So, this leads us to the generalization in \cref{ex:AdjunctGen}, which I restate schematically in \cref{ex:AdjunctGenSchema}.
\ex.* $\langle \left\{ \text{DP}_{i} \left\{ \dots t_i \dots \right\} \right\}, \text{XP}\rangle$\label{ex:AdjunctGenSchema}\\
($^\textsc{ok}\langle \left\{ t_{i} \left\{ \dots t_i \dots \right\} \right\}, \text{XP}\rangle$)

Note that this seems to be violated by sentences like \cref{ex:AbsoluteClause}, where the modifier AC has its subject \textit{in situ}.
\ex. Her order having arrived late, Kinza was in a sour mood.\label{ex:AbsoluteClause}

The AC in this sentence however, is a sort of topic, and, by hypothesis, topical expressions are merged in a Topic projection rather than adjoined.
So, the structure of \cref{ex:AbsoluteClause} is given in \cref{fig:AbsoluteClause}.
\begin{figure}[h]
	\centering
	\begin{forest}
	    nice empty nodes,
	    sn edges,baseline
	    [$\delta$
		    [$\gamma$
			    [DP$_{\varphi}$[Her order,roof]]
			    [$\beta$
				    [Prog$_{\varphi}$]
				    [vP[be late,roof]]
			    ]
		    ]
		    [$\alpha$
			    [Topic]
			    [TP [Kinza was in a sour mood,roof]]
		    ]
	    ]
	\end{forest}
	\caption{The unlabelled structure of \cref{ex:AbsoluteClause}}
	\label{fig:AbsoluteClause}
\end{figure}

The DP \textit{her order} and Prog will Agree, meaning the AC $\gamma$ will be labelled $\langle\varphi,\varphi\rangle$, and its constituent $\beta$ will be labelled Prog.
The process of labelling $\delta$ is a slightly more complicated case, though.
Recall that the labelling algorithm is the process of finding the most prominent element in a syntactic object.
Setting aside the cases that result in a feature-pair label, the most prominent element is the least embedded atomic object.
So, what is the least embedded atomic object in $\delta$?
The likely candidates are Topic, Prog, or D (\textit{her}), of which Topic is the least embedded.
Prog is dominated by 3 nodes ($\beta,\gamma,\delta$), as is D (DP,$\gamma,\delta$).
Topic, on the other hand, is dominated by 2 nodes ($\alpha,\delta$).
Therefore, the label of $\delta$ is Topic, and, since \cref{ex:AbsoluteClause} is grammatical, we can safely assume that Topic is strong enough to be a label.
It follows, then, that the label of $\beta$ (a head-phrase structure) will also be Topic.
The end result of the labelling process, then, is represented in \cref{fig:AbsoluteClauseLabel}.
\begin{figure}[h]
	\centering
	\begin{forest}
	    nice empty nodes,
	    sn edges,baseline
	    [Topic
		    [{$\langle\varphi,\varphi\rangle$}
			    [DP$_{\varphi}$[Her order,roof]]
			    [Prog
				    [Prog$_{\varphi}$]
				    [vP[be late,roof]]
			    ]
		    ]
		    [Topic
			    [Topic]
			    [TP [Kinza was in a sour mood,roof]]
		    ]
	    ]
	\end{forest}
	\caption{The labelled structure of \cref{ex:AbsoluteClause}}
	\label{fig:AbsoluteClauseLabel}
\end{figure}

Turning back to resultatives, we can see that the proposed structure included the adjunction of a resP to a VP.
Therefore, the restriction in \cref{ex:AdjunctGenSchema} would apply to resultatives, and the DP in [Spec, res] would be required to move.
Thus we have an explanation for the ungrammaticality of \cref{ex:double-theme-res}. 
\end{document}

\chapter{Coincidence}\label{sec:coincidence}
%        File: Coincide.tex
%     Created: Sat Apr 14 12:00 PM 2018 E
% Last Change: Sat Apr 14 12:00 PM 2018 E
%
% arara: pdflatex: {options: "-draftmode"}
% arara: biber
% arara: pdflatex: {options: "-draftmode"}
% arara: pdflatex: {options: "-file-line-error-style"}
\documentclass[MilwayThesis]{subfiles}
\begin{document}
In this thesis I have made liberal use of a novel class of sideward movement structure which I schematize below in \cref{fig:SidewardSchema}.
\begin{figure}[h]
	\centering
\[\sbox0{$\begin{array}[]{ccc}
		\begin{forest}
	    nice empty nodes,
	    sn edges,baseline,
	    for tree={
	    calign=fixed edge angles,
	    calign primary angle=-30,calign secondary angle=70}
	    [YP
		    [DP$_i$]
		    [
			    [Y]
			    [ZP]
		    ]
	    ]
	\end{forest}			
	&
	\tikz[baseline=10ex,scale=1] \node[inner sep=0] at (0,-1) {\large,\,};
	&
	\begin{forest}
	    nice empty nodes,
	    sn edges,baseline,
		for tree={
	    calign=fixed edge angles,
	    calign primary angle=-30,calign secondary angle=70}
	    [VP
		    [V]
		    [DP$_i$]
	    ]
	    \end{forest}
		\end{array}$}
\mathopen{\resizebox{1.2\width}{\ht0}{$\Bigg\langle$}}
\usebox{0}
\mathclose{\resizebox{1.2\width}{\ht0}{$\Bigg\rangle$}}
\]
	\caption{A schema of the sideward movement structure}
	\label{fig:SidewardSchema}
\end{figure}
This structure was used as an analysis of adjectival resultatives, depictives, and one available structure for direct perception reports with ACC-ing clauses.
The distinction between these constructions is due to the choice of head Y.
For resultatives, Y is instantiated by res, while for precetion reports, it is instantiated by Prog.
As for depictives, the adjoined phrase YP is a small clause, so Y is either absent or instantiated by a Pred head.
Note that, while the complement of Y, ZP, also varies between the constructions, this variation can be derived from the selectional requirements of Y.

If we take VPs to be event descriptions, and we assume that host-adjunct structures to compose by predicate conjunction, then a VP adjunct (SC for depictives, resP for resultatives, and ProgP for DPRs) must also be event description.
That is, in \cref{fig:SidewardSchema} the VP and YP are each interpreted as a predicate of events, and, because they combine by predicate conjunction, they both describe the same event.

In the case of depictives, the interpretation is mostly straightforward, though not without come complications.
Consider \cref{ex:Depictive}, which an eventuality in which Natasha eats the fish while that fish is raw.
\ex. Natasha ate the fish raw.\label{ex:Depictive}

According to our analysis, this interpretation would be derived from the fact that the eating event and the rawness state are taken to to be identical.
That is the VP in \cref{ex:Depictive} has a logical form  as derived in \cref{ex:DepictiveLF}
\ex. \textsc{sem}(\textit{eat the fish raw}) =\\
$\lambda e [\textsc{sem}(\textit{eat the fish})(e) \, \&\, \textsc{sem}(\textit{the fish raw})(e)] =$\\
$\lambda e [\textbf{eating}(e)\, \&\, \textbf{raw}(e)\, \&\, \textsc{Theme}(\textbf{the\_fish})(e)]$\label{ex:DepictiveLF}

One might object that this LF is incoherent, as eating is an event, while rawness is a state, and an eventuality cannot be both an event and a state.
This objection, however, does not hold up under scrutiny.
Suppose we take the externalist perspective, according to which, the entities that natural language expressions are predicated of are mind-external and -independant entities, and the predicates and concepts of natural language correspond to natural kinds.
From this perspective, eventualities are regions of space-time, some of which are events, while others are states.
So, for instance to utter \cref{ex:EventDesc} truthfully is to refer to a particular region of space-time.
\ex.\label{ex:EventDesc} The officer ticketed the car.

Now, according to the objection at hand, the region of space-time referred to by \cref{ex:EventDesc} is an event, and, therefore, not a state.
However, it is entirely reasonable to assume we could truthfully utter \cref{ex:StateDesc}, a state description, referring to the same space-time region.
\ex.\label{ex:StateDesc} The car was parked illegally.

It seems, then, that, if there is an event/state contrast, it does not originate in the extramental world, or else \cref{ex:EventDesc} and \cref{ex:StateDesc} could not possibly refer to the same region of space-time.

Furthermore, many adverbs describe states, yet may modify event descriptions.
If adverbs are adjuncts, then they are interpreted as conjoined with their host, meaning that they will provide a partial description of an event, rather than a state.
Therefore, there doesn't seem to be any contradiction in my analysis of depictives.

The case of adjunct ACC-ing clauses is slightly more challenging, due to a subtlety in their meanings which I will discuss below.
Ultimately, however, their meaning can be explained from their structure, given in \cref{fig:ACCingPair}, and an assumption about the nature of eventualities.
\begin{figure}[h]
	\centering
\[\sbox0{$\begin{array}[]{ccc}
		\begin{forest}
	    nice empty nodes,
	    sn edges,baseline,
	    for tree={
	    calign=fixed edge angles,
	    calign primary angle=-30,calign secondary angle=70}
	    [ProgP
		    [DP$_i$]
		    [
			    [Prog]
			    [VoiceP[DP$_i$ run,roof]]
		    ]
	    ]
	\end{forest}			
	&
	\tikz[baseline=10ex,scale=1] \node[inner sep=0] at (0,-1) {\large,\,};
	&
	\begin{forest}
	    nice empty nodes,
	    sn edges,baseline,
		for tree={
	    calign=fixed edge angles,
	    calign primary angle=-30,calign secondary angle=70}
	    [VP
		    [saw]
		    [DP$_i$[the dog,roof]]
	    ]
	    \end{forest}
		\end{array}$}
\mathopen{\resizebox{1.2\width}{\ht0}{$\Bigg\langle$}}
\usebox{0}
\mathclose{\resizebox{1.2\width}{\ht0}{$\Bigg\rangle$}}
\]
	\caption{An adjunct ACC-ing structure}
	\label{fig:ACCingPair}
\end{figure}
As a first pass, we can say that the interpretation of the structure in \cref{fig:ACCingPair} is a description of an event of the dog being seen and an event of the dog running.
This alone is not sufficient, as an English speaker's intuition regarding the \cref{ex:ACC-ing} is that the $we$ referent saw both the dog and the event of the dog running.
\ex. We saw the dog running.\label{ex:ACC-ing}

This intuition seems to be inescapable; English speakers cannot seem to entertain an interpretation of \cref{ex:ACC-ing} in which \textit{we} saw the dog but not the running event, or the event but not the dog.
This suggests that both complement ACC-ing and adjunct ACC-ing versions of \cref{ex:ACC-ing} are interpreted as both the individual and the event being seen.
The strong version of UTAH that I assume in this thesis, however, predicts that the interpretation \textit{x was seen} can only be encoded if the expression denoting $x$ is merged as the complement of the verb \textit{see}.
In adjunct ACC-ing analysis of \cref{ex:ACC-ing}, as shown in \cref{fig:ACCingPair}, the event denoting expression, ProgP, is adjoined to VP, yet we interpret it as meaning that the event was seen.
Since this interpretation is not directly encoded, we must infer it from what is directly encoded.

To see how we would infer the perception of the event, consider what is directly encoded.
First, the VP is interpreted as a description of a seeing event which the dog is the theme of.
\ex.\label{ex:VPSEM} \textsc{sem}(VP) = $\lambda e [\textbf{see}(e)\,\&\,\textsc{theme}(\textbf{the\_dog})(e)]$

The interpretation of the ProgP, represented in \cref{ex:ProgPStruct}, however, is more complicated.
\ex.\label{ex:ProgPStruct} [$\langle$the dog$\rangle$, [Prog [$_\text{VoiceP} \langle\text{the dog}\rangle$ run]]]

I will make the simplifying assumption that the copy of \textit{the dog} in [Spec, Prog] is semantically vacuous\footnote{
	At this stage, this is purely stipulative.
	I suspect some version of this assumption is true, but a full investigation and justification of it is beyond the scope ot this thesis.
} and discuss the Prog-VoiceP structure.
The VoiceP is unremarkable, so I assume its meaning is the complete but tenseless event description in \cref{ex:VoicePSEM}.
\ex.\label{ex:VoicePSEM} \textsc{sem}(VoiceP) = $\lambda e [\textbf{run}(e)\,\&\,\textsc{doer}(e)(\textbf{the\_dog})]$

Prog, then, takes this description as an argument, and ascribes progressive aspect to it.
The standard, though perhaps na\"ive analysis of progressive aspect is that Prog takes a description of event $e$ as an argument, introduces a topic time $t$, and asserts that $t$ is included in the run-time of $e$.
This predicts that ProgP encodes the predicate of times in \cref{ex:ProgPSEM1}
\ex.\label{ex:ProgPSEM1} \textsc{SEM}(ProgP) =  $\lambda t \exists e [t \subseteq \textsc{time}(e)\,\&\,\textbf{run}(e)\,\&\,\textsc{doer}(e)(\textbf{the\_dog})]$ (first pass)

However, since ProgP adjoins to VP and the resulting structure is interpreted as a conjunction, ProgP must be interpreted as a predicate of events.
Therefore, I will modify the semantic analysis of Prog such that it introduces an event $e^{\prime}$ and asserts that $e^{\prime}$ is included in $e$.
The final interpretation of the ProgP, is given in \cref{ex:ProgPSEM2}.
\ex.\label{ex:ProgPSEM2} \textsc{sem}(ProgP) =  $\lambda e^{\prime} \exists e [e^{\prime} \subseteq e\,\&\,\textbf{run}(e)\,\&\,\textsc{doer}(e)(\textbf{the\_dog})]$ (second pass)

This interpretation will properly compose with \cref{ex:VPSEM} to yield the interpretation of the host-adjunct structure in \cref{ex:VPProgPSEM}
\ex.\label{ex:VPProgPSEM} \textsc{sem}($\langle$ProgP, VP$\rangle$) = $\lambda e^{\prime} \exists e [ \textbf{see}(e^{\prime})\,\&\,\textsc{theme}(\textbf{the\_dog})(e^{\prime})\,\&\,e^{\prime} \subseteq e\,\&\,\textbf{run}(e)\,\&\,\textsc{doer}(e)(\textbf{the\_dog})]$

So, the event of seeing the dog is included in the event of the dog running, meaning that the seeing occurred in the same space-time region as the running and therefore.
This denotation along with the very nature of seeing and running allows us to infer from \cref{ex:VPProgPSEM} that we saw the running event.

One could argue that this analysis is implausible as it requires that the seeing event is a part of the seemingly independant running event.
On its face, this seems to imply an interdependecy between the two events, and, while it seems reasonable to say that the perception event depends on the perceived event, it is far from obvious that the perceived event depends on the perception event.
This line of argumentation, I believe, confuses the issue at hand.

Saying that the perception of an event is a subpart of that event, does imply that the event is dependent on it being percieved, but it does so in a very weak way.
It is perhaps a truism of set theory and mereology to say that two complex objects are identical only if they consist of the same parts.
So, if $x$ is a part of $e$ but not a part of $e^{\prime}$, the $e \neq e^{\prime}$.
Similarly, a particular running event which is seen by some individual $x$, cannot be identical to a running event which is not seen by $x$.
Note that this does not mean that the unseen running event is not a running event, only that it is a not a seen running event.

\end{document}

\chapter{Conclusion}\label{sec:Conclusion}
%        File: conclusion.tex
%     Created: Wed Nov 28 10:00 AM 2018 E
% Last Change: Wed Nov 28 10:00 AM 2018 E
%
% arara: pdflatex: {options: "-draftmode"}
% arara: biber
% arara: pdflatex: {options: "-draftmode"}
% arara: pdflatex: {options: "-file-line-error-style"}
\documentclass[MilwayThesis]{subfiles}

\begin{document}
The parametric variation of resultative, what I call the resultative parameter, presents a puzzle for linguistic theory:
Children acquire their language's parameter-setting despite a seeming lack of direct evidence in the primary linguistic data.
Since there is no direct evidence, the parameter setting must follow from indirect evidence.
This line of reasoning leads to a two-part research question:
What aspect of a child's PLD provides indirect evidence for the setting of the resultative parameter, and how is the parameter setting deduced from that aspect?
Each part of that question, it turns out, calls for a dissertation-length answer.
The first part is largely answered by William Snyder's dissertation \parencite{snyder1995language} and refined in his later work \parencite{snyder2001nature,snyder2012parameter,snyder2016compound}.
Snyder's answer is that children use the availability of bare stem compounding in their PLD as indirect evidence for the availability of resultatives in their target grammar.
My dissertation begins with this result, and aims to provide an answer to the second part of the question: 
A language may generate both bare stem compounds and adjectival resultatives only if its lexicon has categorizing heads without $\varphi$-features.

Answering the question ``How are resultatives linked to bare stem compounding?'' is a theoretical task, and, as with any theoretical task, it begins with an explicit litany of theoretical assumptions.
I make many of assumptions standardly made in early \nth{21} century generative syntax (Merge, the Y-model of grammar) and a number of non-standard assumptions.
First, I assume that the $\Theta$-criterion does not fully hold, and that an argument may receive multiple $\theta$-roles.
Second, I assume that Merge operates freely, provided that there are two syntactic objects to be combined.
finally, I assume that there is no operation Agree active in the Narrow Syntax.
These assumptions, as I discuss in \cref{sec:nonstandard}, despite being non-standard, actually follow from the logic of the minimalist program.

In order to provide any answer to the question of how resultatives are related to bare stem compounds, we must have an idea of what adjectival resultatives are.
That is, we must give a syntactic analysis of resultatives.
Furthermore, we must provide what I call a parametric analysis---an analysis of how a parameter may be acquired and represented in the grammar.
To that end, I discuss previous analyses in \cref{sec:litreview} before offering my own in \cref{sec:analysis}.
The syntactic analysis I offer, reproduced in \cref{fig:hammer-flat-conc}, is one in which a result phrase is adjoined to the VP and a DP undergoes sideward movement between them.
\begin{figure}[h] 
	\centering
	{\small
	\begin{forest}
	    nice empty nodes,sn edges,baseline,
	    for tree={
	    calign=fixed edge angles,
	    calign primary angle=-35,calign secondary angle=60}
	    [VP
		    [VP
			    [hammer]
			    [DP[the metal,roof,name=compV]]
		    ]
		    [resP
			    [$\langle$DP$\rangle$,name=specRes]
			    [res$^{\prime}$
				    [res]
				    [SC
					    [$\langle$DP$\rangle$,name=SCDP]
					    [flat]
				    ]
			    ]
		    ]
	    ]
	    \draw[->] (SCDP) to[out=south west, in=south] (specRes);
	    \draw[->] (specRes) to[out=south, in=south] (compV);
	\end{forest}
	}
	\caption{The structure of resultatives}
	\label{fig:hammer-flat-conc}
\end{figure}
The parametric analysis, I offer is based on a similar one by \textcite{kratzer2004building}.
According to this analysis, the presence of bare stem compounding in a child's PLD signals that the child's lexicon should admit categorizing heads without $\varphi$-features.

In order to show that resultatives depend on $\varphi$-less heads, we must show that a structure such as \cref{fig:hammer-flat-conc} can be derived only if the lexicon contains $\varphi$-less categorizing heads.
I do so in \cref{sec:deriving}, but only after discussing the latest iteration (at least at the time of this thesis) of Chomsky's syntactic theory---label theory---in \cref{sec:labels}.
According to label theory, a syntactic derivation only converges if the structure it creates can be unambiguously labelled.
In \cref{sec:deriving}, I show that the structure in \cref{fig:hammer-flat-conc} can be derived and labelled if the result adjective \textit{flat} is categorized by a $\varphi$-less head $adj_{\emptyset}$.
I then show that if \textit{flat} is categorized by $adj_{\varphi}$, the derivation either fails or creates an unlabellable structure.
Thus I have answered the question at hand.

In part II, I bring to the forefront the apparently loose theoretical ends left by Part I.
Rather than tie these loose ends up with auxiliary or ad-hoc hypotheses, I investigate how they might inform our theory of the language faculty.
In \cref{sec:FreSC}, I argue that an apparent undergeneration problem of my proposed theory is actually due to the lack of a suitable theory of feature agreement.
Such a theory, I propose, is one in which agreement occurs postsyntactically.

In \cref{sec:ACCing}, I point out an odd fact---that movement from [Spec, res] to [Comp, V] in \cref{fig:hammer-flat-conc} seems to be obligatory---and show that it seems to generalize to other cases of sideward movement---objects in the specifier of adjoined phrases must move to the host phrase.
I argue that this fact is odd only if we make the standard (although often tacit) assumption that grammaticality is determined within the Narrow Syntax.
Under an interface-based theory, such as label theory, this fact can be accounted for.

Finally, in \cref{sec:coincidence} I discuss a semantic question raised by my proposal.
I assume that primary and secondary predicates compose via something like predicate modification. 
This mode of composition leads to what initially seem to be odd interpretations in which the the events described by the two predicates are in fact the same event.
I argue that despite this apparent oddness, there is no other principled way of interpreting the structures in question (\textit{i.e.}, resultatives, depictives, and some direct perception reports), and furthermore, that the apparent oddness is only apparent.
A closer look at both structures and the ontology of eventualities significantly diminishes this oddness, but a closer investigation will be needed to corroborate the predicted interpretation.

There are, of course, loose ends left by this thesis, which will have to be tied up in later investigations.
My starting points for the thesis were works on adjectival resultatatives by \textcite{snyder1995language,snyder2001nature,snyder2016compound} and \textcite{kratzer2004building} who drew a strong correlation between adjectival/nominal inflection and adjectival resultatives.
As with all empirical generalizations, there are exceptions to this correlation.
Exceptions, of course, are tricky things in any scientific inquiry.
They can either strengthen a theory or destroy it, and there is no way to tell which they will do without a full analysis

For instance, Italian, which is one of the prototypical *resultative languages, does seem to generate a form of adjectival resultative, but only under fairly restrictive conditions.
\textcite{napoli1992secondary}, for instance, gives the following examples of Italian adjectival resultatives.
\ex.
\a. Ha dipinto la macchina rossa.\\
``He painted the car red.''
\b. 
	\a. Ho stirato la camicia piatta piatta.\\
	``I ironed the shirt flat flat.''
	\b.* Ho stirato la camicia piatta.
	\z.
\z.

\textcite{folli2005prepositions} adds to this list the following cases, in which the result AP is intensified with \textit{troppo}.
\ex.
\a. Gianni ha cucito la camicia *(troppo) stretta.\\
``John sewed the dress *(too) tight.''
\b. Gianni ha sciolto il cioccolato *(troppo) liquido.\\
``John melted the chocolate *(too) liquid.''

\textcite{napoli1992secondary} suggests a semantic/pragmatic analysis; namely, that Italian only allows resultatives when ``the verb can be interpreted as focusing on the endpoint of its activity'' (p75).
\textcite{folli2005prepositions}, on the other hand suggest a syntactic analysis; Italian only allows resultatives when the result AP is complex.
Neither analysis is complete, though, and the case of Italian resultatives remains a puzzle.

If \citeauthor{napoli1992secondary} is correct, and the case of Italian resultatives is to be given a semantic/pragmatic analysis, then my syntactic explanation of the resultative parameter will face some difficulties.
If, on the other hand, \citeauthor{folli2005prepositions} are correct that this exception is to be given a syntactic analysis, then perhaps it will only strengthen my proposal.

\textcite{whelpton2007building}, addressing Kratzer's (\citeyear{kratzer2004building}) proposal, presents Icelandic as a possible counterexample. 
Recall that Kratzer's analysis was that resultatives could only be derived if the result adjective was uninflected, a proposal that is compatible with mine.
Whelpton shows that, while Icelandic allows resultatives, it also seems to require inflectional morphology on result adjectives as in the following examples.
\ex.
\ag. \'Eg k\'yldi l\"ogguna kalda.\\
I.Nom punched cop.the.FSgAcc cold.FSgAcc\\
``I punched the cop out cold.''
\bg.J\'{a}rnsmi\dh{}urinn hamra\dh{}i \'{a}lminn flatan.\\
blacksmith.the hammered metal.the.MSgAcc flat.MSgAcc\\
``the blacksmith hammered the metal flat.''
\bg. D\'{o}ra \ae{}pti sig h\'{a}sa.\\
D\'{o}ra screamed herself.FSgAcc hoarse.FSgAcc\\
``D\'{o}ra screamed herself hoarse.''

However, Whelpton also notes that Icelandic, unlike a prototypical *resultative	language, does have bare stem compounding.
Indeed, it has bare stem compounding that is interpreted as resultatives as in the following examples where bare result adjectives are compounded with deverbal adjectives.
\ex. 
\a. svart-lita\dh{}ur\\
black-coloured.mSgNom
\b. \th{}unnsneiddu sveppirnir\\
thin-cut.MPlNom mushrooms.the
\b. f\'{i}nmuldu piparkornin\\
fine-ground.NPlNom peppercorns.the
\b. hreinskr\'{u}bbu\dh{}u p\"{o}nnurnar\\
clean-scrubbed.FPlNom pans.the
\b. mj\'{u}kbr\ae{}dda s\'{u}kkula\dh{}i\\
soft-meltedNSgNom chocolate

Whelpton presents this and other data as a rebuttal to Kratzer's (\citeyear{kratzer2004building}) analysis, but offers no deep analysis or counter-proposal.
Without a deeper analysis, it is difficult to estimate the importance of his data as counterevidence to my proposal.
Therefore I leave it to further investigation.

In addition to possible counterexamples, there are a number of phenomena related to adjectival resultatives which may be amenable to an analysis/explanation along the lines of what I propose here.
First off, there is the case of directionalized locatives such as one of the (a) reading of \Next.
\ex. Kate kicked the ball between the posts.
\a. $\approx$ Kate kicked the ball such that it passed/landed between the posts. (directionalized)
\b. $\approx$ Kate stood between the posts and kicked the ball. (plain locative)

While these PPs are standardly assumed to be PathPs like PPs headed by, say, \textit{through} or \textit{around}, I argue elsewhere \parencite{milway20xxmodifying} that such an assumption is unfounded.
Rather, directionalized locatives are perhaps analyzable as PP resultatives based on their semantics.
Furthermore, they show a parametric variation similar to that of adjectival resultatives.
So, Germanic languages seem to have directionalized locatives, but Romance languages do not.
There are, however, reports that certain varieties of Acadian French allow directionalized locatives.
For instance, according to Ruth King and Yves Roberge \parencite[p.c. cited in][253--254]{rooryck1996prepositions} report that sentences like \Next, while they only receive a plain locative reading in Metropolitan and Laurentian French, receive a directionalized locative reading in PEI French.
\ex.La bouteille flottait [sous le pont].\\
The bottle floated under the bridge. \parencite{rooryck1996prepositions}

In previous work \parencite{milway2015generals}, I hypothesized that this could be linked to the fact that, unlike Metropolitan and Laurentian French, Acadian French tends to allow P-stranding.
So, for instance The sentences in \Next are acceptable in PEI French but ungrammatical in most other varieties of French.
\ex.
\ag. Le ciment a \'{e}t\'{e} march\'{e} dedans.\\
the cement has been walked in\\
``The cement was walked in''
\bg. O\'{u} il vient de?\\
where he comes from\\
``Where does he come from?'' \parencite{roberge2013preposition}

A full analysis and explanation would require an in depth empirical study, perhaps of the sort \textcite{snyder1995language} performed on adjectival resultatives.
I leave such a study for future research.

Secondly, there are serial verb constructions (SVCs) of the type studied by \textcite{stewart2013serial,bakerstewart1999double}.
Consider, for example, the Edo SVC in \Next.
\exg. \`{O}z\'{o}  gh\'{a}d\`{i}y\'{a}n    r\`{e}.\\
Ozo  FUT   buy   yam    eat\\
``Ozo will buy yams and eat them.'' \parencite{bakerstewart1999double}

SVCs and resultatives are similar in that both involve a single argument shared between two predicates which are related to each other by more than mere coincidence.
So, in \Last, the buying event is a prerequisite of the eating event, and the latter is, in some sense the goal of the former, and yams are the theme of both events.
Also like resultatives, SVCs are parameterized, though they are rarer typologically than resultatives.
Indeed, \textcite{stewart2013serial} proposes that the two constructions are linked and that the SVC parameter may be a subparameter of the resultative parameter.
Further research would be required to integrate my results with those of \textcite{stewart2013serial,bakerstewart1999double},

Finally, there is the case of Romanian bare noun resultatives\footnote{\textcite{irimia2012secondary} calls these ``bare noun \textit{pseudo}results.''} as discussed by \textcite[220--224]{irimia2012secondary} and \textcite{farkas2011predicative}.
Romanian, like other Romance languages disallows adjectival resultatives as shown in \Next.
\exg. *Femeia a cur\u{a}\cb{t}at casa str\u{a}lucitoare.\footnotemark\\
Woman.the has cleaned.PstPrt house.the spotless.FSg\\
``The woman cleaned the house spotless.''\parencite{irimia2012secondary}

Unlike the other Romance languages, however, Romanian has a bare nominal resultative as shown in \Next.
\footnotetext{\textcite{irimia2012secondary} reports that this sentence is grammatical in Romanian, but only receives a depictive reading.}
\exg. Studentul s -a sup\u{a}rat foc.\\
student-the CL.3ReflAcc has get angry.Perf fire\\
``The student has got so angry that he became as red as fire.''\parencite{farkas2011predicative}

As the name suggests, the result nominal in a bare nominal resultative, despite the fact that Romanian allows nominal inflection.
\ex. a se sup\u{a}ra foc/*focul/*un foc/*focuri/*focurile\\
``to get angry fire/fire-the/a fire/fires/fires-the'' \parencite{farkas2011predicative}

Although this clashes with the generalization that Romance language disallow resultatives, it is entirely consistent with my proposal.
If we propose that the Romanian lexicon has the $n_{\emptyset}$ head but not the $adj_{\emptyset}$ head, then the bare noun resultative can be integrated with my analysis.
This does, however, raise the question of why the bare noun resultative does not show up in other languages.
I leave this question to future research.


The proposals made here are, of course, provisional as is the case for any scientific proposal.
That is, they are subject to revisions, clarifications, and perhaps outright refutation.
That said, I believe that with this thesis I have made two broad contributions to the ongoing study of the human language faculty.
First, I have presented a template for the explanation of parametric variation, especially parametric semantic variation.
Such variation can be explained by first finding a surface correlate of that parameter, and then showing how that correlate can be connected to the parameter.
Second, I have incrementally developed the theory of the language faculty by identifying and fixing flaws in our understanding of such things as the syntax-semantics interface and adjunction.
The flaws were found by applying the logic of the minimalist program to these domains, as were the proposed solutions to those flaws.
I believe my solutions to be intriguing and suggestive, but they may of course be dead ends.
The flaws, themselves, however, are more important; they represent domains that we previously thought we understood.
Finding gaps in our understanding, such as these, is what makes scientific inquiry worth it.
A failure of understanding is merely an opportunity to understand.
\end{document}



\printbibliography[heading=bibintoc]
\end{document}


