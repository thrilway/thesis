% arara: pdflatex: {options: "-draftmode"}
% arara: biber
% arara: pdflatex: {options: "-draftmode"}
% arara: pdflatex: {options: "-file-line-error-style"}
\documentclass[MilwayThesis]{subfiles}
\begin{document}
To understand how adjuncts behave re labelling, let's consider their basic properties: optionality, iterativity, and freedom of order.
These can be demonstrated in the series of sentences in \Next.
\ex. 
\a. The protester was brought to the police station.
\b. The protester was brought to the police station, against her will.
\b. The protester was brought to the police station, against her will, following the demonstration.
\b. The protester was brought to the police station following the demonstration, against her will.
\z.

If we assume that the adjuncts in \Last are adjoined to TP, then the TPs in each of the sentences in \Last, are, in some sense, grammatically indistinguishable.
If we take this much for granted, then we can view the task of developing a theory of adjuncts to be the task of making explicit the sense in which the sentences in \Last are indistinguishable.
Since we are assuming label theory, we can make a fairly trivial explication of the indistinguishability of these sentences: the TPs in \Last are indistinguishable in the sense that they are labelled identically.
So, supposing the sentences in \Last are constructed purely by Merge (and Select, and Copy), then they have the structures in \Next.
\ex.
\a. [$_\alpha$ The protester was brought to the police station].
\b. [$_{\beta} [_{\alpha}$The protester was brought to the police station], [against her will]].
\b. $[_{\gamma}[_{\beta}[_{\alpha}$The protester was brought to the police station], [against her will]], [following the demonstration]].
\b. $[_{\eta} [_{\delta} [_{\alpha}$The protester was brought to the police station] [following the demonstration]], [against her will]].
\z.

\end{document}
