%        File: MyAnalysis.tex
%     Created: Mon Nov 06 10:00 AM 2017 E
% Last Change: Mon Nov 06 10:00 AM 2017 E
%
% arara: pdflatex: {options: "-draftmode"}
% arara: biber
% arara: pdflatex: {options: "-draftmode"}
% arara: pdflatex: {options: "-file-line-error-style"}
\documentclass[MilwayThesis]{subfiles}
\setcounter{chapter}{3}
%\usepackage{atveryend}
%
%\BeforeClearDocument{
%	\printbibliography
%}
\begin{document}
In the previous chapter, I discussed the failings of several previous analyses of adjectival resultatives.
In this chapter I will address two of those analyses -- one structural, and one parametric -- and show how they can be modified to address the concerns raised in the previous chapter.
The structural analysis, that of \textcite{kratzer2004building}, was rejected because it did not comply with UTAH, but has three features which I will retain: a small clause structure, theme raising, and a result head.
The parametric analysis, that of \textcite{snyder1995language,snyder2012parameter}, was rejected because it did not comply with the Lexical Parameterization Hypothesis, but was based on a learnable pattern and so I will be adopting a modified version of it.
\section{Fixing the UTAH problem}
The one issue with Kratzer's analysis is that it seems to violate UTAH.
That is, there is a single $\theta$-relation between \textit{hammer} and \textit{the metal} in both sentences in \Next, that is not represented by a single structural relation.
\ex.
\a. Joe hammered the metal flat.
\b. Joe hammered the metal.

According to Kratzer's analysis, \textit{the metal} is the specifier of \textit{hammer} in \Last[a], but a standard analysis of \Last[b] will place \textit{the metal} as the complement of \textit{hammer}
\ex.
\a. hammer the metal flat \parencite[following][]{kratzer2004building}\\
\begin{forest}
    nice empty nodes,sn edges,baseline,for tree={
    calign=fixed edge angles,
    calign primary angle=-30,calign secondary angle=70}
    [VP
	    [DP[the metal,roof,name=specV]]
	    [
		    [hammer]
		    [resP
			    [res]
			    [SC
				    [$\langle$DP$\rangle$,name=SCDP]
				    [flat]
			    ]
		    ]
	    ]
    ]
    \draw[->] (SCDP) to[out=south west, in=south] (specV);
\end{forest}
\b.hammer the metal\\
\begin{forest}
    nice empty nodes,sn edges,baseline,for tree={
    calign=fixed edge angles,
    calign primary angle=-30,calign secondary angle=70}
    [VP
	    [hammer]
	    [DP[the metal,roof]]
    ]
\end{forest}

If we were to modify Kratzer's analysis, so that \textit{the metal} is the complement of \textit{hammer}, then we would need to attach the result phrase in a different position.
I propose that the result phrase is adjoined to the VP, allowing the DP to merge directly with the verb as shown in \Next.
\ex.\label{tree:hammer-flat}
{\small
\begin{forest}
    nice empty nodes,sn edges,baseline,
%    for tree={
%    calign=fixed edge angles,
%    calign primary angle=-30,calign secondary angle=60}
    [VP
	    [VP
		    [hammer]
		    [DP[the metal,roof,name=compV]]
	    ]
	    [resP
		    [$\langle$DP$\rangle$,name=specRes]
		    [
			    [res]
			    [SC
				    [$\langle$DP$\rangle$,name=SCDP]
				    [flat]
			    ]
		    ]
	    ]
    ]
    \draw[->] (SCDP) to[out=south west, in=south] (specRes);
    \draw[->] (specRes) to[out=south, in=south] (compV);
\end{forest}
}

The modified analysis no longer violates UTAH, but it introduces two issues new issues which I address in the proceeding sections.
The first issue is that the movement operation between [Spec, res] and [Comp, V] does not target a c-commanding positions.
In other words it is a sideward rather than upward movement.
The second issue is that resP and VP are now adjoined, meaning they compose by conjunction.
This is counterintuitive, however, since resultatives are inherently asymmetric, with the verb event causing the adjective state.
I will address each of these in turn below.
\subsection{Sideward movement}
In \ref{tree:hammer-flat}, the object DP moves from [Spec res] to [Comp V].
The movement ``chain'' this operation forms is problematic due to the fact that the head of the chain does not c-command the tail.
Although this type of so-called sideward movement is generally barred, \textcite{nunes2001sideward} argues for a restricted version of sideward movement.
Nunes argues that head movement and parasitic gaps both require a sideward movement operation, as they both create non-c-command dependencies.
\ex.
\a. Head Hovement\\
\begin{forest}
    nice empty nodes,sn edges,baseline,for tree={
    calign=fixed edge angles,
    calign primary angle=-30,calign secondary angle=70}
    [TP
	    [DP]
	    [
		    [T
			    [T]
			    [V,name=head]
		    ]
		    [VP
			    [$\langle$V$\rangle$,name=tail]
			    [DP]
		    ]
	    ]
    ]
    \draw[->] (tail) to[out=south, in=south] (head);
\end{forest}
\b. Parasitic gaps\\
What did Mary hear without seeing.\\
\begin{forest}
    nice empty nodes,sn edges,baseline
    [CP
	    [DP[What,roof,name=specCP]]
	    [
		    [C+T\\did,align=center]
		    [TP
			    [DP[Mary,roof]]
			    [
				    [$\langle$T$\rangle$]
				    [VP
					    [VP
						    [V\\see,align=center]
						    [$\langle$DP$\rangle$,name=CompV1]
					    ]
					    [PP
						    [P\\without,align=center]
						    [VP
							    [V\\seeing,align=center]
							    [$\langle$DP$\rangle$,name=CompV2]
						    ]
					    ]
				    ]
			    ]
		    ]
	    ]
    ]
    \draw[->] (CompV2) to[out=south west, in=south] (CompV1);
    \draw[->] (CompV1) to[out=south, in=south] (specCP);
\end{forest}

According to the standard definition of Merge, sideward movement should be impossible.
The facts of parasitic gaps and head movement, however, suggest that a possibly complex operation with the net effect of sideward movement must be active in the grammar.
I adopt the approach developed by \textcite{nunes1995diss,nunes2001sideward} as follows.

In order to explain sideward movement, Nunes hypothesizes that a movement operation is composed of a Copy operation followed by Merge.
The operation Copy adds an object X to the workspace of a derivation provided that X is contained in an already constructed syntactic object.
\ex. For a workspace W and a syntactic object X, Copy(W, X) = $W\cup \{X\}$ iff there is a syntactic object Z $\in$ W and Z contains X.

Merge, then is a simpler operation which replaces two members of a workspace with the set containing them. 
To see how a Copy+Merge theory of movement works, consider the derivation of passivization in \Next.
\ex. 
\begin{tabular}[t]{lll}
	\textbf{Stage} & \textbf{Workspace} & \\
	\cline{1-2}
	1 & $\{$[T, [ \textsc{Voice}$_{pass}$ [see, [the, boy]]]]$\}$ & Copy([the, boy])\\
	2 & $
		\begin{Bmatrix*}[l]
			\text{[the, boy]},\\
			\text{[T, [ \textsc{Voice}$_{pass}$ [see, [the, boy]]]]}
		\end{Bmatrix*}
		$ & Merge([the, boy], [T \ldots])\\
	3 & $\{$[[the, boy], [T, [ \textsc{Voice}$_{pass}$ [see, [the, boy]]]]]$\}$ &\\
\end{tabular}

The Copy+Merge theory of movement allows us to derive sideward movement by holding the copied object in the workspace while another tree is built as in the derivation of \ref{tree:hammer-flat} in \Next.
\ex.
\begin{tabular}[t]{lll}
	\textbf{Stage} & \textbf{Workspace} & \\
	\cline{1-2}
	1 & $\left\{ \text{[[the, metal], [res, [\dots]]]} \right\}$ & Copy([the, metal])\\
	2 & $
	\begin{Bmatrix*}[l]
		\text{[the, metal]},\\
		\text{[[the, metal], [res, [\dots]]]}
	\end{Bmatrix*}
	$ & Select(hammer)\\
	3 & $
	\begin{Bmatrix*}[l]
		\text{hammer},\\
		\text{[the, metal]},\\
		\text{[[the, metal], [res, [\dots]]]}
	\end{Bmatrix*}
	$ & Merge(hammer, [the, metal])\\
	4 & $
	\begin{Bmatrix*}[l]
		\text{[hammer, [the, metal]]},\\
		\text{[[the, metal], [res, [\dots]]]}
	\end{Bmatrix*}
	$ & Merge$\begin{pmatrix*}[l]\text{[hammer, [the, metal]]},\\ \text{[[the, metal], [res [\dots]]]}\end{pmatrix*}$\\
	5 & \multicolumn{2}{l}{$\left\{\text{[[hammer, [the, metal]], [[the, metal], [res [\dots]]]]}\right\}$}\\
\end{tabular}

Note that at stage 5 of the derivation in \Last the syntactic object in the workspace is representable as \ref{tree:hammer-flat}.

In order to constrain sideward movement, Nunes notes that its immediate, results such as the tree in \ref{tree:hammer-flat}, are unpronounceable.
Assuming that decisions regarding linear order depend on c-command relations, and part of linearization is deciding which copy in a movement chain is to be pronounced, we would be unable to make a definitive linearization statement for the derived structure in \Last.
In order to linearize the movement chain of \textit{the hammer}, there must be a copy which c-commands all other copies, meaning there must be a subsequent move from theme position to grammatical object position, which I represent as [Spec, AgrO]\footnote{
	The choice to include AgrO in my structures does not indicate a particular commitment on my part to the existence of such a head, but rather to the fact that Object position seems to be distinct from internal argument position, and higher than VP.
	Further, I assume that either the movement to object position is covert, or there is head raising of the verb, such that it precedes the object.
}
in \Next.
\ex.
{\small
\begin{forest}
    nice empty nodes,sn edges,baseline,
%    for tree={
%    calign=fixed edge angles,
%    calign primary angle=-30,calign secondary angle=65}
    [AgrOP
	    [DP[the metal,roof,name=obj]]
	    [
		    [AgrO]
    [VP
	    [VP
		    [hammer]
		    [DP,name=compV]
	    ]
	    [resP
		    [$\langle$DP$\rangle$,name=specRes]
		    [
			    [res]
			    [SC
				    [$\langle$DP$\rangle$,name=SCDP]
				    [flat]
			    ]
		    ]
	    ]
    ]
    ]
    ]
    \draw[->] (SCDP) to[out=south west, in=south east] (specRes);
    \draw[->] (specRes) to[out=south, in= south] (compV);
    \draw[->] (compV) to[out=south west, in=south] (obj); 
\end{forest}
}

Since the copy of \textit{the metal} in [Spec, AgrO] c-commands all of the other copies, it will be pronounced and the lower copies will be deleted at the SM interface.
So, assuming some mechanism for sideward movement, we are able to modify Kratzer's (\citeyear{kratzer2004building}) analysis of resultatives to be compatible with UTAH.

Since we have modified Kratzer's analysis, it is worth asking if our version will still compose semantically to give us the desired interpretation.
In the next section, I argue that not only are we able to retain the proper interpretation, but we are able to do so while assuming a simpler compositional system.
\section{Composing resultatives}
\textcite{kratzer2004building} adopts a neo-Davidsonian semantics for resultatives, meaning they are analyzed as descriptions of eventualities rather that merely as relations between entities.
Her syntactic analysis is given in \Next for reference.
\ex. \textbf{Kratzer's (2005) structural analysis of resultatives}\\
\begin{forest}
    nice empty nodes,sn edges,baseline,for tree={
    calign=fixed edge angles,
    calign primary angle=-30,calign secondary angle=70}
    [VP
	    [DP[the metal,roof,name=specV]]
	    [
		    [hammer]
		    [resP
			    [res]
			    [SC
				    [$\langle$DP$\rangle$,name=SCDP]
				    [flat]
			    ]
		    ]
	    ]
    ]
    \draw[->] (SCDP) to[out=south west, in=south] (specV);
\end{forest}

According to this analysis, the small clause \textit{the metal flat} is interpreted as the state description in \Next, where the domain $D_s$ is the domain of eventualities.
\ex. $\llbracket\text{SC}\rrbracket = \lambda s_s \left[ \textsc{state}(s) \& \textbf{flat}(\textbf{the\_metal})(s) \right]$

The verb \textit{hammer} is interpreted as a predicate of events.
\ex. $\llbracket\textit{hammer}\rrbracket = \lambda e_s \left[ \textsc{event}(e) \& \textbf{hammer}(e)\right]$

Note that Kratzer analyses resultative verbs as intransitives, meaning they do not take any entity arguments.
Finally, Kratzer analyses the result head as a higher order function, which expresses a causal relation between the event expressed by the verb and the state expressed by the small clause.
\ex. $\llbracket\textit{res}\rrbracket = \lambda P_{\langle s,t\rangle} \lambda e_s \exists s_s \left[\textsc{event}(e) \& \textsc{state}(s) \& P(s) \& \textsc{Cause}(s)(e)\right]$

So, for Kratzer, the typed LF of \textit{hammer the metal flat} is as in \Next.
\ex.
\begin{forest}
    nice empty nodes,sn edges,baseline,for tree={
    calign=fixed edge angles,
    calign primary angle=-30,calign secondary angle=70}
    [VP$_{\langle s,t\rangle}$
		    [hammer$_{\langle s,t\rangle}$]
		    [resP$_{\langle s,t\rangle}$
			    [res$_{\langle st, st\rangle}$]
			    [SC$_{\langle s,t\rangle}$
				    [DP$_e$]
				    [flat$_{\langle e, st\rangle}$]
			    ]
		    ]
	    ]
\end{forest}

Kratzer proposes that \textit{hammer} and the resP compose by an operation she calls Event Identification \parencite{kratzer1996severing} which, in this instance, is equivalent to Predicate Modification generalized to eventualities.
\ex. \textbf{Predicate Modification (eventuality version)}\\
If $\alpha$ is a branching node with daughters $\beta$ and $\gamma$, both of which are of type $\langle s,t\rangle$, the $\llbracket\alpha\rrbracket = \lambda e_s [\llbracket\beta\rrbracket(e) \& \llbracket\gamma\rrbracket(e)]$

So, the interpretation of the VP in \LLast, can be derived as in \Next.
\ex.
\begin{enumerate}
	\item $\llbracket$VP$\rrbracket$ = \hfill (Predicate Modification)
	\item $\lambda e_s [\llbracket\text{hammer}\rrbracket(e) \& \llbracket\text{resP}\rrbracket(e)]$ = 
	\item $\lambda e_s [ \textbf{hammer}(e) \& \exists s_s[\textsc{Cause}(s)(e) \& \textbf{flat}(\textbf{the\_metal})(s)]]$
\end{enumerate}

So, the hammering event is identical to the event of causing the flatness state.
The same compositional process can be adapted to the sideward movement structure I propose, as the VP and resP in my structures are still predicted to be predicates of eventualities.
\ex.
\begin{forest}
    nice empty nodes,sn edges,baseline,for tree={
    calign=fixed edge angles,
    calign primary angle=-30,calign secondary angle=70}
    [VP$_{\langle s,t\rangle}$
	    [VP$_{\langle s,t\rangle}$
		    [hammer$_{\langle e, st\rangle}$]
		    [DP$_e$]
	    ]
	    [resP$_{\langle s,t\rangle}$
		    [res$_{\langle st, st\rangle}$]
		    [SC$_{\langle s,t\rangle}$
			    [DP$_e$]
			    [flat$_{\langle e, st\rangle}$]
		    ]
	    ]
	    ]
\end{forest}

Thus, with these adaptations, Kratzer's analysis of resultatives can be made UTAH-compliant.
In the remainder of this chapter, I will discuss the parametric analysis of resultatives that I will be assuming.

\section{Where does the resultative parameter come from?}
In the previous chapter, I discussed two desiderata for a parametric analysis.
First, the parameter must be learnable, meaning there must be some variable in the primary linguistic data which the learner can detect and deduce a particular parameter setting from.
Second, the variable must be represented in the lexicon.
For the sake of expediency, I will refer to the variable detectable in the PLD as the surface variable, and its lexical representation as the lexical variable.

To my knowledge, there is only one proposed candidate for the surface variable in the generative literature, that is, Snyder's (\citeyear{snyder1995language,snyder2012parameter}) compounding parameter.
According to the latest version of this parameter, a language allows resultatives iff it allows bare stem compounding.
As I discussed in the previous chapter, Snyder rejects the Lexical Parameterization Hypothesis, meaning he does not propose a lexical variable, instead situating the parameter in the operations of the CI interface.
However, I will propose a lexical variable from which both the (un)availability of bare stem compounding, and the (un)availability of adjectival resultatives can be derived.

To make such a proposal, we must make the intermediate hypothesis that a language allows bare stem compounding iff it allows bare stems, meaning there should be no languages that allow for bare stems but cannot compound them together.
Now, a bare stem is merely an independent word with no inflectional material.
Words are represented in most current theories of syntax as an acategorial root merged with a category-determining functional head (following Marantz \citeyear{marantz1997no}, but see also Borer \citeyear{borer2005name} for a similar proposal).
Since roots are, by definition, featureless, any inflectional features on stems must be due to their category-determining heads.
It follows from this that the (im)possibility of bare stems derives from the presence or absence of inflectional features on category-determining heads in the lexicon.

So, if we represent inflected category-determining heads as $v_\varphi, n_\varphi, adj_\varphi, etc.$ and their bare counterparts as $v_\emptyset, n_\emptyset, adj_\emptyset, etc.$, then the lexical version of Snyder's compounding parameter can be represented as in \Next.
\ex. \textsc{lex} $\left\{ \text{includes, does not include} \right\}$ $v_\emptyset, n_\emptyset, adj_\emptyset, etc.$

Note that this is a fairly weak claim.
A stronger claim would be that compounding languages have only uninflected category-determining heads.
The weak claim, however, is sufficient for present purposes, and is therefore adopted.

This version of the compounding parameter is lexical, and therefore complies with the Lexical Parameterization Hypothesis.
Furthermore, it is learnable from the primary linguistic data, since its external manifestation is the presence or absence of inflectional morphology.
Since the inflectional morphology is detectable on the surface, its absence must also be detectable or at least deducible.
This leaves us with questions regarding the initial state of the lexicon, and which parameter setting is the default, but those questions are beyond the scope of this thesis and will be set aside.


\end{document}
