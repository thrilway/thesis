% arara: pdflatex: {options: "-draftmode"}
% arara: biber
% arara: pdflatex: {options: "-draftmode"}
% arara: pdflatex: {options: "-file-line-error-style"}
\documentclass[MilwayThesis]{subfiles}

\begin{document}
So far, I have hypothesized that a grammar generates bare stem compounds only if its lexicon contains categorizing heads without $\varphi$-features.
Given my theoretical assumptions, I will propose a syntactic analysis

If we restrict ourselves to endocentric bare-stem compounding like \textit{bourbon bar}, for instance, then our analysis must allow for the possibility of endocentricity, and the fact that a language's ability to generate these compounds depends on the properties of the categorizing heads in that language.

I also 
I will discuss three possibilities below:
	one in which a compound is formed by directly merging roots together,
	another in which two categorized roots ($\left\{ cat, \sqrt{\textsc{Root}} \right\}$, or \textit{stems}) are directly merged together,
	and a one in which a stem merges with a root.
As we shall see, only the third analysis will be capable accounting for endocentricity and parametric variation without major stipulation.
\section{Root-root compounding}
The first analysis that I will consider is on in which roots merge directly with each other as in \cref{fig:RootRoot}.
\begin{figure}[h]
	\centering
	\begin{forest}
		[$\beta$
			[$n$]
			[$\alpha$
				[$\sqrt{\textsc{Bourbon}}$]
				[$\sqrt{\textsc{Bar}}$]
			]
		]
	]
	\end{forest}
	\caption{A root-root analysis of compounding}
	\label{fig:RootRoot}
\end{figure}
Immediately, we can see that the symmetrical nature of merge renders endocentricity impossible; \textit{bourbon bar} would be indistinguishable from \textit{bar bourbon}.

Supposing, however, that this is the correct analysis, can it account for the parametic variation?
That is, can we show that \cref{ex:FreRootRoot} crashes, while \cref{ex:EngRootRoot} converges?
\ex.* $[_{\beta}\, n_{F}, [_{\alpha} \sqrt{\textsc{Bourbon}}, \sqrt{\textsc{Bar}}  ]  ]$ \label{ex:FreRootRoot}

\ex. $[_{\beta}\, n_{\emptyset}, [_{\alpha} \sqrt{\textsc{Bourbon}}, \sqrt{\textsc{Bar}}  ]  ]$ \label{ex:EngRootRoot}

Assuming that $\beta$ in either case will be labelled by $n$ assuming it is strong enough to label.
The French $n_{F}$ in \cref{ex:FreRootRoot}, of course will need to be strengthened by Agree, but since this is presumably the case for all nouns in French, I will assume that whether $\beta$ is labellable in \cref{ex:FreRootRoot} depends on noun-external factors.

Since this analysis lacks both necessary properties for compounds, I will set it aside.
\section{Stem-stem compounding}
The second analysis is one in which bare-stem compounds are created by merging two stems (\textit{i.e.}, \{$n$, $\sqrt{\textsc{root}}$\}).
Like the root-root possibility, the stem-stem possibility is symmetrical, as we can see in \cref{fig:StemStem}
\begin{figure}[h]
	\centering
	\begin{forest}
		[$\gamma$
			[$\alpha$
				[$n$]
				[$\sqrt{\textsc{bourbon}}$]
			]
			[$\beta$
				[$n$]
				[$\sqrt{\textsc{bar}}$]
			]
		]
	\end{forest}
	\caption{A stem-stem analysis of compounding}
	\label{fig:StemStem}
\end{figure}
As is the case with the root-root option, the symmetry of this structure precludes endocentricity.
Since it cannot capture endocentricity, we can set this analysis aside now, but for completeness' sake, let's consider how it fares with respect to parametric variation.
<++>

\section{Root-stem compounding}
The final possibility is that compounds are formed by merging a root with a stem, as in \cref{fig:RootStem}.
\begin{figure}[h]
	\centering
	\begin{forest}
		[$\beta$
			[$\sqrt{\textsc{bourbon}}$]
			[$\alpha$
				[$n$]
				[$\sqrt{\textsc{bar}}$]
			]
		]
	\end{forest}
	\caption{A root-stem analysis}
	\label{fig:RootStem}
\end{figure}
We can immediately see that this is an asymmetric structure, and, as such, one that is able to capture endocentricity.
So \textit{bourbon bar} names a type of bar rather than a type of bourbon by virtue of the fact that the \textit{bar} root merges directly with the category-determining head, while \textit{bourbon} does so indirectly.
Compare this with the case of a Saxon genitive, such as \textit{the president's men}, which names some men rather than the president.
This encocentricity can be captured by the structure in \cref{fig:SaxonGenitive}, by virtue of the fact that \textit{men} is the complement of the head determiner, while \textit{the president} is its specifier.
\begin{figure}[h]
	\centering
	\begin{forest}
		[DP
			[DP[the president,roof]]
			[
				['s]
				[$n$P[men,roof]]
			]
		]
	\end{forest}
	\caption{The Saxon genitive}
	\label{fig:SaxonGenitive}
\end{figure}
\end{document}
