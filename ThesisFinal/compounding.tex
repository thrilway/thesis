% arara: pdflatex: {options: "-draftmode"}
% arara: biber
% arara: pdflatex: {options: "-draftmode"}
% arara: pdflatex: {options: "-file-line-error-style"}
\documentclass[MilwayThesis]{subfiles}

\begin{document}
So far, I have demonstrated that, given my theoretical assumptions, resultatives can only be generated if the result adjective is categorized by a featureless $adj_{\emptyset}$ head.
I also proposed that a child acquires featureless categorizing heads if they encounter productive bare-stem compounding (BSC) in their PLD.
In this section I will propose an analysis of bare-stem compounding that is consistent with these proposals and my theoretical assumptions.

If we restrict ourselves to endocentric bare-stem compounding like \textit{bourbon bar}, for instance, then our analysis must be consistent with the possibility of endocentricity.
That is, a proper analysis of \textit{bourbon bar} must naturally explain why it describes a type of bar rather than a type of bourbon.
Furthermore an analysis of BSC must allow for the fact that a language's ability to generate these compounds depends on the properties of the categorizing heads in that language.

I will discuss three possibilities below:
	one in which a compound is formed by directly merging roots together,
	another in which two categorized roots ($\left\{ cat, \sqrt{\textsc{Root}} \right\}$, or \textit{stems}) are directly merged together,
	and a one in which a stem merges with a root.
As we shall see, only the third analysis can account for endocentricity and parametric variation without major stipulation.
\subsection{Root-root compounding}
The first analysis that I will consider is one in which roots merge directly with each other as in \cref{fig:RootRoot}.
\begin{figure}[h]
	\centering
	\begin{forest}
    nice empty nodes,sn edges,baseline,
		[$\beta$
			[$n$]
			[$\alpha$
				[$\sqrt{\textsc{Bourbon}}$]
				[$\sqrt{\textsc{Bar}}$]
			]
		]
	]
	\end{forest}
	\caption{A root-root analysis of compounding}
	\label{fig:RootRoot}
\end{figure}
Immediately, we can see that the symmetrical nature of merge (\textit{i.e.} the fact that merge creates an unordered set) renders endocentricity impossible; \textit{bourbon bar} would be indistinguishable from \textit{bar bourbon}.

Setting this problem aside for the moment, could this analysis account for the parametic variation?
That is, can we show that \cref{ex:FreRootRoot} crashes, while \cref{ex:EngRootRoot} converges?
\ex.* $[_{\beta}\, n_{F}, [_{\alpha} \sqrt{\textsc{Bourbon}}, \sqrt{\textsc{Bar}}  ]  ]$ \label{ex:FreRootRoot}

\ex. $[_{\beta}\, n_{\emptyset}, [_{\alpha} \sqrt{\textsc{Bourbon}}, \sqrt{\textsc{Bar}}  ]  ]$ \label{ex:EngRootRoot}

Since $n$ is the least embedded atomic element in both \cref{ex:FreRootRoot,ex:EngRootRoot}, it would label bothe phrases if it were strong enough.
The French $n_{F}$ in \cref{ex:FreRootRoot} will of course need to be strengthened by Agree, but compare the proposed compound structure with that of a simple noun in \cref{ex:FreNoun}.
\ex. $[_{\alpha} n_{F}, \sqrt{\textsc{bar}}]$\label{ex:FreNoun}

Note that in both the simple noun \cref{ex:FreNoun} and the compound noun \cref{ex:FreRootRoot}, the categorizing head $n_{F}$ is an immediate constituent of the phrase in question.
Furthermore, in both cases, $n_{F}$ is the only possible labeller, as all of the other constituents are roots.
It follows, then, that $\beta$ in \cref{ex:FreRootRoot} and $\alpha$ in \cref{ex:FreNoun} are indistinguishable with respect to labelling.
Therefore, we a root-root analysis of BSC gives us no principled way of ruling out compound nouns in French-like languages, without also ruling out simple nouns.

Since this analysis lacks both necessary properties for compounds, I will set it aside.
\subsection{Stem-stem compounding}
The second analysis is one in which bare-stem compounds are created by merging two stems (\textit{i.e.}, \{$n$, $\sqrt{\textsc{root}}$\}).
Like the root-root possibility, the stem-stem possibility is symmetrical, as we can see in \cref{fig:StemStem}
\begin{figure}[h]
	\centering
	\begin{forest}
    sn edges,baseline,
    [$\gamma$,tier=zero
	    [$\alpha$,tier=one
		    [$n$,tier=two]
				[$\sqrt{\textsc{bourbon}}$,tier=two]
			]
			[$\beta$,tier=one
				[$n$,tier=two]
				[$\sqrt{\textsc{bar}}$,tier=two]
			]
		]
	\end{forest}
	\caption{A stem-stem analysis of compounding}
	\label{fig:StemStem}
\end{figure}
As is the case with the root-root option, the symmetry of this structure precludes endocentricity.
Again setting this problem aside, let's consider how it fares with respect to parametric variation.

If we consider the compound $\gamma$ as a whole, we can see that it is a phrase-phrase structure, and therefore labellable in only two situations.
The first situation is when one of the constituent phrases has moved, which is not the case in a compound.
The second is when there is agreement between the ``heads'' of the two constituent phrases (the two $n$ heads in this case).
In the case of an English-type language, there is certainly no agreement between the two $n$ heads, as they have, by hypothesis, no features, which are required for Agree.
So, a stem-stem analysis of compounds seems to wrongly predict that English-type grammars do not generate bare-stem compounds; another strike against them.
French-like languages, on the other hand, have feature-bearing $n$ heads which are in the ideal structural configuration to agree with each other, but it is unlikely that they would undergo agreement with each other due to their featural endowment.
The standard cases of agreement involve a featural asymmetry between the agreeing heads---an interpretable feature checks an uninterpretable feature; a valued feature values an unvalued feature)---but in the case of the structure in \cref{fig:StemStem}, the would-be agreeing heads are of the same type, and therefore have identical featural endowments.
Since there is no featural asymmetry, it is likely that there can be no agreement between the two $n_{F}$ heads in French-type languages.
If they cannot agree with each other, then, just as in the case of English-type languages, we would expect the structure in \cref{fig:StemStem} to be unlabellable.
If they can agree with each other, then $\gamma$ in \cref{fig:StemStem} should be labelled $\langle F,F\rangle$, and therefore, the structure in \cref{fig:StemStem} should be a licit compound.
So, the stem-stem analysis of compounds may wrongly predict that French-type grammars do generate bare-stem compounds.

Since a stem-stem analysis accounts for neither endocentricity nor the parametric variation with respect to bare-stem compounding, we will set it aside and move on to the next possible analysis.
\subsection{Root-stem compounding}
The final possibility is that compounds are formed by merging a root with a stem, as in \cref{fig:RootStem}.
\begin{figure}[h]
	\centering
	\begin{forest}
    sn edges,baseline,
    [$\beta$,tier=zero
	    [$\sqrt{\textsc{bourbon}}$,tier=one]
	    [$\alpha$,tier=one
		    [$n$,tier=two]
		    [$\sqrt{\textsc{bar}}$,tier=two]
			]
		]
	\end{forest}
	\caption{A root-stem analysis}
	\label{fig:RootStem}
\end{figure}
We can immediately see that this is an asymmetric structure, and, as such, one that is in principle able to capture endocentricity.
So \textit{bourbon bar} names a type of bar rather than a type of bourbon by virtue of the fact that the \textit{bar} root merges directly with the category-determining head, while \textit{bourbon} does so indirectly.
Compare this with the case of a Saxon genitive, such as \textit{the president's men}, which names some men rather than the president.
This endocentricity can be captured by the structure in \cref{fig:SaxonGenitive}, by virtue of the fact that \textit{men} is the complement, merging directly with the determiner \textit{-'s}, while \textit{the president} is its specifier.
\begin{figure}[h]
	\centering
	\begin{forest}
    nice empty nodes,sn edges,baseline,
		[DP
			[DP[the president,roof]]
			[
				['s]
				[$n$P[men,roof]]
			]
		]
	\end{forest}
	\caption{The Saxon genitive}
	\label{fig:SaxonGenitive}
\end{figure}

Now what about the variability?
Let's consider how the structure in \cref{fig:RootStem} would be labelled in the case of an English-type language.
The least embedded atom in $\beta$ is the root \textsc{bourbon}, which is invisible for labelling.
Therefore we must look for the next-least-embedded atom, which, assuming the root \textsc{bar} is invisible, would be $n_{\emptyset}$.
So, the label of $\beta$ would be $n_{\emptyset}$, as it would be for $\alpha$ as well.

Now consider the French-type language, where the featureless $n_{\emptyset}$ is replaced by $n_{F}$ with an incomplete feature set as in \cref{fig:RootStemFrench}.
Recall that $n_{F}$ cannot label a phrase unless it is strengthened to $n_{\langle F,F\rangle}$ by Agree.
So, the structure in \cref{fig:RootStemFrench}, is unlabellable on its own,
\begin{figure}[h]
	\centering
	\begin{forest}
    nice empty nodes,sn edges,baseline,
		[$\beta$
			[$\sqrt{\textsc{bourbon}}$]
			[$\alpha$
				[$n_{F}$]
				[$\sqrt{\textsc{bar}}$]
			]
		]
	\end{forest}
	\caption{A French-like Root-Stem structure}
	\label{fig:RootStemFrench}
\end{figure}
Furthermore, the $n_{F}$ head is embedded beneath a root, making it a more remote target for agreement with some head to be merged later.
If $n_{F}$ is too remote to be agreed with, then it will be too weak to label $\alpha$ or $\beta$, and therefore the structure in \cref{fig:RootStemFrench} would crash.
This difficulty would not, however, emerge in the case of a simple noun (\textit{e.g.}, $\left\{ n_{F}, \sqrt{\textsc{bar}} \right\}$), because the categorizing head $n_{F}$ is not embedded, and therefore readily available for agreement with a higher head.

A Root-Stem analysis of BSC, unlike the other two alternatives described above, can capture the fact that languages without featureless $cat_{\emptyset}$ heads cannot produce bare stem compounds.
Since a Root-Stem analysis of BSC seems to be the only one that can explain both endocentricity and variablity, it is likely the correct analysis.
\end{document}
