%        File: conclusion.tex
%     Created: Wed Nov 28 10:00 AM 2018 E
% Last Change: Wed Nov 28 10:00 AM 2018 E
%
% arara: pdflatex: {options: "-draftmode"}
% arara: biber
% arara: pdflatex: {options: "-draftmode"}
% arara: pdflatex: {options: "-file-line-error-style"}
\documentclass[MilwayThesis]{subfiles}

\begin{document}
The parametric variation of resultative, what I call the resultative parameter, presents a puzzle for linguistic theory:
Children acquire their language's parameter-setting despite a seeming lack of direct evidence in the primary linguistic data.
Since there is no direct evidence, the parameter setting must follow from indirect evidence.
This line of reasoning leads to a two-part research question:
What aspect of a child's PLD provides indirect evidence for the setting of the resultative parameter, and how is the parameter setting deduced from that aspect?
Each part of that question, it turns out, calls for a dissertation-length answer.
The first part is largely answered by William Snyder's dissertation \parencite{snyder1995language} and that answer has been refined in his later work \parencite{snyder2001nature,snyder2012parameter,snyder2016compound}.
Snyder's answer is that children use that availability of bare stem compounding in their PLD as indirect evidence for the availability of resultatives in their target grammar.
My disseration begins with this result, and aims to answer the second part of the question.
\end{document}


