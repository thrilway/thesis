The parametric variation of resultative, what I call the resultative parameter, presents a puzzle for linguistic theory:
Children acquire their language's parameter-setting despite a seeming lack of direct evidence in the primary linguistic data.
Since there is no direct evidence, the parameter setting must follow from indirect evidence.
This line of reasoning leads to a two-part research question:
What aspect of a child's PLD provides indirect evidence for the setting of the resultative parameter, and how is the parameter setting deduced from that aspect?
Each part of that question, it turns out, calls for a dissertation-length answer.
The first part is largely answered by William Snyder's dissertation \parencite{snyder1995language} and refined in his later work \parencite{snyder2001nature,snyder2012parameter,snyder2016compound}.
Snyder's answer is that children use the availability of bare stem compounding in their PLD as indirect evidence for the availability of resultatives in their target grammar.
My dissertation begins with this result, and aims to provide an answer to the second part of the question: 
A language may generate both bare stem compounds and adjectival resultatives only if its lexicon has categorizing heads without $\varphi$-features.\footnote{
	This, of course, grossly oversimplifies the range of possibilities available for lexical variation.
	Restricting ourselves to categorizing heads, we can express the logical range of possibilities as in (i).
	\ex.[(i)] For each category $cat$, a non-empty subset of $\left\{ cat_{\emptyset}, cat_{F} \right\}$ is included in the lexicon.

	Furthermore, the choice of lexicon will certainly affect the grammar in a variety of ways.
	This can be seen by comparing isolating languages such as Niuean, which seem to lack any morphological agreement, to languages such as Italian, which shows a great deal of agreement morphology.
}

Answering the question ``How are resultatives linked to bare stem compounding?'' is a theoretical task, and, as with any theoretical task, it begins with an explicit litany of theoretical assumptions.
I make many of the assumptions standardly made in early \nth{21} century generative syntax (Merge, the Y-model of grammar) and a number of non-standard assumptions.
First, I assume that the $\Theta$-criterion does not fully hold and that an argument may receive multiple $\theta$-roles.
Second, I assume that Merge operates freely, provided that there are two syntactic objects to be combined.
finally, I assume that there is no operation Agree active in the Narrow Syntax.
These assumptions, as I discuss in \cref{sec:nonstandard}, despite being non-standard, actually follow from the logic of the minimalist program.

In order to provide any answer to the question of how resultatives are related to bare stem compounds, we must have an idea of what adjectival resultatives are.
That is, we must give a syntactic analysis of resultatives.
Furthermore, we must provide what I call a parametric analysis---an analysis of how a parameter may be acquired and represented in the grammar.
To that end, I discuss previous analyses in \cref{sec:litreview} before offering my own in \cref{sec:analysis}.
The syntactic analysis I offer, reproduced in \cref{fig:hammer-flat-conc}, is one in which a result phrase is adjoined to the VP and a DP undergoes sideward movement between them.
\begin{figure}[h] 
	\centering
	{\small
	\begin{forest}
	    nice empty nodes,sn edges,baseline,
	    for tree={
	    calign=fixed edge angles,
	    calign primary angle=-35,calign secondary angle=60}
	    [VP
		    [VP
			    [hammer]
			    [DP[the metal,roof,name=compV]]
		    ]
		    [resP
			    [$\langle$DP$\rangle$,name=specRes]
			    [res$^{\prime}$
				    [res]
				    [SC
					    [$\langle$DP$\rangle$,name=SCDP]
					    [flat]
				    ]
			    ]
		    ]
	    ]
	    \draw[->] (SCDP) to[out=south west, in=south] (specRes);
	    \draw[->] (specRes) to[out=south, in=south] (compV);
	\end{forest}
	}
	\caption{The structure of resultatives}
	\label{fig:hammer-flat-conc}
\end{figure}
The parametric analysis, I offer is based on a similar one by \textcite{kratzer2004building}.
According to this analysis, the presence of bare stem compounding in a child's PLD signals that the child's lexicon should admit categorizing heads without $\varphi$-features.

In order to show that resultatives depend on $\varphi$-less heads, we must show that a structure such as \cref{fig:hammer-flat-conc} can be derived only if the lexicon contains $\varphi$-less categorizing heads.
I do so in \cref{sec:deriving}, but only after discussing the latest iteration (at least at the time of this thesis) of Chomsky's syntactic theory---label theory---in \cref{sec:labels}.
According to label theory, a syntactic derivation only converges if the structure it creates can be unambiguously labelled.
In \cref{sec:deriving}, I show that the structure in \cref{fig:hammer-flat-conc} can be derived and labelled if the result adjective \textit{flat} is categorized by a $\varphi$-less head $adj_{\emptyset}$.
I then show that if \textit{flat} is categorized by $adj_{\varphi}$, the derivation either fails or creates an unlabellable structure.
Thus I have answered the question at hand.

In part II, I bring to the forefront the apparently loose theoretical ends left by Part I.
Rather than tie these loose ends up with auxiliary or ad-hoc hypotheses, I investigate how they might inform our theory of the language faculty.
In \cref{sec:FreSC}, I argue that an apparent undergeneration problem of my proposed theory is actually due to the lack of a suitable theory of feature agreement.
Such a theory, I propose, is one in which agreement occurs postsyntactically.

In \cref{sec:ACCing}, I point out an odd fact---that movement from [Spec, res] to [Comp, V] in \cref{fig:hammer-flat-conc} seems to be obligatory---and show that it seems to generalize to other cases of sideward movement---objects in the specifier of adjoined phrases must move to the host phrase.
I argue that this fact is odd only if we make the standard (although often tacit) assumption that grammaticality is determined within the Narrow Syntax.
Under an interface-based theory, such as label theory, this fact can be accounted for.

Finally, in \cref{sec:coincidence} I discuss a semantic question raised by my proposal.
I assume that primary and secondary predicates compose via something like predicate modification. 
This mode of composition leads to what initially seem to be odd interpretations in which the events described by the two predicates are in fact the same event.
I argue that despite this apparent oddness, there is no other principled way of interpreting the structures in question (\textit{i.e.}, resultatives, depictives, and some direct perception reports), and furthermore, that the apparent oddness is only apparent.
A closer look at both structures and the ontology of eventualities significantly diminishes this oddness, but a closer investigation will be needed to corroborate the predicted interpretation.

There are, of course, loose ends left by this thesis, which will have to be tied up in later investigations.
My starting points for the thesis were works on adjectival resultatives by \textcite{snyder1995language,snyder2001nature,snyder2016compound} and \textcite{kratzer2004building} who drew a strong correlation between adjectival/nominal inflection and adjectival resultatives.
As with all empirical generalizations, there are exceptions to this correlation.
Exceptions, of course, are tricky things in any scientific inquiry.
They can either strengthen a theory or destroy it, and there is no way to tell which they will do without a full analysis

For instance, Italian, which is one of the prototypical *resultative languages, does seem to generate a form of adjectival resultative, but only under fairly restrictive conditions.
\textcite{napoli1992secondary}, for instance, gives the following examples of Italian adjectival resultatives.
\ex.
\a. Ha dipinto la macchina rossa.\\
``He painted the car red.''
\b. 
	\a. Ho stirato la camicia piatta piatta.\\
	``I ironed the shirt flat flat.''
	\b.* Ho stirato la camicia piatta.
	\z.
\z.

\textcite{folli2005prepositions} add to this list the following cases, in which the result AP is intensified with \textit{troppo}.
\ex.
\a. Gianni ha cucito la camicia *(troppo) stretta.\\
``John sewed the dress *(too) tight.''
\b. Gianni ha sciolto il cioccolato *(troppo) liquido.\\
``John melted the chocolate *(too) liquid.''

\textcite{napoli1992secondary} suggests a semantic/pragmatic analysis; namely, that Italian only allows resultatives when ``the verb can be interpreted as focusing on the endpoint of its activity'' (p75).
\textcite{folli2005prepositions}, on the other hand, suggest a syntactic analysis; Italian only allows resultatives when the result AP is complex.
Neither analysis is complete, though, and the case of Italian resultatives remains a puzzle.

If \citeauthor{napoli1992secondary} is correct, and the case of Italian resultatives is to be given a semantic/pragmatic analysis, then my syntactic explanation of the resultative parameter will face some difficulties.
If, on the other hand, \citeauthor{folli2005prepositions} are correct that this exception is to be given a syntactic analysis, then perhaps it will only strengthen my proposal.

\textcite{whelpton2007building} presents Icelandic as a possible counterexample to Kratzer's (\citeyear{kratzer2004building}) proposal. 
Recall that Kratzer's analysis was that resultatives could only be derived if the result adjective was uninflected, a proposal that is compatible with mine.
Whelpton shows that, while Icelandic allows resultatives, it also seems to require inflectional morphology on result adjectives as in the following examples.
\ex.
\ag. \'Eg k\'yldi l\"ogguna kalda.\\
I.Nom punched cop.the.FSgAcc cold.FSgAcc\\
``I punched the cop out cold.''
\bg.J\'{a}rnsmi\dh{}urinn hamra\dh{}i \'{a}lminn flatan.\\
blacksmith.the hammered metal.the.MSgAcc flat.MSgAcc\\
``the blacksmith hammered the metal flat.''
\bg. D\'{o}ra \ae{}pti sig h\'{a}sa.\\
D\'{o}ra screamed herself.FSgAcc hoarse.FSgAcc\\
``D\'{o}ra screamed herself hoarse.''

However, Whelpton also notes that Icelandic, unlike a prototypical *resultative	language, does have bare stem compounding.
Indeed, it has bare stem compounding that is interpreted as resultatives as in the following examples where bare result adjectives are compounded with deverbal adjectives.
\ex. 
\a. svart-lita\dh{}ur\\
black-coloured.mSgNom
\b. \th{}unnsneiddu sveppirnir\\
thin-cut.MPlNom mushrooms.the
\b. f\'{i}nmuldu piparkornin\\
fine-ground.NPlNom peppercorns.the
\b. hreinskr\'{u}bbu\dh{}u p\"{o}nnurnar\\
clean-scrubbed.FPlNom pans.the
\b. mj\'{u}kbr\ae{}dda s\'{u}kkula\dh{}i\\
soft-meltedNSgNom chocolate

Whelpton presents this and other data as a rebuttal to Kratzer's (\citeyear{kratzer2004building}) analysis but offers no deep analysis or counter-proposal.\footnote{
	On its face, Snyder's (\citeyear{snyder2012parameter}) analysis of the resultative parameter seems to be able to account for Icelandic resultatives.
	This, however, is only true insofar as Snyder's analysis is theoretically permissible, a proposition that I dispute in \cref{sec:litreview}.
}
Without a deeper analysis, it is difficult to estimate the importance of his data as counterevidence to my proposal.
Therefore I leave it to further investigation.

In addition to possible counterexamples, there are a number of phenomena related to adjectival resultatives which may be amenable to an analysis/explanation along the lines of what I propose here.
First off, there is the case of directionalized locatives such as one of the (a) reading of \Next.
\ex. Kate kicked the ball between the posts.
\a. $\approx$ Kate kicked the ball such that it passed/landed between the posts. (directionalized)
\b. $\approx$ Kate stood between the posts and kicked the ball. (plain locative)

While these PPs are standardly assumed to be PathPs like PPs headed by, say, \textit{through} or \textit{around}, I argue elsewhere \parencite{milway20xxmodifying} that such an assumption is unfounded.
Rather, directionalized locatives are perhaps analyzable as PP resultatives based on their semantics.
Furthermore, they show a parametric variation similar to that of adjectival resultatives.
So, Germanic languages seem to have directionalized locatives, but Romance languages do not.
There are, however, reports that certain varieties of Acadian French allow directionalized locatives.
For instance, according to Ruth King and Yves Roberge \parencite[p.c. cited in][253--254]{rooryck1996prepositions} report that sentences like \Next, while they only receive a plain locative reading in Metropolitan and Laurentian French, receive a directionalized locative reading in PEI French.
\ex.La bouteille flottait [sous le pont].\\
The bottle floated under the bridge. \parencite{rooryck1996prepositions}

In previous work \parencite{milway2015generals}, I hypothesized that this could be linked to the fact that, unlike Metropolitan and Laurentian French, Acadian French tends to allow P-stranding.
So, for instance, the sentences in \Next are acceptable in PEI French but ungrammatical in most other varieties of French.
\ex.
\ag. Le ciment a \'{e}t\'{e} march\'{e} dedans.\\
the cement has been walked in\\
``The cement was walked in''
\bg. O\'{u} il vient de?\\
where he comes from\\
``Where does he come from?'' \parencite{roberge2013preposition}

A full analysis and explanation would require an in-depth empirical study, perhaps of the sort \textcite{snyder1995language} performed on adjectival resultatives.
I leave such a study for future research.

Secondly, there are serial verb constructions (SVCs) of the type studied by \textcite{stewart2013serial,bakerstewart1999double}.
Consider, for example, the Edo SVC in \Next.
\exg. \`{O}z\'{o}  gh\'{a}d\`{i}y\'{a}n    r\`{e}.\\
Ozo  FUT   buy   yam    eat\\
``Ozo will buy yams and eat them.'' \parencite{bakerstewart1999double}

SVCs and resultatives are similar in that both involve a single argument shared between two predicates which are related to each other by more than mere coincidence.
So, in \Last, the buying event is a prerequisite of the eating event, and the latter is, in some sense the goal of the former, and yams are the theme of both events.
Also like resultatives, SVCs are parameterized, though they are rarer typologically than resultatives.
Indeed, \textcite{stewart2013serial} proposes that the two constructions are linked and that the SVC parameter may be a subparameter of the resultative parameter.
Further research would be required to integrate my results with those of \textcite{stewart2013serial,bakerstewart1999double},

Finally, there is the case of Romanian bare noun resultatives\footnote{\textcite{irimia2012secondary} calls these ``bare noun \textit{pseudo}results.''} as discussed by \textcite[220--224]{irimia2012secondary} and \textcite{farkas2011predicative}.
Romanian, like other Romance languages, disallows adjectival resultatives as shown in \Next.
\exg. *Femeia a cur\u{a}\cb{t}at casa str\u{a}lucitoare.\footnotemark\\
Woman.the has cleaned.PstPrt house.the spotless.FSg\\
``The woman cleaned the house spotless.''\parencite{irimia2012secondary}

Unlike the other Romance languages, however, Romanian has a bare nominal resultative as shown in \Next.
\footnotetext{\textcite{irimia2012secondary} reports that this sentence is grammatical in Romanian, but only receives a depictive reading.}
\exg. Studentul s -a sup\u{a}rat foc.\\
student-the CL.3ReflAcc has get angry.Perf fire\\
``The student has got so angry that he became as red as fire.''\parencite{farkas2011predicative}

As the name suggests, the result nominal in a bare nominal resultative, despite the fact that Romanian allows nominal inflection.
\ex. a se sup\u{a}ra foc/*focul/*un foc/*focuri/*focurile\\
``to get angry fire/fire-the/a fire/fires/fires-the'' \parencite{farkas2011predicative}

Although this clashes with the generalization that Romance languages disallow resultatives, it is entirely consistent with my proposal.
If we propose that the Romanian lexicon has the $n_{\emptyset}$ head but not the $adj_{\emptyset}$ head, then the bare noun resultative can be integrated with my analysis.
This does, however, raise the question of why the bare noun resultative does not show up in other languages.
I leave this question to future research.


The proposals made here are, of course, provisional as is the case for any scientific proposal.
That is, they are subject to revisions, clarifications, and perhaps outright refutation.
That said, I believe that with this thesis I have made two broad contributions to the ongoing study of the human language faculty.
First, I have presented a template for the explanation of parametric variation, especially parametric semantic variation.
Such variation can be explained by first finding a surface correlate of that parameter, and then showing how that correlate can be connected to the parameter.
Second, I have incrementally developed the theory of the language faculty by identifying and fixing flaws in our understanding of such things as the syntax-semantics interface and adjunction.
The flaws were found by applying the logic of the minimalist program to these domains, as were the proposed solutions to those flaws.
I believe my solutions to be intriguing and suggestive, but they may, of course, be dead ends.
The flaws, themselves, however, are more important; they represent domains that we previously thought we understood.
Finding gaps in our understanding, such as these, is what makes scientific inquiry worth it.
A failure of understanding is merely an opportunity to understand.
