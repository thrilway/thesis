%        File: intro.tex
%     Created: Thu Jun 22 04:00 PM 2017 E
% Last Change: Thu Jun 22 04:00 PM 2017 E
%
% arara: pdflatex: {options: "-draftmode"}
% arara: biber
% arara: pdflatex: {options: "-draftmode"}
% arara: pdflatex: {options: "-file-line-error-style"}
\documentclass[MilwayThesis]{subfiles}

\begin{document}
This thesis asks a seemingly simple question: Why do some but not all languages allow their users to generate adjectival resultatives?
I write ``seemingly simple'' for reasons that are likely obvious to anyone reading this, but I will clarify the reasons as a way of introducing the actual content of the thesis that follows.
There are, as far as I can tell, two complications inherent in questions of the form \textit{Why P?}: one linguistic, one metaphysical (for lack of a better word).
The linguistic complication is that \textit{Why P?} presupposes that \textit{P}.
So, in order to answer, or even be justified in asking, the question at hand, we must first demonstrate that there is a class of expressions that can be called adjectival resultatives, and that they are not found in every language.
The metaphysical complication is due to the fact that a given \textit{Why} question has an indefinite number of true responses, yet the appropriate response depends on the level of explanation that is sought.
So, In order to answer the stated question, we must clarify the level of explanation we are seeking.

Beginning with the presupposition: is it the case that some but not all natural language grammars generate adjectival resultatives?
The first thing we need to answer that question is a working definition of adjectival resultatives, which I give in \cref{def:AdjRes} and which, in turn, depends on the definition of \textit{secondary predicate} in \cref{def:SecPred}.
\begin{defn}[Adjectival Resultative]\label{def:AdjRes}
	An \textit{adjectival resultative} is a secondary predication structure, whose secondary predicate is an adjective (phrase) and is interpreted as describing the a state directly caused by the event described by the primary predicate.
\end{defn}
\begin{defn}[Secondary Predication]\label{def:SecPred}
	A \textit{secondary predication} structure is a monoclausal structure containing constituent consisting of a verb (phrase) (V) an argument (DP) and a another element (SP) such that\\
	SP is interpreted as a predicate,\\
	and DP is an argument of both V and SP.
\end{defn}

A canonical example of an adjectival resultative is give in (AR).
\AREx

This is a secondary predication structure in the sense that it contains a constituent \textit{hammer the metal flat} which consists of a verbal and adjectival predicate (\textit{hammer} and \textit{flat}, respectively) and a DP \textit{the metal} which is the argument of both predicates.
Furthermore, it is an adjectival resultative because its secondary predicate is the adjective \textit{flat}, which describes a state caused by the hammering event.
Resultatives contrast minimally with depictives, secondary predication structures whose secondary predicate discribes a state not caused by the primary predicate.
\end{document}


