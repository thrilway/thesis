%        File: intro.tex
%     Created: Thu Jun 22 04:00 PM 2017 E
% Last Change: Thu Jun 22 04:00 PM 2017 E
%
% arara: pdflatex: {options: "-draftmode"}
% arara: biber
% arara: pdflatex: {options: "-draftmode"}
% arara: pdflatex: {options: "-file-line-error-style"}
\documentclass[MilwayThesis]{subfiles}

\usepackage{atveryend}

\BeforeClearDocument{
	\printbibliography
}
\addbibresource{../Thesis.bib}
\begin{document}
This thesis asks a seemingly simple question: Why do some but not all languages allow their users to generate adjectival resultatives?
I write ``seemingly simple'' for reasons that are likely obvious to anyone reading this, but I will clarify the reasons as a way of introducing the actual content of the thesis that follows.
There are, as far as I can tell, two complications inherent in questions of the form \textit{Why P?}: one linguistic, one metaphysical (for lack of a better word).
The linguistic complication is that \textit{Why P?} presupposes that \textit{P}.
So, in order to answer, or even be justified in asking, the question at hand, we must first demonstrate that there is a class of expressions that can be called adjectival resultatives, and that they are not found in every language.

\end{document}


