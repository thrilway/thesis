%        File: intro.tex
%     Created: Thu Jun 22 04:00 PM 2017 E
% Last Change: Thu Jun 22 04:00 PM 2017 E
%
% arara: pdflatex: {options: "-draftmode"}
% arara: biber
% arara: pdflatex: {options: "-draftmode"}
% arara: pdflatex: {options: "-file-line-error-style"}
\documentclass[MilwayThesis]{subfiles}

\usepackage{atveryend}

\BeforeClearDocument{
	\printbibliography
}
\addbibresource{../Thesis.bib}
\begin{document}
What follows would likely be placed by contemporary linguists under the broad umbrella of theoretical syntax for what seem to me to be purely negative reasons.
This thesis does not contain novel data gathered from fieldwork, laboratory techniques, or corpora, and, since it does not fit into any of those categories, it is considered ``theoretical.''
This negative definition of theoretical linguistics, however, does not properly characterize the variety of work that it subsumes.
\textcite{chametzky1996theory} argues that three distinct types of work are often subsumed under the name \textit{theoretical linguistics}: metatheoretical work, heoretical work, and analytical work, which he defines as follows.
\begin{quotation}
	\textit{Metatheoretical} work is theory of theory, and divides into two sorts: general and (domain) specific. 
	General metatheoretical work is concerned with developing and investigating adequacy conditions for any theory in any domain\dots
	Specific metatheoretical work is concerned with adequacy conditions for theory in a particular domain\dots

	\textit{Theoretical} work is concerned with developing and investigating primitives, derived concepts and architecture within a particular domain of inquiry.
	This work will also deploy and test concepts developed in metatheoretical work against the results of actual theory construction in a domain, allowing for both evaluating of the domain theory and sharpening of the metatheoretical concepts.
	Note this well: \textbf{deployment} of metatheoretical concepts is \textit{not} metatheoretical work; it is theoretical work.

	\textit{Analytic} work is concerned with investigating the (phenomena of the) domain in question.
	It deploys and tests concepts and architecture developed in theoretical work, allowing for both understanding of the domain and sharpening of the theoretical concepts.
	Note this well: \textbf{deployment} of theoretical concepts is \textit{not} theoretical work, it is analytic work.
	\quelle{(xvii--xviii)}
\end{quotation}
I adopt this trichotomy along with Chametzky's caveat that no actual work in linguistics fits neatly into any one of these categories.
Metatheory and theory bleed into each other as do theory and analysis.
I add a further caveat that, since Chametzky's definitions are all relative to a domain the the phenomena of that domain, the application of those definitions will depend on what we construe to be our empirical domain.
<+MoreHere+>

Generative theories of linguistics are and have always been computational theories, meaning the mental procedures they hypothesize are local and syntactic.
Since the terms \textit{computational}, \textit{local}, and \textit{syntactic}, are all quite ambiguous and none seem to have retained their original primary interpretations, it is worth defining them.
\ex. A procedure $P$ is local/syntactic/computational iff\\
for every object $X$ in the domain of $P$,\\
the value of $P(X)$ depends solely on properties of $X$.

This means that certain aspects of language will not be amenable to a generative theory.
For instance, many aspects of language use are very context-dependent and therefore are likely to be excluded from the domain of generative theories of linguistics.
\end{document}


