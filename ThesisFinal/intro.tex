%        File: intro.tex
%     Created: Thu Jun 22 04:00 PM 2017 E
% Last Change: Thu Jun 22 04:00 PM 2017 E
%
% arara: pdflatex: {options: "-draftmode"}
% arara: biber
% arara: pdflatex: {options: "-draftmode"}
% arara: pdflatex: {options: "-file-line-error-style"}
\documentclass[MilwayThesis]{subfiles}

\begin{document}
This thesis asks a seemingly simple question: Why do some but not all languages allow their users to generate adjectival resultatives?
I write ``seemingly simple'' for reasons that are likely obvious to anyone reading this, but I will clarify the reasons as a way of introducing the actual content of the thesis that follows.
There are, as far as I can tell, two complications inherent in questions of the form \textit{Why P?}: one linguistic, one metaphysical (for lack of a better word).
The linguistic complication is that \textit{Why P?} presupposes that \textit{P}.
So, in order to answer, or even be justified in asking, the question at hand, we must first demonstrate that there is a class of expressions that can be called adjectival resultatives, and that they are not found in every language.
The metaphysical complication is due to the fact that a given \textit{Why} question has an indefinite number of true responses, yet the appropriate response depends on the level of explanation that is sought.
So, In order to answer the stated question, we must clarify the level of explanation we are seeking.

Beginning with the presupposition: is it the case that some but not all natural language grammars generate adjectival resultatives?
The first thing we need to answer that question is a working definition of adjectival resultatives, which I give in \cref{def:AdjRes} and which, in turn, depends on the definition of \textit{secondary predicate} in \cref{def:SecPred}.
\begin{defn}[Adjectival Resultative]\label{def:AdjRes}
	An \textit{adjectival resultative} is a secondary predication structure, whose secondary predicate is an adjective (phrase) and is interpreted as describing the a state directly caused by the event described by the primary predicate.
\end{defn}
\begin{defn}[Secondary Predication]\label{def:SecPred}
	A \textit{secondary predication} structure is a monoclausal structure containing constituent consisting of a verb (phrase) (V) an argument (DP) and another element (SP) such that\\
	SP is interpreted as a predicate,\\
	and DP is an argument of both V and SP.
\end{defn}

A canonical example of an adjectival resultative is give in (AR).
\AREx

This is a secondary predication structure in the sense that it contains a constituent \textit{hammer the metal flat} which consists of a verbal and adjectival predicate (\textit{hammer} and \textit{flat}, respectively) and a DP \textit{the metal} which is the argument of both predicates.
Furthermore, it is an adjectival resultative because its secondary predicate is the adjective \textit{flat}, which describes a state caused by the hammering event.
Resultatives contrast minimally with depictives, secondary predication structures whose secondary predicate discribes a state not caused by the primary predicate.
The sentence in \cref{ex:DepictiveCanon}, is a canonical example of a depictive.
\ex.\label{ex:DepictiveCanon} Heather ate the fish raw.

This is an example of secondary predication, with \textit{ate} being the verb, \textit{raw} being the secondary predicate, and the argument \textit{the fish} being shared between the two.
It is not a resultative because, in the situation it describes, the rawness state is in no way caused by the eating event.
So, part of the presupposition is true: Resultattives exist in a least one language.
\textcite{snyder1995language,snyder2001nature}, however, demonstrates that resultatives exist in a number of other languages, including ASL, Dutch, German, Khmer, Japanese, Korean, Hungarian, Mandarin, and Thai.
\ex. <+Resultatives+>

Furthermore, Snyder demonstrates that a number of languages seem to be incapable of generating resultatives, expressing resultatives periphrastically instead.
\ex. <StarResultatives+>

Our presupposition, then, seems to hold; some, but not all languages generate adjectival resultatives.

Our second issue---that of deciding what we mean by \textit{why}---I believe is a far more interesting one, as answering it requires us to be explicit about the broader goals of our inquiry.
If our interest is historical linguistics or language variation and change, then we might be interested in the migration patterns and language contact situations, and how they do or do not correlate with a language's ability to generate resultatives, or with the social factors linked to resultatives.
This thesis, however, is a work of largely theoretical generative syntax, so our \textit{why} question is actually two questions: What essential property (or properties) do grammars that generate resultatives have that grammars that do not generate resultatives lack? And how is that property (or set of properties) acquirable by children from the primary linguistic data?
Note that I have framed the acquisition question as dependent on the grammatical question---likely a reflection of my training as a syntactician---but I don't believe that one question is \textit{logically} prior to the other.
\textcite{snyder1995language,snyder2001nature}, for instance, takes the grammatical question to be dependent on the acquisition.
The questions, I believe are interdependent, meaning that the correct answer to one should at least be consistent with the correct answer to the other.
The simplest situation, however, would be that the correct answer to each question entails the correct answer; in other words, a single statement would provide an aswer to both questions.

For reasons that have little to do with the content of linguistic theory or it's empirical base, and a great deal to do with the social, cultural, and political atmosphere of modern scientific research, the two questions that I pose above are not commonly treated as interdependent.
Syntacticians focus on the grammatical question, and consider the aquistion question to be secondary at best, while aqcuistionists consider the reverse to be the case.

\end{document}


