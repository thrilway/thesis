%        File: structres.tex
%     Created: Thu Apr 20 10:00 AM 2017 E
% Last Change: Thu Apr 20 10:00 AM 2017 E
%
% arara: pdflatex: {options: "-draftmode"}
% arara: biber
% arara: pdflatex: {options: "-draftmode"}
% arara: pdflatex: {options: "-file-line-error-style"}
\documentclass[MilwayThesis]{subfiles}

\begin{document}
A structural analysis of adjectival resultatives such as (AR) should explain two general facts about them.
\AREx{}

The first fact is that the object of a resultative is interpreted as the theme of both the verb and the adjective; \textit{the metal} is hammered and \textit{the metal} is/becomes flat.
The second fact is that the verb and the adjective interpreted as describing a cause-effect relation; the \textit{hammer}ing event causes the \textit{flat}ness state.
In this section I will consider the current structural analyses of resultatives, and how they account or fail to account for the two general facts.

\subsection{The resultative theme is shared by V and Adj}
The first fact implicates the notion of $\Theta$-roles which has been an important notion in generative syntax since at least as far back as GB theory \parencite{chomsky1981lectures}.
Research on the syntax of $\Theta$-roles has to this day largely been guided by the Uniformity of Theta Hypothesis (UTAH), given, in its original form, below.
\ex. \textbf{The Uniformity of Theta Hypothesis (UTAH)}\\
Identical thematic relationships between items are represented by identical structural relationships between those items at the level of D-structure. \parencite[46]{baker1988incorporation}

Although UTAH, as formulated by Baker, requires clarification to be applied to specific data, it provides a good enough guideline for my discussion of the first fact about resultatives.
If we consider the thematic relations between verbs and the bolded DPs in \Next, and adjectives and the bolded DPs in \NNext we can see that they are identical to the corresponding thematic relations in (AR).
\ex.\label{ex:simple-hammer} 
\a. Natalie hammered \textbf{the metal} for an hour.
\b. \textbf{The metal} was hammered by Natalie.

\ex.\label{ex:simple-flat} 
\a. \textbf{The metal} is flat.
\b. Anaximander considered \textbf{the earth} flat.

Following UTAH then, we expect that the relevant V-DP structural relationships in \LLast should be identical to that in (AR), as should the Adj-DP structural relationships in \Last.
In both cases, the standard assumption is that both structural relationships are local ones (\textit{i.e.} sisterhood relationships).
It follows, then, that in (AR), \textit{the metal} is sister to both \textit{flat} and \textit{hammer}.
The only way this is possible is if \textit{the metal} undergoes sideward movement from AdjP to VP as in \Next.
\ex.\label{fig:hammer-flat-swm1} 
\begin{forest}
nice empty nodes,sn edges,baseline
  [VP
    [
      [hammer]
      [DP [the metal, roof]]
    ]
    [AdjP
      [DP [the metal, roof]]
      [flat]
    ]
  ]
\end{forest}

This proposed sideward movement is generally ruled out in standard minimalist theories of syntax \parencite{chomsky2013problems,collins2016formalization}, but this type of mutiple $\Theta$-marking presents new reason to believe that something like sideward movement exists.\footnote{
  \textcite{nunes2001sideward}, for instance, argues that sideward movement is required for parasitic gaps and head movement.
}
\textcite{nunes2001sideward} and \textcite{hornstein2009theory} propose a modified model of grammar that allows a suitably constrained theory of sideward movement.
According to Nunes and Hornstein, the faculty of language is not only Merge, but a suite of other operations including Copy.
In order to make this proposal compatible with SMT, \textcite[3]{hornstein2009theory} argues that Copy is required for general cognition, and is therefore not an ad-hoc complication of the language faculty that should be eliminated.

In this model of grammar, every instance of Merge is preceded by either Select, which draws a lexical item from a lexical array, or Copy, which creates a copy of a syntactic object which has already been merged.
The former case is external Merge, while the latter is internal Merge.
To see how this model works, consider the derivation \NNext of the abstract tree \Next.
\ex.
\begin{forest}
  nice empty nodes,sn edges,baseline
  [A
    [X]
    [B
      [Y]
      [X]
    ]
  ]
\end{forest}

\ex.
\begin{tabular}[t]{llll}
  & Workspace & Lexical Array & \\
  \hline
  1 & & X, Y & Select(X) \\
  2 & X & Y & Select(Y) \\
  3 & X, Y &  & Merge(X,Y)\\
  4 & $\left[_\text{B} \text{Y, X}  \right]$ & & Copy(X)\\
  5 & $\left[_\text{B} \text{Y, X} \right]$ X & & Merge(X, Z)\\
  6 & $\left[_\text{A} \text{X} \left[_\text{B} \text{Y, X} \right] \right]$ & & \\
\end{tabular}

Sideward movement, then, is generated by copying a syntactic object X contained in Z and merging X with some unconnected object Y.
The result is two separate objects W and Z, each containing a copy of X, which can then be merged together.
The abstract tree \Next, then can be derived as shown in \NNext.
\ex.
\begin{forest}
  nice empty nodes,sn edges,baseline
  [A
    [B
      [Z]
      [X]
    ]
    [C
      [X]
      [Y]
    ]
  ]
\end{forest}

\ex.
\begin{tabular}[t]{llll}
  & Workspace & Lexical Array & \\
  1 & & X, Y, Z & Select(X) \\
  2 & X & Y, Z & Select(Y) \\
  3 & Y, X & Z & Merge(X, Y) \\
  4 & $\left[_\text{C} \text{X, Y} \right]$ & Z & Copy(X)\\
  5 & $\left[_\text{C} \text{X, Y} \right]$, X & Z & Select(Z)\\
  6 & $\left[_\text{C} \text{X, Y} \right]$, X, Z & & Merge(Z, X)\\
  7 & $\left[_\text{C} \text{X, Y} \right]$, $\left[_\text{B} \text{Z, X} \right]$ & & Merge(B, C)\\
  8 & $\left[_\text{A} \left[_\text{B} \text{Z, X} \right], \left[_\text{C} \text{X, Y} \right]\right]$ & &\\
\end{tabular}

Substituting the abstract items X, Y, and Z with the syntactic objects \textit{the metal}, \textit{flat}, and \textit{hammer}, respectively, we can derive \ref{fig:hammer-flat-swm1} as shown in \Last.

As \textcite{nunes2001sideward} correctly notes, sideward movement structures such as \Last and \ref{fig:hammer-flat-swm1} are not linearizable.
The copies (X and \textit{the metal}) do not stand in a c-command relation with each other, and asymmetric c-command is generally assumed to be necessary for linearization decisions.
Nunes, who looks at head movement and parasitic gaps, argues that sideward movement structures are linearizable if some morphological process overrides antisymmetry (as in the case of head movement), or a further movement operation creates a copy that c-commands both extant copies.
In the case of parasitic gaps, this second movement is an a-bar movement, either Wh-movement or topicalization.
\ex. 
\a. [Which book]$_i$ did [everyone review $t_i$] [without reading $t_i$].
\b. Clarissa$_i$, Jennifer spoke to $t_i$ without meeting $t_i$. 

In the case of resultatives, the second move is to grammatical object position, which I represent in \Next as [Spec AgrO].
I assume that movement from theme position to object position occurs in simple transitive clauses, and that the verb raises to a position above AgrO.
\ex.
\begin{forest}
  nice empty nodes,sn edges,baseline
  [AgrOP
    [DP [the metal,roof]]
    [
      [AgrO]
      [VP
	[
	  [hammer]
	  [DP [the metal, roof]]
	]
	[AdjP
	  [DP [the metal, roof]]
	  [flat]
	]
      ]
    ]
  ]	
\end{forest}

So, the fact that the object of a resultative is interpreted as the argument of both the verb and the adjective (the first fact) can be explained if the derivation of a resultative involves sideward movement of the object from AdjP to VP and subsequent movement to [Spec AgrOP].
This results in the structure in \Last.

Furthermore, the logic of UTAH argues against a complex predicate analysis of adjectival resultatives, such as that of \textcite{snyder1995language,irimia2012secondary}.
Consider the complex predicate analysis of (AR) given in \Next.
\ex. 
\begin{forest}
  nice empty nodes,sn edges,baseline
  [VP
	[DP [the metal, roof]]
	[X
		[V\\hammer,align=center]
		[Adj\\flat,align=center]
	]
  ]
\end{forest}

What does our UTAH logic say about the $\Theta$-marking of \textit{the metal}?
If $\Theta$-marking occurs locally, we must ask what $\Theta$-marker is the DP local to.
Depending on what type of object the complex predicate X is, the DP is either local to both V and Adj, or it is local to neither.
If it is local to neither, then it can't be $\Theta$-marked, which rules this option out.
If it is local to both, or, more precisely, an amalgam of the two, then it is $\Theta$-marked by both, which is the desired outcome.

So, let's assume that the DP in \Last is $\Theta$-marked by an amalgam of V and Adj.
The next question is how are V and Adj amalgamated.
There is already very good reason to believe that heads can amalgamate, usually through head movement, for instance V-to-T movement in \textit{e.g.} French.
The operation of head movement is poorly understood, but we can still ask if there are reasons to believe that complex predicates are formed by the same operation as French finite verbs.



The explanation of causativity (the second fact), given below, must be compatible with such a derivation.

\subsection{The verb event causes the adjective event}
The second fact raises the notion of causativity, which is a topic often linked to transitivity alternations such as the causative-inchoative alternation, demonstrated in \Next.
\ex.
\a. \textbf{Inchoative}\\
The toast burned.
\b. \textbf{Causative}\\
Jeff burned the toast.

Specifically, \Last[b] is called a causative alternant because it means that an action of Jeff's caused the event described by \Last[b].
Within generative grammar, \Last[b] is commonly assumed to be in some way derived from \Last[a].\footnote{
  See \textcite{fodor1970three} for a dissenting view.
}
This derivation is generally achieved by augmenting inchoative \textit{burn} with a causative functional head to form causative \textit{burn}.
The content of that functional head is an open question, but the only relevant part is the causativity.
Following \textcite{kratzer2004building} and \textcite{pietroski2003small}, I assume that the causativity is a relation between eventualities, such that the interpretation of \Last[b] is roughly as given below in \Next.
\ex. $\exists e \exists f [ \textsc{Agent}(e)(\textbf{jeff}) \& \textsc{Cause}(f)(e) \& \textbf{burning}(f) \& \textsc{Theme}(f)(\textbf{the\_toast})]$

So, if Jeff burned the toast, then he performed some action and that action caused the event of the toast burning.

I will assume that there is a res head which encodes the \textsc{Cause} relation.
Given our structure so far for resultatives, there are two concievable positions to place this res head: (i) Merged above the VP (c-commanding both the VP and the AdjP) or (ii) merged with the AdjP (c-commanding only the AdjP).
\ex.
\a. High-merged res\\
\begin{forest}
  nice empty nodes,sn edges,baseline
  [resP
    [res]
      [VP
	[
	  [hammer]
	  [DP [the metal, roof]]
	]
	[AdjP
	  [DP [the metal, roof]]
	  [flat]
	]
      ]
    ]
  \end{forest}
\b. Low-merged res\\
\begin{forest}
  nice empty nodes,sn edges,baseline
  [VP
    [
      [hammer]
      [DP [the metal,roof]]
    ]
    [resP
      [res]
	[AdjP
	  [DP [the metal, roof]]
	  [flat]
	]
      ]
    ]
\end{forest}

High-merged res will not be able to mediate the hammering event description and the flatness state, so I will assume the low-merged version.
In this case, the resP is adjoined to the VP, and the interpretation of the combined structure will be the conjunction of the two.
\ex.
\a. $\llbracket\text{resP}\rrbracket = \lambda e \exists f [\textsc{Cause}(f)(e) \& \textbf{flat}(f) \& \textsc{Theme}(f)(\textbf{the\_metal})]$
\b. $\llbracket\text{VP}\rrbracket = \lambda e [\textbf{hammer}(e) \& \textsc{Theme}(e)(\textbf{the\_metal})]$
\c. $\llbracket\left[ \text{VP, resP} \right]\rrbracket = \lambda e [\llbracket\text{VP}\rrbracket(e) \& \llbracket\text{resP}\rrbracket(e)]$

\end{document}


