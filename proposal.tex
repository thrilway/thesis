%        File: proposal.tex
%     Created: Sun Nov 22 01:00 PM 2015 E
% Last Change: Sun Nov 22 01:00 PM 2015 E
%
% arara: pdflatex: {options: "-draftmode"}
% arara: biber
% arara: pdflatex: {options: "-draftmode"}
% arara: pdflatex: {options: "-file-line-error-style"}
\documentclass{beamer}

\usepackage[backend=biber,style=authoryear-comp,useprefix=false]{biblatex}

\DeclareNameFormat{labelname:poss}{% Based on labelname from biblatex.def
  \ifcase\value{uniquename}%
    \usebibmacro{name:last}{#1}{#3}{#5}{#7}%
  \or
    \ifuseprefix
      {\usebibmacro{name:first-last}{#1}{#4}{#5}{#8}}
      {\usebibmacro{name:first-last}{#1}{#4}{#6}{#8}}%
  \or
    \usebibmacro{name:first-last}{#1}{#3}{#5}{#7}%
  \fi
  \usebibmacro{name:andothers}%
  \ifnumequal{\value{listcount}}{\value{liststop}}{'s}{}}

\DeclareFieldFormat{shorthand:poss}{%
  \ifnameundef{labelname}{#1's}{#1}}

\DeclareFieldFormat{citetitle:poss}{\mkbibemph{#1}'s}

\DeclareFieldFormat{label:poss}{#1's}

\newrobustcmd*{\posscitealias}{%
  \AtNextCite{%
    \DeclareNameAlias{labelname}{labelname:poss}%
    \DeclareFieldAlias{shorthand}{shorthand:poss}%
    \DeclareFieldAlias{citetitle}{citetitle:poss}%
    \DeclareFieldAlias{label}{label:poss}}}

\newrobustcmd*{\posscite}{%
  \posscitealias%
  \textcite}

\newrobustcmd*{\Posscite}{\bibsentence\posscite}

\newrobustcmd*{\posscites}{%
  \posscitealias%
  \textcites}

\bibliography{Thesis}

\title{Thesis Title}
\date[Proposal]{Thesis Proposal}
\author[Milway]{Dan Milway}

\begin{document}
\begin{frame}
	\titlepage
\end{frame}
\begin{frame}
	\frametitle{Outline}
	\tableofcontents
\end{frame}
\section{Background}
\begin{frame}{Background}
	\frametitle{Explanatory Adequacy}
	\begin{quote}
		Although even descriptive adequacy on a large scale is by no means easy to approach, it is crucial for the productive development of linguistic theory that much higher goals than this be pursued. 
		To facilitate the clear formulation of deeper questions, it is useful to consider the abstract problem of constructing an “acquisition model” for language, that is, a theory of language learning or grammar construction.
		\parencite[pp.24-25]{chomsky1965aspects}
	\end{quote}
\end{frame}
\section{Timeline}
\begin{frame}
	\frametitle{Timeline}
	????
\end{frame}
\begin{frame}{References}
	\printbibliography
\end{frame}
\end{document}


